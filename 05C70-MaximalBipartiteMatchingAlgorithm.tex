\documentclass[12pt]{article}
\usepackage{pmmeta}
\pmcanonicalname{MaximalBipartiteMatchingAlgorithm}
\pmcreated{2013-03-22 12:40:11}
\pmmodified{2013-03-22 12:40:11}
\pmowner{mathcam}{2727}
\pmmodifier{mathcam}{2727}
\pmtitle{maximal bipartite matching algorithm}
\pmrecord{7}{32943}
\pmprivacy{1}
\pmauthor{mathcam}{2727}
\pmtype{Algorithm}
\pmcomment{trigger rebuild}
\pmclassification{msc}{05C70}

% this is the default PlanetMath preamble.  as your knowledge
% of TeX increases, you will probably want to edit this, but
% it should be fine as is for beginners.

% almost certainly you want these
\usepackage{amssymb}
\usepackage{amsmath}
\usepackage{amsfonts}

% used for TeXing text within eps files
%\usepackage{psfrag}
% need this for including graphics (\includegraphics)
%\usepackage{graphicx}
% for neatly defining theorems and propositions
%\usepackage{amsthm}
% making logically defined graphics
%%%\usepackage{xypic} 

% there are many more packages, add them here as you need them

% define commands here
\begin{document}
The maximal bipartite matching algorithm is similar some ways to the Ford-Fulkerson algorithm for network flow.  This is not a coincidence; network flows and matchings are closely related.  This algorithm, however, avoids some of the overhead associated with finding network flow.

The basic idea behind this algorithm is as follows:
\begin{enumerate}
\item Start with some (not necessarily maximal) matching $M$.
\item Find a path that alternates with an edge $e_1 \not\in M$, followed by an edge $e_2 \in M$, and so on, ending with some edge $e_f \not\in M$.
\item For each edge $e$ in the path, add $e$ to $M$ if $e \not\in M$ or remove $e$ from $M$ if $e \in M$.  Note that this must increase $|M|$ by $1$.
\item Repeat until we can no longer augment the matching in this manner.
\end{enumerate}

The algorithm employs a clever labeling trick to find these paths and to ensure that the set of edges chosen remains a valid matching.

The algorithm as described here uses the matrix form of a bipartite graph.  Translating the matching from a matrix to a graph is straightforward.

There are two phases to this algorithm: labeling and flipping.

\paragraph{Labeling}

We begin with a matrix with $R$ rows and $C$ columns containing $0$s, $1$s, and $1*$s, where a $1*$ indicates in edge in the matching and a $1$ indicates an edge not in the matching.  Number the columns $1 \dots C$ and number the rows $1 \dots R$.

Start by labeling each column that contains no $1*$s with the symbol $\#$.

Now we scan the columns.  Scan each column $i$ that has been labelled but not scanned.  Find each $1$ in column $i$ that is in an unlabelled row; label this row $i$.  Mark column $i$ as scanned.

Next, we scan the rows.  Scan each row $j$ that has been labelled but not scanned.  Find the first $1*$ in row $j$.  Label the column in which it appears $j$, and mark row $j$ as scanned.  If there is no $1*$ in row $j$, proceed to the flipping phase.

Otherwise, go back to column scanning.  Continue scanning and labelling until there are no labelled, unscanned rows or columns; at that point, the set of $1*$s is a maximal matching.

\paragraph{Flipping}

We enter the flip phase when we scan some row $j$ that contains no $1*$.  This row must have some label $c$, and in column $c$, row $j$ of the matrix, there must be a $1$; change this to a $1*$.

Now consider column $c$; it has some label $r$.  If $r$ is $\#$, clear all the labels and mark all rows and columns unscanned, and begin the labeling phase again.  Otherwise, change the $1*$ at column $c$, row $r$ to a $1$.

Move on to row $r$ and scan this row.

\paragraph{Notes}

The algorithm must begin with some matching; we may begin with the empty set (or a single edge), since that is always a matching.  However, each iteration through the process increases the size of the matching by exactly one.  Therefore, we can make a simple optimization by starting with a larger matching.  A na\"ive greedy algorithm can quickly choose a valid matching that is usually close to the size of the maximal matching; we may initalize our matrix with that matching to give the procedure a head start.
%%%%%
%%%%%
\end{document}
