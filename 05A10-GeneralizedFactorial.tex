\documentclass[12pt]{article}
\usepackage{pmmeta}
\pmcanonicalname{GeneralizedFactorial}
\pmcreated{2013-03-22 16:08:31}
\pmmodified{2013-03-22 16:08:31}
\pmowner{gilbert_51126}{14238}
\pmmodifier{gilbert_51126}{14238}
\pmtitle{generalized factorial}
\pmrecord{7}{38220}
\pmprivacy{1}
\pmauthor{gilbert_51126}{14238}
\pmtype{Definition}
\pmcomment{trigger rebuild}
\pmclassification{msc}{05A10}

% this is the default PlanetMath preamble.  as your knowledge
% of TeX increases, you will probably want to edit this, but
% it should be fine as is for beginners.

% almost certainly you want these
\usepackage{amssymb}
\usepackage{amsmath}
\usepackage{amsfonts}

% used for TeXing text within eps files
%\usepackage{psfrag}
% need this for including graphics (\includegraphics)
%\usepackage{graphicx}
% for neatly defining theorems and propositions
%\usepackage{amsthm}
% making logically defined graphics
%%%\usepackage{xypic}

% there are many more packages, add them here as you need them

% define commands here

\begin{document}
\emph{Definition.}
The \PMlinkescapetext{\emph{factorial}} $[x,d]_n$ is defined for any number $x \in \mathbb{C}$, any stepsize $d \in \mathbb{C}$ and any \PMlinkescapetext{length} $n \in \mathbb{Z}$, except for $-x \in \{ d,2d,\dots,nd \}$, by
\begin{eqnarray*}
\displaystyle [x,d]_n := \begin{cases} \prod_{j=0}^{n-1}(x-jd) & n \in \mathbb{N}\\
1 & n = 0 \\
\prod_{j=1}^{-n}\frac{1}{x+jd} & -n \in \mathbb{N}, -x \notin \{ d,2d,\dots,nd \}.
\end{cases}
\end{eqnarray*}
 
If $x = n$, $d = 1$ and $n \in \mathbb{N}$ then 
\[
\displaystyle [n,1]_n = \begin{cases} n(n-1)\dots(2)(1) & n > 0 \\
1 & n = 0, \end{cases}
\]
on which it follows that $[n,1]_n = n!$. This is why the above definition generalizes the notion of the usual factorial.
%%%%%
%%%%%
\end{document}
