\documentclass[12pt]{article}
\usepackage{pmmeta}
\pmcanonicalname{BellNumber}
\pmcreated{2013-03-22 14:47:07}
\pmmodified{2013-03-22 14:47:07}
\pmowner{aoh45}{5079}
\pmmodifier{aoh45}{5079}
\pmtitle{Bell number}
\pmrecord{7}{36436}
\pmprivacy{1}
\pmauthor{aoh45}{5079}
\pmtype{Definition}
\pmcomment{trigger rebuild}
\pmclassification{msc}{05A18}
\pmclassification{msc}{11B73}
\pmrelated{StirlingNumbersSecondKind}

\usepackage{amssymb}
\usepackage{amsmath}
\usepackage{amsfonts}
\usepackage{amsthm}

% need this for including graphics (\includegraphics)
%\usepackage{graphicx}
% making logically defined graphics
%%%\usepackage{xypic}

\begin{document}
The \emph{Bell number}, denoted $B(n)$ is the total number of partitions of a set with $n$ elements.  For $n=0$, we have $B(0)=1$.  For $n\ge 1$, we have 
\[
B(n) = \sum_{k=0}^n S(n,k) \qquad \textrm{for } n \ge 1
\]
where $S(n,k)$ are the Stirling numbers of the second kind.

\newtheorem{bell}{Proposition}
\begin{bell}
\[
B(n+1) = \sum_{k=0}^n \binom{n}{k} B(k)
\]
\end{bell}

\begin{proof}
We count the number of partitions of a set of $n+1$ elements, depending on the size of the block containing the $n+1$st element. If the block has size $j$ for $1 \le j \le n+1$ then we have $\binom{n}{j-1}$ choices for the $j-1$ other elements of the block. The remaining $n+1-j$ elements can be partitioned in $B(n+1-j)$ ways. We have therefore that:
\begin{eqnarray*}
B(n+1) & = & \sum_{j=1}^{n+1}\binom{n}{j-1} B(n+1-j) \\
       & = & \sum_{j=1}^{n+1}\binom{n}{n+1-j} B(n+1-j) \\
       & = & \sum_{k=0}^n \binom{n}{k} B(k)
\end{eqnarray*}
\end{proof}

Using the formula above, one can easily derive the first few Bell numbers.  Starting with $n=0$, the first ten Bell numbers are 1, 1, 2, 5, 15, 52, 203, 877, 4140, 21147.

%%%%%
%%%%%
\end{document}
