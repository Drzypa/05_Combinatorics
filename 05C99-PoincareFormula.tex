\documentclass[12pt]{article}
\usepackage{pmmeta}
\pmcanonicalname{PoincareFormula}
\pmcreated{2013-03-22 13:40:15}
\pmmodified{2013-03-22 13:40:15}
\pmowner{CWoo}{3771}
\pmmodifier{CWoo}{3771}
\pmtitle{Poincar\'e formula}
\pmrecord{11}{34336}
\pmprivacy{1}
\pmauthor{CWoo}{3771}
\pmtype{Theorem}
\pmcomment{trigger rebuild}
\pmclassification{msc}{05C99}
\pmsynonym{Euler-Poincar\'e formula}{PoincareFormula}
\pmsynonym{Euler-Poincare formula}{PoincareFormula}
\pmrelated{EulersPolyhedronTheorem}
\pmrelated{Polytope}

\endmetadata

% this is the default PlanetMath preamble.  as your knowledge
% of TeX increases, you will probably want to edit this, but
% it should be fine as is for beginners.

% almost certainly you want these
\usepackage{amssymb}
\usepackage{amsmath}
\usepackage{amsfonts}

% used for TeXing text within eps files
%\usepackage{psfrag}
% need this for including graphics (\includegraphics)
%\usepackage{graphicx}
% for neatly defining theorems and propositions
%\usepackage{amsthm}
% making logically defined graphics
%%%\usepackage{xypic}

% there are many more packages, add them here as you need them

% define commands here
\begin{document}
Let $K$ be finite oriented simplicial complex of dimension $n$.  Then $$\chi(K)= \sum_{p=0}^n (-1)^p R_p(K),$$ where $\chi(K)$ is the Euler characteristic of $K$, and $R_{p}(K)$ is the $p$-th Betti number of $K$.

This formula also works when $K$ is any finite CW complex.  The Poincar\'e formula is also known as the Euler-Poincar\'e formula, for it is a generalization of the Euler formula for polyhedra.

If $K$ is a compact connected orientable surface with no boundary and with genus h, then $\chi(K)=2-2h$.  If $K$ is non-orientable instead, then $\chi(K)=2-h$.
%%%%%
%%%%%
\end{document}
