\documentclass[12pt]{article}
\usepackage{pmmeta}
\pmcanonicalname{PrincipleOfInclusionexclusion}
\pmcreated{2013-03-22 12:33:25}
\pmmodified{2013-03-22 12:33:25}
\pmowner{Mathprof}{13753}
\pmmodifier{Mathprof}{13753}
\pmtitle{principle of inclusion-exclusion}
\pmrecord{9}{32803}
\pmprivacy{1}
\pmauthor{Mathprof}{13753}
\pmtype{Theorem}
\pmcomment{trigger rebuild}
\pmclassification{msc}{05A99}
\pmsynonym{inclusion-exclusion principle}{PrincipleOfInclusionexclusion}
\pmrelated{SieveOfEratosthenes2}
\pmrelated{BrunsPureSieve}

\endmetadata

% this is the default PlanetMath preamble.  as your knowledge
% of TeX increases, you will probably want to edit this, but
% it should be fine as is for beginners.

% almost certainly you want these
\usepackage{amssymb}
\usepackage{amsmath}
\usepackage{amsfonts}

% used for TeXing text within eps files
%\usepackage{psfrag}
% need this for including graphics (\includegraphics)
%\usepackage{graphicx}
% for neatly defining theorems and propositions
%\usepackage{amsthm}
% making logically defined graphics
%%%\usepackage{xypic} 

% there are many more packages, add them here as you need them

% define commands here
\begin{document}
The \emph{principle of inclusion-exclusion} provides a way of methodically counting the union of possibly non-disjoint sets.

Let $C = \{A_1, A_2, \dots A_N\}$ be a finite collection of finite sets.  Let $I_k$ represent the set of $k$-fold intersections of members of $C$ (e.g., $I_2$ contains all possible intersections of two sets chosen from $C$).

Then
$$\left| \bigcup_{i=1}^{N} A_i \right| = \sum_{j=1}^N \left( (-1)^{(j+1)} \sum_{S \in I_j} |S| \right )$$

For example:
$$|A \cup B| = |A|+|B|-|A \cap B|$$
$$|A \cup B \cup C| = |A|+|B|+|C|-(|A \cap B|+|A \cap C|+|B \cap C|)+|A \cap B \cap C|$$

The principle of inclusion-exclusion, combined with de Morgan's laws, can be used to count the intersection of sets as well.  Let $A$ be some universal set such that $A_k \subseteq A$ for each $k$, and let $\overline{A_k}$ represent the complement of $A_k$ with respect to $A$.  Then we have

$$\left | \bigcap_{i=1}^N A_i \right | = \left |\overline{ \bigcup_{i=1}^N \overline{A_i} }\right |$$

thereby turning the problem of finding an intersection into the problem of finding a union.
%%%%%
%%%%%
\end{document}
