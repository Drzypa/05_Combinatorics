\documentclass[12pt]{article}
\usepackage{pmmeta}
\pmcanonicalname{ProofOfTuransTheorem}
\pmcreated{2013-03-22 12:46:13}
\pmmodified{2013-03-22 12:46:13}
\pmowner{mathwizard}{128}
\pmmodifier{mathwizard}{128}
\pmtitle{proof of Turan's theorem}
\pmrecord{6}{33080}
\pmprivacy{1}
\pmauthor{mathwizard}{128}
\pmtype{Proof}
\pmcomment{trigger rebuild}
\pmclassification{msc}{05C99}

% this is the default PlanetMath preamble.  as your knowledge
% of TeX increases, you will probably want to edit this, but
% it should be fine as is for beginners.

% almost certainly you want these
\usepackage{amssymb}
\usepackage{amsmath}
\usepackage{amsfonts}

% used for TeXing text within eps files
%\usepackage{psfrag}
% need this for including graphics (\includegraphics)
%\usepackage{graphicx}
% for neatly defining theorems and propositions
%\usepackage{amsthm}
% making logically defined graphics
%%%\usepackage{xypic}

% there are many more packages, add them here as you need them

% define commands here
\begin{document}
If the graph $G$ has $n\leq p-1$ vertices it cannot contain any $p$--clique and thus has at most ${n\choose 2}$ edges. So in this case we only have to prove that
$$\frac{n(n-1)}{2}\leq\left( 1-\frac{1}{p-1}\right)\frac{n^2}{2}.$$
Dividing by $n^2$ we get
$$\frac{n-1}{n}=1-\frac{1}{n}\leq 1-\frac{1}{p-1},$$
which is true since $n\leq p-1$.

Now we assume that $n\geq p$ and the set of vertices of $G$ is denoted by $V$. If $G$ has the maximum number of edges possible without containing a $p$--clique it contains a $p-1$--clique, since otherwise we might add edges to get one. So we denote one such clique by $A$ and define $B:=G\backslash A$.

So $A$ has ${p-1\choose 2}$ edges. We are now interested in the number of edges in  $B$, which we will call $e_B$, and in the number of edges connecting $A$ and $B$, which will be called $e_{A,B}$. By induction we get:
$$e_B\leq\frac{1}{2}\left( 1-\frac{1}{p-1}\right)\left(n-p+1\right)^2.$$
Since $G$ does not contain any $p$--clique every vertice of $B$ is connected to at most $p-2$ vertices in $A$ and thus we get:
$$e_{A,B}\leq (p-2)(n-p+1).$$
Putting this together we get for the number of edges $|E|$ of $G$:
$$|E|\leq{p-1\choose 2}+\frac{1}{2}\left(1-\frac{1}{p-1}\right)(n-p+1)^2+(p-2)(n -p+1).$$
And thus we get:
$$|E|\leq\left( 1-\frac{1}{p-1}\right)\frac{n^2}{2}.$$
%%%%%
%%%%%
\end{document}
