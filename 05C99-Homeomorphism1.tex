\documentclass[12pt]{article}
\usepackage{pmmeta}
\pmcanonicalname{Homeomorphism1}
\pmcreated{2013-03-22 12:31:55}
\pmmodified{2013-03-22 12:31:55}
\pmowner{digitalis}{76}
\pmmodifier{digitalis}{76}
\pmtitle{homeomorphism}
\pmrecord{6}{32773}
\pmprivacy{1}
\pmauthor{digitalis}{76}
\pmtype{Definition}
\pmcomment{trigger rebuild}
\pmclassification{msc}{05C99}
\pmrelated{Subdivision}
\pmrelated{Realization}

\endmetadata

% this is the default PlanetMath preamble.  as your knowledge
% of TeX increases, you will probably want to edit this, but
% it should be fine as is for beginners.

% almost certainly you want these
\usepackage{amssymb}
\usepackage{amsmath}
\usepackage{amsfonts}

% used for TeXing text within eps files
%\usepackage{psfrag}
% need this for including graphics (\includegraphics)
%\usepackage{graphicx}
% for neatly defining theorems and propositions
%\usepackage{amsthm}
% making logically defined graphics
%%%\usepackage{xypic} 

% there are many more packages, add them here as you need them

% define commands here
\begin{document}
We say that a graph $G$ is \emph{homeomorphic} to graph $H$ if the
  realization $R(G)$ of $G$ is topologically \PMlinkname{homeomorphic}{Homeomorphism} to $R(H)$ or, equivalently, $G$ and $H$ have isomorphic subdivisions.


\footnotesize{Adapted with permission of the author from \emph{\PMlinkescapetext{Modern Graph Theory}} by B\'{e}la Bollob\'{a}s, published by Springer-Verlag New York, Inc., 1998.}
%%%%%
%%%%%
\end{document}
