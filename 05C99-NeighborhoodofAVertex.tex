\documentclass[12pt]{article}
\usepackage{pmmeta}
\pmcanonicalname{NeighborhoodofAVertex}
\pmcreated{2013-03-22 11:58:03}
\pmmodified{2013-03-22 11:58:03}
\pmowner{digitalis}{76}
\pmmodifier{digitalis}{76}
\pmtitle{neighborhood (of a vertex)}
\pmrecord{10}{30785}
\pmprivacy{1}
\pmauthor{digitalis}{76}
\pmtype{Definition}
\pmcomment{trigger rebuild}
\pmclassification{msc}{05C99}
\pmsynonym{neighborhood}{NeighborhoodofAVertex}
%\pmkeywords{vertex}
%\pmkeywords{graph}
\pmrelated{Graph}

\usepackage{amssymb}
\usepackage{amsmath}
\usepackage{amsfonts}
\usepackage{graphicx}
%%%\usepackage{xypic}
\begin{document}
For a graph $G$, the set of vertices adjacent to a vertex $x \in G$, the \emph{neighborhood} of $x$, is denoted by $\Gamma(x)$. Occasionally one calls $\Gamma(x)$ the \emph{open} neighborhood of $x$, and $\Gamma \cup \{x\}$ the \emph{closed} neighborhood of $x$.


\footnotesize{Adapted with permission of the author from \emph{Modern Graph Theory} by B\'{e}la Bollob\'{a}s, published by Springer-Verlag New York, Inc., 1998.}
%%%%%
%%%%%
%%%%%
\end{document}
