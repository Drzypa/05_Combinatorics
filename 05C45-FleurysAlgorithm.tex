\documentclass[12pt]{article}
\usepackage{pmmeta}
\pmcanonicalname{FleurysAlgorithm}
\pmcreated{2013-03-22 13:35:15}
\pmmodified{2013-03-22 13:35:15}
\pmowner{mathcam}{2727}
\pmmodifier{mathcam}{2727}
\pmtitle{Fleury's algorithm}
\pmrecord{9}{34210}
\pmprivacy{1}
\pmauthor{mathcam}{2727}
\pmtype{Algorithm}
\pmcomment{trigger rebuild}
\pmclassification{msc}{05C45}
\pmrelated{EulerCircuit}

\endmetadata

% this is the default PlanetMath preamble.  as your knowledge
% of TeX increases, you will probably want to edit this, but
% it should be fine as is for beginners.

% almost certainly you want these
\usepackage{amssymb}
\usepackage{amsmath}
\usepackage{amsfonts}

% used for TeXing text within eps files
%\usepackage{psfrag}
% need this for including graphics (\includegraphics)
%\usepackage{graphicx}
% for neatly defining theorems and propositions
%\usepackage{amsthm}
% making logically defined graphics
%%%\usepackage{xypic}

% there are many more packages, add them here as you need them

% define commands here
\begin{document}
Fleury's algorithm constructs an Euler circuit in a graph (if it's possible). \\
\begin{enumerate}
\item Pick any vertex to start
\item From that vertex pick an edge to traverse, considering following rule: never cross a bridge of the reduced graph unless there is no other choice 
\item Darken that edge, as a reminder that you can't traverse it again 
\item Travel that edge, coming to the next vertex 
\item Repeat 2-4 until all edges have been traversed, and you are back at the starting vertex
\end{enumerate}
By ``reduced graph'' we mean the original graph minus the darkened (already used) edges.
%%%%%
%%%%%
\end{document}
