\documentclass[12pt]{article}
\usepackage{pmmeta}
\pmcanonicalname{SumOfPowersOfBinomialCoefficients}
\pmcreated{2013-03-22 14:25:43}
\pmmodified{2013-03-22 14:25:43}
\pmowner{Andrea Ambrosio}{7332}
\pmmodifier{Andrea Ambrosio}{7332}
\pmtitle{sum of powers of binomial coefficients}
\pmrecord{7}{35937}
\pmprivacy{1}
\pmauthor{Andrea Ambrosio}{7332}
\pmtype{Result}
\pmcomment{trigger rebuild}
\pmclassification{msc}{05A10}
\pmclassification{msc}{11B65}

\endmetadata

% this is the default PlanetMath preamble.  as your knowledge
% of TeX increases, you will probably want to edit this, but
% it should be fine as is for beginners.

% almost certainly you want these
\usepackage{amssymb}
\usepackage{amsmath}
\usepackage{amsfonts}

% used for TeXing text within eps files
%\usepackage{psfrag}
% need this for including graphics (\includegraphics)
%\usepackage{graphicx}
% for neatly defining theorems and propositions
%\usepackage{amsthm}
% making logically defined graphics
%%%\usepackage{xypic}

% there are many more packages, add them here as you need them

% define commands here
\begin{document}
Some results exist on sums of powers of binomial coefficients.  Define $A_s$ as follows:

\[ A_s(n) = \sum_{i=0}^n {n \choose i}^s \]

for $s$ a positive integer and $n$ a nonnegative integer.

For $s=1$, the binomial theorem implies that the sum $A_1(n)$ is simply $2^n$.

For $s=2$, the following result on the sum of the squares of the binomial coefficients ${n \choose i}$ holds:

\[ A_2(n) = \sum_{i=0}^n {n \choose i}^2 = {2n \choose n} \]

that is, $A_2(n)$ is the $n$th central binomial coefficient.

{\bf Proof:}
This result follows immediately from the Vandermonde identity:

\[ {p+q \choose k}=\sum_{i=0}^k {p \choose i} {q \choose k-i} \]
upon choosing $p=q=k=n$ and observing that ${n \choose n-i}={n \choose i}$.

Expressions for $A_s(n)$ for larger values of $s$ exist in terms of hypergeometric functions.
%%%%%
%%%%%
\end{document}
