\documentclass[12pt]{article}
\usepackage{pmmeta}
\pmcanonicalname{KuratowskisTheorem}
\pmcreated{2013-03-22 11:57:45}
\pmmodified{2013-03-22 11:57:45}
\pmowner{bbukh}{348}
\pmmodifier{bbukh}{348}
\pmtitle{Kuratowski's theorem}
\pmrecord{12}{30764}
\pmprivacy{1}
\pmauthor{bbukh}{348}
\pmtype{Theorem}
\pmcomment{trigger rebuild}
\pmclassification{msc}{05C10}
%\pmkeywords{planar}
\pmrelated{PlanarGraph}
\pmrelated{WagnersTheorem}

\usepackage{amssymb}
\usepackage{amsmath}
\usepackage{amsfonts}

\makeatletter
\@ifundefined{bibname}{}{\renewcommand{\bibname}{References}}
\makeatother
\begin{document}
A finite graph is planar if and only if it contains no subgraph that is isomorphic to or is a subdivision of $K_5$ or $K_{3,3}$, where $K_5$ is the complete graph of order 5 and $K_{3,3}$ is the complete bipartite graph with 3 vertices in each of the halfs. Wagner's theorem is an equivalent later result.

\begin{thebibliography}{1}

\bibitem{cite:kuratowski_planarity}
Kazimierz Kuratowski.
\newblock Sur le probl{\`e}me des courbes gauches en topologie.
\newblock {\em Fund. Math.}, 15:271--283, 1930.

\end{thebibliography}

%@ARTICLE{cite:kuratowski_planarity,
% author    = {Kazimierz Kuratowski},
% title     = "Sur le Probl{\`e}me des Courbes Gauches en Topologie",
% journal   = {Fund. Math.},
% volume    = 15,
% pages     = {271--283},
% year      = 1930
%}
%%%%%
%%%%%
%%%%%
\end{document}
