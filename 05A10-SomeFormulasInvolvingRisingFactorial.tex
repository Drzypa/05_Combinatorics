\documentclass[12pt]{article}
\usepackage{pmmeta}
\pmcanonicalname{SomeFormulasInvolvingRisingFactorial}
\pmcreated{2013-03-22 17:49:12}
\pmmodified{2013-03-22 17:49:12}
\pmowner{Wkbj79}{1863}
\pmmodifier{Wkbj79}{1863}
\pmtitle{some formulas involving rising factorial}
\pmrecord{4}{40283}
\pmprivacy{1}
\pmauthor{Wkbj79}{1863}
\pmtype{Result}
\pmcomment{trigger rebuild}
\pmclassification{msc}{05A10}

\endmetadata

\usepackage{amssymb}
\usepackage{amsmath}
\usepackage{amsfonts}
\usepackage{pstricks}
\usepackage{psfrag}
\usepackage{graphicx}
\usepackage{amsthm}
%%\usepackage{xypic}

\begin{document}
\PMlinkescapeword{formula}
\PMlinkescapeword{relation}

Recall that, for $a\in\mathbb{C}$ and $n$ a nonnegative integer, the rising factorial $(a)_n$ is defined by
\[
(a)_n=\prod_{k=0}^{n-1}(a+k).
\]

The following results hold regarding the rising factorial:

\begin{itemize}
\item For all $a\in\mathbb{C}$, we have $(a)_0=1$.
\item For all nonnegative integers $n$, we have $(1)_n=n!$.
\item The binomial coefficients are given by
\[
\binom{a}{n}=\frac{(-1)^n(-a)_n}{n!}.
\]
\item The rising factorial relates to the gamma function.  One relation is given by the formula
\[
(a)_n=\frac{\Gamma(a+n)}{\Gamma(a)}.
\]
This formula can be used to extend the definition of rising factorial so that $n$ can be any complex number provided that $a+n$ is not a nonpositive integer.
\item Another relation between the rising factorial and the gamma function is given by
\[
\Gamma(a)=\lim_{n\to\infty} \frac{n!\,n^{a-1}}{(a)_n}.
\]
\end{itemize}
%%%%%
%%%%%
\end{document}
