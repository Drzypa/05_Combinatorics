\documentclass[12pt]{article}
\usepackage{pmmeta}
\pmcanonicalname{HasseDiagram}
\pmcreated{2013-03-22 12:15:23}
\pmmodified{2013-03-22 12:15:23}
\pmowner{bbukh}{348}
\pmmodifier{bbukh}{348}
\pmtitle{Hasse diagram}
\pmrecord{18}{31639}
\pmprivacy{1}
\pmauthor{bbukh}{348}
\pmtype{Definition}
\pmcomment{trigger rebuild}
\pmclassification{msc}{05C90}
\pmrelated{Poset}
\pmrelated{PartialOrder}

\usepackage{amssymb}
\usepackage{amsmath}
\usepackage{amsfonts}
%%%\usepackage{xypic}
\begin{document}
If $(A,\leq)$ is a finite poset, then it can be represented by a \emph{Hasse diagram}, which is a graph whose vertices are elements of $A$ and the edges correspond to the covering relation. More precisely an edge from $x\in A$ to $y\in A$ is present if
\begin{itemize}
\item $x < y$.
\item There is no $z \in A$ such that $x < z$ and $z < y$. (There are no in-between elements.)
\end{itemize}
If $x<y$, then in $y$ is drawn higher than $x$. Because of that, the direction of the edges is never indicated in a Hasse diagram.

\emph{Example:} If $A = \mathcal{P}(\{1,2,3\})$, the power set of $\{1,2,3\}$, and $\leq$
is the subset relation $\subseteq$, then Hasse diagram is
$$\xymatrix{
                               & \{1,2,3\} &    \\
\{1,2\} \ar@{-}[ur]  & \{1,3\} \ar@{-}[u]  & \{2,3\} \ar@{-}[ul] \\
\{1\} \ar@{-}[u] \ar@{-}[ur] & \{2\} \ar@{-}[ul] \ar@{-}[ur] & \{3\} \ar@{-}[ul] \ar@{-}[u] \\
  & \emptyset \ar@{-}[ul] \ar@{-}[u] \ar@{-}[ur] & }
$$


Even though $\{3\} < \{1,2,3\}$ (since $\{3\} \subset \{1,2,3\}$), there is no edge directly between them because there are inbetween elements:
$\{2,3\}$ and $\{1,3\}$. However, there still remains an indirect path from $\{3\}$ to $\{1,2,3\}$.
%%%%%
%%%%%
%%%%%
\end{document}
