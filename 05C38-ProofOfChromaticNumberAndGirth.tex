\documentclass[12pt]{article}
\usepackage{pmmeta}
\pmcanonicalname{ProofOfChromaticNumberAndGirth}
\pmcreated{2013-03-22 14:31:02}
\pmmodified{2013-03-22 14:31:02}
\pmowner{kshum}{5987}
\pmmodifier{kshum}{5987}
\pmtitle{proof of chromatic number and girth}
\pmrecord{9}{36057}
\pmprivacy{1}
\pmauthor{kshum}{5987}
\pmtype{Proof}
\pmcomment{trigger rebuild}
\pmclassification{msc}{05C38}
\pmclassification{msc}{05C15}
\pmclassification{msc}{05C80}

\endmetadata

% this is the default PlanetMath preamble.  as your knowledge
% of TeX increases, you will probably want to edit this, but
% it should be fine as is for beginners.

% almost certainly you want these
\usepackage{amssymb}
\usepackage{amsmath}
\usepackage{amsfonts}

% used for TeXing text within eps files
%\usepackage{psfrag}
% need this for including graphics (\includegraphics)
%\usepackage{graphicx}
% for neatly defining theorems and propositions
%\usepackage{amsthm}
% making logically defined graphics
%%%\usepackage{xypic}

% there are many more packages, add them here as you need them

% define commands here
\begin{document}
\PMlinkescapeword{properties}
\PMlinkescapeword{satisfies}


Let $\alpha(G)$ denote the size of the largest independent set in
$G$, and $\chi(G)$ the chromatic number of $G$. We want to show
that there is a graph $G$ with girth larger than $\ell$ and
$\chi(G)>k$, for any $\ell, k > 0$.

\bigskip

We first prove the following claim.

{\it Claim:} Given a positive integer $\ell$ and a positive real
number $t<1/\ell$, for all sufficiently large $n$, there is a
graph $G$ on $n$ vertices satisfying properties

\begin{enumerate}
\item  the number of cycles of length at most $\ell$ is less than
$n/2$,

\item $\alpha(G) < 3n^{1-t}\log n$.
\end{enumerate}

{\it Proof of claim:} Let $G$ be a random graph on $n$ vertices,
in which each pair of vertices joint by an edge independently with
probability $p=n^{t-1}$. Let $X$ be the number of cycles of length
at most $\ell$ in $G$. The expected value of $X$ is
\begin{align*}
  E[X] &= \sum_{i=3}^\ell \frac{n(n-1)\cdots(n-i+1)}{2i}p^i \\
  &< \sum_{i=3}^\ell \frac{(np)^i}{2i} < \sum_{i=3}^\ell n^{ti} < \ell n^{t\ell}
\end{align*}
By Markov inequality,
\[
  \Pr(X \geq n/2) < 2\ell n^{t\ell-1},
\]
we have
\[ \Pr(X \geq n/2) \rightarrow 0
\]
as $n\rightarrow \infty$, since $t\ell < 1$.

On the other hand, let $y=\lceil (3\log n)/ p \rceil$, and $Y$ be
the number of independent sets of size $y$ in $G$. By Markov
inequality again,
\[
  \Pr(\alpha(G)\geq y) = \Pr(Y\geq 1) \leq E[Y].
\]
However,
\[
 E[Y] = {n \choose y} (1-p)^{y(y-1)/2}
\]
Using the inequalities, ${n \choose y} < n^y$ and $(1-p)\leq
e^{-p}$, we get
\[
 \Pr(\alpha(G)\geq y) < (n e^{-p(y-1)/2})^y
\]
Our choice of $y$ guarantees that $n e^{-p(y-1)/2} < \beta < 1 $
for some $\beta$, and $y\rightarrow \infty$ as $n$ approaches
infinity. Therefore,
\[
 \Pr(\alpha(G)\geq y) \rightarrow 0, \quad \text{as
 $n\rightarrow\infty$}.
\]



We can thus find $n_0$ such that for all $n>n_0$, both  $\Pr(X
\geq n/2)$ and $\Pr(\alpha(G)\geq y)$ are strictly less than
$1/2$. For all $n>n_0$,
\[
 \Pr(X<\ell \text{ and } \alpha(G)<y) > 1-\Pr(X\geq\ell)-\Pr(
 \alpha(G)\geq y)>1.
\]
Therefore there exists a graph that satisfies the two properties
in the claim. This ends the proof of the claim.

\bigskip

 Let $G$ be a graph that satisfies the two properties in the
claim. Remove a vertex from each cycle of length at most $\ell$ in
$G$. The resulting graph $G'$ has girth larger than $\ell$, more
than $n/2$ vertices, and $\alpha(G')\leq \alpha(G)$. Since
\PMlinkid{$\chi(G') \alpha(G') \geq |G'|$}{6037}, we have
\[
  \chi(G') \geq \frac{n/2}{3n^{1-t}\log n} = \frac{n^t}{6 \log n}
\]
We can pick sufficiently large $n$ such that $\chi(G')$ is larger
than $k$. Then the chromatic number of $G'$ is larger than $k$ and
girth is larger than $\ell$.


{\bf Reference}: N. Alon and J. Spencer, {\it The probabilistic method}, 2nd, John Wiley.
%%%%%
%%%%%
\end{document}
