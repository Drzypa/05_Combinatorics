\documentclass[12pt]{article}
\usepackage{pmmeta}
\pmcanonicalname{MultiindexDerivativeOfAPower}
\pmcreated{2013-03-22 13:42:01}
\pmmodified{2013-03-22 13:42:01}
\pmowner{matte}{1858}
\pmmodifier{matte}{1858}
\pmtitle{multi-index derivative of a power}
\pmrecord{9}{34376}
\pmprivacy{1}
\pmauthor{matte}{1858}
\pmtype{Theorem}
\pmcomment{trigger rebuild}
\pmclassification{msc}{05-00}

\endmetadata

% this is the default PlanetMath preamble.  as your knowledge
% of TeX increases, you will probably want to edit this, but
% it should be fine as is for beginners.

% almost certainly you want these
\usepackage{amssymb}
\usepackage{amsmath}
\usepackage{amsfonts}

% used for TeXing text within eps files
%\usepackage{psfrag}
% need this for including graphics (\includegraphics)
%\usepackage{graphicx}
% for neatly defining theorems and propositions
%\usepackage{amsthm}
% making logically defined graphics
%%%\usepackage{xypic}

% there are many more packages, add them here as you need them

% define commands here

\newcommand{\sR}[0]{\mathbb{R}}
\newcommand{\sC}[0]{\mathbb{C}}
\newcommand{\sN}[0]{\mathbb{N}}
\newcommand{\sZ}[0]{\mathbb{Z}}
\begin{document}
{\bf Theorem}
If $i,k$ are multi-indices in $\sN^n$, and $x=(x_1,\ldots, x_n)$, 
then 
\begin{eqnarray*}
\partial^i x^k = \left\{ \begin {array}{ll} 
 \frac{k!}{(k-i)!} x^{k-i} & \mbox{if}\,\, i\le k, \\
                         0 & \mbox{otherwise}.
 \end{array} \right.
\end{eqnarray*}

\emph{Proof.} The proof follows from the corresponding rule for 
the ordinary derivative; if $i,k$ are in $0,1,2,\ldots$, then 
\begin{eqnarray}
\label{usual}
\frac{d^i}{dx^i} x^k = \left\{ \begin {array}{ll}  \frac{k!}{(k-i)!} x^{k-i} & \mbox{if}\,\, i\le k, \\
0 & \mbox{otherwise.} 
 \end{array} \right.
\end{eqnarray}
Suppose $i=(i_1,\ldots, i_n)$, $k=(k_1,\ldots, k_n)$, and 
$x=(x_1,\ldots, x_n)$. 
Then we have that 
\begin{eqnarray*}
\partial^i x^k &=& \frac{\partial^{|i|}}{\partial x_1^{i_1} \cdots \partial x_n^{i_n}} x_1^{k_1} \cdots x_n^{k_n} \\
 &=& \frac{\partial^{i_1}}{\partial x_1^{i_1}} x_1^{k_1} \cdot \cdots \cdot  \frac{\partial^{i_n}}{\partial x_n^{i_n}} x_n^{k_n}.
\end{eqnarray*}
For each $r=1,\ldots, n$, the function $x_r^{k_r}$ only depends on $x_r$. 
In the above, each 
partial differentiation $\partial/\partial x_r$ therefore
reduces to the corresponding 
ordinary differentiation $d/dx_r$. 
Hence, from equation \ref{usual}, it follows that $\partial^i x^k$ vanishes
if $i_r > k_r$ for any  $r=1,\ldots, n$. If this is not the case, i.e., 
if $i\le k$ as multi-indices, then for each $r$, 
$$\frac{d^{i_r}}{dx_r^{i_r}} x_r^{k_r} =  \frac{k_r!}{(k_r-i_r)!} x_r^{k_r-i_r},$$
and the theorem follows. $\Box$
%%%%%
%%%%%
\end{document}
