\documentclass[12pt]{article}
\usepackage{pmmeta}
\pmcanonicalname{Graph}
\pmcreated{2013-03-22 11:57:54}
\pmmodified{2013-03-22 11:57:54}
\pmowner{mathcam}{2727}
\pmmodifier{mathcam}{2727}
\pmtitle{graph}
\pmrecord{34}{30777}
\pmprivacy{1}
\pmauthor{mathcam}{2727}
\pmtype{Definition}
\pmcomment{trigger rebuild}
\pmclassification{msc}{05C99}
%\pmkeywords{multigraph}
%\pmkeywords{digraph}
%\pmkeywords{pseudograph}
%\pmkeywords{tree}
\pmrelated{LoopOfAGraph}
\pmrelated{NeighborhoodOfAVertex}
\pmrelated{EulersPolyhedronTheorem}
\pmrelated{Digraph}
\pmrelated{Tree}
\pmrelated{SpanningTree}
\pmrelated{ConnectedGraph}
\pmrelated{Cycle}
\pmrelated{GraphTheory}
\pmrelated{GraphTopology}
\pmrelated{Subgraph}
\pmrelated{SimplePath}
\pmrelated{EulerPath}
\pmrelated{Diameter3}
\pmrelated{DistanceInAGraph}
\pmrelated{GraphHomomorphism}
\pmrelated{Pseudograph}
\pmrelated{Multigraph}
\pmrelated{OrderOf}
\pmdefines{edge}
\pmdefines{vertex}
\pmdefines{endvertex}
\pmdefines{adjacent}
\pmdefines{incident}
\pmdefines{join}
\pmdefines{vertices}
\pmdefines{simple graph}

% this is the default PlanetMath preamble.  as your knowledge
% of TeX increases, you will probably want to edit this, but
% it should be fine as is for beginners.

% almost certainly you want these
\usepackage{amssymb}
\usepackage{graphicx}
\usepackage{amsmath}
\usepackage{amsfonts}

% used for TeXing text within eps files
%\usepackage{psfrag}
% need this for including graphics (\includegraphics)
%\usepackage{graphicx}
% for neatly defining theorems and propositions
%\usepackage{amsthm}
% making logically defined graphics
%%%%\usepackage{xypic} 

% there are many more packages, add them here as you need them

% define commands here
\begin{document}
A \emph{graph} $G$ is an ordered pair of disjoint sets $(V, E)$ such that $E$ is a subset of the set $V^{(2)}$ of unordered pairs of $V$. $V$ and $E$ are always assumed to be finite, unless explicitly stated otherwise. The set $V$ is the set of \emph{vertices} (sometimes called nodes) and $E$ is the set of \emph{edges}. If $G$ is a graph, then $V = V(G)$ is the vertex set of $G$, and $E = E(G)$ is the edge set.
Typically, $V(G)$ is defined to be nonempty. If $x$ is a vertex of $G$, we sometimes write $x \in G$ instead of $x \in V(G)$.

An edge $\{x, y\}$ (with $x\neq y$) is said to \emph{join} the vertices $x$ and $y$ and is denoted by $xy$. One says that the edges $xy$ and $yx$ are \emph{equivalent}; the vertices $x$ and $y$ are the \emph{endvertices} of this edge. If $xy \in E(G)$, then $x$ and $y$ are \emph{adjacent}, or \emph{neighboring}, vertices of $G$, and the vertices $x$ and $y$ are \emph{incident} with the edge $xy$. Two edges are \emph{adjacent} if they have at least one common endvertex. Also, $x \sim y$ means that the vertex $x$ is adjacent to the vertex $y$.

Notice that this definition allows pairs of the form $\{x,x\}$, which would correspond to a node joining to itself.  Some authors explicitly disallow this in their definition of a graph.

\begin{center}
\includegraphics{grapha.eps} \qquad \qquad
\includegraphics{graphb.eps} \qquad \qquad
\includegraphics{graphc.eps} \\
Some graphs.
\end{center}

Note:  Some authors include multigraphs in their definition of a graph.  In this notation, the above definition corresponds to that of a \emph{simple graph}.  A graph is then \emph{simple} if there is at most one edge joining each pair of nodes.

\footnotesize{Adapted with permission of the author from \emph{\PMlinkescapetext{Modern Graph Theory}} by B\'{e}la Bollob\'{a}s, published by Springer-Verlag New York, Inc., 1998.}
%%%%%
%%%%%
%%%%%
\end{document}
