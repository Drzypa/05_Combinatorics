\documentclass[12pt]{article}
\usepackage{pmmeta}
\pmcanonicalname{ExponentialFactorial}
\pmcreated{2013-03-22 16:01:38}
\pmmodified{2013-03-22 16:01:38}
\pmowner{CompositeFan}{12809}
\pmmodifier{CompositeFan}{12809}
\pmtitle{exponential factorial}
\pmrecord{7}{38068}
\pmprivacy{1}
\pmauthor{CompositeFan}{12809}
\pmtype{Definition}
\pmcomment{trigger rebuild}
\pmclassification{msc}{05A10}
\pmrelated{Factorial}

\endmetadata

% this is the default PlanetMath preamble.  as your knowledge
% of TeX increases, you will probably want to edit this, but
% it should be fine as is for beginners.

% almost certainly you want these
\usepackage{amssymb}
\usepackage{amsmath}
\usepackage{amsfonts}

% used for TeXing text within eps files
%\usepackage{psfrag}
% need this for including graphics (\includegraphics)
%\usepackage{graphicx}
% for neatly defining theorems and propositions
%\usepackage{amsthm}
% making logically defined graphics
%%%\usepackage{xypic}

% there are many more packages, add them here as you need them

% define commands here

\begin{document}
Given a positive integer $n$, the "power tower" $n^{(n - 1)^{(n - 2) \dots }}$ is the {\em exponential factorial} of $n$. The recurrence relation is $a_1 = 1$, $a_n = n^{a_{n - 1}}$ for $n > 1$.

So for example, $9 = 3^{2^1}$, $262144 = 4^{3^{2^1}}$. The exponential factorial for 5 has almost two hundred thousand base 10 digits. The ones that are small enough are listed in sequence A049384 of Sloane's OEIS.

The sum of the reciprocals of the exponential factorials is a Liouville number. $$\sum_{i = 1}^\infty {1 \over a_i} \approx 1.6111149258083767361111111$$
%%%%%
%%%%%
\end{document}
