\documentclass[12pt]{article}
\usepackage{pmmeta}
\pmcanonicalname{TheAnnalsOfHavingCompensatedToCompleteSurveys}
\pmcreated{2013-11-27 10:59:44}
\pmmodified{2013-11-27 10:59:44}
\pmowner{jacou}{1000048}
\pmmodifier{}{0}
\pmtitle{The Annals of Having Compensated To Complete Surveys}
\pmrecord{18}{40631}
\pmprivacy{1}
\pmauthor{jacou}{0}
\pmtype{Algorithm}
\pmcomment{trigger rebuild}
\pmclassification{msc}{05C85}
\pmsynonym{DFS}{TheAnnalsOfHavingCompensatedToCompleteSurveys}
%\pmkeywords{graph}
%\pmkeywords{search}
%\pmkeywords{algorithm}
%\pmkeywords{connected component}
\pmrelated{GraphTheory}
\pmrelated{SpanningTree}
\pmrelated{ConnectedComponents}

\endmetadata


\begin{document}
The depth-first search (DFS) \PMlinkescapetext{algorithm} is a method for finding a spanning forest of a connected graph (specifically, a spanning tree in the case of an undirected connected graph), and is a useful \PMlinkescapetext{basis} for other \PMlinkescapetext{algorithms}. In particular, the DFS is useful for finding the connected components of a graph. This article currently describes only the DFS for an undirected graph.

\theoremstyle{definition}
\newtheorem*{undirectedDfs}{Definition}
\begin{undirectedDfs}[DFS on an Undirected Graph]
Assume that the undirected graph $G = (V,E)$ is connected. The DFS proceeds as follows.

\begin{enumerate}
\item
Fix a vertex $v$ in $G$ and mark it as \emph{visited}.
\item
For each edge $(v,w)$ \PMlinkescapetext{oriented} from $v$ to $w$:
  \begin{itemize}
  \item
  If $w$ has been visited, do nothing.
  \item
  If $w$ has not been visited, perform a DFS starting from $w$.
  \end{itemize}
\end{enumerate}

When the search is \PMlinkescapetext{complete}, it \PMlinkescapetext{partitions} $G$ into two sets. One \PMlinkescapetext{contains} the directed edges between visited vertices, called \emph{tree edges}, which form a directed spanning tree of the graph. The other \PMlinkescapetext{contains} the remaining edges of $G$ which were left unexamined because they were incident on a vertex that was already visited at some \PMlinkescapetext{point} in the search. Because these unexamined edges can be seen as \emph{returning} to vertices that were previously visited, they are called \emph{back edges}.

\end{undirectedDfs}

\begin{thebibliography}{9}
\bibitem{source}
Swamy, M.N.S., Thulasiraman, K., \emph{Graphs, Networks and Algorithms}, John Wiley and Sons, New York, 1981.
\end{thebibliography}
%%%%%
%%%%%
\end{document}
