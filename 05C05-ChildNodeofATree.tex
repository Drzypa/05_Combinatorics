\documentclass[12pt]{article}
\usepackage{pmmeta}
\pmcanonicalname{ChildNodeofATree}
\pmcreated{2013-03-22 12:30:37}
\pmmodified{2013-03-22 12:30:37}
\pmowner{akrowne}{2}
\pmmodifier{akrowne}{2}
\pmtitle{child node (of a tree)}
\pmrecord{4}{32744}
\pmprivacy{1}
\pmauthor{akrowne}{2}
\pmtype{Definition}
\pmcomment{trigger rebuild}
\pmclassification{msc}{05C05}
\pmsynonym{child node}{ChildNodeofATree}
\pmsynonym{child}{ChildNodeofATree}
\pmrelated{ParentNodeInATree}

\usepackage{amssymb}
\usepackage{amsmath}
\usepackage{amsfonts}

%\usepackage{psfrag}
%\usepackage{graphicx}
%%\usepackage{xypic} 
\xyoption{all}
\usepackage{color}
\begin{document}
A \emph{child node} $C$ of a node $P$ in a tree is any node connected to $P$ which has a path distance from the root node $R$ which is one greater than the path distance between $P$ and $R$.

Drawn in the canonical root-at-top manner, a child node of a node $P$ in a tree is simply any node immediately below $P$ which is connected to it.

\begin{center}

$$\xymatrix{
& \bullet \ar@{-}[dl] \ar@{-}[dr] & & & \\
\bullet & & {\color{blue}\bullet} \ar@{-}[dr]\ar@{-}[dl] & & \\
& {\color{red}\bullet} \ar@{-}[dl] & & {\color{red}\bullet} & \\
\bullet & & & & }$$

{\tiny Figure: A node (blue) and its children (red.)}
\end{center}
%%%%%
%%%%%
\end{document}
