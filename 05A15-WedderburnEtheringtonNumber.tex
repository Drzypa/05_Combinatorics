\documentclass[12pt]{article}
\usepackage{pmmeta}
\pmcanonicalname{WedderburnEtheringtonNumber}
\pmcreated{2013-03-22 16:49:32}
\pmmodified{2013-03-22 16:49:32}
\pmowner{PrimeFan}{13766}
\pmmodifier{PrimeFan}{13766}
\pmtitle{Wedderburn-Etherington number}
\pmrecord{6}{39064}
\pmprivacy{1}
\pmauthor{PrimeFan}{13766}
\pmtype{Definition}
\pmcomment{trigger rebuild}
\pmclassification{msc}{05A15}
\pmsynonym{Wedderburn Etherington number}{WedderburnEtheringtonNumber}
\pmsynonym{Etherington-Wedderburn number}{WedderburnEtheringtonNumber}

% this is the default PlanetMath preamble.  as your knowledge
% of TeX increases, you will probably want to edit this, but
% it should be fine as is for beginners.

% almost certainly you want these
\usepackage{amssymb}
\usepackage{amsmath}
\usepackage{amsfonts}

% used for TeXing text within eps files
%\usepackage{psfrag}
% need this for including graphics (\includegraphics)
%\usepackage{graphicx}
% for neatly defining theorems and propositions
%\usepackage{amsthm}
% making logically defined graphics
%%%\usepackage{xypic}

% there are many more packages, add them here as you need them

% define commands here

\begin{document}
The $n$th {\em Wedderburn-Etherington number} counts how many weakly binary trees can be constructed such that each graph vertex (not counting the root vertex) is adjacent to no more than three other such vertices, for a given number $n$ of nodes. The first few Wedderburn-Etherington numbers are 1, 1, 1, 2, 3, 6, 11, 23, 46, 98, 207, 451, 983, etc. listed in A001190 of Sloane's  OEIS. Michael Somos gives the following recurrence relations:

$$a_{2n} = \frac{1}{2} a_n a_{n + 1} + \sum_{i = 1}^n a_i a_{2n - i}$$

and

$$a_{2n - 1} = \sum_{i = 0}^{n - 1} a_{i + 1} a_{2n - i}$$

with $a_1 = a_2 = 1$ in both relations.
%%%%%
%%%%%
\end{document}
