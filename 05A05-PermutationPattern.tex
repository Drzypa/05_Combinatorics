\documentclass[12pt]{article}
\usepackage{pmmeta}
\pmcanonicalname{PermutationPattern}
\pmcreated{2013-03-22 16:24:41}
\pmmodified{2013-03-22 16:24:41}
\pmowner{mps}{409}
\pmmodifier{mps}{409}
\pmtitle{permutation pattern}
\pmrecord{4}{38560}
\pmprivacy{1}
\pmauthor{mps}{409}
\pmtype{Definition}
\pmcomment{trigger rebuild}
\pmclassification{msc}{05A05}
\pmclassification{msc}{05A15}
\pmdefines{pattern avoidance}

% this is the default PlanetMath preamble.  as your knowledge
% of TeX increases, you will probably want to edit this, but
% it should be fine as is for beginners.

% almost certainly you want these
\usepackage{amssymb}
\usepackage{amsmath}
\usepackage{amsfonts}

% used for TeXing text within eps files
%\usepackage{psfrag}
% need this for including graphics (\includegraphics)
%\usepackage{graphicx}
% for neatly defining theorems and propositions
%\usepackage{amsthm}
% making logically defined graphics
%%%\usepackage{xypic}

% there are many more packages, add them here as you need them

% define commands here

\begin{document}
\PMlinkescapeword{contains}
\PMlinkescapeword{contain}
\PMlinkescapeword{representation}
\PMlinkescapeword{length}
\PMlinkescapeword{isomorphism}

A \emph{permutation pattern} is simply a permutation viewed as its representation in one-line notation.  Let $\pi=\pi_1\pi_2\dots\pi_k$ be a permutation pattern on $k$ symbols.  Then for any permutation $\sigma=\sigma_1\sigma_2\dots\sigma_n\in\mathfrak{S}_n$, we say that $\sigma$ \emph{contains} $\pi$ if there is a (not necessarily contiguous) subword of $\sigma$ of length $k$ that is order-isomorphic to $\pi$.  More formally, 
for any subset $J=\{j_1,\dots,j_k\}\subset\{1,\dots,n\}$ of cardinality $k$, write 
\[
\sigma_J = \sigma_{j_1} \sigma_{j_2} \dots \sigma_{j_k}.
\]
There is a \PMlinkescapetext{natural} \PMlinkname{isomorphism}{GroupHomomorphism} $\mathfrak{S}_J\to\mathfrak{S}_k$.  We say that $\sigma$ \emph{contains} $\pi$ if there is some $J\subset\{1,\dots,n\}$ of cardinality $k$ such that $\sigma_J\mapsto \pi$ under the isomorphism.

If a permutation $\sigma$ does not contain $\pi$, then we say that $\sigma$ \emph{avoids} the pattern $\pi$.

For example, let $\pi=132$.  Then the permutation $\sigma=1234$ avoids $\pi$, since $\sigma$ is strictly increasing but $\pi$ has a descent.  On the other hand, $\tau=1423$ contains $\pi$ twice, once as the subword $142$ and once as the subword $143$.

Knuth showed that a permutation is stack-sortable if and only if it avoids the permutation pattern $231$.
%%%%%
%%%%%
\end{document}
