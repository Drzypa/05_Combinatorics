\documentclass[12pt]{article}
\usepackage{pmmeta}
\pmcanonicalname{IntegerPartition}
\pmcreated{2013-03-22 14:17:34}
\pmmodified{2013-03-22 14:17:34}
\pmowner{rm50}{10146}
\pmmodifier{rm50}{10146}
\pmtitle{integer partition}
\pmrecord{9}{35748}
\pmprivacy{1}
\pmauthor{rm50}{10146}
\pmtype{Definition}
\pmcomment{trigger rebuild}
\pmclassification{msc}{05A17}
\pmclassification{msc}{11P99}
\pmsynonym{partition}{IntegerPartition}
\pmsynonym{unordered partition}{IntegerPartition}
\pmrelated{PartitionFunction2}
\pmdefines{dual partition}
\pmdefines{Young diagram}
\pmdefines{Ferrers diagram}

\endmetadata

\usepackage{amssymb}
\usepackage{amsmath}
\usepackage{amsfonts}

\usepackage[all,web]{xypic}

\makeatletter
\@ifundefined{bibname}{}{\renewcommand{\bibname}{References}}
\makeatother

\def\drawsqlat{%
\begin{xy}{
  0;<1.7pc,0pc>:<0pc,1.7pc>::
  \xylattice{0}{6}{0}{6}}
\end{xy}}
\def\drawsq{\ar@{-}c;c+(1,0)\ar@{-}c;c+(0,1)\ar@{-}c+(1,0);c+(1,1)\ar@{-}c+(0,1);c+(1,1)}
\def\drawsqlabel#1{\save c+(0.5,0.5)*\txt<2pc>{#1} \restore}
\begin{document}
\PMlinkescapeword{row}

An (unordered) partition of a natural number $n$ is a way of writing $n$ as a
sum of natural numbers. For example, the following are the all
partitions of $5$:
\begin{align*}
5&=5&5&=2+2+1\\
5&=4+1&5&=2+1+1+1\\
5&=3+2&5&=1+1+1+1+1\\
5&=3+1+1
\end{align*}
Conventionally, parts of a partition are written from the largest
to the smallest. Instead of writing the partition as a sum, it is common to use the multiset notation, such as $\{8,5,5,5,2,1,1\}$ for $27=8+5+5+5+2+1+1$. Another common notation is to write multiplicities as superscripts, such as $1^22^15^38$ for $27=8+5+5+5+2+1+1$. Note that this way of writing partitions typically uses smallest to largest.

Partitions are often drawn as \emph{Young
diagrams}, which are rectangular arrays of boxes in which the $k$'th
row has a number of boxes equal to the $k$'th part of the
partition. Sometimes dots are used instead of boxes, and then the
obtained picture is called a \emph{Ferrers diagram}. For instance,
the partition $10=5+2+2+1$ is drawn

\begin{center}
\parbox[b]{10pc}{
\begin{renewcommand}{\latticebody}{%
\ifnum\latticeA=5 \ifnum\latticeB=4 \drawsq\fi\fi
\ifnum\latticeA=4 \ifnum\latticeB=4 \drawsq\fi\fi
\ifnum\latticeA=3 \ifnum\latticeB=4 \drawsq\fi\fi
\ifnum\latticeA=2 \ifnum\latticeB=4 \drawsq\fi\fi
\ifnum\latticeA=1 \ifnum\latticeB=4 \drawsq\fi\fi
\ifnum\latticeA=2 \ifnum\latticeB=3 \drawsq\fi\fi
\ifnum\latticeA=1 \ifnum\latticeB=3 \drawsq\fi\fi
\ifnum\latticeA=2 \ifnum\latticeB=2 \drawsq\fi\fi
\ifnum\latticeA=1 \ifnum\latticeB=2 \drawsq\fi\fi
\ifnum\latticeA=1 \ifnum\latticeB=1 \drawsq\fi\fi}
\drawsqlat
\end{renewcommand} 
Young diagram}\quad
\parbox[b]{10pc}{\begin{renewcommand}{\latticebody}{%
\ifnum\latticeA=5 \ifnum\latticeB=4 \drawsqlabel{$\bullet$}\fi\fi
\ifnum\latticeA=4 \ifnum\latticeB=4 \drawsqlabel{$\bullet$}\fi\fi
\ifnum\latticeA=3 \ifnum\latticeB=4 \drawsqlabel{$\bullet$}\fi\fi
\ifnum\latticeA=2 \ifnum\latticeB=4 \drawsqlabel{$\bullet$}\fi\fi
\ifnum\latticeA=1 \ifnum\latticeB=4 \drawsqlabel{$\bullet$}\fi\fi
\ifnum\latticeA=2 \ifnum\latticeB=3 \drawsqlabel{$\bullet$}\fi\fi
\ifnum\latticeA=1 \ifnum\latticeB=3 \drawsqlabel{$\bullet$}\fi\fi
\ifnum\latticeA=2 \ifnum\latticeB=2 \drawsqlabel{$\bullet$}\fi\fi
\ifnum\latticeA=1 \ifnum\latticeB=2 \drawsqlabel{$\bullet$}\fi\fi
\ifnum\latticeA=1 \ifnum\latticeB=1 \drawsqlabel{$\bullet$}\fi\fi
}\drawsqlat
\end{renewcommand} Ferrers diagram}
\end{center}


The \emph{dual partition} is the partition obtained by reflecting
the Young diagram along the main diagonal. For example, the Young
diagram of the partition dual to the one above is
\begin{center}
\begin{renewcommand}{\latticebody}{
\ifnum\latticeA=4 \ifnum\latticeB=5 \drawsq\fi\fi
\ifnum\latticeA=3 \ifnum\latticeB=5 \drawsq\fi\fi
\ifnum\latticeA=2 \ifnum\latticeB=5 \drawsq\fi\fi
\ifnum\latticeA=1 \ifnum\latticeB=5 \drawsq\fi\fi
\ifnum\latticeA=3 \ifnum\latticeB=4 \drawsq\fi\fi
\ifnum\latticeA=2 \ifnum\latticeB=4 \drawsq\fi\fi
\ifnum\latticeA=1 \ifnum\latticeB=4 \drawsq\fi\fi
\ifnum\latticeA=1 \ifnum\latticeB=3 \drawsq\fi\fi
\ifnum\latticeA=1 \ifnum\latticeB=2 \drawsq\fi\fi
\ifnum\latticeA=1 \ifnum\latticeB=1 \drawsq\fi\fi }
\drawsqlat
\end{renewcommand}
\end{center}
which is the diagram of $10=4+3+1+1+1$.

\begin{thebibliography}{1}

\bibitem{cite:stanley_enum_one}
Richard~P. Stanley.
\newblock {\em Enumerative Combinatorics}, volume~I.
\newblock Wadsworth \& Brooks, 1986.
\newblock \PMlinkexternal{Zbl 0608.05001}{http://www.emis.de/cgi-bin/zmen/ZMATH/en/quick.html?type=html&an=0608.05001}.

\end{thebibliography}
%%%%%
%%%%%
\end{document}
