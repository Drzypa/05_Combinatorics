\documentclass[12pt]{article}
\usepackage{pmmeta}
\pmcanonicalname{ChromaticNumberOfAMetricSpace}
\pmcreated{2013-03-22 14:06:31}
\pmmodified{2013-03-22 14:06:31}
\pmowner{Mathprof}{13753}
\pmmodifier{Mathprof}{13753}
\pmtitle{chromatic number of a metric space}
\pmrecord{44}{35510}
\pmprivacy{1}
\pmauthor{Mathprof}{13753}
\pmtype{Definition}
\pmcomment{trigger rebuild}
\pmclassification{msc}{05C15}
\pmclassification{msc}{52C10}
%\pmkeywords{chromatic number of the plane}
\pmrelated{ChromaticNumber}
\pmdefines{Moser spindle}

\usepackage{amssymb}
\usepackage{amsmath}
\usepackage{amsfonts}
\usepackage{graphicx}

\usepackage[T2A]{fontenc}
\usepackage[russian,english]{babel}

\usepackage[dvips,all,web]{xypic}

\makeatletter
\@ifundefined{url}{\usepackage{url}}{}
\@ifundefined{latexonly}{\newcommand{\ltxonly}[2]{#1}}{\newcommand{\ltxonly}[2]{#2}}
\makeatother


\makeatletter
\@ifundefined{makeimage}{\newcommand{\mkimg}{}}{\newcommand{\mkimg}[1]{\begin{makeimage}#1\end{makeimage}}}
\makeatother

\makeatletter
\@ifundefined{bibname}{}{\renewcommand{\bibname}{References}}
\makeatother

\newcommand*{\R}{\mathbb{R}}
\newcommand*{\Q}{\mathbb{Q}}

\def\drawhexlat{% 1.2 bigger
\begin{xy}{
0;<1.8pc,1.039230484541326376pc>:<0pc,2.07846096908265275223293560980705pc>::
  \xylattice{-4}{8}{-4}{8}}
\end{xy}}

\def\drawhexd{\ar@{-}c;c+(0.666666666666,-0.333333333333)}
\def\drawhexdr{\ar@{-}c+(0.666666666666,-0.333333333333);c+(1,0)}
\def\drawhexdrhalfi{\ar@{-}c+(0.666666666666,-0.333333333333);c+(0.833333333333,-0.166666666666)}
\def\drawhexdrhalfii{\ar@{-}c+(0.833333333333,-0.166666666666);c+(1,0)}
\def\drawhexur{\ar@{-}c+(1,0);c+(0.666666666666,0.666666666666)}
\def\drawhexurhalfi{\ar@{-}c+(1,0);c+(0.833333333333,0.333333333333)}
\def\drawhexurhalfii{\ar@{-}c+(0.833333333333,0.333333333333);c+(0.666666666666,0.666666666666)}
\def\drawhexu{\ar@{-}c+(0.666666666666,0.666666666666);c+(0,1)}
\def\drawhexuhalfi{\ar@{-}c+(0.666666666666,0.666666666666);c+(0.333333333333,0.833333333333)}
\def\drawhexuhalfii{\ar@{-}c+(0.333333333333,0.833333333333);c+(0,1)}
\def\drawhexul{\ar@{-}c+(0,1);c+(-0.333333333333,0.666666666666)}
\def\drawhexdl{\ar@{-}c+(-0.333333333333,0.666666666666);c}
\def\drawhex{%
\ar@{-}c;c+(0.666666666666,-0.333333333333)%
\ar@{-}c+(0.666666666666,-0.333333333333);c+(1,0)%
\ar@{-}c+(1,0);c+(0.666666666666,0.666666666666)%
\ar@{-}c+(0.666666666666,0.666666666666);c+(0,1)%
\ar@{-}c+(0,1);c+(-0.333333333333,0.666666666666)%
\ar@{-}c+(-0.333333333333,0.666666666666);c}
\def\drawhexlabel#1{\save c+(0.333333333333,0.333333333333)*\txt<2pc>{#1} \restore}

\begin{document}
\PMlinkescapeword{information}
\PMlinkescapeword{blue}
\PMlinkescapeword{argument}

The \emph{\PMlinkescapetext{chromatic number} of a \PMlinkname{metric space}{MetricSpace}} is the minimum number of
colors required to color the space in such a way that no two
points at distance $1$ are assigned the same color. Alternatively,
the chromatic number of a metric space is the
\PMlinkname{chromatic number}{ChromaticNumber} of a graph whose
vertices are points of the space, and two points are connected by
an edge if they are at distance $1$ from each other.

For example, the chromatic number of $\mathbb{R}$ is $2$ because
it is impossible to color $\mathbb{R}$ into one color, but it is
possible to color $\mathbb{R}$ into $2$ colors by coloring each of
the intervals $[k,k+1)$, where $k$ is an integer, red or blue
according to whether $k$ is odd or even.

Unlike $\mathbb{R}$, the chromatic number of $\mathbb{R}^n$ is not
known for any $n\geq 2$. For example, the chromatic number of the
plane is known to be between $4$ and $7$ as the following pictures
show.

\parbox[b]{10pc}{\begin{xy}
  0;<7pc,0pc>:<0pc,7pc>::
@={(-1.,0.),(0.,0.),(0.228714,0.973494),(-0.5,1.65831),(-1.22871,0.973494),(-0.271286,0.684819),(-0.728714,0.684819)}@@{*{\ensuremath{\bullet}}}
  \ar@{-}(-1.,0.);(0.,0.)
  \ar@{-}(0.,0.);(0.228714,0.973494)
  \ar@{-}(0.228714,0.973494);(-0.5,1.65831)
  \ar@{-}(-0.5,1.65831);(-1.22871,0.973494)
  \ar@{-}(-1.22871,0.973494);(-1.,0.)
  \ar@{-}(-1.,0.);(-0.271286,0.684819)
  \ar@{-}(-1.22871,0.973494);(-0.271286,0.684819)
  \ar@{-}(-0.5,1.65831);(-0.271286,0.684819)
  \ar@{-}(0.,0.);(-0.728714,0.684819)
  \ar@{-}(0.228714,0.973494);(-0.728714,0.684819)
  \ar@{-}(-0.5,1.65831);(-0.728714,0.684819)
\end{xy}
The Moser spindle. All edges are of unit length. The chromatic
number is $4$.}\hspace{1pc}
\parbox[b]{15pc}{
\begin{renewcommand}{\latticebody}{%
  \ifnum\latticeA=-4 \ifnum\latticeB=4 \drawhexuhalfi\fi\fi
  \ifnum\latticeA=-4 \ifnum\latticeB=5 \drawhexuhalfi\drawhexlabel{7}\fi\fi
  \ifnum\latticeA=-3 \ifnum\latticeB=2 \drawhexdrhalfii\fi\fi
  \ifnum\latticeA=-3 \ifnum\latticeB=3 \drawhex\drawhexlabel{3}\fi\fi
  \ifnum\latticeA=-3 \ifnum\latticeB=4 \drawhex\drawhexlabel{4}\fi\fi
  \ifnum\latticeA=-3 \ifnum\latticeB=5 \drawhex\drawhexlabel{5}\fi\fi
  \ifnum\latticeA=-3 \ifnum\latticeB=6 \drawhex\drawhexlabel{6}\fi\fi
  \ifnum\latticeA=-3 \ifnum\latticeB=7 \drawhexurhalfi\fi\fi
  \ifnum\latticeA=-2 \ifnum\latticeB=1 \drawhexdrhalfii\fi\fi
  \ifnum\latticeA=-2 \ifnum\latticeB=2 \drawhex\drawhexlabel{7}\fi\fi
  \ifnum\latticeA=-2 \ifnum\latticeB=3 \drawhex\drawhexlabel{1}\fi\fi
  \ifnum\latticeA=-2 \ifnum\latticeB=4 \drawhex\drawhexlabel{2}\fi\fi
  \ifnum\latticeA=-2 \ifnum\latticeB=5 \drawhex\drawhexlabel{3}\fi\fi
  \ifnum\latticeA=-2 \ifnum\latticeB=6 \drawhex\drawhexlabel{4}\fi\fi
  \ifnum\latticeA=-2 \ifnum\latticeB=7 \drawhexurhalfi\fi\fi
  \ifnum\latticeA=-1 \ifnum\latticeB=1 \drawhex\drawhexlabel{4}\fi\fi
  \ifnum\latticeA=-1 \ifnum\latticeB=2 \drawhex\drawhexlabel{5}\fi\fi
  \ifnum\latticeA=-1 \ifnum\latticeB=3 \drawhex\drawhexlabel{6}\fi\fi
  \ifnum\latticeA=-1 \ifnum\latticeB=4 \drawhex\drawhexlabel{7}\fi\fi
  \ifnum\latticeA=-1 \ifnum\latticeB=5 \drawhex\drawhexlabel{1}\fi\fi
  \ifnum\latticeA=-1 \ifnum\latticeB=6 \drawhex\drawhexlabel{2}\fi\fi
  \ifnum\latticeA=0 \ifnum\latticeB=0 \drawhex\drawhexlabel{1}\fi\fi
  \ifnum\latticeA=0 \ifnum\latticeB=1 \drawhex\drawhexlabel{2}\fi\fi
  \ifnum\latticeA=0 \ifnum\latticeB=2 \drawhex\drawhexlabel{3}\fi\fi
  \ifnum\latticeA=0 \ifnum\latticeB=3 \drawhex\drawhexlabel{4}\fi\fi
  \ifnum\latticeA=0 \ifnum\latticeB=4 \drawhex\drawhexlabel{5}\fi\fi
  \ifnum\latticeA=0 \ifnum\latticeB=5 \drawhex\drawhexlabel{6}\fi\fi
  \ifnum\latticeA=0 \ifnum\latticeB=6 \drawhex\drawhexlabel{7}\fi\fi
  \ifnum\latticeA=1 \ifnum\latticeB=-1 \drawhexurhalfii\fi\fi
  \ifnum\latticeA=1 \ifnum\latticeB=0 \drawhex\drawhexlabel{6}\fi\fi
  \ifnum\latticeA=1 \ifnum\latticeB=1 \drawhex\drawhexlabel{7}\fi\fi
  \ifnum\latticeA=1 \ifnum\latticeB=2 \drawhex\drawhexlabel{1}\fi\fi
  \ifnum\latticeA=1 \ifnum\latticeB=3 \drawhex\drawhexlabel{2}\fi\fi
  \ifnum\latticeA=1 \ifnum\latticeB=4 \drawhex\drawhexlabel{3}\fi\fi
  \ifnum\latticeA=1 \ifnum\latticeB=5 \drawhex\drawhexlabel{4}\fi\fi
  %\ifnum\latticeA=1 \ifnum\latticeB=6 \drawhexdr\fi\fi
  \ifnum\latticeA=1 \ifnum\latticeB=6 \drawhexdrhalfi\fi\fi
  \ifnum\latticeA=2 \ifnum\latticeB=-1 \drawhexurhalfii\fi\fi
  \ifnum\latticeA=2 \ifnum\latticeB=0 \drawhex\drawhexlabel{4}\fi\fi
  \ifnum\latticeA=2 \ifnum\latticeB=1 \drawhex\drawhexlabel{5}\fi\fi
  \ifnum\latticeA=2 \ifnum\latticeB=2 \drawhex\drawhexlabel{6}\fi\fi
  \ifnum\latticeA=2 \ifnum\latticeB=3 \drawhex\drawhexlabel{7}\fi\fi
  \ifnum\latticeA=2 \ifnum\latticeB=4 \drawhex\drawhexlabel{1}\fi\fi
  \ifnum\latticeA=2 \ifnum\latticeB=5 \drawhexdrhalfi\fi\fi
  \ifnum\latticeA=3 \ifnum\latticeB=0 \drawhex\drawhexlabel{2}\fi\fi
  \ifnum\latticeA=3 \ifnum\latticeB=1 \drawhex\drawhexlabel{3}\fi\fi
  \ifnum\latticeA=3 \ifnum\latticeB=2 \drawhex\drawhexlabel{4}\fi\fi
  \ifnum\latticeA=3 \ifnum\latticeB=3 \drawhex\drawhexlabel{5}\fi\fi
  \ifnum\latticeA=4 \ifnum\latticeB=0 \drawhexuhalfii\fi\fi
  \ifnum\latticeA=4 \ifnum\latticeB=1 \drawhexuhalfii\drawhexlabel{1}\fi\fi
}
\drawhexlat
\end{renewcommand}
Periodic coloring of the plane
with $7$ colors. The diameter of each hexagonal region is slightly
less than $1$.}

Using \PMlinkname{compactness argument}{GettingModelsIModelsConstructedFromConstants} one can show that if every finite set
of points in the plane can be colored with $4$ colors, then the
whole plane can be colored with $4$. However, if such a coloring
exists, then at least one of the color sets is nonmeasurable with respect to the Lebesgue measure \cite{cite:falconer_msrbl_chrom_Rn}.

If the coloring of the plane is to consist of regions bounded by
Jordan curves, then at least $6$ colors are needed
\cite{cite:woodall_chrom_plane}.  Moreover, under the additional
assumption that no two regions, which are distance less than $1$ apart,
receive the same color, then at least $7$ colors are needed
\cite{cite:thomassen_planechromnum}.

%The chromatic number of $\mathbb{R}^n$ is known to be between
%$(1.239+o(1))^n$ and
%$(3+o(1))^n$\cite{cite:raigorodskii_chrom_rn}.

The following table summarizes known lower and upper bounds on
the chromatic number of some metric spaces.

\begin{tabular}{cccccccc}
&$\R^2$&$\R^3$&$\R^4$&$\R^5$&$\R^6$&$\R^7$&$\R^n$\\
Upper bound&7&15\cite{cite:coulson_chrom_rthree,cite:radoicic_toth_chrom_rthree}& & & & & $(3+o(1))^n$\cite{cite:larman_rogers_chrom_rn}\\
\PMlinkescapetext{Lower bound}&4&6\cite{cite:neuchushtan}&7\cite{cite:cantwell}&%
9\cite{cite:cantwell}&10\cite{cite:larman_rogers_chrom_rn}&%
15\cite{cite:raigorodskii_survey}&$(1.239+o(1))^n$\cite{cite:raigorodskii_survey,cite:raigorodskii_chrom_rn}
\end{tabular}\vspace{1.5ex}

\begin{tabular}{cccccccc}
&$\Q^2$&$\Q^3$&$\Q^4$&$\Q^5$&$\Q^6$&$\Q^7$&$\Q^n$\\
Upper bound&2\cite{cite:woodall_chrom_plane}&2\cite{cite:johnson_chrom_qsmall}&4\cite{cite:benda_perles}& &  & &\\
\PMlinkescapetext{Lower
bound}&2\cite{cite:woodall_chrom_plane}&2\cite{cite:johnson_chrom_qsmall}&4\cite{cite:benda_perles}&7\cite{cite:mann_chrom_qfive}
&10\cite{cite:mann_chrom_qn}%
&13\cite{cite:mann_chrom_qn}&$(1.173+o(1))^n$\cite{cite:raigorodskii_survey}
\end{tabular}

For more information on the lower bounds for the chromatic numbers of
$\R^n$ see \cite{cite:szekely_unitdist_szemtrot}.


\begin{thebibliography}{10}

\bibitem{cite:benda_perles}
M.~Benda and M.~Perles.
\newblock Colorings of metric spaces.
\newblock {\em Geombinatorics}, 9(3):113--126, 2000.
\newblock \PMlinkexternal{Zbl
  0951.05037}{http://www.emis.de/cgi-bin/zmen/ZMATH/en/quick.html?type=html&an=0951.05037}.

\bibitem{cite:coulson_chrom_rthree}
David Coulson.
\newblock A $15$-colouring of $3$-space omitting distance one.
\newblock {\em Discrete Math.}, 256:83--90, 2002.
\newblock \PMlinkexternal{Zbl
  1007.05052}{http://www.emis.de/cgi-bin/zmen/ZMATH/en/quick.html?type=html&an=1007.05052}.

\bibitem{cite:falconer_msrbl_chrom_Rn}
K.~J. Falconer.
\newblock The realization of distances in measurable subsets covering
  $\mathbb{R}^n$.
\newblock {\em J.\ Combin.\ Theory Ser. A}, 31:184--189, 1981.
\newblock \PMlinkexternal{Zbl
  0469.05021}{http://www.emis.de/cgi-bin/zmen/ZMATH/en/quick.html?type=html&an=0469.05021}.

\bibitem{cite:johnson_chrom_qsmall}
Peter~D. Johnson.
\newblock Coloring abelian groups.
\newblock {\em Discrete Math.}, 40:219--223, 1982.
\newblock \PMlinkexternal{Zbl
  0485.05009}{http://www.emis.de/cgi-bin/zmen/ZMATH/en/quick.html?type=html&an=0485.05009}.

\bibitem{cite:larman_rogers_chrom_rn}
D.~G. Larman and C.~A. Rogers.
\newblock The realization of distances within sets in Euclidean space.
\newblock {\em Mathematika}, 19:1--24, 1972.
\newblock \PMlinkexternal{Zbl
  0246.05020}{http://www.emis.de/cgi-bin/zmen/ZMATH/en/quick.html?type=html&an=0246.05020}.

\bibitem{cite:mann_chrom_qfive}
Matthias Mann.
\newblock A new bound for the chromatic number of the rational five-space.
\newblock {\em Geombinatorics}, 11(2):49--53, 2001.
\newblock \PMlinkexternal{Zbl
  0995.05051}{http://www.emis.de/cgi-bin/zmen/ZMATH/en/quick.html?type=html&an=0995.05051}.

\bibitem{cite:mann_chrom_qn}
Matthias Mann.
\newblock Hunting unit-distance graphs in rational $n$-spaces.
\newblock {\em Geombinatorics}, 13(2):86--97, 2003.

\bibitem{cite:radoicic_toth_chrom_rthree}
Rado{\v{s}} Radoi{\v{c}}i{\'c} and G{\'e}za T{\'o}th.
\newblock Note on the chromatic number of the space.
\newblock In {\em Discrete and computational geometry}, volume~25 of {\em
  Algorithms Combin.}, pages 695--698. Springer, Berlin, 2003.
\newblock \PMlinkexternal{Zbl 1071.05527}{http://www.emis.de/cgi-bin/zmen/ZMATH/en/quick.html?type=html&an=1071.05527}.


\bibitem{cite:raigorodskii_chrom_rn}
Andrei~M. Raigorodskii.
\newblock On the chromatic number of a space.
\newblock {\em Russian Math. Surveys}, 55(2):351--352, 2000.
\newblock \PMlinkexternal{Zbl
  0966.05029}{http://www.emis.de/cgi-bin/zmen/ZMATH/en/quick.html?type=html&an=0966.05029}.

\bibitem{cite:raigorodskii_survey}
Andrei~M. Raigorodskii.
\newblock Borsuk's problem and the chromatic numbers of some metric spaces.
\newblock {\em Russian Math. Surveys}, 56(1):103--139, 2001.
\newblock \PMlinkexternal{Zbl
  1008.54018}{http://www.emis.de/cgi-bin/zmen/ZMATH/en/quick.html?type=html&an=1008.54018}.

\bibitem{cite:raigorodskii_lecture}
Andrei~M. Raigorodskii.
\newblock % (garbage????)
\newblock Available at
\PMlinkexternal{http://www.mccme.ru/mmmf-lectures/books/index.php?task=video}{http://www.mccme.ru/mmmf-lectures/books/index.php?task=video}, 2002.

\bibitem{cite:raiskii_borsuk}
D.~E. Raiskii.
\newblock The realization of all distances in a decomposition of the space
  $\mathbb{R}^n$ into $n+1$ parts.
\newblock {\em Math. Notes}, 7:194--196, 1970.
\newblock \PMlinkexternal{Zbl
  0202.21702}{http://www.emis.de/cgi-bin/zmen/ZMATH/en/quick.html?type=html&an=0202.21702}.

\bibitem{cite:szekely_unitdist_szemtrot}
L{\'a}szl{\'o}~A. Sz{\'e}kely.
\newblock Erd{\H{o}}s on unit distances and {S}zemer{\'e}di-{T}rotter theorems.
\newblock Preprint is at
  \PMlinkexternal{http://www.math.sc.edu/~szekely/}{http://www.math.sc.edu/~szekely/}, 2002.

\bibitem{cite:thomassen_planechromnum}
Carsten Thomassen.
\newblock On the {N}elson unit distance coloring problem.
\newblock {\em Amer. Math. Monthly}, 106(9):850--853, 1999.
\newblock \PMlinkexternal{Zbl 0986.05041}{http://www.emis.de/cgi-bin/zmen/ZMATH/en/quick.html?type=html&an=0986.05041}.
\newblock \PMlinkexternal{Available
  online}{http://links.jstor.org/sici?sici=0002-9890\%28199911\%29106\%3A9\%3C\%850\%3AOTNUDC\%3E2.0.CO\%3B2-U} at
  \PMlinkexternal{JSTOR}{http://www.jstor.org}.


\bibitem{cite:woodall_chrom_plane}
D.~R. Woodall.
\newblock Distances realized by sets covering the plane.
\newblock {\em J.\ Combin.\ Theory Ser. A}, 14:187--200, 1973.
\newblock \PMlinkexternal{Zbl
  0251.50003}{http://www.emis.de/cgi-bin/zmen/ZMATH/en/quick.html?type=html&an=0251.50003}.

\bibitem{cite:cantwell}
K. Cantwell.
\newblock Finite Euclidean Ramsey theory.
\newblock {\em J.\ Combin. Theory Ser. A}, 73:273--285, 1996.
\newblock \PMlinkexternal{Zbl 0842.05064}{http://www.emis.de/cgi-bin/zmen/ZMATH/en/quick.html?type=html&an=0842.05064}.

\bibitem{cite:neuchushtan}
O. Neuchushtan.
\newblock On the space chromatic number.
\newblock {\em Discrete Mathematics.} 256:499--507, 2002.
\newblock \PMlinkexternal{Zbl 1009.05058}{http://www.emis.de/cgi-bin/zmen/ZMATH/en/quick.html?type=html&an=1009.05058}.
\end{thebibliography}

%%%%%
%%%%%
\end{document}
