\documentclass[12pt]{article}
\usepackage{pmmeta}
\pmcanonicalname{ExampleOfPigeonholePrinciple}
\pmcreated{2013-03-22 12:41:32}
\pmmodified{2013-03-22 12:41:32}
\pmowner{Mathprof}{13753}
\pmmodifier{Mathprof}{13753}
\pmtitle{example of pigeonhole principle}
\pmrecord{8}{32972}
\pmprivacy{1}
\pmauthor{Mathprof}{13753}
\pmtype{Example}
\pmcomment{trigger rebuild}
\pmclassification{msc}{05-00}

\usepackage{graphicx}
%%%\usepackage{xypic} 
\usepackage{bbm}
\usepackage{amsthm}
\newtheorem*{thm}{Theorem}
\newcommand{\Z}{\mathbbmss{Z}}
\newcommand{\C}{\mathbbmss{C}}
\newcommand{\R}{\mathbbmss{R}}
\newcommand{\Q}{\mathbbmss{Q}}{
\newcommand{\mathbb}[1]{\mathbbmss{#1}}
\newcommand{\figura}[1]{\begin{center}\includegraphics{#1}\end{center}}

\begin{document}
A \PMlinkescapetext{simple} example.
\begin{thm} For any set of $8$ integers, there exist at least two of them 
whose difference is divisible by $7$.
\end{thm}

\small
\begin{proof}
The residue classes modulo $7$ are $0,1,2,3,4,5,6$. 
We have seven \PMlinkescapetext{classes} and eight integers. So it must be the case that 2 integers fall on the same
residue class, and therefore their difference will be divisible by $7$.
\end{proof}
%%%%%
%%%%%
\end{document}
