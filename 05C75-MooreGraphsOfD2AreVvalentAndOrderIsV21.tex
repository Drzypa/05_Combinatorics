\documentclass[12pt]{article}
\usepackage{pmmeta}
\pmcanonicalname{MooreGraphsOfD2AreVvalentAndOrderIsV21}
\pmcreated{2013-03-22 15:11:30}
\pmmodified{2013-03-22 15:11:30}
\pmowner{marijke}{8873}
\pmmodifier{marijke}{8873}
\pmtitle{Moore graphs of $d=2$ are $v$-valent and order is $v^2+1$}
\pmrecord{6}{36948}
\pmprivacy{1}
\pmauthor{marijke}{8873}
\pmtype{Proof}
\pmcomment{trigger rebuild}
\pmclassification{msc}{05C75}

\usepackage{amssymb}
% \usepackage{amsmath}
% \usepackage{amsfonts}

% used for TeXing text within eps files
%\usepackage{psfrag}
% need this for including graphics (\includegraphics)
%\usepackage{graphicx}

% for neatly defining theorems and propositions
%\usepackage{amsthm}
% making logically defined graphics
%%%\usepackage{xypic}

% there are many more packages, add them here as you need them

% define commands here %%%%%%%%%%%%%%%%%%%%%%%%%%%%%%%%%%
% portions from
% makra.sty 1989-2005 by Marijke van Gans %
                                          %          ^ ^
\catcode`\@=11                            %          o o
                                          %         ->*<-
                                          %           ~
%%%% CHARS %%%%%%%%%%%%%%%%%%%%%%%%%%%%%%%%%%%%%%%%%%%%%%

                        %    code    char  frees  for

\let\Para\S             %    \Para     §   \S \scriptstyle
\let\Pilcrow\P          %    \Pilcrow  ¶   \P
\mathchardef\pilcrow="227B

\mathchardef\lt="313C   %    \lt       <   <     bra
\mathchardef\gt="313E   %    \gt       >   >     ket

\let\bs\backslash       %    \bs       \
\let\us\_               %    \us       _     \_  ...

%%%% DIACRITICS %%%%%%%%%%%%%%%%%%%%%%%%%%%%%%%%%%%%%%%%%

%let\udot\d             % under-dot (text mode), frees \d
\let\odot\.             % over-dot (text mode),  frees \.
%let\hacek\v            % hacek (text mode),     frees \v
%let\makron\=           % makron (text mode),    frees \=
%let\tilda\~            % tilde (text mode),     frees \~
\let\uml\"              % umlaut (text mode),    frees \"

%def\trema#1{\discretionary{-}{#1}{\uml #1}}

%%%% amssymb %%%%%%%%%%%%%%%%%%%%%%%%%%%%%%%%%%%%%%%%%%%%

\let\le\leqslant
\let\ge\geqslant
%let\prece\preceqslant
%let\succe\succeqslant

%%%% USEFUL MISC %%%%%%%%%%%%%%%%%%%%%%%%%%%%%%%%%%%%%%%%

%%%% KERNING, SPACING, BREAKING %%%%%%%%%%%%%%%%%%%%%%%%%

\def\comma{,\,\allowbreak}

%%%% LAYOUT %%%%%%%%%%%%%%%%%%%%%%%%%%%%%%%%%%%%%%%%%%%%%

%%%% MATH LAYOUT %%%%%%%%%%%%%%%%%%%%%%%%%%%%%%%%%%%%%%%%

\let\D\displaystyle
\let\T\textstyle
\let\S\scriptstyle
\let\SS\scriptscriptstyle

%%%% MATH SYMBOLS %%%%%%%%%%%%%%%%%%%%%%%%%%%%%%%%%%%%%%%

%def\d{\mathord{\rm d}}                      % d as in dx
%def\e{{\rm e}}                              % e as in e^x

%def\Ell{\hbox{\it\char`\$}}

\def\sfmath#1{{\mathchoice%
{{\sf #1}}{{\sf #1}}{{\S\sf #1}}{{\SS\sf #1}}}}
\def\Stalkset#1{\sfmath{I\kern-.12em#1}}
\def\Bset{\Stalkset B}
\def\Nset{\Stalkset N}
\def\Rset{\Stalkset R}
\def\Hset{\Stalkset H}
\def\Fset{\Stalkset F}
\def\kset{\Stalkset k}
\def\In@set{\raise.14ex\hbox{\i}\kern-.237em\raise.43ex\hbox{\i}}
\def\Roundset#1{\sfmath{\kern.14em\In@set\kern-.4em#1}}
\def\Qset{\Roundset Q}
\def\Cset{\Roundset C}
\def\Oset{\Roundset O}
\def\Zset{\sfmath{Z\kern-.44emZ}}

% \frac overwrites LaTeX's one (use TeX \over instead)
%def\fraq#1#2{{}^{#1}\!/\!{}_{\,#2}}
\def\frac#1#2{\mathord{\mathchoice%
{\T{#1\over#2}}
{\T{#1\over#2}}
{\S{#1\over#2}}
{\SS{#1\over#2}}}}

\mathcode`\<="4268         % < now is \langle, \lt is <
\mathcode`\>="5269         % > now is \rangle, \gt is >

\let\epsi=\varepsilon
\def\omikron{o}

\def\Alpha{{\rm A}}
\def\Beta{{\rm B}}
\def\Epsilon{{\rm E}}
\def\Zeta{{\rm Z}}
\def\Eta{{\rm H}}
\def\Iota{{\rm I}}
\def\Kappa{{\rm K}}
\def\Mu{{\rm M}}
\def\Nu{{\rm N}}
\def\Omikron{{\rm O}}
\def\Rho{{\rm P}}
\def\Tau{{\rm T}}
\def\Ypsilon{{\rm Y}} % differs from \Upsilon
\def\Chi{{\rm X}}

%def\dg{^{\circ}}                   % degrees

%def\1{^{-1}}                       % inverse

\def\*#1{{\bf #1}}                  % boldface e.g. vector
%def\vi{\mathord{\hbox{\bf\i}}}     % boldface vector \i
%def\vj{\mathord{\,\hbox{\bf\j}}}   % boldface vector \j

%def\union{\mathbin\cup}
%def\isect{\mathbin\cap}

\let\so\Longrightarrow
\let\oso\Longleftrightarrow
\let\os\Longleftarrow

% := and :<=>
%def\isdef{\mathrel{\smash{\stackrel{\SS\rm def}{=}}}}
%def\iffdef{\mathrel{\smash{stackrel{\SS\rm def}{\oso}}}}

\def\qed{ ${\S\circ}\!{}^\circ\!{\S\circ}$}
%def\qed{\vrule height 6pt width 6pt depth 0pt}

\def\mod#1{\allowbreak \mkern 10mu({\rm mod}\,\,#1)}
% redefines TeX's one using less space

%def\conj#1{\overline{#1\vphantom1}}
%def\cj#1{\overline{#1\vphantom+}}

%def\forAll{\mathop\forall\limits}
%def\Exists{\mathop\exists\limits}

%%%% PICTURES %%%%%%%%%%%%%%%%%%%%%%%%%%%%%%%%%%%%%%%%%%%

\def\cent{\makebox(0,0)}

%def\node{\circle*4}
%def\nOde{\circle4}

%%%% REFERENCES %%%%%%%%%%%%%%%%%%%%%%%%%%%%%%%%%%%%%%%%%

%def\opcit{[{\it op.\,cit.}]}
\def\bitem#1{\bibitem[#1]{#1}}
\def\name#1{{\sc #1}}
\def\book#1{{\sl #1\/}}
\def\paper#1{``#1''}
\def\mag#1{{\it #1\/}}
\def\vol#1{{\bf #1}}
\def\isbn#1{{\small\tt ISBN\,\,#1}}
\def\seq#1{{\small\tt #1}}
%def\url<{\verb>}
%def\@cite#1#2{[{#1\if@tempswa\ #2\fi}]}

%%%% VERBATIM CODE %%%%%%%%%%%%%%%%%%%%%%%%%%%%%%%%%%%%%%

%def\"{\verb"}

%%%% AD HOC %%%%%%%%%%%%%%%%%%%%%%%%%%%%%%%%%%%%%%%%%%%%%

%%%% WORDS %%%%%%%%%%%%%%%%%%%%%%%%%%%%%%%%%%%%%%%%%%%%%%

%def\oord/{o{\trema o}rdin\-ate}
% usage: C\oord/s, c\oord/.
% output: co\"ord... except when linebreak at co-ord...

%%%%%%%%%%%%%%%%%%%%%%%%%%%%%%%%%%%%%%%%%%%%%%%%%%%%%%%%%

                                          %
                                          %          ^ ^
\catcode`\@=12                            %          ` '
                                          %         ->*<-
                                          %           ~
\begin{document}
\PMlinkescapeword{node}
\PMlinkescapeword{nodes}
\PMlinkescapeword{size}
\PMlinkescapeword{type}
\PMlinkescapeword{line}
\PMlinkescapeword{circle}

{\bf Given} a Moore graph with diameter 2 and girth 5, which implies the
existence of cycles. {\bf To prove} every node (vertex) has the same valency,
$v$ say, and the number $n$ of nodes (vertices) is $v^2+1$.

{\bf Lemma:} a key feature of Moore graphs with diameter 2 is that any nodes X
and \PMlinkescapetext{Z} that are not adjacent have exactly one shared neighbour Y. {\bf Proof of lemma:} at least one Y$_k$ because distance XY is not 1 so must be 2, at most one because XY$_0$ZY$_1$ would be a 4-cycle.~\qed

{\bf Lemma:} every two adjacent nodes B and C lie together on some 5-cycle.
{\bf Proof of lemma:} every node (other than B) is at distance 1 or 2 from B, and every node (other than C) at distance 1 or 2 from C. There are no nodes X
at distance 1 from both (BXC would be a 3-cycle). Suppose the graph only has nodes at distance 1 from B and 2 from C (call them A$_i$), and nodes at distance 1 from C and 2 from B (call them D$_j$). Now no cycle can exist (the only edges are A$_i$B, BC, CD$_j$; any edge of type AA, AC, \PMlinkescapetext{AD}, BD, DD would create a 3-or 4-cycle). But Moore graphs have cycles by definition. So there must be at least one node E at distance 2 from both B and C. Let A be the joint neighbour of B and E, and D that of C and E (note A $\ne$ D, otherwise BAC would be a 3-cycle). Now CDEAB is a 5-cycle with edge BC.~\qed

{\bf Lemma:} two nodes O and Q at distance~2 have the same valency. {\bf Proof
of lemma:} let P be the unique joint neighbour. Let O have $o$ other neighbours
N$_i$ and let Q have $q$ other neighbours R$_j$.

\begin{center}
\begin{picture}(100,110)(-50,-55)

\put(+44,+27){\circle{12}}
\put( -7,+44){\circle*{4}}
\put(-40,  0){\circle*{4}}
\put( -7,-44){\circle*{4}}
\put(+44,-27){\circle{12}}

\put( -7,+44){\line(+3,-1){45}}
\put(-40,  0){\line(+3,+4){33}}
\put(-40,  0){\line(+3,-4){33}}
\put( -7,-44){\line(+3,+1){45}}

\put( +7,+44){\cent[b]{$o$}}
\put(+31,+36){\cent[b]{$1$}}
\put(+44,+27){\cent{$o$}}
\put(+44,-27){\cent{$q$}}
\put(+31,-36){\cent[t]{$1$}}
\put( +7,-44){\cent[t]{$q$}}

\put(+54,+27){\cent[l]{\small N$_i$}}
\put(-10,+48){\cent[rb]{\small O}}
\put(-46,  0){\cent[r]{\small P}}
\put(-10,-48){\cent[rt]{\small Q}}
\put(+54,-27){\cent[l]{\small R$_j$}}

\end{picture}
\end{center}

No N$_i$ can be adjacent to P (3-cycle PON$_i$) so the unique joint neighbours
of any N$_i$ and Q must be among the R$_j$. Different N$_i$ and N$_{i'}$
cannot use the same R$_j$ (4-cycle ON$_i$R$_j$N$_{i'}$) so we have $o\le q$.
By the same argument (swapping O and Q, N$_i$ and R$_j$) also $q\le o$ so we
have $o=q$, both nodes have valency $q+1$.~\qed

{\bf Lemma:} two adjacent nodes O and S have the same valency. {\bf Proof
of lemma:} let OPQRS be the 5-cycle through SO. Calling the valency of Q
again $q+1$, both O and S have that same valency by the previous lemma.~\qed

{\bf Proof of theorem:} the graph is connected. Travel from any node to any
other via adjacent ones, the valency stays the same by the last lemma (let's
keep calling it $q+1$).

\begin{center}
\begin{picture}(100,110)(-50,-55)

\put(-44,+27){\circle*{4}}
\put( +7,+44){\circle{12}}
\put(+40,  0){\circle{20}}
\put( +7,-44){\circle{12}}
\put(-44,-27){\circle*{4}}

\put( +1,+42){\line(-3,-1){45}}
\put(+34, +8){\line(-3,+4){23}}
\put(+34, -8){\line(-3,-4){23}}
\put( +1,-42){\line(-3,+1){45}}
\put(-44,-27){\line( 0,+1){54}}
\put(+50,  0){\line(+4,+1){24}}
\put(+50,  0){\line(+4,-1){24}}
\put(+74,  0){\oval(12,12)[r]}

\put( +7,+44){\cent{$q$}}
\put(+40,  0){\cent{$q^2$}}
\put( +7,-44){\cent{$q$}}

\put(+26,+15){\cent[tr]{$1$}}
\put(+15,+30){\cent[tr]{$q$}}
\put( -6,+35){\cent[t]{$1$}}
\put(-27,+27){\cent[t]{$q$}}
\put(-41,+13){\cent[l]{$1$}}
\put(-41,-13){\cent[l]{$1$}}
\put(-27,-27){\cent[b]{$q$}}
\put( -6,-35){\cent[b]{$1$}}
\put(+15,-30){\cent[br]{$q$}}
\put(+26,-15){\cent[br]{$1$}}
\put(+70, +8){\cent[b]{$q-1$}}

\put(-50,+30){\cent[r]{\small N}}
\put(+12,+56){\cent[b]{\small O$_i$}}
\put(+44,-14){\cent[t]{\small P$_{ij}$}}
\put(+12,-56){\cent[t]{\small Q$_j$}}
\put(-50,-30){\cent[r]{\small R}}

\end{picture}
\end{center}

Now let N and R be any two adjacent nodes. N has $q$ other neighbours O$_i$
and R has $q$ other neighbours Q$_j$. Call the joint neighbour of O$_i$ and
Q$_j$ now P$_{ij}$, these $q^2$ nodes are all distinct (4-cycles of type
NOPO and/or RQPQ otherwise) and none of them coincide with N, R, the Os or Qs
(3- or 4-cycles otherwise). On the other hand, there are no further nodes
(distance $\gt2$ from N or R otherwise). Tally: $q^2+2q+2 = (q+1)^2+1$,
for valency $q+1$.~\qed
%%%%%
%%%%%
\end{document}
