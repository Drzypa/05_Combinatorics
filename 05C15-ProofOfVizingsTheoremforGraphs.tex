\documentclass[12pt]{article}
\usepackage{pmmeta}
\pmcanonicalname{ProofOfVizingsTheoremforGraphs}
\pmcreated{2013-03-22 15:10:42}
\pmmodified{2013-03-22 15:10:42}
\pmowner{marijke}{8873}
\pmmodifier{marijke}{8873}
\pmtitle{proof of Vizing's theorem (for graphs)}
\pmrecord{6}{36932}
\pmprivacy{1}
\pmauthor{marijke}{8873}
\pmtype{Proof}
\pmcomment{trigger rebuild}
\pmclassification{msc}{05C15}
\pmrelated{KempeChain}

\endmetadata

\usepackage{amssymb}
% \usepackage{amsmath}
% \usepackage{amsfonts}

% used for TeXing text within eps files
%\usepackage{psfrag}
% need this for including graphics (\includegraphics)
%\usepackage{graphicx}

% for neatly defining theorems and propositions
%\usepackage{amsthm}
% making logically defined graphics
%%%\usepackage{xypic}

% there are many more packages, add them here as you need them

% define commands here %%%%%%%%%%%%%%%%%%%%%%%%%%%%%%%%%%
% portions from
% makra.sty 1989-2005 by Marijke van Gans %
                                          %          ^ ^
\catcode`\@=11                            %          o o
                                          %         ->*<-
                                          %           ~
%%%% CHARS %%%%%%%%%%%%%%%%%%%%%%%%%%%%%%%%%%%%%%%%%%%%%%

                        %    code    char  frees  for

\let\Para\S             %    \Para     §   \S \scriptstyle
\let\Pilcrow\P          %    \Pilcrow  ¶   \P
\mathchardef\pilcrow="227B

\mathchardef\lt="313C   %    \lt       <   <     bra
\mathchardef\gt="313E   %    \gt       >   >     ket

\let\bs\backslash       %    \bs       \
\let\us\_               %    \us       _     \_  ...

\mathchardef\lt="313C   %    \lt       <   <     bra
\mathchardef\gt="313E   %    \gt       >   >     ket

%%%% DIACRITICS %%%%%%%%%%%%%%%%%%%%%%%%%%%%%%%%%%%%%%%%%

%let\udot\d             % under-dot (text mode), frees \d
\let\odot\.             % over-dot (text mode),  frees \.
%let\hacek\v            % hacek (text mode),     frees \v
%let\makron\=           % makron (text mode),    frees \=
%let\tilda\~            % tilde (text mode),     frees \~
\let\uml\"              % umlaut (text mode),    frees \"

%def\ij/{i{\kern-.07em}j}
%def\trema#1{\discretionary{-}{#1}{\uml #1}}

%%%% amssymb %%%%%%%%%%%%%%%%%%%%%%%%%%%%%%%%%%%%%%%%%%%%

\let\le\leqslant
\let\ge\geqslant
%let\prece\preceqslant
%let\succe\succeqslant

%%%% USEFUL MISC %%%%%%%%%%%%%%%%%%%%%%%%%%%%%%%%%%%%%%%%

%def\C++{C$^{_{++}}$}

%let\writelog\wlog
%def\wl@g/{{\sc wlog}}
%def\wlog{\@ifnextchar/{\wl@g}{\writelog}}

%def\org#1{\lower1.2pt\hbox{#1}} 
% chem struct formulae: \bs, --- /  \org{C} etc. 

%%%% USEFUL INTERNAL LaTeX STUFF %%%%%%%%%%%%%%%%%%%%%%%%

%let\Ifnextchar=\@ifnextchar
%let\Ifstar=\@ifstar
%def\currsize{\@currsize}

%%%% KERNING, SPACING, BREAKING %%%%%%%%%%%%%%%%%%%%%%%%%

%def\qqquad{\hskip3em\relax}
%def\qqqquad{\hskip4em\relax}
%def\qqqqquad{\hskip5em\relax}
%def\qqqqqquad{\hskip6em\relax}
%def\qqqqqqquad{\hskip7em\relax}
%def\qqqqqqqquad{\hskip8em\relax}

%%%% LAYOUT %%%%%%%%%%%%%%%%%%%%%%%%%%%%%%%%%%%%%%%%%%%%%

%%%% COUNTERS %%%%%%%%%%%%%%%%%%%%%%%%%%%%%%%%%%%%%%%%%%%

%let\addtoreset\@addtoreset
%{A}{B} adds A to list of counters reset to 0
% when B is \refstepcounter'ed (see latex.tex)
%
%def\numbernext#1#2{\setcounter{#1}{#2}\addtocounter{#1}{\m@ne}}

%%%% EQUATIONS %%%%%%%%%%%%%%%%%%%%%%%%%%%%%%%%%%%%%%%%%%

%%%% LEMMATA %%%%%%%%%%%%%%%%%%%%%%%%%%%%%%%%%%%%%%%%%%%%

%%%% DISPLAY %%%%%%%%%%%%%%%%%%%%%%%%%%%%%%%%%%%%%%%%%%%%

%%%% MATH LAYOUT %%%%%%%%%%%%%%%%%%%%%%%%%%%%%%%%%%%%%%%%

\let\D\displaystyle
\let\T\textstyle
\let\S\scriptstyle
\let\SS\scriptscriptstyle

% array:
%def\<#1:{\begin{array}{#1}}
%def\>{\end{array}}

% array using [ ] with rounded corners:
%def\[#1:{\left\lgroup\begin{array}{#1}} 
%def\]{\end{array}\right\rgroup}

% array using ( ):
%def\(#1:{\left(\begin{array}{#1}}
%def\){\end{array}\right)}

%def\hh{\noalign{\vskip\doublerulesep}}

%%%% MATH SYMBOLS %%%%%%%%%%%%%%%%%%%%%%%%%%%%%%%%%%%%%%%

%def\d{\mathord{\rm d}}                      % d as in dx
%def\e{{\rm e}}                              % e as in e^x

%def\Ell{\hbox{\it\char`\$}}

\def\sfmath#1{{\mathchoice%
{{\sf #1}}{{\sf #1}}{{\S\sf #1}}{{\SS\sf #1}}}}
\def\Stalkset#1{\sfmath{I\kern-.12em#1}}
\def\Bset{\Stalkset B}
\def\Nset{\Stalkset N}
\def\Rset{\Stalkset R}
\def\Hset{\Stalkset H}
\def\Fset{\Stalkset F}
\def\kset{\Stalkset k}
\def\In@set{\raise.14ex\hbox{\i}\kern-.237em\raise.43ex\hbox{\i}}
\def\Roundset#1{\sfmath{\kern.14em\In@set\kern-.4em#1}}
\def\Qset{\Roundset Q}
\def\Cset{\Roundset C}
\def\Oset{\Roundset O}
\def\Zset{\sfmath{Z\kern-.44emZ}}

% \frac overwrites LaTeX's one (use TeX \over instead)
%def\fraq#1#2{{}^{#1}\!/\!{}_{\,#2}}
\def\frac#1#2{\mathord{\mathchoice%
{\T{#1\over#2}}
{\T{#1\over#2}}
{\S{#1\over#2}}
{\SS{#1\over#2}}}}
%def\half{\frac12}

\mathcode`\<="4268         % < now is \langle, \lt is <
\mathcode`\>="5269         % > now is \rangle, \gt is >

%def\biggg#1{{\hbox{$\left#1\vbox %to20.5\p@{}\right.\n@space$}}}
%def\Biggg#1{{\hbox{$\left#1\vbox %to23.5\p@{}\right.\n@space$}}}

\let\epsi=\varepsilon
\def\omikron{o}

\def\Alpha{{\rm A}}
\def\Beta{{\rm B}}
\def\Epsilon{{\rm E}}
\def\Zeta{{\rm Z}}
\def\Eta{{\rm H}}
\def\Iota{{\rm I}}
\def\Kappa{{\rm K}}
\def\Mu{{\rm M}}
\def\Nu{{\rm N}}
\def\Omikron{{\rm O}}
\def\Rho{{\rm P}}
\def\Tau{{\rm T}}
\def\Ypsilon{{\rm Y}} % differs from \Upsilon
\def\Chi{{\rm X}}

%def\dg{^{\circ}}                   % degrees

%def\1{^{-1}}                       % inverse

\def\*#1{{\bf #1}}                  % boldface e.g. vector
%def\vi{\mathord{\hbox{\bf\i}}}     % boldface vector \i
%def\vj{\mathord{\,\hbox{\bf\j}}}   % boldface vector \j

%def\union{\mathbin\cup}
%def\isect{\mathbin\cap}

%let\so\Longrightarrow
%let\oso\Longleftrightarrow
%let\os\Longleftarrow

% := and :<=>
%def\isdef{\mathrel{\smash{\stackrel{\SS\rm def}{=}}}}
%def\iffdef{\mathrel{\smash{stackrel{\SS\rm def}{\oso}}}}

\def\isdef{\mathrel{\mathop{=}\limits^{\smash{\hbox{\tiny def}}}}}
%def\iffdef{\mathrel{\mathop{\oso}\limits^{\smash{\hbox{\tiny %def}}}}}

%def\tr{\mathop{\rm tr}}            % tr[ace]
%def\ter#1{\mathop{^#1\rm ter}}     % k-ter[minant]

%let\.=\cdot
%let\x=\times                % ח (direct product)

%def\qed{ ${\S\circ}\!{}^\circ\!{\S\circ}$}
%def\qed{\vrule height 6pt width 6pt depth 0pt}

%def\edots{\mathinner{\mkern1mu
%   \raise7pt\vbox{\kern7pt\hbox{.}}\mkern1mu   %  .shorter
%   \raise4pt\hbox{.}\mkern1mu                  %     .
%   \raise1pt\hbox{.}\mkern1mu}}                %        .
%def\fdots{\mathinner{\mkern1mu
%   \raise7pt\vbox{\kern7pt\hbox{.}}            %   . ~45°
%   \raise4pt\hbox{.}                           %     .
%   \raise1pt\hbox{.}\mkern1mu}}                %       .

\def\mod#1{\allowbreak \mkern 10mu({\rm mod}\,\,#1)}
% redefines TeX's one using less space

%def\int{\intop\displaylimits}
%def\oint{\ointop\displaylimits}

%def\intoi{\int_0^1}
%def\intall{\int_{-\infty}^\infty}

%def\su#1{\mathop{\sum\raise0.7pt\hbox{$\S\!\!\!\!\!#1\,$}}}

%let\frakR\Re
%let\frakI\Im
%def\Re{\mathop{\rm Re}\nolimits}
%def\Im{\mathop{\rm Im}\nolimits}
%def\conj#1{\overline{#1\vphantom1}}
%def\cj#1{\overline{#1\vphantom+}}

%def\forAll{\mathop\forall\limits}
%def\Exists{\mathop\exists\limits}

%%%% PICTURES %%%%%%%%%%%%%%%%%%%%%%%%%%%%%%%%%%%%%%%%%%%

\def\cent{\makebox(0,0)}

\def\node{\circle*4}
%def\nOde{\circle4}

%%%% REFERENCES %%%%%%%%%%%%%%%%%%%%%%%%%%%%%%%%%%%%%%%%%

%def\opcit{[{\it op.\,cit.}]}
\def\bitem#1{\bibitem[#1]{#1}}
\def\name#1{{\sc #1}}
\def\book#1{{\sl #1\/}}
\def\paper#1{``#1''}
\def\mag#1{{\it #1\/}}
\def\vol#1{{\bf #1}}
\def\isbn#1{{\small\tt ISBN\,\,#1}}
\def\seq#1{{\small\tt #1}}
%def\url<{\verb>}
%def\@cite#1#2{[{#1\if@tempswa\ #2\fi}]}

%%%% VERBATIM CODE %%%%%%%%%%%%%%%%%%%%%%%%%%%%%%%%%%%%%%

%def\"{\verb"}

%%%% AD HOC %%%%%%%%%%%%%%%%%%%%%%%%%%%%%%%%%%%%%%%%%%%%%

\def\0#1{\hbox{\sc #1}}    % typeset vertex names

%%%% WORDS %%%%%%%%%%%%%%%%%%%%%%%%%%%%%%%%%%%%%%%%%%%%%%

% \hyphenation{pre-sent pre-sents pre-sent-ed pre-sent-ing
% re-pre-sent re-pre-sents re-pre-sent-ed re-pre-sent-ing
% re-fer-ence re-fer-ences re-fer-enced re-fer-encing
% ge-o-met-ry re-la-ti-vi-ty Gauss-ian Gauss-ians
% Des-ar-gues-ian}

%def\oord/{o{\trema o}rdin\-ate}
% usage: C\oord/s, c\oord/.
% output: co\"ord... except when linebreak at co-ord...

%%%%%%%%%%%%%%%%%%%%%%%%%%%%%%%%%%%%%%%%%%%%%%%%%%%%%%%%%

                                          %
                                          %          ^ ^
\catcode`\@=12                            %          ` '
                                          %         ->*<-
                                          %           ~
\begin{document}
\PMlinkescapeword{length}
\PMlinkescapeword{node}
\PMlinkescapeword{terms}
\PMlinkescapeword{time}
\PMlinkescapeword{line}
\PMlinkescapeword{circle}


\subsection*{Proof of Vizing's theorem (for graphs)}

We must prove, for any integer $\rho$, the following: if $G$ is a graph all of
whose vertices have valency $\le \rho$, then its edges can be colored (with
adjacent edges receiving different colors) in no more than $\rho+1$ colors.

This form of stating the theorem allows us to do induction on the
number of edges. A graph with zero edges, for instance, certainly doesn't
need more than $\rho+1$ colors. Now assume (for any given $\rho$) that
all graphs with $m$ edges can be thus colored. For any graph $G$ with $m+1$
edges we choose an edge to remove, color the remaining $m$ edges, and
try putting the edge back.

Another way of looking at this: suppose there was (for a certain $\rho$) a
graph $G^*$ with $m^*$ edges that could not be colored thus. There would then
be a largest number $m$, bounded by $0\le m\lt m^*$, such that all graphs with
$m$ edges can still be colored thus. Then there would also be a $G$ with $m+1$
edges that cannot. Choose one of its edges and proceed as above. This time,
succeeding in coloring it after all would prove by contradiction that there
was no such $m$ and hence no such $G^*$.

Either way we have our $G$ where after picking an edge we can color everything
except that edge, and fitting it in too proves the theorem. Let the edge to
fit run between $\0v$ and $\0w_1$. Some terms:
%
\begin{itemize}

\item With a palette of $\rho+1$ edge colors, and no more than $\rho$ edges
      at any vertex, each vertex has at least one {\bf missing color}.

\item Let a {\bf Kempe chain} $H(a,b)$ be any connected component of the
      subgraph formed by all edges colored $a$ or $b$. This is either a
      cycle of even length, or an open path terminating at two vertices
      where one of $a$ or $b$ is missing.

\end{itemize}
%
{\bf Step 1}. If there is a color missing both at $\0v$ and at $\0w_1$, use
that color for the edge $\0v\0w_1$ and we're finished.

Else, let color $a$ be missing at $\0v$ (but present at $\0w_1$), and color
$b_1$ be missing at $\0w_1$ (but present at $\0v$). If the Kempe chain of
colors $a$ and $b_1$ that terminates at $\0v$ is disjoint from the one
terminating at $\0w_1$, swap the colors in the latter chain, then use
$a$ for edge $\0v\0w_1$ and be done.

Else it's the same $H(a,b_1)$ chain, in which case we proceed as follows.

\begin{center}
\begin{picture}(300,90)(-150,0)

\put(  0, 0){\node}
\put(-90,30){\node}
\put(-30,30){\node}
\put(-90,60){\node}
\put(-30,60){\node}

\put(  0, 0){\line(-1,+1){30}}

\put(-80,60){\oval(20,20)[tl]}
\put(-40,60){\oval(20,20)[tr]}
\put(-80,70){\line(1,0){40}}
\put(  5, 0){\cent[l]{$\0v$}}
\put(-85,30){\cent[l]{$\0w_1$}}
\put(-60,75){\cent[b]{$H(a,b_1)$}}
\linethickness{2pt}
\put(-90,30){\line(0,1){30}}
\put(-30,30){\line(0,1){30}}
\put(-80,70){\line(1,0){10}}
\put(-63,70){\line(1,0){6}}
\put(-50,70){\line(1,0){10}}

\end{picture}
\end{center}

Find the $b_1$-colored edge from $\0v$, call it
$\0v\0w_2$. Steal color $b_1$ from it and give it to edge $\0v\0w_1$, now
$\0v\0w_2$ is the problem edge that needs a color.

\begin{center}
\begin{picture}(300,90)(-150,0)

\put(  0, 0){\node}
\put(-90,30){\node}
\put(-30,30){\node}
\put(-90,60){\node}
\put(-30,60){\node}

\put(  0, 0){\line(-3,+1){90}}

\put(-80,60){\oval(20,20)[tl]}
\put(-40,60){\oval(20,20)[tr]}
\put(-80,70){\line(1,0){40}}
\put(  5, 0){\cent[l]{$\0v$}}
\put(-85,30){\cent[l]{$\0w_1$}}
\put(-25,30){\cent[l]{$\0w_2$}}
\put(-60,75){\cent[b]{$H(a,b_1)$}}
\linethickness{2pt}
\put(-90,30){\line(0,1){30}}
\put(-30,30){\line(0,1){30}}
\put(-80,70){\line(1,0){10}}
\put(-63,70){\line(1,0){6}}
\put(-50,70){\line(1,0){10}}

\end{picture}
\end{center}


Go back to Step~1 calling it {\bf Step~2}, replacing subscripts $_1$ by $_2$
in all text. This involves a color $b_2$ that was missing at $\0w_2$ ($a$
remains the missing color at $\0v$). If $\0v$ and $\0w_2$ are awkward too,
being in the same $H(a,b_2)$ chain

\begin{center}
\begin{picture}(300,90)(-150,0)

\put(  0, 0){\node}
\put(-90,30){\node}
\put(-30,30){\node}
\put(+30,30){\node}
\put(-90,60){\node}
\put(-30,60){\node}
\put(+30,60){\node}

\put(  0, 0){\line(-3,+1){90}}
\put( 0,1.5){\line(+1,+1){30}}
\put(1.5, 0){\line(+1,+1){30}}

\put(-80,60){\oval(20,20)[tl]}
\put(-40,60){\oval(20,20)[tr]}
\put(-21,61){\oval(20,20)[tl]}
\put(+21,61){\oval(20,20)[tr]}
\put(-21,61){\oval(16,16)[tl]}
\put(+21,61){\oval(16,16)[tr]}
\put(-80,70){\line(1,0){40}}
\put(-21,71){\line(1,0){42}}
\put(-21,69){\line(1,0){42}}
\put(  5,-3){\cent[l]{$\0v$}}
\put(-85,30){\cent[l]{$\0w_1$}}
\put(-25,30){\cent[l]{$\0w_2$}}
\put(-60,75){\cent[b]{$H(a,b_1)$}}
\put(  0,75){\cent[b]{$H(a,b_2)$}}
\linethickness{2pt}
\put(-90,30){\line(0,1){30}}
\put(-30,30){\line(0,1){30}}
\put(+30,30){\line(0,1){30}}
\put(-80,70){\line(1,0){10}}
\put(-63,70){\line(1,0){6}}
\put(-50,70){\line(1,0){10}}
\put(-20,70){\line(1,0){10}}
\put( -3,70){\line(1,0){6}}
\put(+10,70){\line(1,0){10}}

\end{picture}
\end{center}

find the $b_2$-colored edge $\0v\0w_3$, give its color to $\0v\0w_2$
making $\0v\0w_3$ the problem:

\begin{center}
\begin{picture}(300,90)(-150,0)

\put(  0, 0){\node}
\put(-90,30){\node}
\put(-30,30){\node}
\put(+30,30){\node}
\put(-90,60){\node}
\put(-30,60){\node}
\put(+30,60){\node}

\put(  0, 0){\line(-3,+1){90}}
\put( 0,1.5){\line(-1,+1){30}}
\put(-1.5,0){\line(-1,+1){30}}

\put(-80,60){\oval(20,20)[tl]}
\put(-40,60){\oval(20,20)[tr]}
\put(-21,61){\oval(20,20)[tl]}
\put(+21,61){\oval(20,20)[tr]}
\put(-21,61){\oval(16,16)[tl]}
\put(+21,61){\oval(16,16)[tr]}
\put(-80,70){\line(1,0){40}}
\put(-21,71){\line(1,0){42}}
\put(-21,69){\line(1,0){42}}
\put(  5, 0){\cent[l]{$\0v$}}
\put(-85,30){\cent[l]{$\0w_1$}}
\put(-25,30){\cent[l]{$\0w_2$}}
\put(+35,30){\cent[l]{$\0w_3$}}
\put(-60,75){\cent[b]{$H(a,b_1)$}}
\put(  0,75){\cent[b]{$H(a,b_2)$}}
\linethickness{2pt}
\put(-90,30){\line(0,1){30}}
\put(-30,30){\line(0,1){30}}
\put(+30,30){\line(0,1){30}}
\put(-80,70){\line(1,0){10}}
\put(-63,70){\line(1,0){6}}
\put(-50,70){\line(1,0){10}}
\put(-20,70){\line(1,0){10}}
\put( -3,70){\line(1,0){6}}
\put(+10,70){\line(1,0){10}}

\end{picture}
\end{center}

Next time round ({\bf Step~3}) increase the subscripts again.
And if still necessary, transfer the color again.

\begin{center}
\begin{picture}(300,90)(-150,0)

\put(  0, 0){\node}
\put(-90,30){\node}
\put(-30,30){\node}
\put(+30,30){\node}
\put(+90,30){\node}
\put(-90,60){\node}
\put(-30,60){\node}
\put(+30,60){\node}
\put(+90,60){\node}

\put(  0, 0){\line(-3,+1){90}}
\put( 0,1.5){\line(-1,+1){30}}
\put(-1.5,0){\line(-1,+1){30}}
\put( 3, 3){\cent.}
\put( 5, 5){\cent.}
\put( 7, 7){\cent.}
\put( 9, 9){\cent.}
\put(11,11){\cent.}
\put(13,13){\cent.}
\put(15,15){\cent.}
\put(17,17){\cent.}
\put(19,19){\cent.}
\put(21,21){\cent.}
\put(23,23){\cent.}
\put(25,25){\cent.}
\put(27,27){\cent.}

\put(-80,60){\oval(20,20)[tl]}
\put(-40,60){\oval(20,20)[tr]}
\put(-21,61){\oval(20,20)[tl]}
\put(+21,61){\oval(20,20)[tr]}
\put(-21,61){\oval(16,16)[tl]}
\put(+21,61){\oval(16,16)[tr]}
\put(-80,70){\line(1,0){40}}
\put(-21,71){\line(1,0){42}}
\put(-21,69){\line(1,0){42}}
\put(+32,66){\cent.}
\put(+34,68){\cent.}
\put(+36,69){\cent.}
\put(+38,70){\cent.}
\put(+52,70){\cent.}
\put(+55,70){\cent.}
\put(+65,70){\cent.}
\put(+68,70){\cent.}
\put(+82,70){\cent.}
\put(+84,69){\cent.}
\put(+86,68){\cent.}
\put(+88,66){\cent.}
\put(+89,64){\cent.}
\put(  5,-2){\cent[l]{$\0v$}}
\put(-85,30){\cent[l]{$\0w_1$}}
\put(-25,30){\cent[l]{$\0w_2$}}
\put(+35,30){\cent[l]{$\0w_3$}}
\put(+95,30){\cent[l]{$\0w_4$}}
\put(-60,75){\cent[b]{$H(a,b_1)$}}
\put(  0,75){\cent[b]{$H(a,b_2)$}}
\put(+60,75){\cent[b]{$H(a,b_3)$}}
\linethickness{2pt}
\put(-90,30){\line(0,1){30}}
\put(-30,30){\line(0,1){30}}
\put(+30,30){\line(0,1){30}}
\put(+90,30){\line(0,1){30}}
\put(-80,70){\line(1,0){10}}
\put(-63,70){\line(1,0){6}}
\put(-50,70){\line(1,0){10}}
\put(-20,70){\line(1,0){10}}
\put( -3,70){\line(1,0){6}}
\put(+10,70){\line(1,0){10}}
\put(+40,70){\line(1,0){10}}
\put(+57,70){\line(1,0){6}}
\put(+70,70){\line(1,0){10}}

\end{picture}
\end{center}

Sooner or later (if we don't get to bail out at one of the Steps) we will run
out of colors, in the following sense: the color missing at some $\0w_{j+1}$
is not some new color $b_{j+1}$ but a color $b_i$ we've seen before (it can't
be $a$, that's present at $\0w_{j+1}$). And not only is $i\lt j+1$ but also
$i\lt j$ because $b_j$ is the color we just stole from $\0w_{j+1}$.

A chain $H(a,b_i)$ runs from $\0v$ to $\0w_{i+1}$ and all its vertices
except $\0w_{i+1}$ have color $b_i$. Now $\0w_{j+1}$ isn't $\0w_{i+1}$
(because $i\lt j$) nor one of the other vertices of the chain (because
it doesn't have color $b_i$). So $\0w_{j+1}$ isn't on this $H(a,b_i)$.

\begin{center}
\begin{picture}(300,90)(-150,0)

\put(  0, 0){\node}
\put(-30,30){\node}
\put(+30,30){\node}
\put(+90,30){\node}
\put(-30,60){\node}
\put(+30,60){\node}

\put( 0,1.5){\line(-1,+1){30}}
\put(-1.5,0){\line(-1,+1){30}}

\put(-21,61){\oval(20,20)[tl]}
\put(+21,61){\oval(20,20)[tr]}
\put(-21,61){\oval(16,16)[tl]}
\put(+21,61){\oval(16,16)[tr]}
\put(-21,71){\line(1,0){42}}
\put(-21,69){\line(1,0){42}}
\put(  5, 0){\cent[l]{$\0v$}}
\put(-25,30){\cent[l]{$\0w_i$}}
\put(+35,30){\cent[l]{$\0w_{i+1}$}}
\put(+95,30){\cent[l]{$\0w_{j+1}$}}
\put(  0,75){\cent[b]{$H(a,b_i)$}}
\linethickness{2pt}
\put(-30,30){\line(0,1){30}}
\put(+30,30){\line(0,1){30}}
\put(-20,70){\line(1,0){10}}
\put( -3,70){\line(1,0){6}}
\put(+10,70){\line(1,0){10}}

\end{picture}
\end{center}

$\0w_{k+1}$ does have color $a$ so it's on some other $H'(a,b_i)$ disjoint
from $H(a,b_i)$. This could be as short as a single $a$-colored edge or
longer, it doesn't matter.

\begin{center}
\begin{picture}(300,90)(-150,0)

\put(  0, 0){\node}
\put(-30,30){\node}
\put(+30,30){\node}
\put(+90,30){\node}
\put(-30,60){\node}
\put(+30,60){\node}
\put(+90,60){\node}

\put( 0,1.5){\line(-1,+1){30}}
\put(-1.5,0){\line(-1,+1){30}}

\put(-21,61){\oval(20,20)[tl]}
\put(+21,61){\oval(20,20)[tr]}
\put(-21,61){\oval(16,16)[tl]}
\put(+21,61){\oval(16,16)[tr]}
\put(-21,71){\line(1,0){42}}
\put(-21,69){\line(1,0){42}}
\put(+89,60){\line(0,1){20}}
\put(+91,60){\line(0,1){20}}
\put(+90,79){\line(1,0){20}}
\put(+90,81){\line(1,0){20}}
\put(  5, 0){\cent[l]{$\0v$}}
\put(-25,30){\cent[l]{$\0w_i$}}
\put(+35,30){\cent[l]{$\0w_{i+1}$}}
\put(+95,30){\cent[l]{$\0w_{j+1}$}}
\put(  0,75){\cent[b]{$H(a,b_i)$}}
\put(120,65){\cent[b]{$H'(a,b_i)$}}
\linethickness{2pt}
\put(-30,30){\line(0,1){30}}
\put(+30,30){\line(0,1){30}}
\put(+90,30){\line(0,1){30}}
\put(-20,70){\line(1,0){10}}
\put( -3,70){\line(1,0){6}}
\put(+10,70){\line(1,0){10}}
\put(+90,80){\line(1,0){10}}
\put(105,80){\line(1,0){3}}

\end{picture}
\end{center}

Swap the colors in that $H'(a,b_i)$. Now use $a$ to finally color
$\0v\0w_{j+1}$ (merging $H(a,b_i)$ and $H'(a,b_i)$ in the process).

\begin{center}
\begin{picture}(300,90)(-150,0)

\put(  0, 0){\node}
\put(-30,30){\node}
\put(+30,30){\node}
\put(+90,30){\node}
\put(-30,60){\node}
\put(+30,60){\node}
\put(+90,60){\node}

\put( 0,1.5){\line(-1,+1){30}}
\put(-1.5,0){\line(-1,+1){30}}
\put(-1.2,0.4){\line(+3,+1){90}}
\put(-0.9,0.3){\line(+3,+1){90}}
\put(-0.6,0.2){\line(+3,+1){90}}
\put(-0.3,0.1){\line(+3,+1){90}}
\put(  0, 0){\line(+3,+1){90}}
\put(0.3,-0.1){\line(+3,+1){90}}
\put(0.6,-0.2){\line(+3,+1){90}}
\put(0.9,-0.3){\line(+3,+1){90}}
\put(1.2,-0.4){\line(+3,+1){90}}

\put(-21,61){\oval(20,20)[tl]}
\put(+21,61){\oval(20,20)[tr]}
\put(-21,61){\oval(16,16)[tl]}
\put(+21,61){\oval(16,16)[tr]}
\put(-21,71){\line(1,0){42}}
\put(-21,69){\line(1,0){42}}
\put(+89,30){\line(0,1){30}}
\put(+91,30){\line(0,1){30}}
\put(+90,79){\line(1,0){20}}
\put(+90,81){\line(1,0){20}}
\put(  5,-5){\cent[l]{$\0v$}}
\put(-25,30){\cent[l]{$\0w_i$}}
\put(+35,30){\cent[l]{$\0w_{i+1}$}}
\put(+95,30){\cent[l]{$\0w_{j+1}$}}
\put(  0,75){\cent[b]{$H(a,b_i)$}}
\put(120,65){\cent[b]{$H'(a,b_i)$}}
\linethickness{2pt}
\put(-30,30){\line(0,1){30}}
\put(+30,30){\line(0,1){30}}
\put(+90,60){\line(0,1){20}}
\put(-20,70){\line(1,0){10}}
\put( -3,70){\line(1,0){6}}
\put(+10,70){\line(1,0){10}}
\put(100,80){\line(1,0){5}}
\put(108,80){\line(1,0){2}}

\end{picture}
\end{center}

This is in essence the proof usually given, as taught to students and found for instance in references [FW77] and [Wil02] of the
\PMlinkid{parent entry}{6930}.
%%%%%
%%%%%
\end{document}
