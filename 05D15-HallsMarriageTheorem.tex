\documentclass[12pt]{article}
\usepackage{pmmeta}
\pmcanonicalname{HallsMarriageTheorem}
\pmcreated{2013-03-22 12:35:14}
\pmmodified{2013-03-22 12:35:14}
\pmowner{mathcam}{2727}
\pmmodifier{mathcam}{2727}
\pmtitle{Hall's marriage theorem}
\pmrecord{6}{32837}
\pmprivacy{1}
\pmauthor{mathcam}{2727}
\pmtype{Theorem}
\pmcomment{trigger rebuild}
\pmclassification{msc}{05D15}
\pmsynonym{marriage theorem}{HallsMarriageTheorem}
\pmrelated{Saturate}

% this is the default PlanetMath preamble.  as your knowledge
% of TeX increases, you will probably want to edit this, but
% it should be fine as is for beginners.

% almost certainly you want these
\usepackage{amssymb}
\usepackage{amsmath}
\usepackage{amsfonts}

% used for TeXing text within eps files
%\usepackage{psfrag}
% need this for including graphics (\includegraphics)
%\usepackage{graphicx}
% for neatly defining theorems and propositions
%\usepackage{amsthm}
% making logically defined graphics
%%%\usepackage{xypic} 

% there are many more packages, add them here as you need them

% define commands here
\begin{document}
Let $S = \{ S_1,S_2,\dots S_n \}$ be a finite collection of finite sets.  There exists a system of distinct representatives of $S$ if and only if the following condition holds for any $T \subseteq S$:
$$\left | \cup T \right | \geq |T|$$

As a corollary, if this condition fails to hold anywhere, then no SDR exists.

This is known as Hall's marriage theorem.  The name arises from a particular application of this theorem.  Suppose we have a finite set of single men/women, and, for each man/woman, a finite collection of women/men to whom this person is attracted.  An SDR for this collection would be a way each man/woman could be (theoretically) married happily.  Hence, Hall's marriage theorem can be used to determine if this is possible.

An application of this theorem to graph theory gives that if $G(V_1, V_2, E)$ is a bipartite graph, then $G$ has a complete matching that saturates every vertex of $V_1$ if and only if $|S|\leq |N(S)|$ for every subset $S\subset V_1$.
%%%%%
%%%%%
\end{document}
