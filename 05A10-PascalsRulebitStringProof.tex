\documentclass[12pt]{article}
\usepackage{pmmeta}
\pmcanonicalname{PascalsRulebitStringProof}
\pmcreated{2013-03-22 12:23:03}
\pmmodified{2013-03-22 12:23:03}
\pmowner{vampyr}{22}
\pmmodifier{vampyr}{22}
\pmtitle{Pascal's rule (bit string proof)}
\pmrecord{5}{32166}
\pmprivacy{1}
\pmauthor{vampyr}{22}
\pmtype{Proof}
\pmcomment{trigger rebuild}
\pmclassification{msc}{05A10}

% this is the default PlanetMath preamble.  as your knowledge
% of TeX increases, you will probably want to edit this, but
% it should be fine as is for beginners.

% almost certainly you want these
\usepackage{amssymb}
\usepackage{amsmath}
\usepackage{amsfonts}

% used for TeXing text within eps files
%\usepackage{psfrag}
% need this for including graphics (\includegraphics)
%\usepackage{graphicx}
% for neatly defining theorems and propositions
%\usepackage{amsthm}
% making logically defined graphics
%%%\usepackage{xypic} 

% there are many more packages, add them here as you need them

% define commands here
\begin{document}
This proof is based on an alternate, but equivalent, definition of the binomial coefficient: $\binom{n}{r}$ is the number of bit strings (finite sequences of $0$s and $1$s) of length $n$ with exactly $r$ ones.

We want to show that
$$\binom{n}{r} = \binom{n-1}{r-1} + \binom{n-1}{r}$$
To do so, we will show that both sides of the equation are counting the same set of bit strings.

The left-hand side counts the set of strings of $n$ bits with $r$ $1$s.  Suppose we take one of these strings and remove the first bit $b$.  There are two cases: either $b = 1$, or $b = 0$.

If $b = 1$, then the new string is $n-1$ bits with $r-1$ ones; there are $\binom{n-1}{r-1}$ bit strings of this nature.

If $b = 0$, then the new string is $n-1$ bits with $r$ ones, and there are $\binom{n-1}{r}$ strings of this nature.

Therefore every string counted on the left is covered by one, but not both, of these two cases.  If we add the two cases, we find that
$$\binom{n}{r} = \binom{n-1}{r-1} + \binom{n-1}{r}$$
%%%%%
%%%%%
\end{document}
