\documentclass[12pt]{article}
\usepackage{pmmeta}
\pmcanonicalname{HeawoodNumber}
\pmcreated{2013-03-22 13:21:21}
\pmmodified{2013-03-22 13:21:21}
\pmowner{bbukh}{348}
\pmmodifier{bbukh}{348}
\pmtitle{Heawood number}
\pmrecord{12}{33876}
\pmprivacy{1}
\pmauthor{bbukh}{348}
\pmtype{Theorem}
\pmcomment{trigger rebuild}
\pmclassification{msc}{05C15}
\pmclassification{msc}{05C10}
\pmrelated{FourColorConjecture}

\usepackage{amssymb}
\usepackage{amsmath}
\usepackage{amsfonts}
%%\usepackage{xypic}

\makeatletter
\@ifundefined{bibname}{}{\renewcommand{\bibname}{References}}
\makeatother
\begin{document}
The Heawood number of a surface is an upper bound for the maximal
number of colors needed to color any graph embedded in the
surface. In 1890 Heawood proved for all surfaces except the sphere
that no more than
\begin{equation*}
H(S)=\left\lfloor\frac{7+\sqrt{49-24 e(S)}}{2}\right\rfloor
\end{equation*}
colors are needed to color any graph embedded in a surface of
Euler characteristic $e(S)$. The case of the sphere is the four-color conjecture which was settled by Appel and Haken in 1976. The number $H(S)$ became
known as Heawood number in 1976. Franklin proved that the chromatic number
of a graph embedded in the Klein bottle can be as large as $6$,
but never exceeds $6$. Later it was proved in the works of Ringel
and Youngs that the complete graph of $H(S)$ vertices can be
embedded in the surface $S$ unless $S$ is the Klein bottle. This
established that Heawood's bound could not be improved.

For example, the complete graph on $7$ vertices can be embedded in the torus as follows:
\begin{equation*}
\begin{xy}{*!C\xybox{\xymatrix@R=0pt{ 1\ar@{-}[r]\ar@{-}[dd]\ar@{-}[dddr]&2\ar@{-}[r]\ar@{-}[ddd]\ar@{-}[dddr]\ar@{-}[ddrr]&3\ar@{-}[r]\ar@{-}[ddr]&1\ar@{-}[dd]\\
\\ 4\ar@{-}[dd]\ar@{-}[dr] & & & 4\ar@{-}[dd]\\
 & 6\ar@{-}[r]\ar@{-}[dddr] & 7\ar@{-}[ddd]\ar@{-}[dddr]\ar@{-}[dr]\ar@{-}[ur] & \\
5\ar@{-}[dd]\ar@{-}[ur]\ar@{-}[ddr]\ar@{-}[ddrr] & & & 5\ar@{-}[dd] \\ \\
1\ar@{-}[r] & 2\ar@{-}[r] & 3\ar@{-}[r] & 1}}}
\end{xy}
\end{equation*}
% Whoever can fix the cropping in LaTeX2HTML for the above diagram, 
% contact me or akrowne

\begin{thebibliography}{1}

\bibitem{cite:bollobas_graphth_intro}
B{\'e}la Bollob{\'a}s.
\newblock {\em Graph Theory: An Introductory Course}, volume~63 of {\em GTM}.
\newblock Springer-Verlag, 1979.
\newblock \PMlinkexternal{Zbl 0411.05032}{http://www.emis.de/cgi-bin/zmen/ZMATH/en/quick.html?type=html&an=0411.05032}.

\bibitem{cite:saaty_kainen_fcc}
Thomas~L. Saaty and Paul~C. Kainen.
\newblock {\em The Four-Color Problem: Assaults and Conquest}.
\newblock Dover, 1986.
\newblock \PMlinkexternal{Zbl 0463.05041}{http://www.emis.de/cgi-bin/zmen/ZMATH/en/quick.html?type=html&an=0463.05041}.

\end{thebibliography}
%@BOOK{cite:saaty_kainen_fcc,
% author = {Thomas L. Saaty and Paul C. Kainen},
% title = {The Four-Color Problem: Assaults and Conquest},
% publisher = {Dover},
% year = 1986,
% note      = {\PMlinkexternal{Zbl %0463.05041}{http://www.emis.de/cgi-bin/zmen/ZMATH/en/quick.html?type=html&an=0463.05041}}
%}
%
%@BOOK{cite:bollobas_graphth_intro,
% author    = {B{\'e}la Bollob{\'a}s},
% title     = {Graph Theory: An Introductory Course},
% year      = 1979,
% publisher = {Springer-Verlag},
% volume    = 63,
% series    = {GTM},
% note      = {\PMlinkexternal{Zbl %0411.05032}{http://www.emis.de/cgi-bin/zmen/ZMATH/en/quick.html?type=html&an=0411.05032}}
%}
%%%%%
%%%%%
\end{document}
