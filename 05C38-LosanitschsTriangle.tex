\documentclass[12pt]{article}
\usepackage{pmmeta}
\pmcanonicalname{LosanitschsTriangle}
\pmcreated{2013-03-22 15:44:09}
\pmmodified{2013-03-22 15:44:09}
\pmowner{CompositeFan}{12809}
\pmmodifier{CompositeFan}{12809}
\pmtitle{Losanitsch's triangle}
\pmrecord{13}{37686}
\pmprivacy{1}
\pmauthor{CompositeFan}{12809}
\pmtype{Definition}
\pmcomment{trigger rebuild}
\pmclassification{msc}{05C38}
\pmsynonym{Lozanic's triangle}{LosanitschsTriangle}
\pmsynonym{Lozani\'c's triangle}{LosanitschsTriangle}

% this is the default PlanetMath preamble.  as your knowledge
% of TeX increases, you will probably want to edit this, but
% it should be fine as is for beginners.

% almost certainly you want these
\usepackage{amssymb}
\usepackage{amsmath}
\usepackage{amsfonts}

% used for TeXing text within eps files
%\usepackage{psfrag}
% need this for including graphics (\includegraphics)
%\usepackage{graphicx}
% for neatly defining theorems and propositions
%\usepackage{amsthm}
% making logically defined graphics
%%%\usepackage{xypic}

% there are many more packages, add them here as you need them

% define commands here
\begin{document}
A triangular arrangement of numbers very similar to Pascal's triangle.

Begin as you would if you were constructing Pascal's triangle, with a 1 in the top row, and that row $k$ numbered 0, and the 1's position $n$ as 0.

\begin{eqnarray*}
\begin{array}{cccccccccccccccccc}
& & & & & & & & & 1 & & & & & & & &\\
& & & & & & & & 1 & & 1 & & & & & & &\\
& & & & & & & 1 & & x & & 1 & & & & & &\\
& & & & &\vdots & & & & \vdots & & & & \vdots& & & & \\
\end{array}
\end{eqnarray*}

Now, for the next value, add up the two values above, but then subtract 

$${{\frac{n}{2}-1} \choose {{k - 1} \over 2}}$$

From this \PMlinkescapetext{point} forward, do the same for every even-numbered position in an even-numbered row. Instead of calculating the binomial coefficient, it can be looked up in Pascal's triangle.

\begin{eqnarray*}
\begin{array}{cccccccccccccccccc}
& & & & & & & & & 1 & & & & & & & &\\
& & & & & & & & 1 & & 1 & & & & & & &\\
& & & & & & & 1 & & 1 & & 1 & & & & & &\\
& & & & & & 1 & & 2 & & 2 & & 1 & & & & &\\
& & & & & 1 & & 2 & & 4 & & 2 & & 1 & & & &\\
& & & & 1 & & 3 & &6 & &6 & & 3 & & 1 & & &\\
& & & 1 & & 3 & &9 & & 10& &9 & & 3 & & 1 & &\\
& & 1 & & 4 & &12 & &19 & &19 & &12 & & 4 & & 1 &\\
& & & & &\vdots & & & & \vdots & & & & \vdots& & & & \\
\end{array}
\end{eqnarray*}
This triangle was first studied by the Serbian chemist Sima Losanitsch, but has since been found to have applications in graph theory and combinatorics.

\begin{thebibliography}{6}
\bibitem{dh} S. M. Losanitsch, {\it Die Isomerie-Arten bei den Homologen der Paraffin-Reihe}, Chem. Ber. 30 (1897), 1917-1926.
\end{thebibliography}
%%%%%
%%%%%
\end{document}
