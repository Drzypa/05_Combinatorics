\documentclass[12pt]{article}
\usepackage{pmmeta}
\pmcanonicalname{CombinationsWithRepeatedElements}
\pmcreated{2013-03-22 17:43:17}
\pmmodified{2013-03-22 17:43:17}
\pmowner{kfgauss70}{18761}
\pmmodifier{kfgauss70}{18761}
\pmtitle{combinations with repeated elements}
\pmrecord{8}{40167}
\pmprivacy{1}
\pmauthor{kfgauss70}{18761}
\pmtype{Definition}
\pmcomment{trigger rebuild}
\pmclassification{msc}{05A10}
%\pmkeywords{Combinatorics}
\pmrelated{BinomialCoefficient}

\endmetadata

% this is the default PlanetMath preamble.  as your knowledge
% of TeX increases, you will probably want to edit this, but
% it should be fine as is for beginners.

% almost certainly you want these
\usepackage{amssymb}
\usepackage{amsmath}
\usepackage{amsfonts}

% used for TeXing text within eps files
%\usepackage{psfrag}
% need this for including graphics (\includegraphics)
%\usepackage{graphicx}
% for neatly defining theorems and propositions
\usepackage{amsthm}
% making logically defined graphics
%%%\usepackage{xypic}

% there are many more packages, add them here as you need them

% define commands here
\newtheorem{thm}{Theorem}
\newtheorem{lemma}{Lemma}
\theoremstyle{definition}
\newtheorem{defn}{Definition}
\theoremstyle{remark}
\newtheorem{note}{Note}

\begin{document}
\theoremstyle{definition}
\begin{defn}
A \emph{$k$-combination with repeated elements} chosen within the set $X=\{x_1,x_2,\ldots x_n\}$ is a multiset with cardinality $k$ having $X$ as the underlying set.
\end{defn}
\begin{note}
The definition is based on the multiset concept and therefore the order of the elements within the combination is irrelevant.
\end{note}
\begin{note}
The definition generalizes the concept of \emph{combination with distinct elements}.
\end{note}
\begin{lemma}
Given $n,k \in \{0,1,2,\ldots\},n \ge k$, the following formula holds:
$$
\binom{n+1}{k+1}=\sum_{i=k}^n\binom{i}{k}.
$$
\end{lemma}
\begin{proof}
The formula is easily demonstrated by repeated application of the Pascal's Rule for the binomial coefficient.
\end{proof}
\begin{thm}
The number $C'_{n,k}$ of the $k$-combinations with repeated elements is given by the formula:
$$
C'_{n,k}=\binom{n+k-1}{k}.
$$
\end{thm}
\begin{proof}
The proof is given by \PMlinkname{finite induction}{PrincipleOfFiniteInduction}.\\
The proof is trivial for $k=1$, since no repetitions can occur and the number of $1$-combinations is $n=\binom{n}{1}$.\\
Let's then prove the formula is true for $k+1$, assuming it holds for $k$. The $k+1$-combinations can be partitioned in $n$ subsets as follows:
\begin{itemize}
\item
combinations that include $x_1$ at least once;
\item
combinations that do not include $x_1$, but include $x_2$ at least once;
\item
combinations that do not include $x_1$ and $x_2$, but include $x_3$ at least once;
\item \ldots
\item
combinations that do not include $x_1$, $x_2$,... $x_{n-2}$ but include $x_{n-1}$ at least once;
\item
combinations that do not include $x_1$, $x_2$,... $x_{n-2}$, $x_{n-1}$ but include $x_n$ only.
\end{itemize}
The number of the subsets is:
$$C'_{n,k}+C'_{n-1,k}+C'_{n-2,k}+\ldots +C'_{2,k}+C'_{1,k}$$
which, by the inductive hypothesis and the lemma, equalizes:
$$
\binom{n+k-1}{k}+\binom{n+k-2}{k}+\binom{n+k-3}{k}+\ldots +\binom{k+1}{k}+\binom{k}{k}=\sum_{i=k}^{n+k-1}\binom{i}{k}=\binom{n+k}{k+1}.
$$
\end{proof}
%%%%%
%%%%%
\end{document}
