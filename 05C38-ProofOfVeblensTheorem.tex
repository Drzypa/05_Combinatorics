\documentclass[12pt]{article}
\usepackage{pmmeta}
\pmcanonicalname{ProofOfVeblensTheorem}
\pmcreated{2013-03-22 13:56:51}
\pmmodified{2013-03-22 13:56:51}
\pmowner{mathcam}{2727}
\pmmodifier{mathcam}{2727}
\pmtitle{proof of Veblen's theorem}
\pmrecord{4}{34711}
\pmprivacy{1}
\pmauthor{mathcam}{2727}
\pmtype{Proof}
\pmcomment{trigger rebuild}
\pmclassification{msc}{05C38}

\endmetadata

\usepackage{amssymb}
\usepackage{amsmath}
\usepackage{amsfonts}
\begin{document}
The proof is very easy by induction on the number of elements of
the set $E$ of edges.
If $E$ is empty, then all the vertices have degree zero, which is even.
Suppose $E$ is nonempty.
If the graph contains no cycle, then some vertex has degree $1$, which is odd.
Finally, if the graph does contain a cycle $C$, then every vertex has
the same degree mod $2$ with respect to $E-C$, as it has with respect
to $E$, and we can conclude by induction.
%%%%%
%%%%%
\end{document}
