\documentclass[12pt]{article}
\usepackage{pmmeta}
\pmcanonicalname{StirlingNumbersOfTheFirstKind}
\pmcreated{2013-03-22 12:33:44}
\pmmodified{2013-03-22 12:33:44}
\pmowner{rmilson}{146}
\pmmodifier{rmilson}{146}
\pmtitle{Stirling numbers of the first kind}
\pmrecord{6}{32809}
\pmprivacy{1}
\pmauthor{rmilson}{146}
\pmtype{Definition}
\pmcomment{trigger rebuild}
\pmclassification{msc}{05A15}
\pmsynonym{Stirling numbers}{StirlingNumbersOfTheFirstKind}
\pmrelated{StirlingNumbersSecondKind}

\endmetadata

\usepackage{amsmath}
\usepackage{amsfonts}
\usepackage{amssymb}
\newcommand{\un}{{\underline{n}}}
\newcommand{\reals}{\mathbb{R}}
\newcommand{\natnums}{\mathbb{N}}
\newcommand{\cnums}{\mathbb{C}}
\newcommand{\znums}{\mathbb{Z}}

\newcommand{\lp}{\left(}
\newcommand{\rp}{\right)}
\newcommand{\lb}{\left[}
\newcommand{\rb}{\right]}

\newcommand{\supth}{^{\text{th}}}


\newtheorem{proposition}{Proposition}
\begin{document}
\paragraph{Introduction.}
The Stirling numbers of the first kind, frequently denoted as
$$s(n,k)= { n\brack k}, \quad k,n\in\natnums,\quad 1\leq k\leq n,$$
are the integer
coefficients of the falling factorial polynomials.  To be more precise,
the defining relation for the Stirling numbers of the first kind is:
$$x^\un = x(x-1)(x-2)\ldots(x-n+1) = \sum_{k=1}^n s(n,k)x^k.$$
Here is the table of some initial values.
\bigskip

\begin{tabular}{lrrrrr}
$n\backslash k$ & 1 & 2 & 3  & 4 & 5\\
1 & 1 \\
2 & -1 &  1 \\
3 & 2 & -3 & 1 \\
4 & -6 & 11 & -6 & 1 \\
5 & 24 & -50& 35& -10 & 1
\end{tabular}

\paragraph{Recurrence Relation.}
The evident observation that
$$x^{\underline{n+1}} = x x^\un -n x^\un.$$
leads to the following equivalent characterization of the $s(n,k)$, in
terms of a 2-place
recurrence formula:
$$s(n+1,k) = s(n,k-1) - ns(n,k),\qquad 1\leq k < n,$$
subject to the
following initial conditions:
$$s(n,0) = 0,\qquad s(1,1) = 1.$$

\paragraph{Generating Function.}
There is also a strong connection with the generalized binomial
formula, which furnishes us with the following generating function:
$$(1+t)^x = \sum_{n=0}^\infty \sum_{k=1}^n s(n,k) x^k
\frac{t^n}{n!}.$$
This generating function implies a number of identities.
Taking the derivative of both sides with respect to $t$ and equating
powers, leads to the recurrence relation described above.
Taking the derivative of both sides with respect to $x$ gives
$$
(k+1) s(n,k+1) = \sum_{j=k}^n (-1)^{n-j}(n-j)!\binom{n+1}{j}
s(j,k)$$
This is because the derivative of the left side of the
generating funcion equation with respect to $x$ is
$$
(1+t)^x \ln(1+t) =(1+t)^x \sum_{k=1}^\infty (-1)^{k-1}
\frac{t^k}{k}.$$



The relation
$$(1+t)^{x_1} (1+t)^{x_2} = (1+t)^{x_1+x_2}$$
yields the following family of summation identities. For any given
$k_1,k_2,d\geq 1$ we have 
$$
\binom{k_1+k_2}{k_1} s(d+k_1+k_2,k_1+k_2) =
\sum_{d_1+d_2=d} \binom{d+k_1+k_2}{k_1+d_1} s(d_1+k_1,k_1)
s(d_2+k_2,k_2).$$

\paragraph{Enumerative interpretation.}
The absolute value of the Stirling number of the first kind, $s(n,k)$,
counts the number of permutations of $n$ objects with exactly $k$
orbits (equivalently, with exactly $k$ cycles).  For example, $s(4,2)=11$, corresponds to the fact that
the symmetric group on 4 objects has $3$ permutations of the form
$$(* *)(* *)\quad \mbox{ --- 2 orbits of size 2 each},$$
and $8$ permutations
of the form 
$$(* * *)\quad \mbox{ --- 1 orbit of size 3, and 1 orbit of size
  1},$$
(see the entry on cycle notation for the meaning of the above
expressions.)

Let us prove this.  First, we can remark that the unsigned Stirling
numbers of the first are characterized by the following recurrence
relation:
$$|s(n+1,k)| = |s(n,k-1)| + n|s(n,k)|,\qquad 1\leq k < n.$$ 

To see why the above recurrence relation matches the count of
permutations with $k$ cycles, consider forming a permutation of $n+1$
objects from a permutation of $n$ objects by adding a distinguished
object.  There are exactly two ways in which this can be accomplished.
We could do this by forming a singleton cycle, i.e. leaving the extra
object alone.  This accounts for the $s(n,k-1)$ term in the recurrence
formula.  We could also insert the new object into one of the existing
cycles.  Consider an arbitrary permutation of $n$ object with $k$
cycles, and label the objects $a_1,\ldots,a_n$, so that the
permutation is represented by
$$\underbrace{(a_1 \ldots a_{j_1})(a_{j_1+1} \ldots
  a_{j_2})\ldots(a_{j_{k-1}+1} \ldots a_n)}_{\displaystyle k \mbox{
    cycles}}.$$
To form a new permutation of $n+1$ objects and $k$ cycles one must
insert the new object into this array.  There are, evidently $n$ ways
to perform this insertion.  This explains the $n\,s(n,k)$ term of the
recurrence relation.  Q.E.D.
%%%%%
%%%%%
\end{document}
