\documentclass[12pt]{article}
\usepackage{pmmeta}
\pmcanonicalname{VandermondeIdentity}
\pmcreated{2013-03-22 14:08:49}
\pmmodified{2013-03-22 14:08:49}
\pmowner{mps}{409}
\pmmodifier{mps}{409}
\pmtitle{Vandermonde identity}
\pmrecord{8}{35562}
\pmprivacy{1}
\pmauthor{mps}{409}
\pmtype{Theorem}
\pmcomment{trigger rebuild}
\pmclassification{msc}{05A19}
\pmrelated{PascalsRule}

\endmetadata

% this is the default PlanetMath preamble.  as your knowledge
% of TeX increases, you will probably want to edit this, but
% it should be fine as is for beginners.

% almost certainly you want these
\usepackage{amssymb}
\usepackage{amsmath}
\usepackage{amsfonts}

% used for TeXing text within eps files
%\usepackage{psfrag}
% need this for including graphics (\includegraphics)
%\usepackage{graphicx}
% for neatly defining theorems and propositions
\usepackage{amsthm}
% making logically defined graphics
%%%\usepackage{xypic}

% there are many more packages, add them here as you need them

% define commands here
\newtheorem{Theorem}{Theorem}
\begin{document}
\PMlinkescapeword{restriction}
\PMlinkescapeword{completes}
\PMlinkescapeword{decide}
\PMlinkescapeword{size}
\PMlinkescapeword{formula}
\begin{Theorem}[\cite{AS} 24.1.1 formula A. II.]
For any $p$ and $q$ and any $k$ with $0\le k\le p+q$,
\begin{equation}
\binom{p+q}{k}=\sum_{i=0}^k\binom{p}{i}\binom{q}{k-i}.\tag{*}
\end{equation}
\end{Theorem}

\begin{proof}
Let $P$ and $Q$ be disjoint sets with $|P|=p$ and $|Q|=q$.  Then the 
left-hand side
of Equation~(*) is equal to the number of subsets of $P\cup Q$ of size $k$.
To build a subset of $P\cup Q$ of size $k$, we first decide how many elements, say $i$ with $0\le i\le k$,
we will select from $P$.  We can then select those elements in $\binom{p}{i}$
ways.  Once we have done so, we must select the
remaining $k-i$ elements from $Q$, which we can do in $\binom{q}{k-i}$ ways.  Thus there are $\binom{p}{i}\binom{q}{k-i}$ ways to select a subset of $P\cup Q$ of size $k$ subject to the restriction that exactly $i$ elements come from $P$.  Summing over all possible $i$ completes the proof.
\end{proof}

\begin{thebibliography}{9}
\bibitem{AS}
Abramowitz, M., and I. A. Stegun, eds.  {\it Handbook of Mathematical Functions}.  National Bureau of Standards, Dover, New York, 1974.
\end{thebibliography}
%%%%%
%%%%%
\end{document}
