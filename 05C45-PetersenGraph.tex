\documentclass[12pt]{article}
\usepackage{pmmeta}
\pmcanonicalname{PetersenGraph}
\pmcreated{2013-03-22 11:52:55}
\pmmodified{2013-03-22 11:52:55}
\pmowner{drini}{3}
\pmmodifier{drini}{3}
\pmtitle{Petersen graph}
\pmrecord{10}{30478}
\pmprivacy{1}
\pmauthor{drini}{3}
\pmtype{Definition}
\pmcomment{trigger rebuild}
\pmclassification{msc}{05C45}
\pmclassification{msc}{46L05}
\pmclassification{msc}{82-00}
\pmclassification{msc}{83-00}
\pmclassification{msc}{81-00}
\pmrelated{Traceable}
\pmrelated{HamiltonianPath}
\pmrelated{HamiltonianGraph}

\usepackage{amssymb}
\usepackage{amsmath}
\usepackage{amsfonts}
\usepackage{graphicx}
%%%%\usepackage{xypic}
\begin{document}
\emph{Petersen's graph}. An example of graph that is traceable but not Hamiltonian. That is, it has a Hamiltonian path but doesn't have a Hamiltonian cycle.

\begin{center}
\includegraphics{petersen}
\end{center}

This is also the canonical example of a hypohamiltonian graph.
%%%%%
%%%%%
%%%%%
%%%%%
\end{document}
