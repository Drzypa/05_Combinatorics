\documentclass[12pt]{article}
\usepackage{pmmeta}
\pmcanonicalname{LeafNodeofATree}
\pmcreated{2013-03-22 12:30:28}
\pmmodified{2013-03-22 12:30:28}
\pmowner{akrowne}{2}
\pmmodifier{akrowne}{2}
\pmtitle{leaf node (of a tree)}
\pmrecord{5}{32741}
\pmprivacy{1}
\pmauthor{akrowne}{2}
\pmtype{Definition}
\pmcomment{trigger rebuild}
\pmclassification{msc}{05C05}
\pmsynonym{leaf node}{LeafNodeofATree}
\pmsynonym{leaf}{LeafNodeofATree}

\usepackage{amssymb}
\usepackage{amsmath}
\usepackage{amsfonts}

%\usepackage{psfrag}
%\usepackage{graphicx}
%%\usepackage{xypic} 
\xyoption{all}
\usepackage{color}
\begin{document}
A \emph{leaf} of a tree is any node which has degree of exactly 1. Put another way, a leaf node of a rooted tree is any node which has no child nodes.

\begin{center}

$$\xymatrix{
& \bullet \ar@{-}[dl] \ar@{-}[dr] & & & \\
{\color{red}\bullet} & & \bullet \ar@{-}[dr]\ar@{-}[dl] & & \\
& \bullet \ar@{-}[dl] & & {\color{red}\bullet} & \\
{\color{red}\bullet} & & & & }$$

{\tiny Figure: A tree with leaf nodes highlighted in red.}
\end{center}
%%%%%
%%%%%
\end{document}
