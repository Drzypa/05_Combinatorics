\documentclass[12pt]{article}
\usepackage{pmmeta}
\pmcanonicalname{CentralBinomialCoefficient}
\pmcreated{2013-03-22 14:25:40}
\pmmodified{2013-03-22 14:25:40}
\pmowner{rspuzio}{6075}
\pmmodifier{rspuzio}{6075}
\pmtitle{central binomial coefficient}
\pmrecord{8}{35936}
\pmprivacy{1}
\pmauthor{rspuzio}{6075}
\pmtype{Definition}
\pmcomment{trigger rebuild}
\pmclassification{msc}{05A10}
\pmclassification{msc}{11B65}
\pmrelated{BinomialCoefficient}
\pmrelated{CatalanNumbers}

% this is the default PlanetMath preamble.  as your knowledge
% of TeX increases, you will probably want to edit this, but
% it should be fine as is for beginners.

% almost certainly you want these
\usepackage{amssymb}
\usepackage{amsmath}
\usepackage{amsfonts}

% used for TeXing text within eps files
%\usepackage{psfrag}
% need this for including graphics (\includegraphics)
%\usepackage{graphicx}
% for neatly defining theorems and propositions
%\usepackage{amsthm}
% making logically defined graphics
%%%\usepackage{xypic}

% there are many more packages, add them here as you need them

% define commands here
\newtheorem{thm}{Theorem}
\begin{document}
The $n$th {\em central binomial coefficient} is defined to be

\[ {2n \choose n} = \frac{(2n)!}{(n!)^2} \]

where ${2n \choose n}$ is a binomial coefficient.  These numbers have the generating function

\[ \frac{1}{\sqrt{1-4x}} = 1 + 2x + 6x^2 + 20x^3 + 70x^4 + 252x^5 + \cdots \]

They are closely related to the Catalan sequence, in that 
\[ C_n = \frac{1}{n+1} {2n \choose n}  \]

{\bf Alternate definition}\\

A less frequently-encountered definition for the $n$th central binomial coefficient is ${ n \choose {\lfloor \frac{n}{2} \rfloor} }$.

Note that the set of these numbers meeting this alternate criterion is a superset of those meeting the first criterion, since for $n = 2m$ we have
\[ {n \choose {\lfloor \frac{n}{2} \rfloor} } =
   {2m \choose {\lfloor \frac{2m}{2} \rfloor} } = {2m \choose m} \]

By cancelling terms of one of the $n!$'s against terms of the $2n!$, one may rewrite the central binomial 
coefficient as follows:
\[
{2n \choose n} = 
{2n (2n-1) \cdots (n+2) (n+1) \over
 n (n-1) \cdots 3 \cdot 2 \cdot 1}.
\]
Alternatively, one may cancel each term of the $n!$ against twice itself, leaving $2$'s in the numerator:
\[
{2n \choose n} = 
2^n {(2n-1) (2n-3) \cdots 5 \cdot 3 \cdot 1 \over
n (n-1) \cdots 3 \cdot 2 \cdot 1}
\]
Doubling the terms in the denominator, we obtain an expression for the central binomial coeficient
in terms of a quotient of successive odd numbers by successive even numbers:
\[
{2n \choose n} = 4^n
{(2n-1) (2n-3) \cdots 5 \cdot 3 \cdot 1 \over
 2n (2n-2) \cdots 6 \cdot 4 \cdot 2}
\]
By means of these formulae, one may derive some important properties of the central
binomial coeficients.  By examining the first two formulae, one may deduce results
about the prime factors of central binomial coefficients  (for proofs, please see the
attachments to this entry):

\begin{thm}
If $n \ge 3$ is an integer and $p$ is a prime number such that $n < p < 2n$, then
$p$ divides ${2n \choose n}$.
\end{thm}

\begin{thm}
If $n \ge 3$ is an integer and $p$ is a prime number such that $2n/3 < p \le n$, then
$p$ does not divide ${2n \choose n}$.
\end{thm}

In conjunction with Wallis' formula for $\pi$, the third formula for the central
binomial coefficient may be used to derive an asymptotic expression, as is done in
an attachment to this entry:
\[
 {2n \choose n} \approx
 \sqrt{2 \over \pi}
 {4^n \over \sqrt{2n+1}}
\]
%%%%%
%%%%%
\end{document}
