\documentclass[12pt]{article}
\usepackage{pmmeta}
\pmcanonicalname{MinimumWeightedPathLength}
\pmcreated{2013-03-22 12:32:12}
\pmmodified{2013-03-22 12:32:12}
\pmowner{mathwizard}{128}
\pmmodifier{mathwizard}{128}
\pmtitle{minimum weighted path length}
\pmrecord{6}{32779}
\pmprivacy{1}
\pmauthor{mathwizard}{128}
\pmtype{Definition}
\pmcomment{trigger rebuild}
\pmclassification{msc}{05C05}
\pmrelated{WeightedPathLength}
\pmrelated{ExtendedBinaryTree}
\pmrelated{HuffmansAlgorithm}
\pmrelated{HuffmanCoding}

\usepackage{amssymb}
\usepackage{amsmath}
\usepackage{amsfonts}
\usepackage{graphicx}
\begin{document}
Given a list of weights, $W := \left\{ w_1, w_2, \dots, w_n \right\}$, the
\emph{minimum weighted path length} is the minimum of the weighted path length of all extended binary trees that have $n$ external nodes with weights taken from $W$.  There may be multiple possible trees that give this minimum path length, and quite often finding this tree is more important than determining the path length.

\subsubsection*{Example}

Let $W := \left\{ 1, 2, 3, 3, 4 \right\}$.  The minimum weighted path length is
$29$.  A tree that gives this weighted path length is shown below.

\begin{center}
\includegraphics{tree.10}
\end{center}

\subsubsection*{Applications}

Constructing a tree of minimum weighted path length for a given set of weights has several applications, particularly dealing with optimization problems.
A simple and elegant algorithm for constructing such a tree is Huffman's algorithm.
Such a tree can give the most optimal algorithm for merging $n$ sorted sequences (optimal merge).  It can also provide a means of compressing data (Huffman coding), as well as lead to optimal searches.
%%%%%
%%%%%
\end{document}
