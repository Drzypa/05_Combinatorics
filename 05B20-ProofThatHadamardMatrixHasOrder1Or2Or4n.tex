\documentclass[12pt]{article}
\usepackage{pmmeta}
\pmcanonicalname{ProofThatHadamardMatrixHasOrder1Or2Or4n}
\pmcreated{2013-03-22 16:50:56}
\pmmodified{2013-03-22 16:50:56}
\pmowner{Mathprof}{13753}
\pmmodifier{Mathprof}{13753}
\pmtitle{proof that Hadamard matrix has order 1 or  2 or 4n}
\pmrecord{10}{39095}
\pmprivacy{1}
\pmauthor{Mathprof}{13753}
\pmtype{Proof}
\pmcomment{trigger rebuild}
\pmclassification{msc}{05B20}
\pmclassification{msc}{15-00}

\endmetadata

% this is the default PlanetMath preamble.  as your knowledge
% of TeX increases, you will probably want to edit this, but
% it should be fine as is for beginners.

% almost certainly you want these
\usepackage{amssymb}
\usepackage{amsmath}
\usepackage{amsfonts}

% used for TeXing text within eps files
%\usepackage{psfrag}
% need this for including graphics (\includegraphics)
%\usepackage{graphicx}
% for neatly defining theorems and propositions
%\usepackage{amsthm}
% making logically defined graphics
%%%\usepackage{xypic}

% there are many more packages, add them here as you need them

% define commands here

\begin{document}
Let $m$ be the order of a Hadamard matrix. The matrix $[1]$ shows that order 1 
is possible, and the \PMlinkescapetext{parent} entry has a $2 \times 2$ Hadamard matrix
, so assume $m>2$. 

We can assume that the first row of the matrix is all 1's by multiplying
selected columns by $-1$. Then permute columns as needed to arrive at a
matrix whose first three rows have the following form, where $P$ denotes a submatrix of one row
and all 1's and $N$ denotes a submatrix of one row and all $-1$'s.

$$\begin{matrix}
\begin{matrix}
x \quad  &\quad y & \quad z & \quad w
\end{matrix} &
\begin{matrix}
\quad
\end{matrix}
\\
\left[ \begin{matrix}
\overbrace{P} & \overbrace{P} &  \overbrace{P} & \overbrace{P} \\
P & P & N & N \\
P & N & P & N \\
\end{matrix} \right] 
\end{matrix}
$$

Since the rows are orthogonal and there are $m$ columns we have
\begin{center}
$\begin{cases}
x + y + z +w &= m \\
x + y - z - w &= 0 \\
x - y + z -w &= 0 \\
x - y - z + w &= 0.
\end{cases}$
\end{center}
Adding the 4 equations together we get 
$$
4x = m.
$$
so that $m$ must be divisible by 4.
%%%%%
%%%%%
\end{document}
