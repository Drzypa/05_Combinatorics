\documentclass[12pt]{article}
\usepackage{pmmeta}
\pmcanonicalname{ParentNodeinATree}
\pmcreated{2013-03-22 12:30:32}
\pmmodified{2013-03-22 12:30:32}
\pmowner{akrowne}{2}
\pmmodifier{akrowne}{2}
\pmtitle{parent node (in a tree)}
\pmrecord{5}{32742}
\pmprivacy{1}
\pmauthor{akrowne}{2}
\pmtype{Definition}
\pmcomment{trigger rebuild}
\pmclassification{msc}{05C05}
\pmsynonym{parent node}{ParentNodeinATree}
\pmsynonym{parent}{ParentNodeinATree}
\pmrelated{ChildNodeOfATree}

\endmetadata

\usepackage{amssymb}
\usepackage{amsmath}
\usepackage{amsfonts}

%\usepackage{psfrag}
%\usepackage{graphicx}
%%\usepackage{xypic}
\xyoption{all}
\usepackage{color}
\begin{document}
A \emph{parent} node $P$ of a node $C$ in a tree is the first node which lies along the path from $C$ to the root of the tree, $R$.

Drawn in the canonical root-at-top manner, the parent node of a node $C$ in a tree is simply the node immediately above $C$ which is connected to it.

\begin{center}

$$\xymatrix{
& \bullet \ar@{-}[dl] \ar@{-}[dr] & & & \\
\bullet & & {\color{red}\bullet} \ar@{-}[dr]\ar@{-}[dl] & & \\
& \bullet \ar@{-}[dl] & & {\color{blue}\bullet} & \\
\bullet & & & & }$$

{\tiny Figure: A node (blue) and its parent (red.)}
\end{center}
%%%%%
%%%%%
\end{document}
