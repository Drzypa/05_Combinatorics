\documentclass[12pt]{article}
\usepackage{pmmeta}
\pmcanonicalname{EquivalenceBetweenTheMinorAndTopologicalMinorOfK5OrK33}
\pmcreated{2013-03-22 17:47:15}
\pmmodified{2013-03-22 17:47:15}
\pmowner{jwaixs}{18148}
\pmmodifier{jwaixs}{18148}
\pmtitle{equivalence between the minor and topological minor of $K_5$ or $K_{3,3}$}
\pmrecord{11}{40248}
\pmprivacy{1}
\pmauthor{jwaixs}{18148}
\pmtype{Proof}
\pmcomment{trigger rebuild}
\pmclassification{msc}{05C83}
\pmclassification{msc}{05C10}

% this is the default PlanetMath preamble.  as your knowledge
% of TeX increases, you will probably want to edit this, but
% it should be fine as is for beginners.

% almost certainly you want these
\usepackage{amssymb}
\usepackage{amsmath}
\usepackage{amsfonts}

% used for TeXing text within eps files
%\usepackage{psfrag}
% need this for including graphics (\includegraphics)
%\usepackage{graphicx}
% for neatly defining theorems and propositions
%\usepackage{amsthm}
% making logically defined graphics
%%\usepackage{xypic}

% there are many more packages, add them here as you need them

% define commands here

\begin{document}
$ \Leftarrow )$ Remark that if a graph G contains $ K_5 $ or $ K_{3,3} $ as a \PMlinkname{topological minor}{subdivision} it is easy to see that it also contains respectively $ K_5 $ or $ K_{3,3} $ as a minor.


$ \Rightarrow )$ Same goes for $ K_{3,3} $ as minor. If graph G contains a minor of $ K_{3,3} $ as minor it also contains a \PMlinkname{topological minor}{subdivision} of $ K_{3,3} $ in G.
So it's suffices to show that if graph G contains a $ K_5 $ as minor, it also contains a $ K_5 $ or $ K_{3,3} $ as \PMlinkname{topological minor}{subdivision}.

Suppose that $ G \succeq K_5 $, let $ K \subseteq G $ be minimal such that $ K $ is a minor of $ G $. Because $K$ is minimal every branch set of $K$ induces a tree in $K$ and every two branch sets have exactly one edge between them.
For every branch tree $ V_x $ we can add all four edges joining it to the other branches, this way we obtain a tree $ T_x $.
If each of the trees $ T_x $ is a \PMlinkname{topological minor}{subdivision} of $ K_{1,4} $, $ K $ will be a \PMlinkname{topological minor}{subdivision} of $ K_5 $. This image should make clear how our $ K $ looks like. The big circles are presenting different sets of vertices.

\begin{center}
$ \xymatrix {
  & \bigcirc & & \bigcirc \ar@{.}[ll] \\
  \bigcirc \ar@{.}[ur] \ar@{.}[urrr] & & & & \bigcirc \ar@{.}[llll] \ar@{.}[ul] \ar@{.}[ulll] \\
  \bullet \ar@{-}[u] & \bullet \ar@{-}[uu] & & \bullet \ar@{-}[uu] & \bullet  \ar@{-}[u] & T_x = MK_{1,4} \\
  & & \bigcirc \ar@{-}[urr] \ar@{-}[ur] \ar@{-}[ul] \ar@{-}[ull] & V_x
} $
\end{center}

If one of the trees is not a $ TK_{1,4} $ it has exactly two vertices of degree 3, which means that $ K $ is a \PMlinkname{topological minor}{subdivision} of $ K_{3,3} $. $ K $ now looks like this. The big circles and triangles are presenting different sets of vertices.

\begin{center}
$ \xymatrix {
  & \bigtriangleup & \bigcirc \ar@{.}[l] \\
  \bigtriangleup \ar@{.}[urr] & & & \bigcirc \ar@{.}[lll] \ar@{.}[ull] \\
  \bullet \ar@{-}[u] & \bullet \ar@{-}[uu] & \bullet \ar@{-}[uu] & \bullet  \ar@{-}[u] & T_x \ne MK_{1,4} \\
  & \bigcirc \ar@{-}[ul] \ar@{-}[u] & \bigtriangleup \ar@{-}[u] \ar@{-}[ur] \ar@{-}[l] & V_x
} $
\end{center}

Thus if $ G \succeq K_5 $ then $ G $ contains a \PMlinkname{topological minor}{subdivision} of $ K_{3,3} $ or $ K_5 $. Which proofs this theorem.
%%%%%
%%%%%
\end{document}
