\documentclass[12pt]{article}
\usepackage{pmmeta}
\pmcanonicalname{DeBruijnDigraph}
\pmcreated{2013-03-22 12:16:11}
\pmmodified{2013-03-22 12:16:11}
\pmowner{Mathprof}{13753}
\pmmodifier{Mathprof}{13753}
\pmtitle{de Bruijn digraph}
\pmrecord{8}{31699}
\pmprivacy{1}
\pmauthor{Mathprof}{13753}
\pmtype{Definition}
\pmcomment{trigger rebuild}
\pmclassification{msc}{05C20}
\pmrelated{KautzGraph}
\pmrelated{LineGraph}

% this is the default PlanetMath preamble.  as your knowledge
% of TeX increases, you will probably want to edit this, but
% it should be fine as is for beginners.

% almost certainly you want these
\usepackage{amssymb}
\usepackage{amsmath}
\usepackage{amsfonts}

% used for TeXing text within eps files
%\usepackage{psfrag}
% need this for including graphics (\includegraphics)
%\usepackage{graphicx}
% for neatly defining theorems and propositions
%\usepackage{amsthm}
% making logically defined graphics
%%%%\usepackage{xypic}
\input xy
\xyoption{all}

% there are many more packages, add them here as you need them

% define commands here
\begin{document}
The vertices of the \emph{de Bruijn digraph} $B(n,m)$ are all possible words of length $m-1$ chosen from an alphabet of size $n$.

$B(n,m)$ has $n^{m}$ edges consisting of each possible word of length $m$ from an alphabet of size $n$.  The edge $a_1a_2\dots a_n$ connects the vertex $a_1a_2\dots a_{n-1}$ to the vertex $a_2a_3\dots a_n$.

For example, $B(2,4)$ could be drawn as:
$$\xymatrix{
& 000 \ar@(ur,ul)[]_{0000} \ar[dr]^{0001} & \\
100 \ar[ur]^{1000} \ar[rr]^{1001} & & 001 \ar[dl]^{0010} \ar[ddd]^{0011}\\
& 010 \ar[ul]^{0100} \ar@(ur,dr)[d]^{0101} & \\
& 101 \ar[dr]^{1011} \ar@(dl,ul)[u]^{1010} & \\
110 \ar[uuu]^{1100} \ar[ur]^{1101} & & 011 \ar[ll]_{0110} \ar[dl]^{0111}\\
& 111 \ar@(dl,dr)[]_{1111} \ar[ul]^{1110} & 
}$$

Notice that an Euler cycle on $B(n,m)$ represents a shortest sequence of characters from an alphabet of size $n$ that includes every possible subsequence of $m$ characters.  For example, the sequence $000011110010101000$ includes all 4-bit subsequences.  Any de Bruijn digraph must have an Euler cycle, since each vertex has in degree and out degree of $n$.
%%%%%
%%%%%
%%%%%
\end{document}
