\documentclass[12pt]{article}
\usepackage{pmmeta}
\pmcanonicalname{AlgebraicConnectivityOfAGraph}
\pmcreated{2013-03-22 17:04:37}
\pmmodified{2013-03-22 17:04:37}
\pmowner{Mathprof}{13753}
\pmmodifier{Mathprof}{13753}
\pmtitle{algebraic connectivity of a graph}
\pmrecord{9}{39370}
\pmprivacy{1}
\pmauthor{Mathprof}{13753}
\pmtype{Definition}
\pmcomment{trigger rebuild}
\pmclassification{msc}{05C50}
\pmdefines{algebraic connectivity}

% this is the default PlanetMath preamble.  as your knowledge
% of TeX increases, you will probably want to edit this, but
% it should be fine as is for beginners.

% almost certainly you want these
\usepackage{amssymb}
\usepackage{amsmath}
\usepackage{amsfonts}

% used for TeXing text within eps files
%\usepackage{psfrag}
% need this for including graphics (\includegraphics)
%\usepackage{graphicx}
% for neatly defining theorems and propositions
%\usepackage{amsthm}
% making logically defined graphics
%%%\usepackage{xypic}

% there are many more packages, add them here as you need them

% define commands here

\begin{document}
Let $L(G)$ be the \PMlinkname{Laplacian matrix}{LaplacianMatrixOfAGraph}
 of a finite connected graph $G$ with $n$ vertices. Let the eigenvalues
of $L(G)$ be denoted by
$\lambda_1 \le \lambda_2 \le \cdots \le \lambda_n$, which
is the usual notation in spectral graph theory.
The \emph{\PMlinkescapetext{algebraic} connectivity} of $G$ is $\lambda_2$.
The usual notation for the algebraic connectivity is $a(G)$.
The parameter is a measure of how well the graph is connected.
For example, $a(G) \not = 0$ if and only if $G$ is connected.


\begin{thebibliography}{99}
\bibitem{MF} Fieldler, M. Algebraic connectivity of graphs, \emph{Czech. Math. J.} \textbf{23} (98) (1973) 
pp. 298-305.
\bibitem{RM} Merris, R. Laplacian matrices of graphs: a survey, 
\emph{Lin. Algebra and its Appl.} \textbf{197/198} (1994) pp. 143-176.
\end{thebibliography}


%%%%%
%%%%%
\end{document}
