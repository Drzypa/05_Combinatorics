\documentclass[12pt]{article}
\usepackage{pmmeta}
\pmcanonicalname{PlanarGraph}
\pmcreated{2013-03-22 12:17:37}
\pmmodified{2013-03-22 12:17:37}
\pmowner{archibal}{4430}
\pmmodifier{archibal}{4430}
\pmtitle{planar graph}
\pmrecord{12}{31826}
\pmprivacy{1}
\pmauthor{archibal}{4430}
\pmtype{Definition}
\pmcomment{trigger rebuild}
\pmclassification{msc}{05C10}
\pmsynonym{planar}{PlanarGraph}
\pmrelated{WagnersTheorem}
\pmrelated{KuratowskisTheorem}
\pmrelated{CrossingLemma}
\pmrelated{CrossingNumber}
\pmrelated{FourColorConjecture}
\pmdefines{plane graph}

\endmetadata

\usepackage{amssymb}
\usepackage{amsmath}
\usepackage{amsthm}
\usepackage{amsfonts}

%\usepackage{psfrag}
%\usepackage{graphicx}
%%\usepackage{xypic}
\xyoption{all}

\theoremstyle{definition}
\newtheorem*{defn}{Definition}
\begin{document}
\PMlinkescapeword{divide}
\PMlinkescapeword{map}
\PMlinkescapeword{outer}
\PMlinkescapeword{terms}

\section*{Description}

A \emph{planar graph} is a graph which can be drawn on a plane (a flat 2-d surface) or on a sphere, with no edges crossing. When drawn on a sphere, the edges divide its area in a number of regions called faces (or ``countries'', in the context of map coloring). When drawn on a plane, there is one \emph{outer country} taking up all the space outside the drawing. Every graph drawn on a sphere can be drawn on a plane (puncture the sphere in the interior of any one of the countries) and vice versa. Statements on map coloring are often simpler in terms of a spherical map because the outer country is no longer a special case.

The number of faces (countries) equals $c+1$ where $c$ is the \emph{cyclomatic number}, $c=m-n+k$ (where $m$ is the number of edges, $n$ the number of vertices, and $k$ the number of connected components of the graph). All this holds equally for planar multigraphs and pseudographs.

No complete graphs above $K_4$ are planar.  $K_4$, drawn without crossings, looks like :

$$ 
\xymatrix{
  & A\ar@{-}[d] \ar@{-}[ddl] \ar@{-}[ddr] &    \\
  & B\ar@{-}[dl] \ar@{-}[dr] &    \\
C \ar@{-}[rr] &   & D } 
$$

Hence it is planar (try this for $K_5$.)

A \PMlinkescapetext{straight line} drawing of a planar graph is a drawing in which each edge is drawn as a \PMlinkescapetext{straight} line segment. Every planar graph has a \PMlinkescapetext{straight line} drawing. This result was found independently by Wagner, F\'ary and Stein. Schnyder improved this further by showing how to draw any planar graph with $n$ vertices on an integer grid of $O(n^2)$ area.

\section*{Definition}

Ideally, this definition would just formalize what was described above.  It will not, exactly.  It will formalize the notion of a graph \emph{with an embedding into the plane}.

\begin{defn}
Let $M$ be a topological manifold.  Then a \emph{graph on $M$} is a pair $(G,\iota)$, where 
\begin{enumerate}
\item $G$ is a multigraph, 
\item $\iota$ is a function from the graph topology of $G$ into $M$, and
\item $\iota$ is a homeomorphism onto its image.
\end{enumerate}
A \emph{plane graph} is a graph on $\mathbb{R}^2$. 

A \emph{planar graph} is a graph $G$ that has an embedding $\iota$ making $(G,\iota)$ into a plane graph.
\end{defn}

Normally, the only manifolds $M$ that are of interest are two-dimensional.  

The most usual question for which this definition finds a use is ``can the following graph $G$ be made into a graph on $M$?''.  When $M$ is the plane, this is usually phrased as ``Is $G$ a planar graph?''.  Wagner's theorem provides a \PMlinkescapetext{simple} criterion for answering this question.  When $M$ is a torus, the answer changes: the complete bipartite graph $K_{3,3}$ can be made into a graph on the torus. 

A graph on a manifold has a notion of ``face'' as well as the usual graph notions of vertex and edge.
%%%%%
%%%%%
\end{document}
