\documentclass[12pt]{article}
\usepackage{pmmeta}
\pmcanonicalname{LineGraph}
\pmcreated{2013-03-22 16:24:15}
\pmmodified{2013-03-22 16:24:15}
\pmowner{mps}{409}
\pmmodifier{mps}{409}
\pmtitle{line graph}
\pmrecord{4}{38552}
\pmprivacy{1}
\pmauthor{mps}{409}
\pmtype{Definition}
\pmcomment{trigger rebuild}
\pmclassification{msc}{05C75}
\pmrelated{DeBruijnDigraph}

% this is the default PlanetMath preamble.  as your knowledge
% of TeX increases, you will probably want to edit this, but
% it should be fine as is for beginners.

% almost certainly you want these
\usepackage{amssymb}
\usepackage{amsmath}
\usepackage{amsfonts}

% used for TeXing text within eps files
%\usepackage{psfrag}
% need this for including graphics (\includegraphics)
%\usepackage{graphicx}
% for neatly defining theorems and propositions
\usepackage{amsthm}
% making logically defined graphics
%%\usepackage{xypic}

% there are many more packages, add them here as you need them

% define commands here
\newtheorem*{proposition*}{Proposition}
\theoremstyle{definition}
\newtheorem{example}{Example}
\begin{document}
\PMlinkescapeword{connected}
\PMlinkescapeword{right}
\PMlinkescapeword{label}
\PMlinkescapeword{observation}
Let $G$ be a graph.  The \emph{line graph} of $G$ is denoted by $L(G)$ and is defined as follows.  The vertices of $L(G)$ are the edges of $G$.  Two vertices of $L(G)$ are connected by an edge if and only if the corresponding edges in $G$ form a path (which must be a directed path if $G$ is a digraph).

\begin{example}
Let $G$ be the undirected graph on the vertex set $V(G)=\{a,b,c,d\}$ and edge set $E(G)=\{ab, bc, cd, bd\}$.  Then the line graph $L(G)$ has vertex set $V(L(G))=\{ab, bc, cd, bd\}$.  Since $ab$ and $bc$ are incident in $G$, they are connected by an edge in $L(G)$.  We label this edge by the unique walk from $a$ to $c$ determined by these edges, namely $abc$.  With this notation, the edge set is $E(L(G))=\{abc, abd, bcd, bdc, cbd\}$.
\[\xymatrix{
& a\ar@{-}[d]^{ab}                     & &                          & ab\ar@{-}[dl]_{abc}\ar@{-}[dr]^{abd} & \\
                     & b\ar@{-}[dl]_{bc}\ar@{-}[dr]^{bd} &  & bc\ar@{-}[rr]^{bcd}\ar@{-}[dr]_{cbd} &                          & bd\ar@{-}[dl]^{bdc} \\
c\ar@{-}[rr]_{cd} & & d &                         & cd
}\]
Above we display the graphs $G$ (left) and $L(G)$ (right).
\end{example}

Part of the reason for interest in line graphs is the following observation:

\begin{proposition*}
If a graph is Eulerian, then its line graph is Hamiltonian.
\end{proposition*}

This follows immediately from the fact that a sequence of incident edges in $G$ is a sequence of incident vertices in $L(G)$.

\begin{example}
Let $B_n$ be the de Bruijn digraph on binary strings of length $n$.  Then for $n\ge 2$, the next de Bruijn digraph can be obtained by taking the line graph of the previous one, that is, $B_{n+1}=L(B_n)$.
\end{example}
%%%%%
%%%%%
\end{document}
