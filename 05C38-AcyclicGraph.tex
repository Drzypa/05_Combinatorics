\documentclass[12pt]{article}
\usepackage{pmmeta}
\pmcanonicalname{AcyclicGraph}
\pmcreated{2013-03-22 12:30:39}
\pmmodified{2013-03-22 12:30:39}
\pmowner{Logan}{6}
\pmmodifier{Logan}{6}
\pmtitle{acyclic graph}
\pmrecord{8}{32746}
\pmprivacy{1}
\pmauthor{Logan}{6}
\pmtype{Definition}
\pmcomment{trigger rebuild}
\pmclassification{msc}{05C38}
\pmsynonym{acyclic}{AcyclicGraph}
\pmsynonym{DAG}{AcyclicGraph}
\pmrelated{Graph}
\pmrelated{Cycle}
\pmrelated{BetheLattice}
\pmdefines{directed acyclic graph}

\endmetadata

\usepackage{amssymb}
\usepackage{amsmath}
\usepackage{amsfonts}
\usepackage[all]{xy}
\begin{document}
Any graph that contains no cycles is an \emph{acyclic graph}.  A directed acyclic graph is often called a DAG for short.

For example, the following graph and digraph are acyclic.

$$
\begin{array}{cc}
\xymatrix{&A\ar@{-}[dl]\ar@{-}[dr]\\B&&C}
\quad
&
\quad
\xymatrix{&A\ar[dr]\\B\ar[ur]\ar[rr]&&C}
\end{array}
$$

In contrast, the following graph and digraph are \emph{not} acyclic, because
each contains a cycle.

$$
\begin{array}{cc}
\xymatrix{&A\ar@{-}[dl]\ar@{-}[dr]\\B\ar@{-}[rr]&&C}
\quad
&
\quad
\xymatrix{&A\ar[dr]\\B\ar[ur]&&C\ar[ll]}
\end{array}
$$
%%%%%
%%%%%
\end{document}
