\documentclass[12pt]{article}
\usepackage{pmmeta}
\pmcanonicalname{ProofOfPascalsRule}
\pmcreated{2013-03-22 15:03:11}
\pmmodified{2013-03-22 15:03:11}
\pmowner{drini}{3}
\pmmodifier{drini}{3}
\pmtitle{proof of Pascal's rule}
\pmrecord{5}{36770}
\pmprivacy{1}
\pmauthor{drini}{3}
\pmtype{Proof}
\pmcomment{trigger rebuild}
\pmclassification{msc}{05A19}

\usepackage{amsmath}
\begin{document}
The definition of $\binom{n}{k}$ is the number of $k$-subsets out from an $n$-set. Using this combinatorial meaning the proof is straightforward.

Let $x$ a distinct element from the $n$-set. As previously stated, $\binom{n}{k}$ counts the number of subsets with $k$ elements, chosen from the set with $n$ elements. Now, some of these subsets will contain $x$ and some others don't.

The number of $k$-subsets not containing $x$ is $\binom{n-1}{k}$, since we need to choose $k$ elements from the $n-1$ elements different from $x$.

The number of $k$-subsets containing $x$ is $\binom{n-1}{k-1}$, because if it is given that $x$ is in the subset, we only need to choose the remaining $k-1$ elements from the $n-1$ elements that are different from $x$.

Thus
\[
\binom{n}{k} = \binom{n-1}{k} + \binom{n-1}{k-1}.
\]
%%%%%
%%%%%
\end{document}
