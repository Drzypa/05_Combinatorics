\documentclass[12pt]{article}
\usepackage{pmmeta}
\pmcanonicalname{RamseyNumbers}
\pmcreated{2013-03-22 16:16:32}
\pmmodified{2013-03-22 16:16:32}
\pmowner{wdsmith}{13774}
\pmmodifier{wdsmith}{13774}
\pmtitle{Ramsey numbers}
\pmrecord{10}{38387}
\pmprivacy{1}
\pmauthor{wdsmith}{13774}
\pmtype{Definition}
\pmcomment{trigger rebuild}
\pmclassification{msc}{05D10}
\pmrelated{RamseysTheorem2}
\pmdefines{Ramsey numbers}

% this is the default PlanetMath preamble.  as your knowledge
% of TeX increases, you will probably want to edit this, but
% it should be fine as is for beginners.

% almost certainly you want these
\usepackage{amssymb}
\usepackage{amsmath}
\usepackage{amsfonts}

% used for TeXing text within eps files
%\usepackage{psfrag}
% need this for including graphics (\includegraphics)
%\usepackage{graphicx}
% for neatly defining theorems and propositions
%\usepackage{amsthm}
% making logically defined graphics
%%%\usepackage{xypic}

% there are many more packages, add them here as you need them

% define commands here

\begin{document}
\PMlinkescapeword{blue}
\PMlinkescapeword{edges}
\PMlinkescapeword{contains}
\PMlinkescapeword{integer}

Define $R(a,b)$ to be the least integer $N$ such that, in any red-blue 2-coloring of the edges of a $N$-vertex complete graph $K_N$,
there must exist either an all-red $K_a$ or an all-blue $K_b$.

Frank Ramsey proved these numbers always exist.
He famously pointed out that among any 6 people, some three are mutual friends
or some three mutual non-friends.  That is, $R(3,3)\le 6$.
Since a red pentagon with a blue pentagram drawn inside it has no monochromatic triangle,
$R(3,3)\ge 6$.  So $R(3,3)=6$.

Special attention is usually paid to the diagonal $R(k,k)$, which is often just written $R(k)$.

One can also generalize this in various ways,
e.g. consider $R(a,b,c)$ for {\it three}-colorings of edges, etc (any number of
arguments), and allow general sets of graphs, not just pairs of complete ones.
 
Ramsey numbers are very difficult to determine.
To prove lower bounds, construct good edge-colorings of some $K_N$ and, use a clique-finder
to find the largest mono-colored cliques.
To prove upper bounds, the main tool has been $R(a,b)\le R(a-1,b)+R(b-1,a)$
which implies $R(a,b) \le {{a+b-2} \choose {a-1}}$ and then $R(k) \le [1+o(1)] 4^k / (4 \sqrt{\pi k}$
when $k \to \infty$.
From considering random colorings and using a probabilistic nonconstructive existence
argument, one may show $R(k) \ge k 2^{k/2} [ o(1) + \sqrt{2} / e ]$.
It is known that $R(1)=1$, $R(2)=2$, $R(3)=6$, $R(4)=18$, and $43 \le R(5) \le 49$.
For a survey of the best upper and lower bounds available on small
Ramsey numbers, see 
\PMlinkexternal{Radziszowski's survey}{http://www.combinatorics.org/Surveys/ds1.pdf}
(\PMlinkexternal{alternate link}{http://www.cs.rit.edu/~spr/}).
Another kind of Ramsey-like number which has not gotten as much attention as it deserves,
are Ramsey numbers for {\it directed} graphs.
Let $\vec{R}(n)$ denote the least integer $N$ so that any tournament (complete directed graph
with singly-directed arcs) with $\ge N$ vertices contains an acyclic (also called ``transitive'')
$n$-node tournament.  (Analogies: 2-color the edges $\to$ two directions for arcs.
Monochromatic $\to$ acyclic, i.e. all arcs ``point one way.'')

Again, to prove lower bounds, construct good tournaments and apply something like
a clique-finder (but instead aimed at trying to find the largest acyclic induced subgraph).
To prove upper bounds, the main tool is $\vec{R}(n+1) \le 2 \vec{R}(n)$.
That can be used to show the upper bound, and
random-orientation arguments combined with a nonconstructive probabilistic existence argument
show the lower bound, in $[1+o(1)] 2^{n+1)/2} \le \vec{R}(n) \le 55 \cdot 2^{n-7}$.
It is known that $\vec{R}(1)=1$, 
$\vec{R}(2)=2$,
$\vec{R}(3)=4$,
$\vec{R}(4)=8$,
$\vec{R}(5)=14$,
$\vec{R}(6)=28$,
and
$32 \le \vec{R}(7) \le 55$.
For a full survey of directed graph Ramsey numbers includng proofs and refererences,
see \PMlinkexternal{Smith's survey}{http://www.rangevoting.org/PuzzRamsey.html}.

%\begin{thebibliography}{9}
%\end{thebibliography}

\end{document}

%%%%%
%%%%%
\end{document}
