\documentclass[12pt]{article}
\usepackage{pmmeta}
\pmcanonicalname{AlternateProofOfMantelsTheorem}
\pmcreated{2013-03-22 17:56:35}
\pmmodified{2013-03-22 17:56:35}
\pmowner{lieven}{1075}
\pmmodifier{lieven}{1075}
\pmtitle{alternate proof of Mantel's theorem}
\pmrecord{5}{40440}
\pmprivacy{1}
\pmauthor{lieven}{1075}
\pmtype{Proof}
\pmcomment{trigger rebuild}
\pmclassification{msc}{05C75}
\pmclassification{msc}{05C69}

% this is the default PlanetMath preamble.  as your knowledge
% of TeX increases, you will probably want to edit this, but
% it should be fine as is for beginners.

% almost certainly you want these
\usepackage{amssymb}
\usepackage{amsmath}
\usepackage{amsfonts}

% used for TeXing text within eps files
%\usepackage{psfrag}
% need this for including graphics (\includegraphics)
%\usepackage{graphicx}
% for neatly defining theorems and propositions
%\usepackage{amsthm}
% making logically defined graphics
%%%\usepackage{xypic}

% there are many more packages, add them here as you need them

% define commands here

\begin{document}
Let $G$ be a triangle-free graph of order $n$. For each edge $xy$ of $G$ we consider the 
\PMlinkname{neighbourhoods}{NeighborhoodOfAVertex} 
$\Gamma(x)$ and $\Gamma(y)$ of $G$.
Since $G$ is triangle-free, these are disjoint. 

This is only possible if the sum of the degrees of $x$ and $y$ 
is less 
than or equal to $n$. So for each edge $xy$ we get the inequality

\[
\delta(x)+\delta(y)\leq n
\]

Summing these inequalities for all edges of $G$ gives us
\[
\Sigma_{x\in V(G)} (\delta(x))^2 \leq n |E(G)|
\]

(The left hand side is a sum of $\delta(x)$ where each edge incident with $x$ 
contributes one term and their are $\delta(x)$ such edges.)

Since $\Sigma_{x\in V(G)} \delta(x) = 2 |E(G)|$, we get $4 |E(G)|^2 = 
(\Sigma_{x\in V(G)} \delta(x))^2 $ and applying the Cauchy-Schwarz inequality gives 
$4 |E(G)|^2 \leq n \Sigma_{x\in V(G)} (\delta(x))^2 \leq n^2 |E(G)|$.

So we conclude that for a triangle-free graph $G$, 
\[|E(G)|\leq\frac{n^2}{4} \]
%%%%%
%%%%%
\end{document}
