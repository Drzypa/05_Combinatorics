\documentclass[12pt]{article}
\usepackage{pmmeta}
\pmcanonicalname{AssociationScheme}
\pmcreated{2013-03-22 19:15:03}
\pmmodified{2013-03-22 19:15:03}
\pmowner{CWoo}{3771}
\pmmodifier{CWoo}{3771}
\pmtitle{association scheme}
\pmrecord{10}{42176}
\pmprivacy{1}
\pmauthor{CWoo}{3771}
\pmtype{Definition}
\pmcomment{trigger rebuild}
\pmclassification{msc}{05E30}
\pmdefines{associate class}
\pmdefines{rank}
\pmdefines{intersection number}
\pmdefines{valency}

\endmetadata

\usepackage{amssymb,amscd}
\usepackage{amsmath}
\usepackage{amsfonts}
\usepackage{mathrsfs}

% used for TeXing text within eps files
%\usepackage{psfrag}
% need this for including graphics (\includegraphics)
%\usepackage{graphicx}
% for neatly defining theorems and propositions
\usepackage{amsthm}
% making logically defined graphics
%%\usepackage{xypic}
\usepackage{pst-plot}

% define commands here
\newcommand*{\abs}[1]{\left\lvert #1\right\rvert}
\newtheorem{prop}{Proposition}
\newtheorem{thm}{Theorem}
\newtheorem{ex}{Example}
\newcommand{\real}{\mathbb{R}}
\newcommand{\pdiff}[2]{\frac{\partial #1}{\partial #2}}
\newcommand{\mpdiff}[3]{\frac{\partial^#1 #2}{\partial #3^#1}}
\begin{document}
Association schemes were introduced by statisticians in the 1950's to analyze designs of statistical experiments.  Today, association schemes are useful not only in experimental designs, but in other areas of mathematics such as combinatorics (coding theory) and group theory (permutation groups).

There are several equivalent ways to define an association schemes.  Three useful ones are illustrated here:

Let $\Omega$ be a non-empty set with $n$ elements, and $s$ a positive integer.

\textbf{Definition 1}.  An \emph{association scheme} $\mathscr{Q}$ on $\Omega$ is a partition on $\Omega\times \Omega$ into sets $C_0,C_1,\ldots, C_s$ called \emph{associate classes}, such that
\begin{itemize}
\item each $C_i$ is a symmetric relation on $\Omega$, and $C_0$ in particular is the diagonal relation,
\item for $i,j,k\in \lbrace 0,1,\ldots,s\rbrace$, there is an integer $p_{ij}^k$ such that, for any $(a,b)\in C_k$, $$|\lbrace c \in \Omega \mid (a,c)\in C_i \mbox{ and } (c,b)\in C_j \rbrace|=p_{ij}^k.$$
\end{itemize}

If we write $C(a,b;i,j)$ for the set $\lbrace c \in \Omega \mid (a,c)\in C_i \mbox{ and } (c,b)\in C_j \rbrace$, then the second condition says that for any $(a,b)\in C_k$, the value $|C(a,b;i,j)|$ is a constant, depending only on $i,j$ and $k$, and not on the particular elements of $C_k$.  This implies that, for any $i,j\in \lbrace 0,1,\ldots, n\rbrace$, the relation $C_i\circ C_j$ is a union of (some of) the $C_k$'s.

The definition above can be restated in graph theoretic terminology.  First, think of $\Omega$ is a set of vertices, and two-element subsets of $\Omega$ are edges.  The complete graph on $\Omega$ is just the set of all two-element subsets of $\Omega$.  We may color the edges of the graph.  Say there are colors labeled $1$ through $s$.  For each color $i$, let $C_i$ be the set of edges with color $i$.  Then each $C_i$ is just a symmetric relation on $\Omega$, and that all the $C_i$'s, together with the diagonal relation, partition the set $\Omega\times\Omega$.  This is basically the first condition of the definition above.  In this regard, we can redefine an association scheme graph theoretically, as follows:

\textbf{Definition 2}.  An \emph{association scheme} $\mathscr{Q}$ is a surjective coloring on the edge set of a complete graph whose vertex set is $\Omega$, by a set of $s$ colors (numbered $1$ through $s$), such that
\begin{quote}
for any $i,j,k\in \lbrace 1,\ldots,s\rbrace$, there is an integer $p_{ij}^k$ such that if $\mathscr{Q}(a,b)=k$ (the edge $\lbrace a,b\rbrace$ has color $k$), then 
$$|\lbrace c \in \Omega \mid \mathscr{Q}(a,c)=i \mbox{ and } \mathscr{Q}(c,b)=j \rbrace|=p_{ij}^k.$$
\end{quote}

In words, the definition says that, for any color $k$, and any given edge $e$ with color $k$, the number of triangles (a triangle in a graph is a cycle consisting of three edges) with $e$ as an edge, and two other edges with colors $i,j$ respectively, is $p_{ij}^k$.

The first definition can also be viewed in terms of matrices, and adjacency matrices more specifically.  Given a finite set $\Omega$, a binary relation $R$ on $\Omega$ naturally corresponds to matrix $A$ called the adjacency matrix of $R$.  Entry $(i,j)$ is $1$ if the $i$-th element and the $j$-th element are related by $R$, and $0$ otherwise.  If $R$ is reflexive, then $A$ has all $1$'s in its diagonal, and if $R$ is symmetric, then so is $A$.  Also, it is easy to see that the composition of two binary relations is the same as the product of their corresponding adjacency matrices.  Then the comment in the paragraph after the first definition is the same as saying that the adjacency matrix of $C_i\circ C_j$ is a linear combination of the adjacency matrices of $C_0,C_1,\ldots, C_s$.  This gives us the third definition below:

\textbf{Definition 3}.  An \emph{association scheme} is a finite set $\mathscr{Q}$ of $n\times n$ non-zero matrices $A_0, A_1,\ldots,A_s$ whose entries are $0$'s and $1$'s, such that
\begin{itemize}
\item each $A_i$ is a symmetric matrix, with $A_0=I_n$, the identity matrix,
\item $A_0+A_1+\cdots + A_s = J_n$, the matrix whose entries are all $1$'s, and
\item for any $i,j\in \lbrace 0,\ldots, s\rbrace$, $A_iA_j$ is a linear combination of $A_0,A_1,\ldots, A_s$.
\end{itemize}

By the definitions of the matrices $A_i$ and the second condition, for every pair $(r,s)$, exactly one of the $s+1$ matrices has $1$ in cell $(r,s)$, and all others have $0$ in the corresponding cell.  As a result, the $s+1$ matrices are linearly independent.  

Also, in view of the discussion above, it is easy to see that 
$$A_iA_j=p_{ij}^0 A_0 + p_{ij}^1 A_1 + \cdots + p_{ij}^s A_s.$$

\textbf{Some terminology}.  $s$ is called the \emph{rank} of the association scheme $\mathscr{Q}$.  Any $a\in \Omega$, an element $c\in \mathscr{Q}$ is said to be an \emph{$i$-associate} of $a$ if $(a,c)\in C_i$.  For each $a\in \Omega$, define: $$C_i(a):=\lbrace c\in \Omega\mid (a,c)\in C_i\rbrace.$$ So $C_i(a)$ is the set of all $i$-associates of $a$.  Then 
$$|C_i(a)\cap C_j(b)|=p_{ij}^k,\mbox{ whenever }(a,b)\in C_k,$$
and because of the above equation, each $p_{ij}^k$ is called an \emph{intersection number} of $\mathscr{Q}$.  For each $i$, the intersection number $p_{ii}^0$ is called the \emph{valency} of $C_i$, denoted by $a_i$.

Some basic properties of the intersection numbers:
\begin{itemize}
\item $p_{ij}^k=p_{ji}^k$
\item 
\begin{displaymath}
p_{i0}^k = \left\{
\begin{array}{ll}
1 & \textrm{if }i=k\\
0 & \textrm{otherwise.}
\end{array}
\right.
\end{displaymath}
\item $|C_i(c)|=p_{ii}^0=a_i$ for all $c\in \Omega$.
\end{itemize}

\begin{thebibliography}{7}
\bibitem{RB} R. A. Bailey, {\it Association Schemes, Designed Experiments, Algebra and Combinatorics}, Cambridge University Press (2004)
\end{thebibliography}
%%%%%
%%%%%
\end{document}
