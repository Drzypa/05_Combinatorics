\documentclass[12pt]{article}
\usepackage{pmmeta}
\pmcanonicalname{Tournament}
\pmcreated{2013-03-22 13:05:49}
\pmmodified{2013-03-22 13:05:49}
\pmowner{yark}{2760}
\pmmodifier{yark}{2760}
\pmtitle{tournament}
\pmrecord{7}{33518}
\pmprivacy{1}
\pmauthor{yark}{2760}
\pmtype{Definition}
\pmcomment{trigger rebuild}
\pmclassification{msc}{05C20}
\pmsynonym{directed complete graph}{Tournament}
%\pmkeywords{complete graph}
\pmrelated{CompleteGraph}
\pmrelated{GraphTheory}
\pmdefines{paradoxical tournament}
\pmdefines{transitive tournament}

\usepackage[all]{xypic}
\begin{document}
A \emph{tournament} is a directed graph obtained by choosing a
direction for each edge in an undirected complete graph.
For example, here is a tournament on 4 vertices:
\[ \xymatrix{
        1 \ar[dr] \ar[r] &2 \ar[d]\\
        3 \ar[u] \ar[ur] &4 \ar[l]}
\]
Any tournament on a finite number $n$ of vertices contains a Hamiltonian path, i.e., directed path on all $n$ vertices. This is easily shown by induction on $n$:
suppose that the statement holds for $n$, and consider any
tournament $T$ on $n+1$ vertices. Choose a vertex $v_0$ of $T$ and
consider a directed path $v_1,v_2,\ldots,v_n$ in $T\setminus \{v_0\}$.
Now let $i \in \{0,\ldots,n\}$ be maximal such that $v_j \rightarrow
v_0$ for all $j$ with $1 \leq j \leq i$. Then
\[ v_1,\ldots,v_i,v_0,v_{i+1},\ldots,v_n \]
is a directed path as desired.

The name ``tournament'' originates from such a graph's interpretation as
the outcome of some sports competition in which every player encounters
every other player exactly once, and in which no draws occur; let us say
that an arrow points from the winner to the loser. A player who wins
all games would naturally be the tournament's winner. However, as the
above example shows, there might not be such a player; a tournament for
which there isn't is called a {\em $1$-paradoxical} tournament. More
generally, a tournament $T=(V,E)$ is called {\em $k$-paradoxical} if
for every $k$-subset $V'$ of $V$ there is a $v_0 \in V \setminus V'$
such that $v_0 \rightarrow v$ for all $v \in V'$. By means of the
probabilistic method Erd\H{o}s showed that if $|V|$ is sufficiently
large, then almost every tournament on $V$ is $k$-paradoxical.

A \emph{transitive tournament} is a tournament in which, for all vertices $v_0$, $v_1$ and $v_2$, if there is an edge from $v_0$ to $v_1$ and an edge from $v_1$ to $v_2$ then there is also an edge from $v_0$ to $v_2$.
%%%%%
%%%%%
\end{document}
