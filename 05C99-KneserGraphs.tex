\documentclass[12pt]{article}
\usepackage{pmmeta}
\pmcanonicalname{KneserGraphs}
\pmcreated{2013-03-22 14:16:49}
\pmmodified{2013-03-22 14:16:49}
\pmowner{justice}{4961}
\pmmodifier{justice}{4961}
\pmtitle{Kneser graphs}
\pmrecord{5}{35732}
\pmprivacy{1}
\pmauthor{justice}{4961}
\pmtype{Definition}
\pmcomment{trigger rebuild}
\pmclassification{msc}{05C99}
%\pmkeywords{Kneser graph}
%\pmkeywords{Petersen graph}
\pmdefines{Petersen graph}

% this is the default PlanetMath preamble.  as your knowledge
% of TeX increases, you will probably want to edit this, but
% it should be fine as is for beginners.

% almost certainly you want these
\usepackage{amssymb}
\usepackage{amsmath}
\usepackage{amsfonts}

% used for TeXing text within eps files
%\usepackage{psfrag}
% need this for including graphics (\includegraphics)
\usepackage{graphicx}
% for neatly defining theorems and propositions
%\usepackage{amsthm}
% making logically defined graphics
%%%\usepackage{xypic}

% there are many more packages, add them here as you need them

% define commands here
\begin{document}
Let $k,n$ be positive integers with $k\leq n$. The \emph{Kneser graph} $K_{n:k}$ has as its vertex set the $n\choose k$ $k$-element subsets of $\{1,2,\dots,n\}$. Two vertices are adjacent if and only if they correspond to disjoint subsets.

The graph $K_{n:1}$ is the complete graph on $n$ vertices. Another well-known Kneser graph is $K_{5:2}$, better known as the \emph{Petersen graph}, which is shown in figure \ref{petersen}. The Petersen graph is often a counterexample to graph-theoretical conjectures.

\begin{figure}
\centering
\scalebox{.5}{\includegraphics{PetersenGraph}}
\caption{\label{petersen}The Petersen graph}
\end{figure}
%%%%%
%%%%%
\end{document}
