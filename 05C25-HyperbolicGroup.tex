\documentclass[12pt]{article}
\usepackage{pmmeta}
\pmcanonicalname{HyperbolicGroup}
\pmcreated{2013-03-22 17:11:43}
\pmmodified{2013-03-22 17:11:43}
\pmowner{Wkbj79}{1863}
\pmmodifier{Wkbj79}{1863}
\pmtitle{hyperbolic group}
\pmrecord{6}{39514}
\pmprivacy{1}
\pmauthor{Wkbj79}{1863}
\pmtype{Definition}
\pmcomment{trigger rebuild}
\pmclassification{msc}{05C25}
\pmclassification{msc}{20F06}
\pmclassification{msc}{54E35}
\pmsynonym{hyperbolicity}{HyperbolicGroup}
\pmrelated{RealTree}

\usepackage{amssymb}
\usepackage{amsmath}
\usepackage{amsfonts}
\usepackage{pstricks}
\usepackage{psfrag}
\usepackage{graphicx}
\usepackage{amsthm}
%%\usepackage{xypic}

\begin{document}
A finitely generated group $G$ is \emph{hyperbolic} if, for some finite set of generators $A$ of $G$, the Cayley graph $\Gamma(G,A)$, considered as a metric space with $d(x,y)$ being the minimum number of edges one must traverse to get from $x$ to $y$, is a hyperbolic metric space.

Hyperbolicity is a group-theoretic property.  That is, if $A$ and $B$ are finite sets of generators of a group $G$ and $\Gamma(G,A)$ is a hyperbolic metric space, then $\Gamma(G,B)$ is a hyperbolic metric space.

\PMlinkescapetext{Simple} examples of hyperbolic groups include finite groups and free groups.  If $G$ is a finite group, then for any $x,y \in G$, we have that $d(x,y) \le |G|$.  (See the entry \PMlinkname{Cayley graph of $S_3$}{CayleyGraphOfS_3} for a pictorial example.)  If $G$ is a free group, then its Cayley graph is a real tree.
%%%%%
%%%%%
\end{document}
