\documentclass[12pt]{article}
\usepackage{pmmeta}
\pmcanonicalname{ExternalPathLength}
\pmcreated{2013-03-22 12:32:03}
\pmmodified{2013-03-22 12:32:03}
\pmowner{Logan}{6}
\pmmodifier{Logan}{6}
\pmtitle{external path length}
\pmrecord{4}{32776}
\pmprivacy{1}
\pmauthor{Logan}{6}
\pmtype{Definition}
\pmcomment{trigger rebuild}
\pmclassification{msc}{05C05}
\pmrelated{ExtendedBinaryTree}
\pmrelated{WeightedPathLength}
\pmdefines{external path length}
\pmdefines{internal path length}

\usepackage{amssymb}
\usepackage{amsmath}
\usepackage{amsfonts}
\usepackage{graphicx}
\begin{document}
Given a binary tree $T$, construct its extended binary tree $T'$.
The \emph{external path length} of $T$ is then defined to be the sum of the lengths of the paths to each of the external nodes.

For example, let $T$ be the following tree.

\begin{center}
\includegraphics{tree.3}
\end{center}

The extended binary tree of $T$ is

\begin{center}
\includegraphics{tree.4}
\end{center}

The external path length of $T$ (denoted $E$) is

$$
E = 2 + 3 + 3 + 3 + 3 + 3 + 3 = 20
$$

The \emph{internal path length} of $T$ is defined to be the sum of the lengths of the paths to each of the internal nodes.  The internal path length of our example tree (denoted $I$) is

$$
I = 1 + 2 + 0 + 2 + 1 + 2 = 8
$$

Note that in this case $E = I + 2n$, where $n$ is the number of internal nodes.  This happens to hold for all binary trees.
%%%%%
%%%%%
\end{document}
