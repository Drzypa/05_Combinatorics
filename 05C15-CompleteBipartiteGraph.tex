\documentclass[12pt]{article}
\usepackage{pmmeta}
\pmcanonicalname{CompleteBipartiteGraph}
\pmcreated{2013-03-22 12:17:13}
\pmmodified{2013-03-22 12:17:13}
\pmowner{yark}{2760}
\pmmodifier{yark}{2760}
\pmtitle{complete bipartite graph}
\pmrecord{7}{31784}
\pmprivacy{1}
\pmauthor{yark}{2760}
\pmtype{Definition}
\pmcomment{trigger rebuild}
\pmclassification{msc}{05C15}

\endmetadata

\usepackage{amssymb}
\usepackage{amsmath}
\usepackage{amsfonts}

%%\usepackage{xypic}
\begin{document}
The \emph{complete bipartite graph} $K_{n,m}$ is a graph with two sets of vertices, one with $n$ members and one with $m$, such that each vertex in one set is adjacent to every vertex in the other set and to no vertex in its own set.  As the name implies, $K_{n,m}$ is bipartite.

Examples of complete bipartite graphs:

$K_{2,5}$:

$$\xymatrix{
  & C \\
A \ar@{-}[ur] \ar@{-}[r] \ar@{-}[dr] \ar@{-}[ddr] \ar@{-}[dddr] & D \\
  & E \\
B \ar@{-}[uuur] \ar@{-}[uur] \ar@{-}[ur] \ar@{-}[r] \ar@{-}[dr] & F \\
  & G
}$$

$K_{3,3}$:

$$\xymatrix{
A \ar@{-}[r] \ar@{-}[dr] \ar@{-}[ddr] & D \\
B \ar@{-}[ur] \ar@{-}[r] \ar@{-}[dr] & E \\
C \ar@{-}[uur] \ar@{-}[ur] \ar@{-}[r] & F 
}$$
%%%%%
%%%%%
\end{document}
