\documentclass[12pt]{article}
\usepackage{pmmeta}
\pmcanonicalname{BalancedTree}
\pmcreated{2013-03-22 12:29:11}
\pmmodified{2013-03-22 12:29:11}
\pmowner{Mathprof}{13753}
\pmmodifier{Mathprof}{13753}
\pmtitle{balanced tree}
\pmrecord{7}{32706}
\pmprivacy{1}
\pmauthor{Mathprof}{13753}
\pmtype{Data Structure}
\pmcomment{trigger rebuild}
\pmclassification{msc}{05C05}
\pmrelated{Tree}
\pmrelated{BalancedBinaryTree}
\pmrelated{Heap}
\pmrelated{BinaryTree}

\usepackage{amssymb}
\usepackage{amsmath}
\usepackage{amsfonts}
\begin{document}
A \emph{balanced tree } is a rooted tree where no leaf is much farther 
away from the root than any other leaf. 
Different balancing \PMlinkescapetext{schemes} allow different definitions of "much farther" and 
different amounts of work to keep them balanced. For an example, see binary tree.



%%%%%
%%%%%
\end{document}
