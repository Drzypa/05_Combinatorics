\documentclass[12pt]{article}
\usepackage{pmmeta}
\pmcanonicalname{Branch}
\pmcreated{2013-03-22 12:52:22}
\pmmodified{2013-03-22 12:52:22}
\pmowner{Henry}{455}
\pmmodifier{Henry}{455}
\pmtitle{branch}
\pmrecord{4}{33211}
\pmprivacy{1}
\pmauthor{Henry}{455}
\pmtype{Definition}
\pmcomment{trigger rebuild}
\pmclassification{msc}{05C05}
\pmclassification{msc}{03E05}
\pmrelated{TreeSetTheoretic}
\pmrelated{ExampleOfTreeSetTheoretic}
\pmdefines{branch}
\pmdefines{cofinal branch}

\endmetadata

% this is the default PlanetMath preamble.  as your knowledge
% of TeX increases, you will probably want to edit this, but
% it should be fine as is for beginners.

% almost certainly you want these
\usepackage{amssymb}
\usepackage{amsmath}
\usepackage{amsfonts}

% used for TeXing text within eps files
%\usepackage{psfrag}
% need this for including graphics (\includegraphics)
%\usepackage{graphicx}
% for neatly defining theorems and propositions
%\usepackage{amsthm}
% making logically defined graphics
%%%\usepackage{xypic}

% there are many more packages, add them here as you need them

% define commands here
%\PMlinkescapeword{theory}
\begin{document}
A subset $B$ of a tree $(T,<_T)$ is a \emph{branch} if $B$ is a maximal linearly ordered subset of $T$.  That is:
\begin{itemize}

\item $<_T$ is a linear ordering of $B$

\item If $t\in T\setminus B$ then $B\cup \{t\}$ is not linearly ordered by $<_T$.

\end{itemize}

This is the same as the intuitive conception of a branch: it is a set of nodes starting at the root and going all the way to the tip (in infinite sets the conception is more complicated, since there may not be a tip, but the idea is the same).  Since branches are maximal there is no way to add an element to a branch and have it remain a branch.

A \emph{cofinal branch} is a branch which intersects every level of the tree.
%%%%%
%%%%%
\end{document}
