\documentclass[12pt]{article}
\usepackage{pmmeta}
\pmcanonicalname{ExampleOfPlanarGraphWithTwoDifferentEmbeddingsIntoThePlane}
\pmcreated{2013-03-22 14:17:03}
\pmmodified{2013-03-22 14:17:03}
\pmowner{archibal}{4430}
\pmmodifier{archibal}{4430}
\pmtitle{example of planar graph with two different embeddings into the plane}
\pmrecord{5}{35738}
\pmprivacy{1}
\pmauthor{archibal}{4430}
\pmtype{Example}
\pmcomment{trigger rebuild}
\pmclassification{msc}{05C10}

% this is the default PlanetMath preamble.  as your knowledge
% of TeX increases, you will probably want to edit this, but
% it should be fine as is for beginners.

% almost certainly you want these
\usepackage{amssymb}
\usepackage{amsmath}
\usepackage{amsfonts}

% used for TeXing text within eps files
%\usepackage{psfrag}
% need this for including graphics (\includegraphics)
\usepackage{graphicx}
% for neatly defining theorems and propositions
%\usepackage{amsthm}
% making logically defined graphics
%%%\usepackage{xypic}

% there are many more packages, add them here as you need them

% define commands here

\newtheorem{theorem}{Theorem}
\newtheorem{defn}{Definition}
\newtheorem{prop}{Proposition}
\newtheorem{lemma}{Lemma}
\newtheorem{cor}{Corollary}
\begin{document}
Consider the following two plane graphs:
\begin{center}
\includegraphics{g1}

The plane graph $G_1$.
\end{center}
\begin{center}
\includegraphics{g2}

The plane graph $G_2$.
\end{center}
Clearly the multigraph associated with each of these plane graphs is the same, but the two plane graphs are not isomorphic: in $G_1$, every face is adjacent to the central region, while in $G_2$ there is no face adjacent to every other. This also implies that their dual graphs are non-isomorphic.
%%%%%
%%%%%
\end{document}
