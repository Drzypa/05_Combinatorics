\documentclass[12pt]{article}
\usepackage{pmmeta}
\pmcanonicalname{RuleOfProduct}
\pmcreated{2013-03-22 19:13:02}
\pmmodified{2013-03-22 19:13:02}
\pmowner{pahio}{2872}
\pmmodifier{pahio}{2872}
\pmtitle{rule of product}
\pmrecord{6}{42137}
\pmprivacy{1}
\pmauthor{pahio}{2872}
\pmtype{Definition}
\pmcomment{trigger rebuild}
\pmclassification{msc}{05A05}
\pmclassification{msc}{03-00}
\pmsynonym{multiplication principle}{RuleOfProduct}
\pmrelated{CartesianProduct}
\pmrelated{Combinatorics}
\pmrelated{Cardinality}
\pmrelated{Number}
\pmrelated{Product}

\endmetadata

% this is the default PlanetMath preamble.  as your knowledge
% of TeX increases, you will probably want to edit this, but
% it should be fine as is for beginners.

% almost certainly you want these
\usepackage{amssymb}
\usepackage{amsmath}
\usepackage{amsfonts}

% used for TeXing text within eps files
%\usepackage{psfrag}
% need this for including graphics (\includegraphics)
%\usepackage{graphicx}
% for neatly defining theorems and propositions
 \usepackage{amsthm}
% making logically defined graphics
%%%\usepackage{xypic}

% there are many more packages, add them here as you need them

% define commands here

\theoremstyle{definition}
\newtheorem*{thmplain}{Theorem}

\begin{document}
\PMlinkescapeword{combination}

If a process $A$ can have altogether $m$ different results and another process $B$ altogether $n$ different results, then the two processes can have altogether $mn$ different combined results.\, Putting it to set-theoretical form,
$$\mbox{card}(A\!\times\!B) \;=\; m\!\cdot\!n.$$

The \emph{rule of product} is true also for the combination of several processes:\, If the processes $A_i$ can have $n_i$ possible results ($i = 1,\,2,\,\ldots,\,k$), then their combined process has $n_1n_2\!\cdots\!n_k$ possible results.\, I.e.,
$$\mbox{card}(A_1\!\times\!A_2\!\times\ldots\times\!A_k) \;=\; n_1n_2\!\cdots\!n_k.$$\\


\textbf{Example.}\, Arranging $n$ elements, the first one may be chosen freely from all the $n$ elements, the second from the remaining $n\!-\!1$ elements, the third from the remaining $n\!-\!2$, and so on, the penultimate one from two elements and the last one from the only remaining element; thus by the rule of product, there are in all 
$$n(n\!-\!1)(n\!-\!2)\!\cdots\!2\!\cdot\!1 \;=\; n!$$ 
different arrangements, i.e. permutations, as the result.
%%%%%
%%%%%
\end{document}
