\documentclass[12pt]{article}
\usepackage{pmmeta}
\pmcanonicalname{MinimumSpanningTree}
\pmcreated{2013-03-22 12:29:22}
\pmmodified{2013-03-22 12:29:22}
\pmowner{mathcam}{2727}
\pmmodifier{mathcam}{2727}
\pmtitle{minimum spanning tree}
\pmrecord{7}{32710}
\pmprivacy{1}
\pmauthor{mathcam}{2727}
\pmtype{Definition}
\pmcomment{trigger rebuild}
\pmclassification{msc}{05C05}
\pmsynonym{smallest spanning tree}{MinimumSpanningTree}
\pmrelated{SpanningTree}

\endmetadata

\usepackage{amssymb}
\usepackage{amsmath}
\usepackage{amsfonts}
\usepackage[all]{xy}
\begin{document}
Given a graph $G$ with weighted edges, a \emph{minimum spanning tree} is a spanning tree with minimum weight, where the weight of a spanning tree is the sum of the weights of its edges.  There may be more than one minimum spanning tree for a graph, since it is the weight of the spanning tree that must be minimum.

For example, here is a graph $G$ of weighted edges and a minimum spanning tree $T$ for that graph.  The edges of $T$ are drawn as solid lines, while edges in $G$ but not in $T$ are drawn as dotted lines.

$$
\xymatrix{
&&\bullet\ar@{-}[dll]|3\ar@{-}[dd]|4\ar@{.}[drr]|7 \\
\bullet\ar@{.}[dd]|8\ar@{.}[drr]|4&&&&\bullet\ar@{-}[dll]|2\ar@{-}[dd]|5 \\
&&\bullet\ar@{.}[dll]|5\ar@{-}[dd]|3\ar@{.}[drr]|7 \\
\bullet\ar@{-}[drr]|2&&&&\bullet\ar@{.}[dll]|6 \\
&&\bullet
}
$$

Prim's algorithm or Kruskal's algorithm can compute the minimum spanning tree of a graph.
%%%%%
%%%%%
\end{document}
