\documentclass[12pt]{article}
\usepackage{pmmeta}
\pmcanonicalname{ProofOfAlgebraicIndependenceOfElementarySymmetricPolynomials}
\pmcreated{2013-03-22 17:38:28}
\pmmodified{2013-03-22 17:38:28}
\pmowner{lalberti}{18937}
\pmmodifier{lalberti}{18937}
\pmtitle{proof of algebraic independence of elementary symmetric polynomials}
\pmrecord{4}{40063}
\pmprivacy{1}
\pmauthor{lalberti}{18937}
\pmtype{Proof}
\pmcomment{trigger rebuild}
\pmclassification{msc}{05E05}

\endmetadata

% this is the default PlanetMath preamble.  as your knowledge
% of TeX increases, you will probably want to edit this, but
% it should be fine as is for beginners.

% almost certainly you want these
\usepackage{amssymb}
\usepackage{amsmath}
\usepackage{amsfonts}

% used for TeXing text within eps files
%\usepackage{psfrag}
% need this for including graphics (\includegraphics)
%\usepackage{graphicx}
% for neatly defining theorems and propositions
%\usepackage{amsthm}
% making logically defined graphics
%%%\usepackage{xypic}

% there are many more packages, add them here as you need them

% define commands here
\begin{document}
Geometric proof, works when R is a division ring.\\

Consider the quotient field Q of R and then the algebraic closure $K$ of $Q$.

Consider the substitution map that associates to values $t_1,\ldots,t_n \in K^n$ the symmetric functions in these variables $s_1,\ldots,s_n$.
$$\begin{array}{lccr}
\phi: & K^n & \to & K^n\\
      &  (t_i) & \mapsto & (s_i)
\end{array}
$$
Because $K$ is algebraic closed this map is surjective. Indeed, fix values $v_i$, then on an algebraic closed field there are roots $t_i$ such that 
$$X^n + \sum_i v_i X^i = \Pi_i (X+t_i)$$
And by developing the right-hand side we get $v_i = s_i$.

Then we consider the transposition morphism of algebras $\phi^*$ :
$$\begin{array}{lccr}
\phi^*: & R[S_1,\ldots,S_n] & \to & R[T_1,\ldots,T_n]\\
      & f & \mapsto & f\circ\phi
\end{array}
$$
The capital letters are there to emphasize the $S_i$ and $T_i$ are variables and $R[S_1,\ldots,S_n]$ and $R[T_1,\ldots,T_n]$ are regarded as function algebras over $K^n$.

The theorem stating that the symmetric functions are algebraically independent is no more than saying that this morphism is injective.
As a matter of fact, $\phi^*(S_i)$ is the $i^{th}$ symmetric function in the $T_i$, and $\phi^*$ is clearly a morphism of algebras.\\
The conclusion is then straightforward from the surjectivity of $\phi$ because if $f\circ\phi = 0$ for some $f$, then by surjectivity of $\phi$ it means that $f$ was zero in the first place. In other words the kernel of $\phi^*$ is reduced to 0.
%%%%%
%%%%%
\end{document}
