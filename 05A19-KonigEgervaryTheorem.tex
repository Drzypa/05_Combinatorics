\documentclass[12pt]{article}
\usepackage{pmmeta}
\pmcanonicalname{KonigEgervaryTheorem}
\pmcreated{2013-03-22 16:33:47}
\pmmodified{2013-03-22 16:33:47}
\pmowner{PrimeFan}{13766}
\pmmodifier{PrimeFan}{13766}
\pmtitle{K\"onig-Egervary theorem}
\pmrecord{4}{38752}
\pmprivacy{1}
\pmauthor{PrimeFan}{13766}
\pmtype{Definition}
\pmcomment{trigger rebuild}
\pmclassification{msc}{05A19}
\pmsynonym{Konig-Egervary theorem}{KonigEgervaryTheorem}

\endmetadata

% this is the default PlanetMath preamble.  as your knowledge
% of TeX increases, you will probably want to edit this, but
% it should be fine as is for beginners.

% almost certainly you want these
\usepackage{amssymb}
\usepackage{amsmath}
\usepackage{amsfonts}

% used for TeXing text within eps files
%\usepackage{psfrag}
% need this for including graphics (\includegraphics)
%\usepackage{graphicx}
% for neatly defining theorems and propositions
%\usepackage{amsthm}
% making logically defined graphics
%%%\usepackage{xypic}

% there are many more packages, add them here as you need them

% define commands here

\begin{document}
The {\em K\"onig-Egervary theorem} states that in a finite matrix of 0's and 1's, the maximum numbers of 1's such that no two are in a line, equals the minimum number of lines which collectively contain all the 1's. Here line means row or column.

Take this matrix, for example,

$$\begin{bmatrix}
1 & 0 & 0 & 0 & 0 & 1 \\
1 & 0 & 0 & 0 & 0 & 1 \\
0 & 1 & 0 & 0 & 0 & 1 \\
0 & 1 & 0 & 0 & 0 & 1 \\
1 & 0 & 1 & 1 & 1 & 0 \\
\end{bmatrix}$$

Here the max and min numbers (always equal) are 4.

\begin{thebibliography}{1}
\bibitem{ac} A. Chandra Babu, P. V. Ramakrishnan, ``New Proofs of Konig-Egervary Theorem And Maximal Flow-Minimal Cut Capacity Theorem Using O. R. Techniques'' {\it Indian J. Pure 
Appl. Math.} {\bf 22}(11) (1991): 905 - 911
\end{thebibliography}
%%%%%
%%%%%
\end{document}
