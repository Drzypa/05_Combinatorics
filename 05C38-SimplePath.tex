\documentclass[12pt]{article}
\usepackage{pmmeta}
\pmcanonicalname{SimplePath}
\pmcreated{2013-03-22 12:30:42}
\pmmodified{2013-03-22 12:30:42}
\pmowner{mps}{409}
\pmmodifier{mps}{409}
\pmtitle{simple path}
\pmrecord{6}{32747}
\pmprivacy{1}
\pmauthor{mps}{409}
\pmtype{Definition}
\pmcomment{trigger rebuild}
\pmclassification{msc}{05C38}
\pmrelated{Path}
\pmrelated{Cycle}
\pmrelated{Graph}
\pmrelated{PathConnected}

\endmetadata

% this is the default PlanetMath preamble.  as your knowledge
% of TeX increases, you will probably want to edit this, but
% it should be fine as is for beginners.

% almost certainly you want these
\usepackage{amssymb}
\usepackage{amsmath}
\usepackage{amsfonts}

% used for TeXing text within eps files
%\usepackage{psfrag}
% need this for including graphics (\includegraphics)
%\usepackage{graphicx}
% for neatly defining theorems and propositions
%\usepackage{amsthm}
% making logically defined graphics
%%%\usepackage{xypic}

% there are many more packages, add them here as you need them

% define commands here
\begin{document}
A \emph{simple path} in a graph is a path $P=v_0 e_0 v_1\dots e_{n-1} v_n$
such that no vertex occurs twice in $P$. 
Some authors relax this condition by permitting $v_0=v_n$.  In this case the path is usually called a cycle.
%A \emph{simple path} in a graph is a path $v_0 e_0 v_1 e_1\dots e_{n-1} v_n$
%such that no vertex is repeated, except that possibly $v_0=v_1$.  In the latter %case, the simple path is called a cycle.
%A \emph{simple path} in a graph is a path that contains no vertex more than once.  
%By definition, cycles are particular instances of simple paths.
%%%%%
%%%%%
\end{document}
