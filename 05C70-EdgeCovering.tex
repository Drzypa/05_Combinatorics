\documentclass[12pt]{article}
\usepackage{pmmeta}
\pmcanonicalname{EdgeCovering}
\pmcreated{2013-03-22 12:40:02}
\pmmodified{2013-03-22 12:40:02}
\pmowner{vampyr}{22}
\pmmodifier{vampyr}{22}
\pmtitle{edge covering}
\pmrecord{8}{32940}
\pmprivacy{1}
\pmauthor{vampyr}{22}
\pmtype{Definition}
\pmcomment{trigger rebuild}
\pmclassification{msc}{05C70}
\pmrelated{Matching}
\pmdefines{minimal edge covering}

\endmetadata

% this is the default PlanetMath preamble.  as your knowledge
% of TeX increases, you will probably want to edit this, but
% it should be fine as is for beginners.

% almost certainly you want these
\usepackage{amssymb}
\usepackage{amsmath}
\usepackage{amsfonts}

% used for TeXing text within eps files
%\usepackage{psfrag}
% need this for including graphics (\includegraphics)
%\usepackage{graphicx}
% for neatly defining theorems and propositions
%\usepackage{amsthm}
% making logically defined graphics
%%%\usepackage{xypic} 

% there are many more packages, add them here as you need them

% define commands here
\begin{document}
Let $G$ be a graph.  An \emph{edge covering} $C$ on $G$ is a subset of the vertices of $G$ such that each edge in $G$ is incident with at least one vertex in $C$.

For any graph, the \PMlinkescapeword{entire} vertex set is a trivial edge covering.  Generally, we are more interested in \emph{minimal coverings}.  A minimal edge covering is simply an edge covering of the least possible \PMlinkescapeword{size}.
%%%%%
%%%%%
\end{document}
