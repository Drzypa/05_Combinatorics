\documentclass[12pt]{article}
\usepackage{pmmeta}
\pmcanonicalname{RootedTree}
\pmcreated{2013-03-22 19:32:02}
\pmmodified{2013-03-22 19:32:02}
\pmowner{CWoo}{3771}
\pmmodifier{CWoo}{3771}
\pmtitle{rooted tree}
\pmrecord{38}{42511}
\pmprivacy{1}
\pmauthor{CWoo}{3771}
\pmtype{Definition}
\pmcomment{trigger rebuild}
\pmclassification{msc}{05C05}
\pmdefines{labeled rooted tree}

\endmetadata

\usepackage{amssymb,amscd}
\usepackage{amsmath}
\usepackage{amsfonts}
\usepackage{mathrsfs}

% used for TeXing text within eps files
%\usepackage{psfrag}
% need this for including graphics (\includegraphics)
%\usepackage{graphicx}
% for neatly defining theorems and propositions
\usepackage{amsthm}
% making logically defined graphics
%%\usepackage{xypic}
\usepackage{pst-plot}
\usepackage{pstricks,pst-tree}

% define commands here
\newcommand*{\abs}[1]{\left\lvert #1\right\rvert}
\newtheorem{prop}{Proposition}
\newtheorem{thm}{Theorem}
\newtheorem{ex}{Example}
\newcommand{\real}{\mathbb{R}}
\newcommand{\pdiff}[2]{\frac{\partial #1}{\partial #2}}
\newcommand{\mpdiff}[3]{\frac{\partial^#1 #2}{\partial #3^#1}}

\begin{document}
A \emph{rooted tree} is a pair $(T,v)$, where $T$ is a tree and $v$ is a vertex of $T$ called the \emph{root} of $T$.  For example, the following diagram represents a rooted tree with root $k$
$$
\pstree[treemode=U,radius=3pt,levelsep=7ex]
{\Tc{3pt}~[tnpos=b,tndepth=0pt,radius=4pt]{$k$}}
{
\pstree{\TC~[tnpos=l]{$i$}}
{
\pstree{\TC~[tnpos=l]{$e$}}
{\TC~[tnpos=r]{$a$}}
\TC~[tnpos=r]{$f$}}
\pstree{\TC~[tnpos=r]{$j$}}
{\TC~[tnpos=l]{$g$}
\pstree{\TC~[tnpos=r]{$h$}}
{\TC~[tnpos=l]{$b$}\TC~[tnpos=r]{$c$}\TC~[tnpos=r]{$d$}}
}
}
$$

Any rooted tree $T$ has a natural partial ordering defined on its vertex set $V(T)$.  Let $rt(T)$ be the root of $T$.  For any pair of vertices $x,y$ of $T$, we define $x\le y$ if $x$ is on the path from $rt(T)$ to $y$.  It is easy to see that $\le$ is a partial order on $V(T)$, and $rt(T)$ is the unique minimum element.

Given a vertex $v$, a \emph{parent} of $v$ is a vertex $w$ that is covered by $v$.  In other words, $w< v$ and if $w\le w'< v$, then $w'= w$.  Since the path connecting $rt(T)$ to $v$ is unique, the parent of a vertex is always unique.  In the example above, vertex $i$ is the parent of vertices $e$ and $f$.  The root $rt(T)$ has no parent.

A \emph{child} of a vertex $v$ is a vertex $u$ that covers $v$.  In other words, $v< u$, and if $v< u' \le u$, then $u'=u$.  Again, from the tree above, vertices $b,c$, and $d$ are children of vertex $h$.  A vertex that is maximal with respect to $\le$ is called a \emph{leaf}.  A leaf has no children.  Vertices $a,b,c,d$ above are leaves.  In addition, $u$ is the parent of $v$ iff $v$ is a child of $u$.  It is possible for a vertex to have neither children nor a parent (consider the linear ordering on the non-negative reals).

It is easy to see that every path on a rooted tree is a linear ordering, or a chain.  A \emph{branch} of a rooted tree is a maximal chain.  For example, the path $a,e,i,k$ is a branch.  Every branch of $T$ has at least one end-point: $rt(T)$.  If it has another end point, that point must be a leaf.  Every vertex lies on a unique branch.  If each path of a rooted tree $T$ is in addition well-ordered, then $T$ is tree in the sense of set theory.

A subtree $T'$ of a rooted tree $T$ need not have a root, and even if it is rooted, its root may not be the same as the root of $T$.

Given a rooted tree $T$, if take the disjoint union of $V(T)$ and $\lbrace x\rbrace$, and add $\lbrace rt(T),x\rbrace$ to the edge set of $T$, then we obtain another rooted tree $T'$ whose root is $x$.  This construction can be generalized: if $T_i$ are rooted trees indexed by $i\in I$, then we can take the disjoint union of their vertex sets $V(T_i)$, together with a singleton $\lbrace x\rbrace$ (disjoint union here as well).  Next, add the set $\lbrace \lbrace rt(T_i),x\rbrace \mid i\in I\rbrace$ to the disjoing union of the edge sets $E(T_i)$.  What we have is a rooted tree with root $x$.  This rooted tree is denoted by $\bigoplus T_i$.

A \emph{labeled rooted tree} is a rooted tree where there is a function $f: V(T)\to L$.  In other words, each vertex $v$ has a label $f(v) \in L$.

%%%%%
%%%%%
\end{document}
