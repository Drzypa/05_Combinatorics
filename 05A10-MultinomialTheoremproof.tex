\documentclass[12pt]{article}
\usepackage{pmmeta}
\pmcanonicalname{MultinomialTheoremproof}
\pmcreated{2013-03-22 13:41:55}
\pmmodified{2013-03-22 13:41:55}
\pmowner{Koro}{127}
\pmmodifier{Koro}{127}
\pmtitle{multinomial theorem (proof)}
\pmrecord{4}{34374}
\pmprivacy{1}
\pmauthor{Koro}{127}
\pmtype{Proof}
\pmcomment{trigger rebuild}
\pmclassification{msc}{05A10}

% this is the default PlanetMath preamble.  as your knowledge
% of TeX increases, you will probably want to edit this, but
% it should be fine as is for beginners.

% almost certainly you want these
\usepackage{amssymb}
\usepackage{amsmath}
\usepackage{amsfonts}

% used for TeXing text within eps files
%\usepackage{psfrag}
% need this for including graphics (\includegraphics)
%\usepackage{graphicx}
% for neatly defining theorems and propositions
%\usepackage{amsthm}
% making logically defined graphics
%%%\usepackage{xypic}

% there are many more packages, add them here as you need them

% define commands here

\newcommand{\sR}[0]{\mathbb{R}}
\newcommand{\sC}[0]{\mathbb{C}}
\newcommand{\sN}[0]{\mathbb{N}}
\newcommand{\sZ}[0]{\mathbb{Z}}
\begin{document}
\emph{Proof.} The below proof of the multinomial theorem uses 
the binomial theorem and  induction on $k$. 
In addition, we shall use multi-index notation.

First, 
for $k=1$, both sides equal $x_1^n$. For the induction step,
suppose the multinomial theorem holds for $k$. 
Then the binomial theorem and the induction assumption yield
\begin{eqnarray*}
(x_1+\cdots + x_k\,+\,x_{k+1})^n &=& \sum_{l=0}^n {n \choose l} (x_1+\cdots + x_k)^l x_{k+1}^{n-l}\\
&=& \sum_{l=0}^n {n \choose l} l! \sum_{|i|=l} \frac{x^i}{i!} x_{k+1}^{n-l}\\
&=& n! \sum_{l=0}^n \sum_{|i|=l} \frac{x^i x_{k+1}^{n-l}}{i! (n-l)!} \\
\end{eqnarray*}
where $x=(x_1,\ldots, x_k)$ and $i$ is a multi-index in $I^k_+$. To complete the proof, we need to show that
the sets 
\begin{eqnarray*}
A&=&\{ (i_1,\ldots,i_k, n-l)\in I^{k+1}_+ \mid l=0,\ldots, n,\, |(i_1,\ldots, i_k)|=l \}, \\
B&=&\{j \in I^{k+1}_+ \mid |j|=n \} 
\end{eqnarray*}
are equal.
The inclusion $A \subset B$ is clear since 
$$ |(i_1,\ldots,i_k, n-l)| = l + n-l = n.$$
For $B \subset A$, suppose $j=(j_1,\ldots, j_{k+1}) \in I^{k+1}_+$,
and $|j|=n$. Let $l=|(j_1,\ldots, j_k)|$. Then $l=n-j_{k+1}$, 
so $j_{k+1} = n-l$ for some $l=0,\ldots, n$. 
It follows that that $A=B$.

Let us define $y=(x_1,\cdots, x_{k+1})$ and let 
$j=(j_1,\ldots, j_{k+1})$ be a multi-index in $I_+^{k+1}$. 
Then 
\begin{eqnarray*}
(x_1+\cdots + x_{k+1})^n &=& n! \sum_{|j|=n} \frac{x^{(j_1,\ldots, j_k)} x_{k+1}^{j_{k+1}}}{(j_1,\ldots, j_k)! j_{k+1}!} \\
&=& n! \sum_{|j|=n} \frac{y^j}{j!}.
\end{eqnarray*}
This completes the proof. $\Box$
%%%%%
%%%%%
\end{document}
