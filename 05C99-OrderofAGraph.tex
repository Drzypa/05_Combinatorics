\documentclass[12pt]{article}
\usepackage{pmmeta}
\pmcanonicalname{OrderofAGraph}
\pmcreated{2013-03-22 12:31:23}
\pmmodified{2013-03-22 12:31:23}
\pmowner{Mathprof}{13753}
\pmmodifier{Mathprof}{13753}
\pmtitle{order (of a graph)}
\pmrecord{8}{32762}
\pmprivacy{1}
\pmauthor{Mathprof}{13753}
\pmtype{Definition}
\pmcomment{trigger rebuild}
\pmclassification{msc}{05C99}
\pmsynonym{order}{OrderofAGraph}
\pmrelated{Graph}
\pmrelated{SizeOfAGraph}
\pmrelated{MantelsTheorem}

\endmetadata

% this is the default PlanetMath preamble.  as your knowledge
% of TeX increases, you will probably want to edit this, but
% it should be fine as is for beginners.

% almost certainly you want these
\usepackage{amssymb}
\usepackage{amsmath}
\usepackage{amsfonts}

% used for TeXing text within eps files
%\usepackage{psfrag}
% need this for including graphics (\includegraphics)
%\usepackage{graphicx}
% for neatly defining theorems and propositions
%\usepackage{amsthm}
% making logically defined graphics
%%%\usepackage{xypic} 

% there are many more packages, add them here as you need them

% define commands here
\begin{document}
The \emph{order} of a graph $G$ is the number of vertices in $G$; it is denoted by $|G|$. The same notation is used for the number of elements (cardinality) of a set. Thus, $|G| = |V(G)|$. We write $G^n$ for an \emph{arbitrary graph of order n}. Similarly, $G(n,m)$ denotes an \emph{arbitrary graph of order n and size m}.


\footnotesize{Adapted with permission of the author from \emph{\PMlinkescapetext{Modern Graph Theory}} by B\'{e}la Bollob\'{a}s, published by Springer-Verlag New York, Inc., 1998.}
%%%%%
%%%%%
\end{document}
