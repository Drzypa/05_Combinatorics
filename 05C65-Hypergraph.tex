\documentclass[12pt]{article}
\usepackage{pmmeta}
\pmcanonicalname{Hypergraph}
\pmcreated{2013-03-22 13:05:29}
\pmmodified{2013-03-22 13:05:29}
\pmowner{CWoo}{3771}
\pmmodifier{CWoo}{3771}
\pmtitle{hypergraph}
\pmrecord{24}{33508}
\pmprivacy{1}
\pmauthor{CWoo}{3771}
\pmtype{Definition}
\pmcomment{trigger rebuild}
\pmclassification{msc}{05C65}
\pmsynonym{metacategory}{Hypergraph}
\pmsynonym{multi-graph}{Hypergraph}
\pmsynonym{colored graph}{Hypergraph}
%\pmkeywords{finite and infinite hypergraphs}
%\pmkeywords{edge sets}
%\pmkeywords{vertices}
%\pmkeywords{matrix representations of hypergraphs}
\pmrelated{SteinerSystem}
\pmrelated{IncidenceStructures}
\pmrelated{ToposAxioms}
\pmrelated{TopicEntryOnTheAlgebraicFoundationsOfMathematics}
\pmrelated{GraphTheory}
\pmrelated{ETAS}
\pmrelated{AxiomsOfMetacategoriesAndSupercategories}
\pmdefines{incidence matrix}
\pmdefines{uniform hypergraph}
\pmdefines{regular hypergraph}

\endmetadata

\usepackage{graphicx}
\usepackage{amsmath}
\usepackage{bbm}
\newcommand{\Z}{\mathbbmss{Z}}
\newcommand{\C}{\mathbbmss{C}}
\newcommand{\R}{\mathbbmss{R}}
\newcommand{\Q}{\mathbbmss{Q}}
\newcommand{\mathbb}[1]{\mathbbmss{#1}}
\newcommand{\figura}[1]{\begin{center}\includegraphics{#1}\end{center}}
\newcommand{\figuraex}[2]{\begin{center}\includegraphics[#2]{#1}\end{center}}

\begin{document}
A \emph{hypergraph} $\mathcal{H}$ is an ordered pair $(V, \mathcal{E})$ where $V$ is a set of \emph{vertices} and $\mathcal{E}$ is a set of edges such that $\mathcal{E} \subseteq \mathcal{P}(V)$.  In other words, an \emph{edge} is nothing more than a set of vertices. A hypergraph is the same thing as a simple incidence structure, but with terminology that focuses on the relationship with graphs.

Sometimes it is desirable to restrict this definition more.  The empty hypergraph is not very interesting, so we usually accept that $V \not= \emptyset$.  Singleton edges are allowed in general, but not the empty one.  Most applications consider only finite hypergraphs, but occasionally it is also useful to allow $V$ to be infinite.

Many of the definitions of graphs carry verbatim to hypergraphs.  

$\mathcal{H}$ is said to be $k$-\emph{uniform} if every edge $e \in \mathcal{E}$ has cardinality $k$, and is \emph{uniform} if it is $k$-uniform for some $k$.  An ordinary graph is merely a $2-$uniform hypergraph.

The \emph{degree} of a vertex $v$ is the number of edges in $\mathcal{E}$ that contain this vertex, often denoted $d(v)$.  $\mathcal{H}$ is $k$-\emph{regular} if every vertex has degree $k$, and is said to be \emph{regular} if it is $k$-regular for some $k$.

Let $V = \{v_1, v_2, ~\ldots, ~ v_n\}$ and $\mathcal{E} = \{e_1, e_2, ~ \ldots, ~ e_m\}$.  Associated to any hypergraph is the $n \times m$ \emph{incidence matrix} $A = (a_{ij})$ where
\[a_{ij} = 
\begin{cases}
1 &\text{ if } ~ v_i \in e_j \\
0 &\text{ otherwise }
\end{cases}\]
For example, let $\mathcal{H}=(V,\mathcal{E})$, where $V=\lbrace a,b,c\rbrace$ and $\mathcal{E}=\lbrace \lbrace a\rbrace, \lbrace a,b\rbrace, \lbrace a,c\rbrace, \lbrace a,b,c\rbrace\rbrace$.  Defining $v_i$ and $e_j$ in the obvious manner (as they are listed in the sets), we have
\begin{center}$A =
\begin{pmatrix}
1 & 1 & 1 & 1 \\
0 & 1 & 0 & 1 \\
0 & 0 & 1 & 1
\end{pmatrix}$
\end{center}
Notice that the sum of the entries in any column is the cardinality of the corresponding edge.  Similarly, the sum of the entries in a particular row is the degree of the corresponding vertex.

The transpose $A^t$ of the incidence matrix also defines a hypergraph $\mathcal{H}^*$, the \emph{dual} of $\mathcal{H}$, in an obvious manner.  To be explicit, let $\mathcal{H}^* = (V^*, \mathcal{E}^*)$ where $V^*$ is an $m$-element set and $\mathcal{E}^*$ is an $n$-element set of subsets of $V^*$.  For $v^*_j \in V^*$ and $e^*_i \in \mathcal{E}^*, ~ v^*_j \in e^*_i$ if and only if $a_{ij} = 1$.

Continuing from the example above, the dual $\mathcal{H}^*$ of $\mathcal{H}$ consists of $V^*=\lbrace x,y,z,t\rbrace$ and $\mathcal{E}^*=\lbrace \lbrace x,y,z,t\rbrace, \lbrace y,t\rbrace, \lbrace z,t\rbrace \rbrace$, where $v^*_j$ and $e^*_i$ are defined in the obvious manner.

By the remark immediately after the definition of the incidence matrix of a hypergraph, it is easy to see that the dual of a uniform hypergraph is regular and vice-versa.  It is not rare to see fruitful results emerge by considering the dual of a hypergraph.

\textbf{Note}
A finite hypergraph is also related to a metacategory, and can be shown to be a special case of a supercategory, and can be thus defined as a mathematical interpretation of ETAS axioms.

%%%%%
%%%%%
\end{document}
