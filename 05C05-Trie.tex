\documentclass[12pt]{article}
\usepackage{pmmeta}
\pmcanonicalname{Trie}
\pmcreated{2013-03-22 12:28:36}
\pmmodified{2013-03-22 12:28:36}
\pmowner{mathcam}{2727}
\pmmodifier{mathcam}{2727}
\pmtitle{trie}
\pmrecord{12}{32684}
\pmprivacy{1}
\pmauthor{mathcam}{2727}
\pmtype{Definition}
\pmcomment{trigger rebuild}
\pmclassification{msc}{05C05}
\pmclassification{msc}{68P20}
\pmrelated{DigitalTree}

\usepackage{amssymb}
\usepackage{amsmath}
\usepackage{amsfonts}
\usepackage[all]{xy}
\newcommand{\concat}{{\ensuremath{+\hspace{-1ex}+}}}
\begin{document}
\PMlinkescapeword{edge}
\PMlinkescapeword{edges}
\PMlinkescapeword{equivalent}
\PMlinkescapeword{name}
\PMlinkescapeword{node}
\PMlinkescapeword{nodes}

A \emph{trie} is a digital tree for storing a set of strings in which there is one node for every prefix of every string in the set.  The name comes from the word
re\emph{trie}val, and thus is pronounced the same as \emph{tree} (which leads to much confusion when spoken aloud).  The word retrieval is stressed, because a trie has a lookup time that is equivalent to the length of the string being looked up.

If a trie is to store some set of strings $S\subseteq\Sigma^*$ (where $\Sigma$ is an alphabet),
then it takes the following form.
Each edge leading to non-leaf nodes in the trie is labelled by an element of $\Sigma$.  Any edge leading to a leaf node is labelled by $\$$ (some symbol \emph{not} in $\Sigma$).  For every string $s\in S$, there is a path from the root of the trie to a leaf, the labels of which when concatenated form $s\concat\$$ (where $\concat$ is the string concatenation operator).  For every path from the root of the trie to a leaf, the labels of the edges concatenated form some string in $S$.

\subsubsection*{Example}

Suppose we wish to store the set of strings $S := \left\{ alpha, beta, bear, beast, beat \right\}$.  The trie that stores $S$ would be

$$
\xymatrix{
    &&\bullet \ar@{-}[dl]_a \ar@{-}[dr]^b \\
    &\bullet \ar@{-}[dl]_l && \bullet \ar@{-}[dr]^e \\
    \bullet \ar@{-}[d]_p &&&& \bullet \ar@{-}[dl]_a \ar@{-}[dr]^t \\
    \bullet \ar@{-}[d]_h &&& \bullet \ar@{-}[dl]_r \ar@{-}[d]^s \ar@{-}[dr]^t && \bullet \ar@{-}[dr]^a \\
    \bullet \ar@{-}[d]_a && \bullet \ar@{-}[dl]_{\$} & \bullet \ar@{-}[d]^t & \bullet \ar@{-}[dr]^{\$} && \bullet \ar@{-}[dr]^{\$} \\
    \bullet \ar@{-}[d]_{\$} & \bullet && \bullet \ar@{-}[d]^{\$} && \bullet && \bullet \\
    \bullet &&& \bullet
}
$$
%%%%%
%%%%%
\end{document}
