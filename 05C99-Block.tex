\documentclass[12pt]{article}
\usepackage{pmmeta}
\pmcanonicalname{Block}
\pmcreated{2013-03-22 12:32:00}
\pmmodified{2013-03-22 12:32:00}
\pmowner{digitalis}{76}
\pmmodifier{digitalis}{76}
\pmtitle{block}
\pmrecord{4}{32775}
\pmprivacy{1}
\pmauthor{digitalis}{76}
\pmtype{Definition}
\pmcomment{trigger rebuild}
\pmclassification{msc}{05C99}
\pmrelated{Cutvertex}
\pmrelated{Bridge}

\endmetadata

% this is the default PlanetMath preamble.  as your knowledge
% of TeX increases, you will probably want to edit this, but
% it should be fine as is for beginners.

% almost certainly you want these
\usepackage{amssymb}
\usepackage{amsmath}
\usepackage{amsfonts}

% used for TeXing text within eps files
%\usepackage{psfrag}
% need this for including graphics (\includegraphics)
%\usepackage{graphicx}
% for neatly defining theorems and propositions
%\usepackage{amsthm}
% making logically defined graphics
%%%\usepackage{xypic} 

% there are many more packages, add them here as you need them

% define commands here
\begin{document}
A subgraph $B$ of a graph $G$ is a \emph{block of} $G$ if either it is a bridge (together with the vertices incident with the bridge) or else it is a maximal 2-connected subgraph of $G$.

Any two blocks of a graph $G$ have at most one vertex in common. Also, every vertex belonging to at least two blocks is a cutvertex of $G$, and, conversely, every cutvertex belongs to at least two blocks.


\footnotesize{Adapted with permission of the author from \emph{\PMlinkescapetext{Modern Graph Theory}} by B\'{e}la Bollob\'{a}s, published by Springer-Verlag New York, Inc., 1998.}
%%%%%
%%%%%
\end{document}
