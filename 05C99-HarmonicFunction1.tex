\documentclass[12pt]{article}
\usepackage{pmmeta}
\pmcanonicalname{HarmonicFunction1}
\pmcreated{2013-03-22 15:09:27}
\pmmodified{2013-03-22 15:09:27}
\pmowner{mathcam}{2727}
\pmmodifier{mathcam}{2727}
\pmtitle{harmonic function}
\pmrecord{5}{36906}
\pmprivacy{1}
\pmauthor{mathcam}{2727}
\pmtype{Definition}
\pmcomment{trigger rebuild}
\pmclassification{msc}{05C99}

% this is the default PlanetMath preamble.  as your knowledge
% of TeX increases, you will probably want to edit this, but
% it should be fine as is for beginners.

% almost certainly you want these
\usepackage{amssymb}
\usepackage{amsmath}
\usepackage{amsfonts}
\usepackage{amsthm}

% used for TeXing text within eps files
%\usepackage{psfrag}
% need this for including graphics (\includegraphics)
%\usepackage{graphicx}
% for neatly defining theorems and propositions
%\usepackage{amsthm}
% making logically defined graphics
%%%\usepackage{xypic}

% there are many more packages, add them here as you need them

% define commands here

\newcommand{\mc}{\mathcal}
\newcommand{\mb}{\mathbb}
\newcommand{\mf}{\mathfrak}
\newcommand{\ol}{\overline}
\newcommand{\ra}{\rightarrow}
\newcommand{\la}{\leftarrow}
\newcommand{\La}{\Leftarrow}
\newcommand{\Ra}{\Rightarrow}
\newcommand{\nor}{\vartriangleleft}
\newcommand{\Gal}{\text{Gal}}
\newcommand{\GL}{\text{GL}}
\newcommand{\Z}{\mb{Z}}
\newcommand{\R}{\mb{R}}
\newcommand{\Q}{\mb{Q}}
\newcommand{\C}{\mb{C}}
\newcommand{\<}{\langle}
\renewcommand{\>}{\rangle}
\begin{document}
A real or complex-valued function $f:V\to\mathbb{R}$ or $f:V\to\mathbb{C}$ defined on the vertices~$V$ of a graph $G=(V,E)$ is called \emph{harmonic} at $v\in V$ if its value at~$v$ is its average value at the neighbours of~$v$:
$$f(v) = \frac{1}{\operatorname{deg}(v)} \sum_{\{u,v\}\in E} f(u).$$
It is called harmonic \emph{except on~$A$}, for some $A\subseteq V$, if it is harmonic at each $v\in V\setminus A$, and harmonic if it is harmonic at each $v\in V$.

Any harmonic $f:\mathbb{Z}^n\to\mathbb{R}$, where $\mathbb{Z}^n$ is the $n$-dimensional grid, is \PMlinkescapetext{constant} if \PMlinkescapetext{bounded} below (or above).  However, this is not necessarily true on other graphs.
%%%%%
%%%%%
\end{document}
