\documentclass[12pt]{article}
\usepackage{pmmeta}
\pmcanonicalname{Loop}
\pmcreated{2013-03-22 12:14:08}
\pmmodified{2013-03-22 12:14:08}
\pmowner{drini}{3}
\pmmodifier{drini}{3}
\pmtitle{loop}
\pmrecord{10}{31615}
\pmprivacy{1}
\pmauthor{drini}{3}
\pmtype{Definition}
\pmcomment{trigger rebuild}
\pmclassification{msc}{05C99}
\pmclassification{msc}{20N05}
\pmrelated{Graph}
\pmrelated{Pseudograph}
\pmrelated{Quasigroup}

\endmetadata

\usepackage{amssymb}
\usepackage{amsmath}
\usepackage{amsfonts}
\begin{document}
\PMlinkescapeword{joins}
\PMlinkescapeword{contain}
In graph theory, a \emph{loop} is an edge which joins a vertex
to itself, rather than to some other vertex. By definition,
a graph cannot contain a loop; a pseudograph, however, may contain
both multiple edges and multiple loops. Note that by some definitions,
a multigraph may contain multiple edges and no loops, while other texts
define a multigraph as a graph
allowing multiple edges and multiple loops.

In algebra, a \emph{loop} is a quasigroup which contains an identity element.
%%%%%
%%%%%
%%%%%
\end{document}
