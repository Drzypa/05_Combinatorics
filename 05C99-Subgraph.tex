\documentclass[12pt]{article}
\usepackage{pmmeta}
\pmcanonicalname{Subgraph}
\pmcreated{2013-03-22 12:30:59}
\pmmodified{2013-03-22 12:30:59}
\pmowner{CWoo}{3771}
\pmmodifier{CWoo}{3771}
\pmtitle{subgraph}
\pmrecord{10}{32754}
\pmprivacy{1}
\pmauthor{CWoo}{3771}
\pmtype{Definition}
\pmcomment{trigger rebuild}
\pmclassification{msc}{05C99}
%\pmkeywords{graph}
\pmrelated{Graph}
\pmrelated{Pseudograph}
\pmrelated{Multigraph}
\pmdefines{induced}
\pmdefines{spanned by}
\pmdefines{spanning}
\pmdefines{spanning subgraph}
\pmdefines{induced subgraph}

% this is the default PlanetMath preamble.  as your knowledge
% of TeX increases, you will probably want to edit this, but
% it should be fine as is for beginners.

% almost certainly you want these
\usepackage{amssymb}
\usepackage{amsmath}
\usepackage{amsfonts}

% used for TeXing text within eps files
%\usepackage{psfrag}
% need this for including graphics (\includegraphics)
%\usepackage{graphicx}
% for neatly defining theorems and propositions
%\usepackage{amsthm}
% making logically defined graphics
%%%\usepackage{xypic} 

% there are many more packages, add them here as you need them

% define commands here
\begin{document}
We say that $G' = (V',E')$ is a \emph{subgraph} of $G = (V,E)$ if $V' \subseteq V$ and $E' \subseteq E$. In this case we write $G' \subseteq G$.

If $G'$ contains \emph{all edges} of $G$ that join two vertices in $V'$ then $G'$ is said to be the subgraph \emph{induced} or \emph{spanned by} $V'$ and is denoted by $G[V']$. Thus, a subgraph $G'$ of $G$ is an induced subgraph if $G' = G[V(G')]$. If $V' = V$, then $G'$ is said to be a \emph{spanning} subgraph of $G$.

Often, new graphs are constructed from old ones by deleting or adding some vertices and edges. If $W \subset V(G)$, then $G - W = G[V\setminus W]$ is the subgraph of $G$ obtained by deleting the vertices in $W$ \emph{and all edges incident with them}. Similarly, if $E' \subseteq E(G)$, then $G - E' = (V(G),E(G) \setminus E')$. If $W = {w}$ and $E' = {xy}$, then this notation is simplified to $G - w$ and $G - xy$. Similarly, if $x$ and $y$ are nonadjacent vertices of $G$, then $G + xy$ is obtained from $G$ by joining $x$ to $y$.


\footnotesize{Adapted with permission of the author from \emph{\PMlinkescapetext{Modern Graph Theory}} by B\'{e}la Bollob\'{a}s, published by Springer-Verlag New York, Inc., 1998.}
%%%%%
%%%%%
\end{document}
