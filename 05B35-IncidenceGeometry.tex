\documentclass[12pt]{article}
\usepackage{pmmeta}
\pmcanonicalname{IncidenceGeometry}
\pmcreated{2013-03-22 15:26:13}
\pmmodified{2013-03-22 15:26:13}
\pmowner{CWoo}{3771}
\pmmodifier{CWoo}{3771}
\pmtitle{incidence geometry}
\pmrecord{26}{37284}
\pmprivacy{1}
\pmauthor{CWoo}{3771}
\pmtype{Definition}
\pmcomment{trigger rebuild}
\pmclassification{msc}{05B35}
\pmclassification{msc}{06C10}
\pmclassification{msc}{51A05}
\pmsynonym{lies on}{IncidenceGeometry}
\pmsynonym{lying on}{IncidenceGeometry}
\pmsynonym{passes through}{IncidenceGeometry}
\pmsynonym{passing through}{IncidenceGeometry}
\pmrelated{SemimodularLattice}
\pmdefines{incident}
\pmdefines{incidence relation}
\pmdefines{type function}
\pmdefines{plane incidence geometry}
\pmdefines{solid incidence geometry}
\pmdefines{incidence axiom}
\pmdefines{shadow}
\pmdefines{pass through}
\pmdefines{lie on}
\pmdefines{line}
\pmdefines{plane}
\pmdefines{projective incidence geometry}
\pmdefines{affine incidence geometry}
\pmdefines{Playfair's axiom}
\pmdefines{coplanar}

\usepackage{amssymb,amscd}
\usepackage{amsmath}
\usepackage{amsfonts}

% used for TeXing text within eps files
%\usepackage{psfrag}
% need this for including graphics (\includegraphics)
%\usepackage{graphicx}
% for neatly defining theorems and propositions
%\usepackage{amsthm}
% making logically defined graphics
%%%\usepackage{xypic}

% define commands here
\begin{document}
\PMlinkescapeword{addition}
\PMlinkescapeword{almost all}
\PMlinkescapeword{angle}
\PMlinkescapeword{areas}
\PMlinkescapeword{blocks}
\PMlinkescapeword{block}
\PMlinkescapeword{branches}
\PMlinkescapeword{dimension}
\PMlinkescapeword{dimensions}
\PMlinkescapeword{equivalent}
\PMlinkescapeword{feasible}
\PMlinkescapeword{flat}
\PMlinkescapeword{flats}
\PMlinkescapeword{generate}
\PMlinkescapeword{generated}
\PMlinkescapeword{generated by}
\PMlinkescapeword{generates}
\PMlinkescapeword{hyperplane}
\PMlinkescapeword{hyperplanes}
\PMlinkescapeword{information}
\PMlinkescapeword{mean}
\PMlinkescapeword{reflexive}
\PMlinkescapeword{states}
\PMlinkescapeword{straight}
\PMlinkescapeword{symmetric}
\PMlinkescapeword{terms}
\PMlinkescapeword{type}
\PMlinkescapeword{types}
\PMlinkescapeword{varieties}
\PMlinkescapeword{words}

Incidence geometry is essentially geometry based on the first postulate in Euclid's \emph{The Elements}.  Basically, the first postulate states that we can draw a straight line from one point to another point.  At the end of the 19th century, David Hilbert extended and axiomatized this postulate by adding several companion ``incidence axioms'' in his famous ``Grundlagen der Geometrie'' (\PMlinkescapetext{Foundations of Geometry}).  Since Hilbert's publication, his axioms of incidence have been characterized by alternative but \PMlinkname{equivalent}{Equivalent3} versions, as well as generalized as to include areas of interests from other branches of mathematics, especially, in combinatorics.  In this entry, we will define incidence geometry using abstract notions of sets, functions, and relations (specifically, an incidence relation) and then briefly discuss how this definition is related to the axioms of incidence that we know from high school and college.
\\\\
\textbf{Definition}.  Let $P$ be a set and $n$ a positive integer. An \emph{incidence geometry} on $P$ consists of
\begin{itemize}
\item an onto function $t\colon P\to\lbrace0,1,\ldots,n\rbrace$ called a \emph{type function}.  If we define $P_i:=t^{-1}(i)$, then $P$ can be partitioned into a finite number of subsets:

$$P=P_0\cup\cdots\cup P_n,\mbox{ with }P_i\cap P_j=\varnothing\mbox{ for }i\neq j.$$

Elements of $P_i$ are variously known as \emph{blocks} or \emph{varieties} of \emph{type $i$}.  Sometimes, they are also called \emph{flats} of \emph{dimension $i$}.  For this discussion, we will use the latter terminology.  Flats of specific dimensions have further conditions:

\begin{enumerate}
\item $P_0\neq\varnothing$.  Flats of dimension 0 are called \textbf{\emph{points}}.
\item $P_n\neq\varnothing$ and consists of one element $S$, called the
\emph{space}.
\end{enumerate}

\item a \PMlinkname{reflexive}{Reflexive} and symmetric relation $I\subseteq P\times P$ on $P$, called an \emph{incidence relation}, with the following conditions (or axioms):

\begin{enumerate}
\item for any $a,b$ such that $(a,b)\in I$ and $t(a)=t(b)$, then $a=b$;
\item suppose $t(a)\leq t(b)\leq t(c)$ with $(a,b)\in I$ and $(b,c)\in I$, then $(a,c)\in I$;
\item given that $t(a)<n$, there is a point $p$ such that $(p,a)\notin I$;
\item given $t(a)<n$ and a point $p$ with $(p,a)\notin I$, there is a unique $b$ with $t(b)=t(a)+1$, such that $(a,b)\in I$ and $(p,b)\in I$; furthermore, if, in addition, there is a $c$ such that $(p,c)\in I$ and $(a,c)\in I$, then $(b,c)\in I$ as well;
\item given that $t(a)>0$, there is a pair of a point $p$ and a flat $b$, with $t(b)=t(a)-1$ and $(p,b)\notin I$, such that $(b,a)\in I$ and $(p,a)\in I$;
\item given that $0<t(a)=t(b)=i<n$, a point $p$ with $(p,a),(p,b)\in I$, and a flat $d$ with $t(d)=i+1$, $(a,d),(b,d)\in I$, then there is a $c$ with $t(c)=i-1$ such that $(a,c),(b,c)\in I$.
\end{enumerate}
\end{itemize}

An incidence geometry is often written as a triple $(P,n,I)$.

\textbf{Remarks}.
\begin{itemize}
\item Flats of dimensions 1 and 2 are commonly called \textbf{\emph{lines}} and \textbf{\emph{planes}}, respectively.  Flats of dimension $n-1$ are called \emph{hyperplanes}.
\item When $(a,b)\in I$, we say that $a$ is \emph{incident} with $b$. Since $I$ is symmetric, we also say that $a$ and
$b$ are incident without any ambiguity.  Furthermore, if $a$ and $b$ are incident with $t(a)\leq t(b)$, then we say that $a$ \textbf{\emph{lies on}} $b$, or $b$ \textbf{\emph{passes through}} $a$.
\item Condition 2 is known as the ``weak'' \PMlinkescapetext{transitive property}, or the ``transversality'' of $I$.  Basically, $(a,b)\in I$ and $(b,c)\in I$ imply $(a,c)\in I$ whenever the dimensions of $a,b$ and
$c$ are in a non-decreasing order.
\item From Conditions 4 and 5 above, given any flat $a$ of dimension $i>0$, there exist a point $p$ and a flat $b$ of dimension $i-1$ with the properties that

\begin{enumerate}
\item $p$ is not incident with $b$,
\item $a$ is incident with $p$, and
\item $a$ is incident with $b$,
\end{enumerate}

then the unique flat of dimension $i$ mentioned in Condition 4 is $a$.  We say that $a$ is \emph{generated by} $p$ and $b$, or that $p$ and $b$ generate $a$, and we write $a=\langle p, b\rangle$.  When $b$ is a point, $a$ is often written as $\overleftrightarrow{pb}$.  In addition, if we were to pick a different pair of a point $p^{\prime}$ and a flat $b^{\prime}$ of dimension $i-1$ satisfying the above three properties, then $a=\langle p^{\prime},b^{\prime}\rangle$ as well.
\item From the second part of Condition 4, $\langle p,b\rangle$ is, in a sense, the smallest block (in terms of its type number), such that the above three properties hold. In other words, for any block $a$ with $(p,a)\in I$ and $(b,a)\in I$, then $(\langle p,b\rangle,a)\in I$.  It is easy to see that $t(a)\geq t(\langle p,b\rangle)$. For otherwise, $t(a)<t(\langle p,b\rangle)$, which means $t(a)\leq t(b)$.  This inequality together with $(p,a)\in I$ and $(a,b)$ imply
that $(p,b)\in I$, a contradiction.
\item In Condition 6, if $a\neq b$, then $c$ is necessarily incident with $p$.  Otherwise, $a=\langle p,c\rangle=b$.  Also, without much trouble, one can show that $d$ in the condition must be unique.
\item $P_i\neq\varnothing$ for all $0\leq i\leq n$.  In other words, there exists at least one flat of every dimension.  To see this, we first observe that $P_0\neq\varnothing$, there is at least one point $p$.  With $p$, there is a point $q$ such that $(p,q)\notin I$.  Therefore, there is a (unique) line $\ell$ that is incident with
both $p$ and $q$.  Continue this way until we reach $i=n$.
\item Any flat of dimension $i$ is incident with at least $i+1$ points.
\item Every flat is incident with at least one flat of every dimension.
As a result, the space $S$ is incident with every flat of every dimension.
\end{itemize}

\textbf{Shadow}.  For any $a\in P$, define $I(a)=\lbrace b\in P\mid (a,b)\in I\rbrace$, $I^{-}(a)=\lbrace b\in I(a)\mid t(b)\leq t(a)\rbrace$, and $I^{+}(a)=\lbrace b\in I(a)\mid t(b)\geq t(a)\rbrace$.  For specific type $k$, we also define $I_k(a)=\lbrace b\mid t(b)=k\mbox{ and }(b,a)\in I\rbrace$.  When $k=0$, $I_0(a)$, the set of all points incident with $a$, is referred to as the \emph{shadow} of $a$. We have that $I_0(a)\subseteq
I^{-}(a)\subseteq I(a)$. We also have $I^{-}(a)\cap I^{+}(a)=I_{t(a)}(a)=\lbrace a\rbrace$.
\\\\
\textbf{Remark}. It is possible to show that $(a,b)\in I$ if and only if  $I_0(a)\subseteq I_0(b)$ or $I_0(b)\subseteq I_0(a)$.  Furthermore, if $I_0(a)\subseteq I_0(b)$, then $t(a)\leq t(b)$.  In particular, $a=b$ if and only if  $I_0(a)=I_0(b)$.  From the last remark above, $I(S)=P$, and in particular $P_k=I_k(S)\subset I(S)$.  We also have for any flat $a$, $I_0(a)\subseteq I_0(S)$.  This says that every singleton subset of $I_0(S)$ is of the form $I_0(p)$ for some $p$.  The discussion so far suggests the following simpler, more intuitive, formulation of incidence geometry:
\\\\
Let $A$ be a set.  An incidence geometry on $A$ is a subset $P$ of the power set of $A$ such that $P$ can be partitioned into $n+1$ finite subsets $P_0,\ldots,P_n$ with the following \emph{axioms}:

\begin{enumerate}
\item $P_0$ consists of all singleton subsets of $A$ and $P_0$ is non-empty; elements of $P_0$ are called points of $A$.  Since there is an obvious one-to-one correspondence between $A$ and $P_0$, we shall follow by convention and call elements of $A$ points of $A$ instead;
\item $P_n=\lbrace A\rbrace$; $A$ is called the space;
\item for every element $a$ of $P_i$, where $i<n$, there is a point $p$ such that $p\notin a$;
\item for every $a\in P_i$, where $i<n$, and point $p$ such that $p\notin A$, there is a unique $b\in P_{i+1}$ such that $a\subset b$ and $p\in b$; furthermore, if there is a $c$ with $a\subset c$ and $p\in c$, then $b\subseteq c$;
\item for every $a\in P_i$, where $i>0$, then there is a point $p$ and a $b\in P_{i-1}$ with $p\notin b$, such that $p\in a$ and $b\subset a$;
\item if $a,b\in P_i$ and $d\in P_{i+1}$, where $0<i<n$, with a point $p$ such that $p\in a\subset d$ and $p\in b\subset d$, then there is a $c\in P_{i-1}$ such that $c\subset a$ and $c\subset b$.
\end{enumerate}

If we define $I$ on $P$ to be $(a,b)\in I$ if and only if  there is a symmetrized inclusion relation between $a$ and $b$ ($a\subseteq b$ or $b\subseteq a$), it is not hard to verify that $I$ is an incidence relation on $P$.
\\\\
\textbf{Remarks}.  Elements of $P_1$ are called lines of $A$ and elements of $P_2$ are called planes of $A$.  Whenever $a,b\in P$ such that $a\subset b$, then we say that $a$ lies on $b$ or $b$ passes through $a$.  Two special types of incidence geometries are worth mentioning:

\begin{itemize}
\item If $n=2$, an incidence geometry on $A$ is called a \textbf{\emph{plane incidence geometry}}.  In a plane incidence geometry, Axioms 1 through 3, 5 and first part of 4 are necessary.  Axiom 2 says that the space is \emph{the} unique plane of $A$.  Axiom 3 enumerates elements of $P$.  Axiom 4 is the heart of the incidence geometry; it
says that two distinct points lie on a unique line.  Furthermore, Axiom 4, together with Axiom 3, say that any line is a subset of the plane.  Second part of Axiom 4 is redundant in a plane incidence geometry.  If any element of $P$ that passes through two distinct points must be either a line or the plane. If it is a line, it must be the unique line determined by the two points, or the plane, which, clearly includes the unique line.  Axiom 5 says that there is
only \PMlinkescapeword{one way} to create lines (and the plane), namely, via Axiom 4.  Axiom 6 is trivial too (let $c=p$).
\item If $n=3$, an incidence geometry on $A$ is called a \textbf{\emph{solid incidence geometry}}.  Axioms 1 through 3, 5, and the first part of Axiom 4 here play the same role as they do in a plane incidence geometry.  First part of Axiom 4 also says that a line and a point not lying on it determine a unique plane.  The second part of Axiom 4 and Axiom 6 play an equally important role as the other Axioms.  Without the second part of Axiom 4, we would not be able to show, for example, that given a plane $\pi$ and a point $p$ lying on $\pi$, there is a line $\ell$ lying on $\pi$ but not passing through $p$.  Axiom 6 is decidedly non-trivial in solid incidence geometry.  It basically says that two planes passing through a common point must pass through a line.  Without it, it is possible to find an example such that two planes ``\emph{intersect}'' at exactly one point.
\item Several familiar concepts concerning particular incidence properties of flat can be defined: points are \emph{collinear} if they lie on the same line; points and lines are \emph{coplanar} if they lie on the same plane; a \emph{pencil} is a collection of flats of the same dimension sharing a common incidence property which, in most cases, states they have the same ``intersection''.
\end{itemize}

Speaking of intersections, it would be proper to formally define what it means for two hyperplanes to ``intersect''.
\\\\
\textbf{Intersection}.  Let $a,b\in P$.  An \emph{intersection} of $a$ and $b$ is a flat $c$, if it exists, such that $I_0(c)=I_0(a)\cap I_0(b)$.
\\\\
Immediately, we see that, if an intersection of $a$ and $b$ exists, it must be unique.  For if $I_0(c)=I_0(a)\cap I_0(b)=I_0(d)$, then $c=d$.  We shall abuse the use of set-theoretic intersection to mean incidental intersection: if $a$ and $b$ are two flats, then $a\cap b$ denotes their intersection.  Furthermore, if no intersection exists, we write $a\cap b=\varnothing$.
\\\\
\textbf{Remarks}.

\begin{itemize}
\item It is easy to show that if $a$ and $b$ be flats with $i=t(a)\le t(b)$ and $a\cap b=d\ne \varnothing$, then $(d,a)\in I$ and $(d,b)\in I$. In addition, if $(a,b)\in I$, then $a=d$.  Also, the unique $c$ in Condition 6 above is the intersection or $a$ and $b$.
\item In light of the introduction of the concept of the intersection (of two flats), it seems feasible to toss in an additional element, called the \emph{empty block} or \emph{empty flat}, $\varnothing$, into the underlying set $P$ of the incidence geometry: $P_{-1}:=\lbrace \varnothing\rbrace$ and $P^{\prime}=P_{-1}\cup P$. If we next define a binary relation $I^{\prime}$ on $P^{\prime}$ to be:
$$(a,b)\in I^{\prime}\textrm{ if }\left\{ \begin{array}{ll} (a,b)\in I,\\ a=\varnothing\textrm{, or} \\ b=\varnothing. \end{array}
\right.$$ then $I^{\prime}$ becomes an incidence relation on $P^{\prime}$
if we restrict flat $a$ in Condition 3 to be non-empty only.  Furthermore, $P^{\prime}$, together with $I^{\prime}$ have almost all the ingredients of being an incidence geometry, except that the range of the type function has now been extended to include $-1$.
\item For every pair of non-empty $a,b\in P^{\prime}$, $U=I^{+}(a)\cap I^{+}(b)$ is a non-empty set since the space $S$ is in it.  In addition, since $n$ is finite, $U$ has a minimal element $c$ if we order its elements by their corresponding type numbers.  Moreover, $c$ is unique.  We denote this $c$ by $\langle a,b\rangle$.  This definition is consistent with our earlier definition of $\langle \cdot,\cdot \rangle$ when the first coordinate is a point and the second coordinate is a flat not passing through the point.
\item Collecting all the data above, it is now easy to see that $P^{\prime}$, together with the intersection operator $\cap$, and the angle bracket operator $\langle \cdot,\cdot \rangle$ form a semimodular lattice, if we set $a\wedge b:=a\cap b$ and $a\vee b:=\langle a,b\rangle$.
\item \textbf{\emph{parallelism}}.  Let $a$ be a flat.  A flat $b$ is said be parallel to $a$ if

\begin{enumerate}
\item $t(b)=t(a)$, \item $a\cap b$ is either $a$ or $\varnothing$.
\end{enumerate}

We write $a\parallel b$.  Note that if $a\cap b=a$, then $a=b$, since $t(b)=t(a)$.  So if $a$ is parallel to $b$, $b$ is parallel to $a$, and we may say that $a$ and $b$ are parallel.  Parallelism is a reflexive and symmetric relation.  However, it is not \PMlinkname{transitive}{Transitive3} (as in the case of a hyperbolic geometry).  Condition 6 above can now be restated as: if two flats of dimensions $i$, both lying in a flat of dimension $i+1$, are not parallel, then their intersection is a flat of dimension $i-1$.
\item An incidence geometry with the condition (or axiom) that every pair
of (non-empty) flats of dimensions $>0$ have non-empty intersection is called a \textbf{\emph{projective incidence geometry}}.
\item An incidence geometry with the condition (or axiom) that for every (non-empty) flat $a$ of dimension $i$ with $i<n$, and any point $p$ not lying on $a$, there is a flat $b$ passing through $p$, such that $a\parallel b$, is called an \textbf{\emph{affine incidence geometry}}.  The condition just stated is known as the \emph{Playfair's Axiom}.
\end{itemize}

Note to reader: the historical background of this entry is weak.  Any additional historical information on this is welcome!

\begin{thebibliography}{6}
\bibitem{fb} {\it Handbook of Incidence Geometry}, edited by Francis Buekenhout,
Elsevier Science Publishing Co. (1995)
\bibitem{dh} D. Hilbert, {\it Foundations of Geometry}, Open Court Publishing Co. (1971)
\bibitem{bs} K. Borsuk and W. Szmielew, {\it Foundations of Geometry}, North-Holland Publishing Co. Amsterdam (1960)
\bibitem{rh} R. Hartshorne, {\it Geometry: Euclid and Beyond}, Springer (2000)
\end{thebibliography}
%%%%%
%%%%%
\end{document}
