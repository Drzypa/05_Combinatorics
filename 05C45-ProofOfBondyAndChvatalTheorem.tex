\documentclass[12pt]{article}
\usepackage{pmmeta}
\pmcanonicalname{ProofOfBondyAndChvatalTheorem}
\pmcreated{2013-03-22 14:48:33}
\pmmodified{2013-03-22 14:48:33}
\pmowner{taxipom}{3607}
\pmmodifier{taxipom}{3607}
\pmtitle{proof of Bondy and Chv\'atal theorem}
\pmrecord{8}{36466}
\pmprivacy{1}
\pmauthor{taxipom}{3607}
\pmtype{Proof}
\pmcomment{trigger rebuild}
\pmclassification{msc}{05C45}

\endmetadata

% this is the default PlanetMath preamble.  as your knowledge
% of TeX increases, you will probably want to edit this, but
% it should be fine as is for beginners.

% almost certainly you want these
\usepackage{amssymb}
\usepackage{amsmath}
\usepackage{amsfonts}
\usepackage{amsthm}

% used for TeXing text within eps files
%\usepackage{psfrag}
% need this for including graphics (\includegraphics)
%\usepackage{graphicx}
% for neatly defining theorems and propositions
%\usepackage{amsthm}
% making logically defined graphics
%%%\usepackage{xypic}

% there are many more packages, add them here as you need them

% define commands here
\begin{document}
\begin{proof}
The sufficiency of the condition is obvious and we shall prove the necessity by
contradiction.

Assume that $G+uv$ is Hamiltonian but $G$ is not.
Then $G+uv$ has a Hamiltonian cycle containing the edge $uv$. Thus there exists a path $P=(x_1,\dots,x_n)$ in $G$ from $x_1=u$ to $x_n=v$ meeting all the vertices of $G$. If $x_i$ is adjacent to $x_1$ ($2\leq i\leq n$) then $x_{i-1}$ is not adjacent to $x_n$, for otherwise
$(x_1,x_i,x_{i+1},\dots,x_n,x_{i-1},x_{i-2},\dots,x_1)$ is a Hamiltonian cycle of $G$. Thus $d(x_n)\leq (n-1)-d(x_1)$, that is $d(u)+d(v)\leq n-1$, a contradiction
\end{proof}
%%%%%
%%%%%
\end{document}
