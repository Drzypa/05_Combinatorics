\documentclass[12pt]{article}
\usepackage{pmmeta}
\pmcanonicalname{IncidenceMatrixWithRespectToAnOrientation}
\pmcreated{2013-05-16 21:09:13}
\pmmodified{2013-05-16 21:09:13}
\pmowner{Mathprof}{13753}
\pmmodifier{unlord}{1}
\pmtitle{incidence matrix with respect to an orientation}
\pmrecord{7}{39382}
\pmprivacy{1}
\pmauthor{Mathprof}{1}
\pmtype{Definition}
\pmcomment{trigger rebuild}
\pmclassification{msc}{05C50}
\pmdefines{orientation}

% this is the default PlanetMath preamble.  as your knowledge
% of TeX increases, you will probably want to edit this, but
% it should be fine as is for beginners.

% almost certainly you want these
\usepackage{amssymb}
\usepackage{amsmath}
\usepackage{amsfonts}

% used for TeXing text within eps files
%\usepackage{psfrag}
% need this for including graphics (\includegraphics)
%\usepackage{graphicx}
% for neatly defining theorems and propositions
%\usepackage{amsthm}
% making logically defined graphics
%%%\usepackage{xypic}

% there are many more packages, add them here as you need them

% define commands here

\begin{document}
Let $G$ be a finite graph with $n$ vertices, $\{v_1, \ldots, v_n\}$
and $m$ edges, $\{e_1, \ldots, e_m\}$. 
For each edge $e = (v_i,v_j)$ of $G$ choose one vertex
to be the positive end and the other to be the negative end. In this way,
we assign an \emph{orientation} to $G$. The \emph{\PMlinkescapetext{incidence matrix}}
of $G$ with respect an orientation is an $n \times m$ matrix 
$D=(d_{ij})$
where
\begin{displaymath}
d_{ij} = \left\{ \begin{array}{ll}
+1 & \textrm{if $v_i$ is the positive end of $e_j$} \\
-1 & \textrm{if $v_i$ is the negative end of $e_j$} \\
0 & \textrm{otherwise}.
\end{array} \right.
\end{displaymath}

%%%%%
%%%%% rerender
\end{document}
