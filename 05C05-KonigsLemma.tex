\documentclass[12pt]{article}
\usepackage{pmmeta}
\pmcanonicalname{KonigsLemma}
\pmcreated{2013-03-22 15:38:26}
\pmmodified{2013-03-22 15:38:26}
\pmowner{mps}{409}
\pmmodifier{mps}{409}
\pmtitle{K\"onig's lemma}
\pmrecord{6}{37572}
\pmprivacy{1}
\pmauthor{mps}{409}
\pmtype{Theorem}
\pmcomment{trigger rebuild}
\pmclassification{msc}{05C05}
\pmsynonym{Koenig's lemma}{KonigsLemma}

% this is the default PlanetMath preamble.  as your knowledge
% of TeX increases, you will probably want to edit this, but
% it should be fine as is for beginners.

% almost certainly you want these
\usepackage{amssymb}
\usepackage{amsmath}
\usepackage{amsfonts}

% used for TeXing text within eps files
%\usepackage{psfrag}
% need this for including graphics (\includegraphics)
%\usepackage{graphicx}
% for neatly defining theorems and propositions
\usepackage{amsthm}
% making logically defined graphics
%%%\usepackage{xypic}

% there are many more packages, add them here as you need them

% define commands here
\newtheorem*{theorem*}{Theorem}
\newtheorem*{lemma*}{Lemma}
\begin{document}
\PMlinkescapeword{lemma}
\begin{theorem*}[K\"onig's lemma]
Let $T$ be a rooted directed tree.  If each vertex has finite 
degree but there are arbitrarily long rooted paths in $T$, 
then $T$ contains an infinite path.
\end{theorem*}

\begin{proof}
For each $n\ge 1$, let $P_n$ be a rooted path in $T$ of length $n$,
and let $c_n$ be the child of the root appearing in $P_n$.
By assumption, the set $\{c_n\mid n\ge 1\}$ is finite.  
Since the set $\{P_n\mid n\ge 1\}$ is infinite, the 
pigeonhole principle implies that there is a child $c$ of the root
such that $c=c_n$ for infinitely many $n$.  

Now let us look at the subtree $T_c$ of $T$ rooted at $c$.  Each vertex
has finite degree, and since there are paths $P_n$ of arbitrarily 
long length in $T$ passing through $c$, there are arbitrarily long
paths in $T_c$ rooted at $c$.  Hence if $T$ satisfies the hypothesis
of the lemma, the root has a child $c$ such that $T_c$ also satisfies
the hypothesis of the lemma.  Hence we may inductively build up a
path in $T$ of infinite length, at each stage selecting a child so
that the subtree rooted at that vertex still has arbitrarily long
paths.
\end{proof}

% Proof

% The proof is based upon induction and the pigeonhle principle.  By
% hypothesis there are a finite number of edges emanating from the
% root.  By the other hypothesis, there must exist an infinite number of
% paths emanating at the root.  Each of these paths must include one of
% the edges emanating from the root

\begin{thebibliography}{9}
\bibitem{cite:K}
Kleene, Stephen., \emph{Mathematical Logic}, New York: Wiley, 1967.
\end{thebibliography}
%%%%%
%%%%%
\end{document}
