\documentclass[12pt]{article}
\usepackage{pmmeta}
\pmcanonicalname{ProofOfMantelsTheorem}
\pmcreated{2013-03-22 13:03:04}
\pmmodified{2013-03-22 13:03:04}
\pmowner{mps}{409}
\pmmodifier{mps}{409}
\pmtitle{proof of Mantel's theorem}
\pmrecord{6}{33455}
\pmprivacy{1}
\pmauthor{mps}{409}
\pmtype{Proof}
\pmcomment{trigger rebuild}
\pmclassification{msc}{05C75}
\pmclassification{msc}{05C69}

\endmetadata

% this is the default PlanetMath preamble.  as your knowledge
% of TeX increases, you will probably want to edit this, but
% it should be fine as is for beginners.

% almost certainly you want these
\usepackage{amssymb}
\usepackage{amsmath}
\usepackage{amsfonts}

% used for TeXing text within eps files
%\usepackage{psfrag}
% need this for including graphics (\includegraphics)
%\usepackage{graphicx}
% for neatly defining theorems and propositions
%\usepackage{amsthm}
% making logically defined graphics
%%%\usepackage{xypic}

% there are many more packages, add them here as you need them

% define commands here
\newcommand{\supp}{\mathop{\mathrm{supp}}}
\begin{document}
\PMlinkescapeword{weight}
\PMlinkescapeword{push}
\PMlinkescapeword{support}

Let $G$ be a triangle-free graph.  We may assume that $G$ has at least three vertices and at least one edge; otherwise, there is nothing to prove.  Consider the set $P$ of all functions $c\colon V(G)\to\mathbb{R}_+$ such that $\sum_{v\in V(G)} c(v)=1$.  Define the total weight $W(c)$ of such a function by
\[
W(c) = \sum_{uv\in E(G)} c(u)\cdot c(v).
\]
By declaring that $c\le c^*$ if and only if $W(c)\le W(c^*)$ we make $P$ into a poset.

Consider the function $c_0\in P$ which takes the constant value $\frac{1}{|V(G)|}$ on each vertex.  The total weight of this function is
\[
W(c_0) = \sum_{uv\in E(G)} \frac{1}{|V(G)|}\cdot\frac{1}{|V(G)|}=\frac{|E(G)|}{|V(G)|^2},
\]
which is positive because $G$ has an edge.  So if $c\ge c_0$ in $P$, then $c$ has support on an induced subgraph of $G$ with at least one edge.

We claim that a maximal element of $P$ above $c_0$ is supported on a copy of $K_2$ inside $G$.  To see this, suppose $c\ge c_0$ in $P$.  If $c$ has support on a subgraph larger than $K_2$, then there are nonadjacent vertices $u$ and $v$ such that $c(u)$ and $c(v)$ are both positive.  Without loss of generality, suppose that
\begin{equation*}
\sum_{uw\in E(G)} c(w) \ge \sum_{vw\in E(G)} c(w).\tag{*}
\end{equation*}
Now we push the function off $v$.  To do this, define a function $c^*\colon V(G)\to\mathbb{R}_+$ by
\[
c^*(w) = \begin{cases}
c(u)+c(v) & w=u \\
0         & w=v \\
c(w)      & \text{otherwise.}
\end{cases}
\]
Observe that $\sum_{w\in V(G)} c^*(w) = 1$, so $c^*$ is still in the poset $P$.
Furthermore, by inequality~(*) and the definition of $c^*$,
\begin{align*}
W(c^*)
&= \sum_{uw\in E(G)} c^*(u)\cdot c^*(w) + \sum_{vw\in E(G)} c^*(v)\cdot c^*(w) + \sum_{wz\in E(G)} c^*(w)\cdot c^*(z) \\
&= \sum_{uw\in E(G)} [c(u) + c(v)]\cdot c(w) + 0 + \sum_{wz\in E(G)} c(w)\cdot c(z) \\
&= \sum_{uw\in E(G)} c(u) \cdot c(w) + \sum_{vw} c(v)\cdot c(w) + \sum_{wz\in E(G)} c(w) \cdot c(z) \\
&= W(c).
\end{align*}
Thus $c^*\ge c$ in $G$ and is supported on one less vertex than $c$ is.  So let $c$ be a maximal element of $P$ above $c_0$.  We have just seen that $c$ must be supported on adjacent vertices $u$ and $v$.  The weight $W(c)$ is just $c(u)\cdot c(v)$; since $c(u)+c(v)=1$ and $c$ has maximal weight, it must be that $c(u)=c(v)=\frac{1}{2}$.  Hence
\[
\frac{1}{4}=W(c)\ge W(c_0)=\frac{|E(G)|}{|V(G)|^2},
\]
which gives us the desired inequality: $|E(G)|\le\frac{|V(G)|^2}{4}$.

%%%%%
%%%%%
\end{document}
