\documentclass[12pt]{article}
\usepackage{pmmeta}
\pmcanonicalname{PascalsTriangleIsSymmetricalAlongItsCentralColumn}
\pmcreated{2013-03-22 19:00:14}
\pmmodified{2013-03-22 19:00:14}
\pmowner{PrimeFan}{13766}
\pmmodifier{PrimeFan}{13766}
\pmtitle{Pascal's triangle is symmetrical along its central column}
\pmrecord{5}{41872}
\pmprivacy{1}
\pmauthor{PrimeFan}{13766}
\pmtype{Corollary}
\pmcomment{trigger rebuild}
\pmclassification{msc}{05A19}

% this is the default PlanetMath preamble.  as your knowledge
% of TeX increases, you will probably want to edit this, but
% it should be fine as is for beginners.

% almost certainly you want these
\usepackage{amssymb}
\usepackage{amsmath}
\usepackage{amsfonts}

% used for TeXing text within eps files
%\usepackage{psfrag}
% need this for including graphics (\includegraphics)
%\usepackage{graphicx}
% for neatly defining theorems and propositions
%\usepackage{amsthm}
% making logically defined graphics
%%%\usepackage{xypic}

% there are many more packages, add them here as you need them

% define commands here

\begin{document}
As a consequence of Pascal's rule, we see that Pascal's triangle is symmetrical along its central column (the column containing the central binomial coefficients). Expressing individual values in Pascal's triangle $T$ as $T(n, k)$, with $n$ and $k$ being integers obeying the relation $-1 < k \leq n$, this means that each $T(n, k) = T(n, n - k)$.

Since Pascal's triangle is essentially a table in which to look up binomial coefficients, $$T(n, k) = \binom{n}{k}.$$ From Pascal's rule it follows that $T(n, k) = T(n - 1, k - 1) + T(n - 1, k)$.

Obviously $T(0, k) = 1$ because there is only one way to choose no items from a collection of $k$ items; likewise, $T(k, k) = 1$ because there is only one way to choose $k$ items from a collection of $k$ items. Therefore, the leftmost and rightmost column of Pascal's triangle are filled with 1's. Almost as obvious is the fact that $T(1, k) = k$ because there are $k$ ways to choose just one item from a collection of $k$ items; likewise, $T(k - 1, k) = k$ because there are $k$ ways to choose all but one item from a collection of $k$ items since leaving out one item in turn can only be done $k$ times in such a collection.

From the foregoing, row 1 of Pascal's triangle is {1, 1}, row 2 is {1, 2, 1} and row 3 is {1, 3, 3, 1}. From Pascal's rule it follows that even-numbered rows (with an odd number of columns, and their highest, central value at $T(\frac{k}{2}, k)$) will be symmetrical along the central value if the previous row was also symmetrical, while odd-numbered rows (with an even number of columns, and the highest, central value at both $T(\frac{k - 1}{2}, k)$ and $T(\frac{k + 1}{2}, k)$ will be symmetrical about the central values if the previous row was symmetrical. Since the first three rows are symmetrical, all the following rows are also symmetrical.
%%%%%
%%%%%
\end{document}
