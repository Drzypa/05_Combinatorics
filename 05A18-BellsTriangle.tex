\documentclass[12pt]{article}
\usepackage{pmmeta}
\pmcanonicalname{BellsTriangle}
\pmcreated{2013-03-22 16:49:16}
\pmmodified{2013-03-22 16:49:16}
\pmowner{PrimeFan}{13766}
\pmmodifier{PrimeFan}{13766}
\pmtitle{Bell's triangle}
\pmrecord{5}{39059}
\pmprivacy{1}
\pmauthor{PrimeFan}{13766}
\pmtype{Data Structure}
\pmcomment{trigger rebuild}
\pmclassification{msc}{05A18}
\pmclassification{msc}{11B73}
\pmsynonym{Bell triangle}{BellsTriangle}
\pmsynonym{Aitken's array}{BellsTriangle}
\pmsynonym{Aitken array}{BellsTriangle}
\pmsynonym{Peirce triangle}{BellsTriangle}
\pmsynonym{Peirce's triangle}{BellsTriangle}

% this is the default PlanetMath preamble.  as your knowledge
% of TeX increases, you will probably want to edit this, but
% it should be fine as is for beginners.

% almost certainly you want these
\usepackage{amssymb}
\usepackage{amsmath}
\usepackage{amsfonts}

% used for TeXing text within eps files
%\usepackage{psfrag}
% need this for including graphics (\includegraphics)
%\usepackage{graphicx}
% for neatly defining theorems and propositions
%\usepackage{amsthm}
% making logically defined graphics
%%%\usepackage{xypic}

% there are many more packages, add them here as you need them

% define commands here

\begin{document}
{\em Bell's triangle} or {\em Aitken's array} or {\em Peirce triangle} is a triangular arrangement of integers in which the top row has a single 1, and each subsequent row begins with the last number of the previous row, and each remaining number in a row is the sum of the number to the left to the number above left.

The first eight rows are:

$$\begin{array}{cccccccccccccccccc}
& & & & & & & & & 1 & & & & & & & &\\
& & & & & & & & 1 & & 2 & & & & & & &\\
& & & & & & & 2 & & 3 & & 5 & & & & & &\\
& & & & & & 5 & & 7 & & 10 & & 15 & & & & &\\
& & & & & 15 & & 20 & & 27 & & 37 & & 52 & & & &\\
& & & & 52 & & 67 & & 87 & & 114 & & 151 & & 203 & & &\\
& & & 203 & & 255 & & 322 & & 409 & & 523 & & 674 & & 877 & &\\
& & 877 & & 1080 & & 1335 & & 1657 & & 2066 & & 2589 & & 3263 & & 4140 &\\
& & & & &\vdots & & & & \vdots & & & & \vdots& & & & \\
\end{array}$$

These are listed in A011971 of Sloane's OEIS.

The left outermost diagonal gives the Bell numbers (1, 2, 5, 15, 52, 203, 877, 4140, etc.) in order.
%%%%%
%%%%%
\end{document}
