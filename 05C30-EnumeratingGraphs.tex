\documentclass[12pt]{article}
\usepackage{pmmeta}
\pmcanonicalname{EnumeratingGraphs}
\pmcreated{2013-03-22 15:50:07}
\pmmodified{2013-03-22 15:50:07}
\pmowner{Algeboy}{12884}
\pmmodifier{Algeboy}{12884}
\pmtitle{enumerating graphs}
\pmrecord{6}{37810}
\pmprivacy{1}
\pmauthor{Algeboy}{12884}
\pmtype{Theorem}
\pmcomment{trigger rebuild}
\pmclassification{msc}{05C30}
\pmrelated{EnumeratingGroups}

\usepackage{latexsym}
\usepackage{amssymb}
\usepackage{amsmath}
\usepackage{amsfonts}
\usepackage{amsthm}

%%\usepackage{xypic}

%-----------------------------------------------------

%       Standard theoremlike environments.

%       Stolen directly from AMSLaTeX sample

%-----------------------------------------------------

%% \theoremstyle{plain} %% This is the default

\newtheorem{thm}{Theorem}

\newtheorem{coro}[thm]{Corollary}

\newtheorem{lem}[thm]{Lemma}

\newtheorem{lemma}[thm]{Lemma}

\newtheorem{prop}[thm]{Proposition}

\newtheorem{conjecture}[thm]{Conjecture}

\newtheorem{conj}[thm]{Conjecture}

\newtheorem{defn}[thm]{Definition}

\newtheorem{remark}[thm]{Remark}

\newtheorem{ex}[thm]{Example}



%\countstyle[equation]{thm}



%--------------------------------------------------

%       Item references.

%--------------------------------------------------


\newcommand{\exref}[1]{Example-\ref{#1}}

\newcommand{\thmref}[1]{Theorem-\ref{#1}}

\newcommand{\defref}[1]{Definition-\ref{#1}}

\newcommand{\eqnref}[1]{(\ref{#1})}

\newcommand{\secref}[1]{Section-\ref{#1}}

\newcommand{\lemref}[1]{Lemma-\ref{#1}}

\newcommand{\propref}[1]{Prop\-o\-si\-tion-\ref{#1}}

\newcommand{\corref}[1]{Cor\-ol\-lary-\ref{#1}}

\newcommand{\figref}[1]{Fig\-ure-\ref{#1}}

\newcommand{\conjref}[1]{Conjecture-\ref{#1}}


% Normal subgroup or equal.

\providecommand{\normaleq}{\unlhd}

% Normal subgroup.

\providecommand{\normal}{\lhd}

\providecommand{\rnormal}{\rhd}
% Divides, does not divide.

\providecommand{\divides}{\mid}

\providecommand{\ndivides}{\nmid}


\providecommand{\union}{\cup}

\providecommand{\bigunion}{\bigcup}

\providecommand{\intersect}{\cap}

\providecommand{\bigintersect}{\bigcap}
\begin{document}
\indent\textbf{How many graphs are there?}

\begin{prop}\label{prop:graph}
The number of non-isomorphic simple graphs on $N$ vertices, denoted $gph(N)$, satisfies
	\[\frac{2^{\binom{N}{2}}}{N!} \leq gph(N) \leq 2^{\binom{N}{2}}.\]
\end{prop}
\begin{proof}
\begin{itemize}
\item[($\leq$)]
Every simple graph on $N$ vertices can be encoded by a $N\times N$ symmetric matrix 
with all entries from $\{0,1\}$ and with $0$'s on the diagonal.  Therefore the information
can be encoded by strictly upper triangular $N\times N$ matrices over $\{0,1\}$.  This
gives $2^{\binom{N}{2}}$ possibilities.

\item[($\geq$)]
The number of isomorphisms from a graph $G$ to a graph $H$ is equal to the number of
automorphisms of $G$ (or $H$).  Therefore the total number of isomorphisms between two
graphs is no more than $N!$ -- the total number of permutations of the $N$ vertices of
$G$.  Therefore the total number of non-isomorphic simple graphs on $N$ vertices is
at least $2^{\binom{N}{2}}/N!$.
\end{itemize}
\end{proof}

\begin{itemize}
\item The gap between the bounds is superexponential in $N$:
that is: $O\left(2^{\binom{N}{2}}\right)$.  However logarithmically the estimate is tight:
 \[\log gph(N)\in \Omega(N^2-N\log N)\intersect O(N^2)=\Theta(N^2).\] 
\item
$gph(N)$ is closer to the lower bound; yet, most graphs have few automorphisms.

\item $gph$ is monotonically increasing.
\[
\begin{xy}
(0,10) *\hbox{Graphs on $N$ vertices},
(0,0) *[o]=<40pt>\hbox{G}*\frm{o}, (5,0) *\hbox{$\bullet$},
 (-3,-3) *\hbox{$\bullet$},  (-4,2) *\hbox{$\bullet$};
\end{xy}
\qquad
\hookrightarrow
\qquad
\begin{xy}
(0,10) *\hbox{Graphs on $N+1$ vertices},
(0,0) *[o]=<40pt>\hbox{G}*\frm{o}, (5,0) *\hbox{$\bullet$},
 (-3,-3) *\hbox{$\bullet$},  (-4,2) *\hbox{$\bullet$},
 (10,0) *\hbox{$\bullet$};
\end{xy}
\]
\end{itemize}


\bibliographystyle{amsplain}
\providecommand{\bysame}{\leavevmode\hbox to3em{\hrulefill}\thinspace}
\providecommand{\MR}{\relax\ifhmode\unskip\space\fi MR }
% \MRhref is called by the amsart/book/proc definition of \MR.
\providecommand{\MRhref}[2]{%
  \href{http://www.ams.org/mathscinet-getitem?mr=#1}{#2}
}
\providecommand{\href}[2]{#2}
\begin{thebibliography}{10}


\bibitem{HP}
Harary, Frank and Palmer, Edgar M.
     \emph{Graphical enumeration}
 Academic Press, New York, 1973, pp. xiv+271, 05C30, \MR{MR0357214} (50 \#9682)


\end{thebibliography}
%%%%%
%%%%%
\end{document}
