\documentclass[12pt]{article}
\usepackage{pmmeta}
\pmcanonicalname{EulerCircuit}
\pmcreated{2013-03-22 12:02:06}
\pmmodified{2013-03-22 12:02:06}
\pmowner{CWoo}{3771}
\pmmodifier{CWoo}{3771}
\pmtitle{Euler circuit}
\pmrecord{12}{31044}
\pmprivacy{1}
\pmauthor{CWoo}{3771}
\pmtype{Definition}
\pmcomment{trigger rebuild}
\pmclassification{msc}{05C45}
\pmsynonym{Euler cycle}{EulerCircuit}
%\pmkeywords{Euler path}
%\pmkeywords{Euler circuit}
\pmrelated{EulerPath}
\pmrelated{FleurysAlgorithm}

\endmetadata

\usepackage{amssymb}
\usepackage{amsmath}
\usepackage{amsfonts}
\usepackage{graphicx}
%%%\usepackage{xypic}
\begin{document}
An Euler circuit is a connected graph such that starting at a vertex $a$, one can traverse along every edge of the graph once to each of the other vertices and return to vertex $a$. In other words, an Euler circuit is an Euler path that is a circuit. Thus, using the properties of odd and even \PMlinkid{degree}{788} vertices given in the definition of an Euler path, an Euler circuit exists if and only if every vertex of the graph has an even degree.

\begin{center}
  \includegraphics[width=1in,height=1in]{ecircuit}
\end{center}

This graph is an Euler circuit as all vertices have degree 2.

\begin{center}
\includegraphics[width=1in,height=1in]{epath}
\end{center}

This graph is not an Euler circuit.
%%%%%
%%%%%
%%%%%
\end{document}
