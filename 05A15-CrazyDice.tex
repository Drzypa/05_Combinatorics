\documentclass[12pt]{article}
\usepackage{pmmeta}
\pmcanonicalname{CrazyDice}
\pmcreated{2013-03-22 15:01:43}
\pmmodified{2013-03-22 15:01:43}
\pmowner{mathcam}{2727}
\pmmodifier{mathcam}{2727}
\pmtitle{crazy dice}
\pmrecord{7}{36738}
\pmprivacy{1}
\pmauthor{mathcam}{2727}
\pmtype{Definition}
\pmcomment{trigger rebuild}
\pmclassification{msc}{05A15}

\endmetadata

% this is the default PlanetMath preamble.  as your knowledge
% of TeX increases, you will probably want to edit this, but
% it should be fine as is for beginners.

% almost certainly you want these
\usepackage{amssymb}
\usepackage{amsmath}
\usepackage{amsfonts}
\usepackage{amsthm}

% used for TeXing text within eps files
%\usepackage{psfrag}
% need this for including graphics (\includegraphics)
%\usepackage{graphicx}
% for neatly defining theorems and propositions
%\usepackage{amsthm}
% making logically defined graphics
%%%\usepackage{xypic}

% there are many more packages, add them here as you need them

% define commands here

\newcommand{\mc}{\mathcal}
\newcommand{\mb}{\mathbb}
\newcommand{\mf}{\mathfrak}
\newcommand{\ol}{\overline}
\newcommand{\ra}{\rightarrow}
\newcommand{\la}{\leftarrow}
\newcommand{\La}{\Leftarrow}
\newcommand{\Ra}{\Rightarrow}
\newcommand{\nor}{\vartriangleleft}
\newcommand{\Gal}{\text{Gal}}
\newcommand{\GL}{\text{GL}}
\newcommand{\Z}{\mb{Z}}
\newcommand{\R}{\mb{R}}
\newcommand{\Q}{\mb{Q}}
\newcommand{\C}{\mb{C}}
\newcommand{\<}{\langle}
\renewcommand{\>}{\rangle}
\begin{document}
It is a standard exercise in elementary combinatorics to \PMlinkescapetext{calculate} the number of ways of rolling any given value with 2 fair 6-sided dice (by taking the \PMlinkescapetext{sum} of the two rolls).  The below table \PMlinkescapetext{enumerates} the number of such ways of rolling a given value $n$:

\begin{center}
\begin{tabular}{|c|c|}\hline
$n$&\# of ways\\
2&1\\
3&2\\
4&3\\
5&4\\
6&5\\
7&6\\
8&5\\
9&4\\
10&3\\
11&2\\
12&1\\
\end{tabular}
\end{center}

A somewhat (un?)natural question is to ask whether or not there are any other ways of re-labeling the faces of the dice with positive integers that \PMlinkescapetext{generate} these sums with the same frequencies.  The surprising answer to this question is that there does indeed exist such a re-labeling, via the labeling
\begin{align*}
\mbox{Die }1&=\{1,2,2,3,3,4\}\\
\mbox{Die }2&=\{1,3,4,5,6,8\}\\
\end{align*}
and a pair of dice with this labeling are called a set of \emph{crazy dice}.  It is straight-forward to verify that the various possible \PMlinkescapetext{sums} occur with the same frequencies as given by the above table.
%%%%%
%%%%%
\end{document}
