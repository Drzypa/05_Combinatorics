\documentclass[12pt]{article}
\usepackage{pmmeta}
\pmcanonicalname{InternalNodeofATree}
\pmcreated{2013-03-22 12:30:25}
\pmmodified{2013-03-22 12:30:25}
\pmowner{akrowne}{2}
\pmmodifier{akrowne}{2}
\pmtitle{internal node (of a tree)}
\pmrecord{6}{32740}
\pmprivacy{1}
\pmauthor{akrowne}{2}
\pmtype{Definition}
\pmcomment{trigger rebuild}
\pmclassification{msc}{05C05}
\pmsynonym{internal node}{InternalNodeofATree}

\endmetadata

\usepackage{amssymb}
\usepackage{amsmath}
\usepackage{amsfonts}

%\usepackage{psfrag}
%\usepackage{graphicx}
\usepackage{color}
%%\usepackage{xypic}
\xyoption{all}
\begin{document}
An \emph{internal node} of a tree is any node which has degree greater than one.  Or, phrased in rooted tree terminology, the internal nodes of a tree are the nodes which have at least one child node.

\begin{center}

$$\xymatrix{
& {\color{red}\bullet} \ar@{-}[dl] \ar@{-}[dr] &  &  & \\
\bullet &  & {\color{red}\bullet} \ar@{-}[dr]\ar@{-}[dl] &  & \\
& {\color{red}\bullet} \ar@{-}[dl] &  & \bullet  & \\
\bullet &  &  &   & }$$

{\tiny Figure: A tree with internal nodes highlighted in red.}
\end{center}
%%%%%
%%%%%
\end{document}
