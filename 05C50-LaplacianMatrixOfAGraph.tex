\documentclass[12pt]{article}
\usepackage{pmmeta}
\pmcanonicalname{LaplacianMatrixOfAGraph}
\pmcreated{2013-03-22 17:04:40}
\pmmodified{2013-03-22 17:04:40}
\pmowner{Mathprof}{13753}
\pmmodifier{Mathprof}{13753}
\pmtitle{Laplacian matrix of a graph}
\pmrecord{8}{39371}
\pmprivacy{1}
\pmauthor{Mathprof}{13753}
\pmtype{Definition}
\pmcomment{trigger rebuild}
\pmclassification{msc}{05C50}
\pmdefines{Laplacian matrix}

% this is the default PlanetMath preamble.  as your knowledge
% of TeX increases, you will probably want to edit this, but
% it should be fine as is for beginners.

% almost certainly you want these
\usepackage{amssymb}
\usepackage{amsmath}
\usepackage{amsfonts}

% used for TeXing text within eps files
%\usepackage{psfrag}
% need this for including graphics (\includegraphics)
%\usepackage{graphicx}
% for neatly defining theorems and propositions
%\usepackage{amsthm}
% making logically defined graphics
%%%\usepackage{xypic}

% there are many more packages, add them here as you need them

% define commands here

\begin{document}
Let $G$ be a finite graph with $n$ vectices and let $D$ be the \PMlinkname{incidence matrix}{
IncidenceMatrixWithRespectToAnOrientation} of $G$ with respect to some orientation.
The \emph{Laplacian matrix} of $G$ is defined to be $DD^T$.

If we let $A$ be the adjacency matrix of $G$ then it can be shown
that $DD^T = \Delta - A$,
where $\Delta=\textrm{diag}(\delta_1, \ldots, \delta_n)$ and 
$\delta_i$ is the degree of
the vertex $v_i$. As a result, the Laplacian matrix is independent of what orientation is
chosen for $G$. 

The  Laplacian matrix is usually denoted by $L(G)$. It is a positive semidefinite
singular matrix, so that the   smallest eigenvalue is 0.
%%%%%
%%%%%
\end{document}
