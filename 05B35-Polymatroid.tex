\documentclass[12pt]{article}
\usepackage{pmmeta}
\pmcanonicalname{Polymatroid}
\pmcreated{2013-03-22 13:56:43}
\pmmodified{2013-03-22 13:56:43}
\pmowner{mathcam}{2727}
\pmmodifier{mathcam}{2727}
\pmtitle{polymatroid}
\pmrecord{4}{34707}
\pmprivacy{1}
\pmauthor{mathcam}{2727}
\pmtype{Definition}
\pmcomment{trigger rebuild}
\pmclassification{msc}{05B35}

\endmetadata

\usepackage{amssymb}
\usepackage{amsmath}
\usepackage{amsfonts}
\begin{document}
The \emph{polymatroid} defined by a given matroid $(E,r)$ is the set of
all functions $w:E\to\mathbb{R}$ such that
$$w(e)\ge 0\qquad\text{for all }e\in E$$
$$\sum_{e\in S}w(e)\le r(S)\qquad\text{for all }S\subset E\;.$$
Polymatroids are related to the convex polytopes seen in linear programming,
and have similar uses.
%%%%%
%%%%%
\end{document}
