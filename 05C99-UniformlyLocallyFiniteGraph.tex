\documentclass[12pt]{article}
\usepackage{pmmeta}
\pmcanonicalname{UniformlyLocallyFiniteGraph}
\pmcreated{2013-03-22 16:00:54}
\pmmodified{2013-03-22 16:00:54}
\pmowner{sjm1979}{13837}
\pmmodifier{sjm1979}{13837}
\pmtitle{uniformly locally finite graph}
\pmrecord{6}{38050}
\pmprivacy{1}
\pmauthor{sjm1979}{13837}
\pmtype{Definition}
\pmcomment{trigger rebuild}
\pmclassification{msc}{05C99}
\pmrelated{LocallyFiniteGraph}

\endmetadata

% this is the default PlanetMath preamble.  as your knowledge
% of TeX increases, you will probably want to edit this, but
% it should be fine as is for beginners.

% almost certainly you want these
\usepackage{amssymb}
\usepackage{amsmath}
\usepackage{amsfonts}

% used for TeXing text within eps files
%\usepackage{psfrag}
% need this for including graphics (\includegraphics)
%\usepackage{graphicx}
% for neatly defining theorems and propositions
%\usepackage{amsthm}
% making logically defined graphics
%%%\usepackage{xypic}

% there are many more packages, add them here as you need them

% define commands here

\begin{document}
A \PMlinkescapetext{{\sl uniformly locally finite graph\/}} is a \PMlinkname{locally finite graph}{LocallyFiniteGraph} $ (V,E) $ such that there exists an $ M \in \mathbb{N} $ such that for every $ x \in V $ we have that the degree of $ x $, also denoted $ \rho(x) $, is at most $ M $.  In other words there exists an $ M \in \mathbb{N} $ such that for every $ x \in V \, , \, \rho(x) \le M $.  

Note that the examples provided in \PMlinkname{locally finite graph}{LocallyFiniteGraph} are also examples of a uniformly locally finite graph since both graphs are \PMlinkname{regular}{RegularGraph} and have finite \PMlinkname{degree}{Degree7} at each vertex.
%%%%%
%%%%%
\end{document}
