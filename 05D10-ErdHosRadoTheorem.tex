\documentclass[12pt]{article}
\usepackage{pmmeta}
\pmcanonicalname{ErdHosRadoTheorem}
\pmcreated{2013-03-22 12:59:30}
\pmmodified{2013-03-22 12:59:30}
\pmowner{Henry}{455}
\pmmodifier{Henry}{455}
\pmtitle{Erd\H{o}s-Rado theorem}
\pmrecord{9}{33366}
\pmprivacy{1}
\pmauthor{Henry}{455}
\pmtype{Theorem}
\pmcomment{trigger rebuild}
\pmclassification{msc}{05D10}
\pmclassification{msc}{03E05}
\pmrelated{arrowsrelation}
\pmrelated{ArrowsRelation}

% this is the default PlanetMath preamble.  as your knowledge
% of TeX increases, you will probably want to edit this, but
% it should be fine as is for beginners.

% almost certainly you want these
\usepackage{amssymb}
\usepackage{amsmath}
\usepackage{amsfonts}

% used for TeXing text within eps files
%\usepackage{psfrag}
% need this for including graphics (\includegraphics)
%\usepackage{graphicx}
% for neatly defining theorems and propositions
%\usepackage{amsthm}
% making logically defined graphics
%%%\usepackage{xypic}

% there are many more packages, add them here as you need them

% define commands here
%\PMlinkescapeword{theory}
\begin{document}
Repeated exponentiation for cardinals is denoted $\operatorname{exp}_i(\kappa)$, where $i<\omega$.  It is defined by:

$$\operatorname{exp}_0(\kappa)=\kappa$$

and

$$\operatorname{exp}_{i+1}(\kappa)=2^{\operatorname{exp}_{i}(\kappa)}$$

The Erd\H{o}s-Rado theorem states that:

$$\operatorname{exp}_i(\kappa)^+\rightarrow(\kappa^+)^{i+1}_\kappa$$

That is, if $f:[\operatorname{exp}_i(\kappa)^+]^{i+1}\rightarrow\kappa$ then there is a homogeneous set of size $\kappa^+$.

As special cases, $(2^\kappa)^+\rightarrow(\kappa^+)^2_\kappa$ and $(2^{\aleph_0})^+\rightarrow(\aleph_1)^2_{\aleph_0}$.
%%%%%
%%%%%
\end{document}
