\documentclass[12pt]{article}
\usepackage{pmmeta}
\pmcanonicalname{CompleteGraph}
\pmcreated{2013-03-22 12:16:57}
\pmmodified{2013-03-22 12:16:57}
\pmowner{vampyr}{22}
\pmmodifier{vampyr}{22}
\pmtitle{complete graph}
\pmrecord{10}{31757}
\pmprivacy{1}
\pmauthor{vampyr}{22}
\pmtype{Definition}
\pmcomment{trigger rebuild}
\pmclassification{msc}{05C99}
\pmsynonym{complete}{CompleteGraph}
\pmsynonym{clique}{CompleteGraph}
\pmrelated{Tournament}

\endmetadata

% this is the default PlanetMath preamble.  as your knowledge
% of TeX increases, you will probably want to edit this, but
% it should be fine as is for beginners.

% almost certainly you want these
\usepackage{amssymb}
\usepackage{amsmath}
\usepackage{amsfonts}

% used for TeXing text within eps files
%\usepackage{psfrag}
% need this for including graphics (\includegraphics)
\usepackage{graphicx}
% for neatly defining theorems and propositions
%\usepackage{amsthm}
% making logically defined graphics
%%%%\usepackage{xypic} 

% there are many more packages, add them here as you need them

% define commands here
\begin{document}
The \emph{complete graph} with $n$ vertices, denoted $K_n$, contains all possible edges; that is, any two vertices are adjacent.

The complete graph of $4$ vertices, or $K_4$ looks like this:

\begin{center}
\includegraphics[scale=1.0]{k4.eps}
\end{center}

The number of edges in $K_n$ is the $n-1$th triangular number.  Every vertex in $K_n$ has degree $n-1$; therefore $K_n$ has an Euler circuit if and only if $n$ is odd.  A complete graph always has a Hamiltonian path, and the chromatic number of $K_n$ is always $n$.
%%%%%
%%%%%
%%%%%
\end{document}
