\documentclass[12pt]{article}
\usepackage{pmmeta}
\pmcanonicalname{ProofOfWagnersTheorem}
\pmcreated{2014-01-16 17:37:41}
\pmmodified{2014-01-16 17:37:41}
\pmowner{Ziosilvio}{18733}
\pmmodifier{Ziosilvio}{18733}
\pmtitle{proof of Wagner's theorem}
\pmrecord{14}{40551}
\pmprivacy{1}
\pmauthor{Ziosilvio}{18733}
\pmtype{Proof}
\pmcomment{trigger rebuild}
\pmclassification{msc}{05C99}

\endmetadata

% this is the default PlanetMath preamble.  as your knowledge
% of TeX increases, you will probably want to edit this, but
% it should be fine as is for beginners.

% almost certainly you want these
\usepackage{amssymb}
\usepackage{amsmath}
\usepackage{amsfonts}

% used for TeXing text within eps files
%\usepackage{psfrag}
% need this for including graphics (\includegraphics)
%\usepackage{graphicx}
% for neatly defining theorems and propositions
%\usepackage{amsthm}
% making logically defined graphics
%%%\usepackage{xypic}

% there are many more packages, add them here as you need them

% define commands here
\newcommand{\ie}{\textit{i.e.}}
\begin{document}
It is sufficient to prove that
the planarity condition given by Wagner's theorem
is equivalent to the one given by Kuratowski's theorem,
\ie, that a graph $G=(V,E)$
has $K_{5}$ or $K_{3,3}$ as a minor,
if and only if it has a subgraph
homeomorphic to either $K_{5}$ or $K_{3,3}$.
It is not restrictive to suppose that $G$ is simple and 2-connected.

First, suppose that $G$ has a subgraph
homeomorphic to either $K_{5}$ or $K_{3,3}$.
Then there exists $U\subseteq V$ such that the subgraph induced by $U$
can be transformed into either $K_{5}$ or $K_{3,3}$
via a sequence of simple subdivisions
and simple contractions through vertices of degree 2.
Since none of these operations
can alter the number of vertices of degree $d\neq 2$,
and neither $K_{5}$ nor $K_{3,3}$ have vertices of degree 2,
none of the simple subdivisions is necessary,
and the homeomorphic subgraph is actually a minor.

For the other direction, we prove the following.
\begin{enumerate}
\item \label{item:k33}
If $G$ has $K_{3,3}$ as a minor,
then $G$ has a subgraph homeomorphic to $K_{3,3}$.
\item \label{item:k5}
If $G$ has $K_{5}$ as a minor,
then $G$ has a subgraph homeomorphic to either $K_{5}$ or $K_{3,3}$.
\end{enumerate}

\textit{Proof of point~\ref{item:k33}}

If $G$ has $K_{3,3}$ as a minor,
then there exist $U_{1},U_{2},U_{3},W_{1},W_{2},W_{3}\subseteq V$
that are pairwise disjoint,
induce connected subgraphs of $G$,
and such that, for each $i$ and $j$,
there exist $u_{i,j}\in U_{i}$ and $w_{i,j}\in W_{j}$
such that $(u_{i,j},w_{i,j})\in E$.
Consequently, for each $i$, there exists a subtree of $G$ with three leaves,
one leaf in each of the $W_{j}$'s,
and all of its other nodes inside $U_{i}$;
the situation with the $j$'s is symmetrical.

As a consequence of the handshake lemma,
a tree with three leaves is homeomorphic to $K_{1,3}$.
Thus, $G$ has a subgraph homeomorphic to six copies of $K_{1,3}$
connected three by three,
\ie, to $K_{3,3}$.

\textit{Proof of point~\ref{item:k5}}

If $G$ has $K_{5}$ as a minor,
then there exist pairwise disjoint $U_{1},\ldots,U_{5}\subseteq V$
that induce connected subgraphs of $G$
and such that, for every $i\neq j$,
there exist $u_{i;\{i,j\}}\in U_{i}$ and $u_{j;\{i,j\}}\in U_{j}$
such that $(u_{i;\{i,j\}},u_{j;\{i,j\}})\in E$.
Consequently, for each $i$,
there exists a subtree $T_i$ of $G$ with four leaves,
one leaf in each of the $U_{j}$'s for $i\neq j$,
and with all of its other nodes inside $U_{i}$.

As a consequence of the handshake lemma,
a tree with four leaves is homeomorphic to either $K_{1,4}$
or two joint copies of $K_{1,3}$.
If all of the trees above are homeomorphic to $K_{1,4}$,
then $G$ has a subgraph homeomorphic to
five copies of $K_{1,4}$, each joint to the others:
\ie, to $K_5$.
Otherwise, a subgraph homeomorphic to $K_{3,3}$
can be obtained via the following procedure.
\begin{enumerate}
\item
Choose one of the $T_i$'s
which is homeomorphic to two joint copies of $K_{1,3}$,
call them $T_{i,r}$ and $T_{i,b}$.
\item
Color red the nodes of $T_{i,r}$,
except its two leaves, which are colored blue.
\item
Color blue the nodes of $T_{i,b}$,
except its two leaves, which are colored red.
\item
Color blue the nodes of the $T_j$'s
containing the leaves of $T_{i,r}$.
\item
Color red the nodes of the $T_j$'s
containing the leaves of $T_{i,b}$.
\item
Remove the edges joining nodes with same color in different $T_j$'s.
\\
This ``prunes'' the $T_j$'s so that they have three leaves,
each in a subgraph of a color different than the rest of their vertices.
\end{enumerate}
The graph formed by the red and blue nodes,
together with the remaining edges,
is then isomorphic to $K_{3,3}$.

\begin{thebibliography}{99}

\bibitem{ag06}
Geir Agnarsson, Raymond Greenlaw.
\textit{Graph Theory: Modeling, Applications and Algorithms.}
Prentice Hall, 2006.

\end{thebibliography}
%%%%%
%%%%%
\end{document}
