\documentclass[12pt]{article}
\usepackage{pmmeta}
\pmcanonicalname{ExampleOfAProbabilisticProof}
\pmcreated{2013-03-22 13:06:18}
\pmmodified{2013-03-22 13:06:18}
\pmowner{bbukh}{348}
\pmmodifier{bbukh}{348}
\pmtitle{example of a probabilistic proof}
\pmrecord{13}{33530}
\pmprivacy{1}
\pmauthor{bbukh}{348}
\pmtype{Example}
\pmcomment{trigger rebuild}
\pmclassification{msc}{05C80}

\usepackage{amssymb, amsmath, amsthm, alltt, setspace}
\newtheorem{thm}{Theorem}

\theoremstyle{definition}
\newtheorem*{defn}{Definition}
\theoremstyle{definition}
\newtheorem*{rem}{Remark}

\theoremstyle{definition}
\newtheorem*{nott}{Notation}

\newtheorem{lemma}{Lemma}
\newtheorem{cor}{Corollary}
\newtheorem*{eg}{Example}
\newtheorem*{ex}{Exercise}
\newtheorem*{prop}{Proposition}


\newcommand{\RR}{\mathbb{R}}
\newcommand{\QQ}{\mathbb{Q}}
\newcommand{\ZZ}{\mathbb{Z}}
\newcommand{\NN}{\mathbb{N}}
\newcommand{\leftbb}{[ \! [}
\newcommand{\rightbb}{] \! ]}
\newcommand{\bt}{\begin{thm}}
\newcommand{\et}{\end{thm}}
\newcommand{\Rel}{\mathbf{R}}
\newcommand{\er}{\thicksim}
\newcommand{\sqle}{\sqsubseteq}
\newcommand{\floor}[1]{\lfloor{#1}\rfloor}
\newcommand{\ceil}[1]{\lceil{#1}\rceil}
\begin{document}
\PMlinkescapeword{arrow}
\PMlinkescapeword{size}
\PMlinkescapeword{fixed}

Our example is the existence of $k$-paradoxical tournaments.  The proof hinges upon the following basic probabilistic inequality, for any events $A$ and $B$,
\[
P\left(A \cup B \right) \leq P(A) + P(B)
\]
\begin{thm}
For every $k$, there exists a $k$-paradoxical tournament.
\end{thm}
\begin{proof}
We will construct a tournament $T$ on $n$ vertices.  We will show that for $n$ large enough, a $k$-paradoxical tournament must exist.  The probability space in question is all possible directions of the arrows of $T$, where each arrow can point in either direction with probability $1/2$, independently of any other arrow.

We say that a set $K$ of $k$ vertices is \emph{arrowed} by a vertex $v_0$ outside  the set if every arrow between $v_0$ to $w_i \in K$ points from $v_0$ to $w_i$, for $i = 1, ~\ldots, k$.  Consider a fixed set $K$ of $k$ vertices and a fixed vertex $v_0$ outside $K$.  Thus, there are $k$ arrows from $v_0$ to $K$, and only one arrangement of these arrows permits $K$ to be arrowed by $v_0$, thus
\[
P(K ~\mbox{ is arrowed by }~ v_0) = \frac{1}{2^k}.
\]
The complementary event, is therefore,
\[
P(K ~\mbox{ is \emph{not} arrowed by }~ v_0) = 1 - \frac{1}{2^k}.
\]
By independence, and because there are $n - k$ vertices outside of $K$,
\begin{equation}
\label{kk}
P(K ~\mbox{ is not arrowed by \emph{any} vertex}) = \left(1 - \frac{1}{2^k}\right)^{n-k}.
\end{equation}
Lastly, since there are $\binom{n}{k}$ sets of cardinality $k$ in $T$, we employ the inequality mentioned above to obtain that for the union of all events of the form in equation (\ref{kk})
\[
P(\text{Some set of $k$ vertices is not arrowed by any vertex}) \leq \binom{n}{k}\left(1 - \frac{1}{2^k}\right)^{n-k}.
\]
If the probability of this last event is less than $1$ for some $n$, then there must exist a $k$-paradoxical tournament of $n$ vertices.  Indeed there is such an $n$, since
\begin{eqnarray*}
\binom{n}{k}\left(1 - \frac{1}{2^k}
            \right)^{n-k} &=&  \frac{1}{k!} n(n-1)
                               \cdots (n-k+1) \left(1 - \frac{1}{2^k}
                               \right)^{n-k} \\
                          &<& \frac{1}{k!} n^k \left(1 - \frac{1}{2^k}
                               \right)^{n-k} 
\end{eqnarray*}

Therefore, regarding $k$ as fixed while $n$ tends to infinity, the right-hand-side above tends to zero.  In particular, for some $n$ it is less than $1$, and the result follows.
\end{proof}
%%%%%
%%%%%
\end{document}
