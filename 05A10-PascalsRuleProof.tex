\documentclass[12pt]{article}
\usepackage{pmmeta}
\pmcanonicalname{PascalsRuleProof}
\pmcreated{2013-03-22 11:47:14}
\pmmodified{2013-03-22 11:47:14}
\pmowner{akrowne}{2}
\pmmodifier{akrowne}{2}
\pmtitle{Pascal's rule proof}
\pmrecord{10}{30259}
\pmprivacy{1}
\pmauthor{akrowne}{2}
\pmtype{Proof}
\pmcomment{trigger rebuild}
\pmclassification{msc}{05A10}
\pmclassification{msc}{81T13}
\pmclassification{msc}{53C80}
\pmclassification{msc}{82-00}
\pmclassification{msc}{83-00}
\pmclassification{msc}{81-00}

\usepackage{amssymb}
\usepackage{amsmath}
\usepackage{amsfonts}
\usepackage{graphicx}
%%%%\usepackage{xypic}
\begin{document}
We need to show 
\begin{eqnarray*}
\binom{n}{k} + \binom{n}{k-1} & = & \binom{n+1}{k}
\end{eqnarray*}
Let us begin by writing the left-hand side as $$ \frac{n!}{k!(n-k)!} + \frac{n!}{(k-1)!(n-(k-1))!}$$
Getting a common denominator and simplifying, we have 
\begin{eqnarray*}
\frac{n!}{k!(n-k)!} + \frac{n!}{(k-1)!(n-k+1)!} & = & \frac{(n-k+1)n!}{(n-k+1)k!(n-k)!}+\frac{kn!}{k(k-1)!(n-k+1)!} \\ 
& = & \frac{(n-k+1)n!+kn!}{k!(n-k+1)!} \\
& = & \frac{(n+1)n!}{k!((n+1)-k)!} \\
& = & \frac{(n+1)!}{k!((n+1)-k)!} \\
& = & \binom{n+1}{k}
\end{eqnarray*}
%%%%%
%%%%%
%%%%%
%%%%%
\end{document}
