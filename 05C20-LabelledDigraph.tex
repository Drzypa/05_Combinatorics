\documentclass[12pt]{article}
\usepackage{pmmeta}
\pmcanonicalname{LabelledDigraph}
\pmcreated{2013-03-22 15:15:07}
\pmmodified{2013-03-22 15:15:07}
\pmowner{GrafZahl}{9234}
\pmmodifier{GrafZahl}{9234}
\pmtitle{labelled digraph}
\pmrecord{4}{37033}
\pmprivacy{1}
\pmauthor{GrafZahl}{9234}
\pmtype{Definition}
\pmcomment{trigger rebuild}
\pmclassification{msc}{05C20}
\pmclassification{msc}{05C12}
\pmclassification{msc}{05C78}
\pmclassification{msc}{05C90}
%\pmkeywords{graph}
%\pmkeywords{travelling salesman}
%\pmkeywords{TSP}
\pmdefines{label}
\pmdefines{weighted digraph}
\pmdefines{weight}
\pmdefines{vertex-weighted digraph}
\pmdefines{edge-weighted digraph}

\endmetadata

% this is the default PlanetMath preamble.  as your knowledge
% of TeX increases, you will probably want to edit this, but
% it should be fine as is for beginners.

% almost certainly you want these
\usepackage{amssymb}
\usepackage{amsmath}
\usepackage{amsfonts}

% used for TeXing text within eps files
%\usepackage{psfrag}
% need this for including graphics (\includegraphics)
%\usepackage{graphicx}
% for neatly defining theorems and propositions
%\usepackage{amsthm}
% making logically defined graphics
%%%\usepackage{xypic}

% there are many more packages, add them here as you need them

% define commands here
\newcommand{\Prod}{\prod\limits}
\newcommand{\Sum}{\sum\limits}
\newcommand{\mbb}{\mathbb}
\newcommand{\mbf}{\mathbf}
\newcommand{\mc}{\mathcal}
\newcommand{\ol}{\overline}

% Math Operators/functions
\DeclareMathOperator{\Frob}{Frob}
\DeclareMathOperator{\cwe}{cwe}
\DeclareMathOperator{\we}{we}
\DeclareMathOperator{\wt}{wt}
\begin{document}
\PMlinkescapeword{time}
\PMlinkescapeword{times}
\PMlinkescapeword{complete}
\PMlinkescapeword{size}
\PMlinkescapeword{sizes}
\PMlinkescapeword{associates}
A triple $(V,E,w)$ is called a \emph{labelled digraph}, if $(V,E)$ is a
digraph and $w$ is an association of elements from some set $L$, the
\emph{labels}, to some of the edges and vertices of the digraph. In
other words, $w$ is a mapping from a subset $A\subseteq V\cup E$ to
$L$. Most often, $L$ is a subset of the real numbers, in which case
$(V,E,w)$ is called a \emph{weighted} digraph and its labels are
called \emph{weights}. Typically, either $A=V$ or $A=E$, in which case
$(V,E,w)$ is called either a \emph{vertex-weighted digraph} or an
\emph{edge-weighted digraph}, respectively.

\subsection*{Application examples}

We give two typical ``real life'' examples. The first features an
edge-weighted digraph, while the second requires the implementation of
a vertex-weighted digraph.

\subsubsection*{Railway network}

A railway network consists of railway stations connected by rails. A
train needs a certain time (measured in minutes) to fare from one
station to another. In a formalisation, $V$ is the set of train
stations, $E$ the set of direct connexions between them and $w\colon
E\to\mbb{N}$ a weighting corresponding to the journey times, so
$(V,E,w)$ is an edge-weighted digraph. Although typically $(V,E)$ is a
\PMlinkid{symmetric digraph}{1702}, $w$ does not need to be symmetric: for example, the
journey from $a$ to $b$ might take longer than the return journey
because $b$ is located on a mountain.

An important optimisation problem is the efficient determination of
the fastest way from one station to another. An even harder problem is
to find the fastest round trip (usually called a \emph{tour}) via a
given number of stations. This is the \emph{travelling salesman
problem}.

\subsubsection*{Dependency graph}

A software bundle consists of a number of packages each of which is
either installed or not. An installed package occupies a certain
amount of bytes on a storage medium. Packages may depend on other
packages, that is installation of a package may require other packages
to be installed first, which in turn may require still other packages
and so forth. One is interested in the complete storage requirement
incurred by the installation of one package and all its dependencies.

In a formalisation, the packages are vertices of a digraph $(V,E)$, and an
edge $(a,b)\in E$ means ``$a$ depends on $b$''. Such a digraph is typically
not symmetric. The weighting $w\colon V\to\mbb{N}$ associates sizes to
packages. A subset $W$ of $V$ is \emph{dependency-closed}, if for any
$w\in W$, all dependencies of $w$ are in $W$. Given a to-be-installed
package $v$, the storage requirement incurred by the installation of
$v$ and all its dependencies is the sum of the vertex weights of the
smallest dependency-closed subset of $V$ containing $v$.
%%%%%
%%%%%
\end{document}
