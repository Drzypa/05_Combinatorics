\documentclass[12pt]{article}
\usepackage{pmmeta}
\pmcanonicalname{SpanningTree}
\pmcreated{2013-03-22 12:29:19}
\pmmodified{2013-03-22 12:29:19}
\pmowner{mathcam}{2727}
\pmmodifier{mathcam}{2727}
\pmtitle{spanning tree}
\pmrecord{6}{32709}
\pmprivacy{1}
\pmauthor{mathcam}{2727}
\pmtype{Definition}
\pmcomment{trigger rebuild}
\pmclassification{msc}{05C05}
\pmrelated{Tree}
\pmrelated{Graph}
\pmrelated{MinimumSpanningTree}
\pmrelated{DepthFirstSearch2}
\pmrelated{DepthFirstSearch}

\usepackage{amssymb}
\usepackage{amsmath}
\usepackage{amsfonts}
\usepackage[all]{xy}
\begin{document}
A \emph{spanning tree} of a (connected) graph $G$ is a connected, acyclic subgraph of $G$ that contains all of the vertices of $G$.  Below is an example of a spanning tree $T$, where the edges in $T$ are drawn as solid lines and the edges in $G$ but not in $T$ are drawn as dotted lines.

$$
\xymatrix{
\bullet \ar@{.}[dddd] \ar@{-}[ddr] \ar@{.}[rrr] &&& \bullet \ar@{.}[ddll] \ar@{-}[dd] \ar@{.}[drr] \\
&&&&&\bullet \ar@{-}[dll] \ar@{.}[d] \\
&\bullet\ar@{-}[ddl]\ar@{-}[dr]\ar@{.}[rr] && \bullet\ar@{-}[dl]\ar@{.}[dr]\ar@{-}[rr] &&\bullet\ar@{-}[dl] \\
&&\bullet\ar@{.}[dll]\ar@{-}[dr]\ar@{.}[rr]&&\bullet\ar@{.}[dl] \\
\bullet\ar@{.}[rrr]&&&\bullet
}
$$

For any tree there is exactly one spanning tree: the tree itself.
%%%%%
%%%%%
\end{document}
