\documentclass[12pt]{article}
\usepackage{pmmeta}
\pmcanonicalname{StirlingPolynomial}
\pmcreated{2013-03-22 15:38:36}
\pmmodified{2013-03-22 15:38:36}
\pmowner{kronos}{12218}
\pmmodifier{kronos}{12218}
\pmtitle{Stirling polynomial}
\pmrecord{9}{37575}
\pmprivacy{1}
\pmauthor{kronos}{12218}
\pmtype{Definition}
\pmcomment{trigger rebuild}
\pmclassification{msc}{05A15}
%\pmkeywords{Stirling}
%\pmkeywords{Bernoulli}

% this is the default PlanetMath preamble.  as your knowledge
% of TeX increases, you will probably want to edit this, but
% it should be fine as is for beginners.

% almost certainly you want these
\usepackage{amssymb}
\usepackage{amsmath}
\usepackage{amsfonts}

% used for TeXing text within eps files
%\usepackage{psfrag}
% need this for including graphics (\includegraphics)
%\usepackage{graphicx}
% for neatly defining theorems and propositions
%\usepackage{amsthm}
% making logically defined graphics
%%%\usepackage{xypic}

% there are many more packages, add them here as you need them

% define commands here
\begin{document}
\emph{Stirling's polynomials}  $S_k(x)$ are defined by the generating function $$\left( {t \over {1-e^{-t}}} \right) ^{x+1}= \sum_{k=0} {S_k(x) \over k!} t^k.$$

The sequence $S_k(x-1)$ is of binomial type, since $S_k(x+y-1)= \sum_{i=0}^k {k \choose i} S_i(x-1) S_{k-i}(y-1)$. Moreover, this basic recursion holds: $S_k(x)= (x-k) {S_k(x-1) \over x} + k S_{k-1}(x+1)$. 

These are the first polynomials:
\begin{enumerate}
 \item $S_0(x)=1$;
 \item $S_1(x)= {1 \over 2}(x+1)$;
 \item $S_2(x)= {1 \over 12} (3 x^2 + 5x+ 2)$;
 \item $S_3(x)= {1 \over 8} (x^3+2 x^2+x)$;
 \item $S_4(x)= {1 \over 240} (15 x^4+ 30 x^3+5 x^2- 18x-8)$.
\end{enumerate}

In addition we have these special values:
\begin{enumerate}
 \item $S_k(-m)= {(-1)^k \over {k+m-1 \choose k}} S_{k+m-1,m-1}$, where $S_{m,n}$ denotes Stirling numbers of the second kind. Conversely, $S_{n,m}=(-1)^{n-m} {n \choose m} S_{n-m}(-m-1)$;
 \item $S_k(-1)= \delta_{k,0}$;
 \item $S_k(0)= (-1)^k B_k$, where $B_k$ are Bernoulli's numbers;
 \item $S_k(1)= (-1)^{k+1} ((k-1) B_k+ k B_{k-1})$;
 \item $S_k(2)= {(-1)^{k}\over 2} ((k-1)(k-2) B_k+ 3 k(k-2) B_{k-1}+ 2 k(k-1) B_{k-2})$;
 \item $S_k(k)= k!$;
 \item $S_k(m)= {(-1)^k \over {m \choose k}} s_{m+1, m+1-k}$, where $s_{m,n}$ are Stirling numbers of the first kind. They may be recovered by $s_{n,m}= (-1)^{n-m} {n-1 \choose n-m} S_{n-m}(n-1)$.
\end{enumerate}


Explicit representations involving Stirling numbers can be deduced with Lagrange's interpolation formula: $$S_k(x)= \sum_{n=0}^k (-1)^{k-n} S_{k+n,n} {{x+n \choose n} {x+k+1 \choose k-n} \over {k+n \choose n}} = \sum_{n=0}^k (-1)^n s_{k+n+1,n+1} {{x-k \choose n} {x-k-n-1 \choose k-n} \over {k+n \choose k}}.$$

These following formulae hold as well:
$${k+m \choose k} S_k(x-m)= \sum_{i=0}^k (-1)^{k-i} {k+m \choose i} S_{k-i+m,m} S_i(x),$$
$${k-m \choose k} S_k(x+m)= \sum_{i=0}^k {k-m \choose i} s_{m,m-k+i} S_i(x).$$
%%%%%
%%%%%
\end{document}
