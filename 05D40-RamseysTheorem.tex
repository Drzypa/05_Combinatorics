\documentclass[12pt]{article}
\usepackage{pmmeta}
\pmcanonicalname{RamseysTheorem}
\pmcreated{2013-03-22 13:53:09}
\pmmodified{2013-03-22 13:53:09}
\pmowner{bbukh}{348}
\pmmodifier{bbukh}{348}
\pmtitle{Ramsey's theorem}
\pmrecord{18}{34630}
\pmprivacy{1}
\pmauthor{bbukh}{348}
\pmtype{Theorem}
\pmcomment{trigger rebuild}
\pmclassification{msc}{05D40}
\pmclassification{msc}{05D10}
\pmrelated{RamseysTheorem}
\pmrelated{ProbabilisticMethod}
\pmdefines{Ramsey number}

\endmetadata

\usepackage{amssymb}
\usepackage{amsmath}
\usepackage{amsfonts}

% used for TeXing text within eps files
%\usepackage{psfrag}
% need this for including graphics (\includegraphics)
%\usepackage{graphicx}
% for neatly defining theorems and propositions
%\usepackage{amsthm}
% making logically defined graphics
%%%\usepackage{xypic}

\makeatletter
\@ifundefined{bibname}{}{\renewcommand{\bibname}{References}}
\makeatother

\newcommand*{\abs}[1]{\left\lvert #1\right\rvert}
\begin{document}
\PMlinkescapeword{states}
\PMlinkescapeword{terms}
\PMlinkescapeword{even}
\PMlinkescapeword{partition}
\PMlinkescapeword{classes}
\PMlinkescapeword{similar}
\PMlinkescapeword{argument}
\PMlinkescapeword{matching}
\PMlinkescapeword{combination}
\PMlinkescapeword{behavior}

The original version of \emph{Ramsey's theorem} states that for
every positive integers $k_1$ and $k_2$ there is $n$ such that if
edges of a complete graph on $n$ vertices are colored in two
colors, then there is either a $k_1$-clique of the first color or
$k_2$-clique of the second color.

The standard proof proceeds by induction on $k_1+k_2$. If
$k_1,k_2\leq 2$, then the theorem holds trivially. To prove
induction step we consider the graph $G$ that contains no cliques
of the desired kind, and then consider any vertex $v$ of $G$.
Partition the rest of the vertices into two classes $C_1$ and
$C_2$ according to whether edges from $v$ are of color $1$ or $2$
respectively. By inductive hypothesis $\abs{C_1}$ is bounded since
it contains no $k_1-1$-clique of color $1$ and no $k_2$-clique of
color $2$. Similarly, $\abs{C_2}$ is bounded. QED.

Similar argument shows that for any positive integers
$k_1,k_2,\dotsc,k_t$ if we color the edges of a sufficiently large
graph in $t$ colors, then we would be able to find either
$k_1$-clique of color $1$, or $k_2$-clique or color $2$,\dots, or
$k_t$-clique of color $t$.

The minimum $n$ whose existence stated in Ramsey's theorem is
called \emph{Ramsey number} and denoted by $R(k_1,k_2)$ (and
$R(k_1,k_2,\dotsc,k_t)$ for multicolored graphs). The above proof
shows that $R(k_1,k_2)\leq R(k_1,k_2-1)+R(k_1-1,k_2)$. From that
it is not hard to deduce by induction that $R(k_1,k_2)\leq
\binom{k_1+k_2-2}{k_1-1}$. In the most interesting case
$k_1=k_2=k$ this yields approximately $R(k,k)\leq (4+o(1))^k$. The
lower bounds can be established by means of probabilistic
construction as follows.

Take a complete graph on $n$ vertices and color its edges at random,
choosing the color of each edge uniformly independently of all other edges.
The probability that any given set of $k$ vertices is a
monochromatic clique is $2^{1-k}$. Let $I_r$ be the random
variable which is $1$ if $r$'th set of $k$ elements is
monochromatic clique and is $0$ otherwise. The sum $I_r$'s 
over all $k$-element sets is
simply the number of monochromatic $k$-cliques. Therefore by
linearity of expectation $E(\sum I_r)=\sum
E(I_r)=2^{1-k}\binom{n}{k}$. If the expectation is less than $1$,
then there exists a coloring which has no monochromatic cliques. A
little exercise in calculus shows that if we choose $n$ to be no more
than $(2+o(1))^{k/2}$ then the expectation is indeed less than
$1$. Hence, $R(k,k)\geq (\sqrt{2}+o(1))^{k}$.

The gap between the lower and upper bounds has been \PMlinkname{open}{Open3} for
several decades. There have been a number of improvements in
$o(1)$ terms, but nothing better than $\sqrt{2}+o(1)\leq R(k,k)^{1/k}\leq
4+o(1)$ is known. It is not even known whether $\lim_{k\to \infty}
R(k,k)^{1/k}$ exists.

The behavior of $R(k,x)$ for fixed $k$ and large $x$ is equally
mysterious. For this case Ajtai, Koml\'os and
Szemer\'edi\cite{cite:aks_ramsey} proved that $R(k,x)\leq
c_{k}x^{k-1}/(\ln)^{k-2}$. The matching lower bound has only
recently been established for $k=3$ by Kim
\cite{cite:kim_ramsey_three}. Even in this case the asymptotics is
unknown. The combination of results of Kim and improvement of
Ajtai, Koml\'os and Szemer\'edi's result by Shearer \cite{cite:shearer_independence} yields
$\tfrac{1}{162}(1+o(1)) k^2/\log k \leq R(3,k)\leq (1+o(1))k^2/\log k$.

A lot of machine and human time has been spent trying to
determine Ramsey numbers for small $k_1$ and $k_2$. An up-to-date
summary of our knowledge about small Ramsey numbers can be found
in \cite{cite:radziszowski_smallramsey}.

%\PMlinktofile{anchor}{pmath.sty}

\begin{thebibliography}{1}

\bibitem{cite:aks_ramsey}
Mikl{\'o}s Ajtai, J{\'a}nos Koml{\'o}s, and Endre Szemer{\'e}di.
\newblock A note on {Ramsey} numbers.
\newblock {\em J. Combin. Theory Ser. A}, 29(3):354--360, 1980.
\newblock \PMlinkexternal{Zbl 0455.05045}{http://www.emis.de/cgi-bin/zmen/ZMATH/en/quick.html?type=html&an=0455.05045}.

\bibitem{cite:alon_spencer_probmethod}
Noga Alon and Joel~H. Spencer.
\newblock {\em The probabilistic method}.
\newblock John Wiley \& Sons, Inc., second edition, 2000.
\newblock \PMlinkexternal{Zbl 0996.05001}{http://www.emis.de/cgi-bin/zmen/ZMATH/en/quick.html?type=html&an=0996.05001}.

\bibitem{cite:grs_ramsey_book}
Ronald~L. Graham, Bruce~L. Rothschild, and Joel~H. Spencer.
\newblock {\em Ramsey Theory}.
\newblock Wiley-Interscience series in discrete mathematics. 1980.
\newblock \PMlinkexternal{Zbl 0455.05002}{http://www.emis.de/cgi-bin/zmen/ZMATH/en/quick.html?type=html&an=0455.05002}.

\bibitem{cite:kim_ramsey_three}
Jeong~Han Kim.
\newblock The {Ramsey} number {$R(3,t)$} has order of magnitude {$t^2/\log t$}.
\newblock {\em Random Structures \& Algorithms}, 7(3):173--207, 1995.
\newblock \PMlinkexternal{Zbl 0832.05084}{http://www.emis.de/cgi-bin/zmen/ZMATH/en/quick.html?type=html&an=0832.05084}.
\newblock Preprint is \PMlinkexternal{available online}{http://research.microsoft.com/research/theory/jehkim/papers/ramsey5.pdf}.

\bibitem{cite:radziszowski_smallramsey}
Stanislaw Radziszowski.
\newblock Small Ramsey numbers.
\newblock {\em Electronic Journal of Combinatorics}, Dynamical Survey, 2002.
\newblock \PMlinkexternal{Available
  online}{http://www.combinatorics.org/Surveys/ds1.pdf}.


\bibitem{cite:shearer_independence}
James~B. Shearer.
\newblock A note on the independence number of triangle-free graphs.
\newblock {\em Discrete Mathematics}, 46:83--87, 1983.
\newblock \PMlinkexternal{Zbl 0516.05053}{http://www.emis.de/cgi-bin/zmen/ZMATH/en/quick.html?type=html&an=0516.05053}.
\newblock Abstract is \PMlinkexternal{available online}{http://www.research.ibm.com/people/s/shearer/abstracts/intfg.html}

\end{thebibliography}
%%%%%
%%%%%
\end{document}
