\documentclass[12pt]{article}
\usepackage{pmmeta}
\pmcanonicalname{MultinomialTheorem}
\pmcreated{2013-03-22 13:13:05}
\pmmodified{2013-03-22 13:13:05}
\pmowner{bshanks}{153}
\pmmodifier{bshanks}{153}
\pmtitle{multinomial theorem}
\pmrecord{12}{33683}
\pmprivacy{1}
\pmauthor{bshanks}{153}
\pmtype{Theorem}
\pmcomment{trigger rebuild}
\pmclassification{msc}{05A10}
%\pmkeywords{multinomial}
\pmrelated{BinomialFormula}
\pmrelated{BinomialCoefficient}
\pmrelated{GeneralizedLeibnizRule}
\pmrelated{NthDerivativeOfADeterminant}
\pmdefines{multinomial}
\pmdefines{multinomial coefficient}

\endmetadata

% this is the default PlanetMath preamble.  as your knowledge
% of TeX increases, you will probably want to edit this, but
% it should be fine as is for beginners.

% almost certainly you want these
\usepackage{amssymb}
\usepackage{amsmath}
\usepackage{amsfonts}

% used for TeXing text within eps files
%\usepackage{psfrag}
% need this for including graphics (\includegraphics)
%\usepackage{graphicx}
% for neatly defining theorems and propositions
%\usepackage{amsthm}
% making logically defined graphics
%%%\usepackage{xypic}

% there are many more packages, add them here as you need them

% define commands here
\begin{document}
A multinomial is a mathematical expression consisting of two or more terms, e.g.
$$a_1 x_1 + a_2 x_2 + \ldots + a_k x_k.$$
The multinomial theorem provides the general form of the expansion of  the powers of this
expression, in the process specifying the multinomial coefficients which are found in that expansion. The expansion is:
\begin{equation}
(x_1 + x_2 + \ldots + x_k)^n =
\sum \frac{n!}{n_1! n_2! \dotsb n_k!} x_1^{n_1} x_2^{n_2} \cdots x_k^{n_k}
\end{equation}
where the sum is taken over all multi-indices $(n_1, \ldots n_k)\in\mathbb{N}^k$  that 
sum to $n$.

The expression $\frac{n!}{n_1! n_2! \cdots n_k!}$ occurring in the expansion is called \emph{multinomial coefficient} and is denoted by
\begin{equation*}
\binom{n}{n_1, n_2, \dotsc, n_k}.
\end{equation*}
%%%%%
%%%%%
\end{document}
