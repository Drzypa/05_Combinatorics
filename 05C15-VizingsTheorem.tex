\documentclass[12pt]{article}
\usepackage{pmmeta}
\pmcanonicalname{VizingsTheorem}
\pmcreated{2013-03-22 15:10:36}
\pmmodified{2013-03-22 15:10:36}
\pmowner{marijke}{8873}
\pmmodifier{marijke}{8873}
\pmtitle{Vizing's theorem}
\pmrecord{11}{36930}
\pmprivacy{1}
\pmauthor{marijke}{8873}
\pmtype{Theorem}
\pmcomment{trigger rebuild}
\pmclassification{msc}{05C15}
%\pmkeywords{graph}
\pmrelated{TaitColoring}
\pmdefines{edge coloring}
\pmdefines{edge-${k}$-coloring}
\pmdefines{chromatic index}
\pmdefines{edge-chromatic number}
\pmdefines{class I}
\pmdefines{class II}
\pmdefines{class 1}
\pmdefines{class 2}
\pmdefines{edge-critical graph}
\pmdefines{enlarged valency}
\pmdefines{snark}

\endmetadata

\usepackage{amssymb}
% \usepackage{amsmath}
% \usepackage{amsfonts}

% used for TeXing text within eps files
%\usepackage{psfrag}
% need this for including graphics (\includegraphics)
%\usepackage{graphicx}

% for neatly defining theorems and propositions
%\usepackage{amsthm}
% making logically defined graphics
%%%\usepackage{xypic}

% there are many more packages, add them here as you need them

% define commands here %%%%%%%%%%%%%%%%%%%%%%%%%%%%%%%%%%
% portions from
% makra.sty 1989-2005 by Marijke van Gans %
                                          %          ^ ^
\catcode`\@=11                            %          o o
                                          %         ->*<-
                                          %           ~
%%%% CHARS %%%%%%%%%%%%%%%%%%%%%%%%%%%%%%%%%%%%%%%%%%%%%%

                        %    code    char  frees  for

\let\Para\S             %    \Para     §   \S \scriptstyle
\let\Pilcrow\P          %    \Pilcrow  ¶   \P
\mathchardef\pilcrow="227B

\mathchardef\lt="313C   %    \lt       <   <     bra
\mathchardef\gt="313E   %    \gt       >   >     ket

\let\bs\backslash       %    \bs       \
\let\us\_               %    \us       _     \_  ...

\mathchardef\lt="313C   %    \lt       <   <     bra
\mathchardef\gt="313E   %    \gt       >   >     ket

%%%% DIACRITICS %%%%%%%%%%%%%%%%%%%%%%%%%%%%%%%%%%%%%%%%%

%let\udot\d             % under-dot (text mode), frees \d
%let\odot\.             % over-dot (text mode),  frees \.
%let\hacek\v            % hacek (text mode),     frees \v
%let\makron\=           % makron (text mode),    frees \=
%let\tilda\~            % tilde (text mode),     frees \~
\let\uml\"              % umlaut (text mode),    frees \"

%def\ij/{i{\kern-.07em}j}
%def\trema#1{\discretionary{-}{#1}{\uml #1}}

%%%% amssymb %%%%%%%%%%%%%%%%%%%%%%%%%%%%%%%%%%%%%%%%%%%%

\let\le\leqslant
\let\ge\geqslant
%let\prece\preceqslant
%let\succe\succeqslant

%%%% USEFUL MISC %%%%%%%%%%%%%%%%%%%%%%%%%%%%%%%%%%%%%%%%

%def\C++{C$^{_{++}}$}

%let\writelog\wlog
%def\wl@g/{{\sc wlog}}
%def\wlog{\@ifnextchar/{\wl@g}{\writelog}}

%def\org#1{\lower1.2pt\hbox{#1}} 
% chem struct formulae: \bs, --- /  \org{C} etc. 

%%%% USEFUL INTERNAL LaTeX STUFF %%%%%%%%%%%%%%%%%%%%%%%%

%let\Ifnextchar=\@ifnextchar
%let\Ifstar=\@ifstar
%def\currsize{\@currsize}

%%%% KERNING, SPACING, BREAKING %%%%%%%%%%%%%%%%%%%%%%%%%

%def\qqquad{\hskip3em\relax}
%def\qqqquad{\hskip4em\relax}
%def\qqqqquad{\hskip5em\relax}
%def\qqqqqquad{\hskip6em\relax}
%def\qqqqqqquad{\hskip7em\relax}
%def\qqqqqqqquad{\hskip8em\relax}

%%%% LAYOUT %%%%%%%%%%%%%%%%%%%%%%%%%%%%%%%%%%%%%%%%%%%%%

%%%% COUNTERS %%%%%%%%%%%%%%%%%%%%%%%%%%%%%%%%%%%%%%%%%%%

%let\addtoreset\@addtoreset
%{A}{B} adds A to list of counters reset to 0
% when B is \refstepcounter'ed (see latex.tex)
%
%def\numbernext#1#2{\setcounter{#1}{#2}\addtocounter{#1}{\m@ne}}

%%%% EQUATIONS %%%%%%%%%%%%%%%%%%%%%%%%%%%%%%%%%%%%%%%%%%

%%%% LEMMATA %%%%%%%%%%%%%%%%%%%%%%%%%%%%%%%%%%%%%%%%%%%%

%%%% DISPLAY %%%%%%%%%%%%%%%%%%%%%%%%%%%%%%%%%%%%%%%%%%%%

%%%% MATH LAYOUT %%%%%%%%%%%%%%%%%%%%%%%%%%%%%%%%%%%%%%%%

\let\D\displaystyle
\let\T\textstyle
\let\S\scriptstyle
\let\SS\scriptscriptstyle

% array:
%def\<#1:{\begin{array}{#1}}
%def\>{\end{array}}

% array using [ ] with rounded corners:
%def\[#1:{\left\lgroup\begin{array}{#1}} 
%def\]{\end{array}\right\rgroup}

% array using ( ):
%def\(#1:{\left(\begin{array}{#1}}
%def\){\end{array}\right)}

%def\hh{\noalign{\vskip\doublerulesep}}

%%%% MATH SYMBOLS %%%%%%%%%%%%%%%%%%%%%%%%%%%%%%%%%%%%%%%

%def\d{\mathord{\rm d}}                      % d as in dx
%def\e{{\rm e}}                              % e as in e^x

%def\Ell{\hbox{\it\char`\$}}

\def\sfmath#1{{\mathchoice%
{{\sf #1}}{{\sf #1}}{{\S\sf #1}}{{\SS\sf #1}}}}
\def\Stalkset#1{\sfmath{I\kern-.12em#1}}
\def\Bset{\Stalkset B}
\def\Nset{\Stalkset N}
\def\Rset{\Stalkset R}
\def\Hset{\Stalkset H}
\def\Fset{\Stalkset F}
\def\kset{\Stalkset k}
\def\In@set{\raise.14ex\hbox{\i}\kern-.237em\raise.43ex\hbox{\i}}
\def\Roundset#1{\sfmath{\kern.14em\In@set\kern-.4em#1}}
\def\Qset{\Roundset Q}
\def\Cset{\Roundset C}
\def\Oset{\Roundset O}
\def\Zset{\sfmath{Z\kern-.44emZ}}

% \frac overwrites LaTeX's one (use TeX \over instead)
%def\fraq#1#2{{}^{#1}\!/\!{}_{\,#2}}
\def\frac#1#2{\mathord{\mathchoice%
{\T{#1\over#2}}
{\T{#1\over#2}}
{\S{#1\over#2}}
{\SS{#1\over#2}}}}
%def\half{\frac12}

\mathcode`\<="4268         % < now is \langle, \lt is <
\mathcode`\>="5269         % > now is \rangle, \gt is >

%def\biggg#1{{\hbox{$\left#1\vbox %to20.5\p@{}\right.\n@space$}}}
%def\Biggg#1{{\hbox{$\left#1\vbox %to23.5\p@{}\right.\n@space$}}}

\let\epsi=\varepsilon
\def\omikron{o}

\def\Alpha{{\rm A}}
\def\Beta{{\rm B}}
\def\Epsilon{{\rm E}}
\def\Zeta{{\rm Z}}
\def\Eta{{\rm H}}
\def\Iota{{\rm I}}
\def\Kappa{{\rm K}}
\def\Mu{{\rm M}}
\def\Nu{{\rm N}}
\def\Omikron{{\rm O}}
\def\Rho{{\rm P}}
\def\Tau{{\rm T}}
\def\Ypsilon{{\rm Y}} % differs from \Upsilon
\def\Chi{{\rm X}}

%def\dg{^{\circ}}                   % degrees

%def\1{^{-1}}                       % inverse

\def\*#1{{\bf #1}}                  % boldface e.g. vector
%def\vi{\mathord{\hbox{\bf\i}}}     % boldface vector \i
%def\vj{\mathord{\,\hbox{\bf\j}}}   % boldface vector \j

%def\union{\mathbin\cup}
%def\isect{\mathbin\cap}

%let\so\Longrightarrow
%let\oso\Longleftrightarrow
%let\os\Longleftarrow

% := and :<=>
%def\isdef{\mathrel{\smash{\stackrel{\SS\rm def}{=}}}}
%def\iffdef{\mathrel{\smash{stackrel{\SS\rm def}{\oso}}}}

\def\isdef{\mathrel{\mathop{=}\limits^{\smash{\hbox{\tiny def}}}}}
%def\iffdef{\mathrel{\mathop{\oso}\limits^{\smash{\hbox{\tiny %def}}}}}

%def\tr{\mathop{\rm tr}}            % tr[ace]
%def\ter#1{\mathop{^#1\rm ter}}     % k-ter[minant]

%let\.=\cdot
%let\x=\times                % ח (direct product)

%def\qed{ ${\S\circ}\!{}^\circ\!{\S\circ}$}
%def\qed{\vrule height 6pt width 6pt depth 0pt}

%def\edots{\mathinner{\mkern1mu
%   \raise7pt\vbox{\kern7pt\hbox{.}}\mkern1mu   %  .shorter
%   \raise4pt\hbox{.}\mkern1mu                  %     .
%   \raise1pt\hbox{.}\mkern1mu}}                %        .
%def\fdots{\mathinner{\mkern1mu
%   \raise7pt\vbox{\kern7pt\hbox{.}}            %   . ~45°
%   \raise4pt\hbox{.}                           %     .
%   \raise1pt\hbox{.}\mkern1mu}}                %       .

\def\mod#1{\allowbreak \mkern 10mu({\rm mod}\,\,#1)}
% redefines TeX's one using less space

%def\int{\intop\displaylimits}
%def\oint{\ointop\displaylimits}

%def\intoi{\int_0^1}
%def\intall{\int_{-\infty}^\infty}

%def\su#1{\mathop{\sum\raise0.7pt\hbox{$\S\!\!\!\!\!#1\,$}}}

%let\frakR\Re
%let\frakI\Im
%def\Re{\mathop{\rm Re}\nolimits}
%def\Im{\mathop{\rm Im}\nolimits}
%def\conj#1{\overline{#1\vphantom1}}
%def\cj#1{\overline{#1\vphantom+}}

%def\forAll{\mathop\forall\limits}
%def\Exists{\mathop\exists\limits}

%%%% PICTURES %%%%%%%%%%%%%%%%%%%%%%%%%%%%%%%%%%%%%%%%%%%

%def\cent{\makebox(0,0)}

%def\node{\circle*4}
%def\nOde{\circle4}

%%%% REFERENCES %%%%%%%%%%%%%%%%%%%%%%%%%%%%%%%%%%%%%%%%%

%def\opcit{[{\it op.\,cit.}]}
\def\bitem#1{\bibitem[#1]{#1}}
\def\name#1{{\sc #1}}
\def\book#1{{\sl #1\/}}
\def\paper#1{``#1''}
\def\mag#1{{\it #1\/}}
\def\vol#1{{\bf #1}}
\def\isbn#1{{\small\tt ISBN\,\,#1}}
\def\seq#1{{\small\tt #1}}
%def\url<{\verb>}
%def\@cite#1#2{[{#1\if@tempswa\ #2\fi}]}

%%%% VERBATIM CODE %%%%%%%%%%%%%%%%%%%%%%%%%%%%%%%%%%%%%%

%def\"{\verb"}

%%%% AD HOC %%%%%%%%%%%%%%%%%%%%%%%%%%%%%%%%%%%%%%%%%%%%%

\def\0#1{\hbox{\sc #1}}  % typesetting for nodes (vertices)

%%%% WORDS %%%%%%%%%%%%%%%%%%%%%%%%%%%%%%%%%%%%%%%%%%%%%%

% \hyphenation{pre-sent pre-sents pre-sent-ed pre-sent-ing
% re-pre-sent re-pre-sents re-pre-sent-ed re-pre-sent-ing
% re-fer-ence re-fer-ences re-fer-enced re-fer-encing
% ge-o-met-ry re-la-ti-vi-ty Gauss-ian Gauss-ians
% Des-ar-gues-ian}

%def\oord/{o{\trema o}rdin\-ate}
% usage: C\oord/s, c\oord/.
% output: co\"ord... except when linebreak at co-ord...

%%%%%%%%%%%%%%%%%%%%%%%%%%%%%%%%%%%%%%%%%%%%%%%%%%%%%%%%%

                                          %
                                          %          ^ ^
\catcode`\@=12                            %          ` '
                                          %         ->*<-
                                          %           ~
\begin{document}
\PMlinkescapeword{bound}
\PMlinkescapephrase{class number}
\PMlinkescapeword{contains}
\PMlinkescapeword{covers}
\PMlinkescapeword{cyclic}
\PMlinkescapeword{decompositions}
\PMlinkescapeword{embeddings}
\PMlinkescapeword{Fix}
\PMlinkescapeword{length}
\PMlinkescapeword{link}
\PMlinkescapeword{links}
\PMlinkescapeword{node}
\PMlinkescapeword{nodes}
\PMlinkescapeword{odd}
\PMlinkescapeword{range}
\PMlinkescapeword{ranges}
\PMlinkescapeword{size}
\PMlinkescapeword{sizes}
\PMlinkescapeword{structure}
\PMlinkescapeword{structures}
\PMlinkescapeword{term}
\PMlinkescapeword{time}


\subsection*{Definitions and notation}

In a graph or multigraph $G$, let $\rho(\0v)$ denote the valency of vertex
(node) $\0v$, and let $\rho(G)$ denote the largest valency in $G$ (often written
just $\rho$ on its own, $\Delta(G)$ is another common notation).

Let the multiplicity $\mu(\0v,\0w)$ of vertices (nodes) $\0v$ and $\0w$
be the number of parallel edges that link them, and here too let $\mu(G)$ be
the largest multiplicity in $G$. A graph is a multigraph for which $\mu(G)=1$.

An {\bf edge coloring} of a (multi)graph $G$ is a mapping from the
set $E$ of its edges to a set $K$ of items called colors, in such a way
that at any vertex $\0v$, the $\rho(\0v)$ edges there all have a different
color. An {\bf edge-$k$-coloring} is an edge-coloring where $|K|=k$.

Note that a loop (an edge joining $\0v$ to itself) accounts for two of the
$\rho(\0v)$ edges at $\0v$ and cannot have a color different from itself,
so edge-colorings as defined here do not exist for pseudographs (structures
that are allowed to have loops).

The {\bf chromatic index} (aka {\bf edge-chromatic number}) ${\chi\,}'(G)$ is
the smallest number $k$ for which an edge-$k$-coloring of $G$ exists. This now
standard notation is analogous to the {\bf chromatic number} $\chi(G)$ which
refers to vertex coloring.

\subsection*{Vizing's theorem}

For any multigraph $G$ (this includes graphs), we have of course
$$
  {\chi\,}'(G) \;\ge\; \rho(G)
$$
immediately from the definition. Surprisingly though, we also have

{\bf Theorem (Vizing)}$\quad$
For any graph $G$,
$$
  {\chi\,}'(G) \;\le\; \rho(G)+1
$$
This celebrated theorem was proved by V.~G.~Vizing in 1964 while still a
graduate student (at Novosibirsk). It is a special case of

{\bf Theorem (Vizing)}$\quad$
For any multigraph $G$,
$$
  {\chi\,}'(G) \;\le\; \rho(G)+\mu(G)
$$
(edge-coloring of multigraphs is the subject of Vizing's doctoral thesis, 1965).
The theorem was proved independently by R.~P.~Gupta in 1966; nowadays various
versions of the proofs exist. A key r\^ole is played by connected subgraphs $H(a,b)$ of colors $a$ and $b$ that cannot be extended further. They are analogous to Kempe chains in face or vertex colorings, but have a simpler structure: they are either paths or closed paths (the latter of even length). \PMlinkid{See here for a proof}{6932} (in the case of graphs).

Ore~\cite{Ore67} gives a sharper bound for multigraphs. Let the
{\bf enlarged valency} $\rho^+(\0v)$ be given by
$$
  \rho^+(\0v) \;\isdef\; \rho(\0v) + \max_{\rm w} \mu(\0v,\0w)
$$
where $\0w$ can be taken over all vertices of the graph; it effectively only
ranges over those adjacent to $\0v$ (as $\mu(\0v,\0w)$ is zero for the others).
Again, let $\rho^+(G)$ denote the maximum enlarged valency occurring in
$G$. Now

{\bf Theorem (Ore)}$\quad$ For any multigraph $G$,
$$
  {\chi\,}'(G) \;\le\; \rho^+(G)
$$

\subsection*{Shannon's theorem}

The following theorem was proven in 1949 by
Claude E.~Shannon. He also gives examples, for any value of $\rho$, of
multigraphs that actually attain the bound,
the so-called Shannon graphs: they have three vertices, with
$\mu(\0a,\0b)=\mu(\0a,\0c)=\lfloor\rho/2\rfloor$ and
$\mu(\0b,\0c)=\lceil\rho/2\rceil$.

{\bf Theorem (Shannon)}$\quad$ For any multigraph $G$,
$$
  {\chi\,}'(G) \;\le\; \lfloor\frac32\rho(G)\rfloor
$$
While giving a much worse bound for graphs, it gives for some multigraphs a
better bound than Vizing's theorem. Nevertheless, it is also possible to prove
it from the latter~\cite{FW77}.

Only in the context of graph colorings is {\em Shannon's theorem\/} understood
to refer to the one here; in the wider world the term tends to refer to any of
his fundamental theorems in information theory.

Here too Ore~\cite{Ore67} gives a sharper bound based on the maximum of
a local expression. Let $\sigma(\0v)$ be 0 if $\0v$ has fewer than two
neighbours, and otherwise
$$
  \sigma(\0v) \;\isdef\;
  \max_{{\rm v}',\,{\rm v}''} (\rho(\0v) + \rho(\0v') + \rho(\0v''))
$$
where $\0v'$ and $\0v''$ range over all pairs of distinct neighbours of $\0v$. Again, let $\sigma(G)$ be its
largest value in the graph.

{\bf Theorem (Ore)}$\quad$ For any multigraph $G$,
$$
  {\chi\,}'(G) \;\le\; \max(\rho(G),\lfloor\frac12\sigma(G)\rfloor)
$$

\subsection*{Chromatic class}

Vizing's theorem has the effect of placing each multigraph $G$ in one of the
classes 0, 1, 2, \dots\ $\mu(G)$ where the class number is
${\chi\,}'(G)-\rho(G)$.

For graphs it means they split into just two classes: those that can be
edge-colored in $\rho(G)$ colors and those that need $\rho(G)+1$
colors. The logical name for them would be class~0 and class~1; unfortunately
the standard terminology is class~1 and 2 (or I and II).

{\bf Class I} (graphs that can be edge-$\rho$-colored) contains among others
%
\begin{itemize}

\item single $n$-cycles for even $n$ $^\dagger$

\item complete graphs $K_n$ for even $n$ $^\ddagger$

\item bipartite graphs (this is K\uml{o}nig's theorem, 1916)

\item bridgeless planar trivalent graphs (four-color theorem via Tait coloring)

\item planar graphs with $\rho(G)\ge8$ (by another theorem of Vizing)

\item planar graphs with $\rho(G)\ge6$?? (conjecture of Vizing)

\end{itemize}
%
{\bf Class II} (graphs that need $\rho+1$ colors) contains among others
%
\begin{itemize}

\item single $n$-cycles for odd $n$ $^\dagger$

\item complete graphs $K_n$ for odd $n$ $^\ddagger$

\item $K_n$ for odd $n$ with a few edges missing (by a couple of theorems)

\item trivalent graphs with a bridge $^\pilcrow$

\end{itemize}
%
but the general classification problem has thus far eluded the best efforts of
Vizing and many others. There are interesting links here with polyhedral
decompositions (aka cyclic double covers) and embeddings in surfaces.

A(n {\bf edge-}){\bf critical graph} is a connected graph of class II but such
that removing any of its edges makes it class I. As often in graph theory,
such a minimality condition imposes a certain amount of structure on the graph.
There are conjectures\dots

\subsection*{Almost all graphs are in class I}

Let $\#G_n$ be the number of graphs with $n$ vertices,
and $\#G^{\,\rm I}_n$ the number of them in class~I.

{\bf Theorem (P.~Erd\H{o}s and R.~J.~Wilson)}$\quad$
$$
  \lim_{n\to\infty} {\#G^{\,\rm I}_n \over \#G_n} = 1
$$
So graphs of class II get {\em relatively\/} rarer for larger graph sizes. The {\em absolute\/} numbers do still increase. For cubic graphs for instance, we saw every one with a bridge is in class II. Bridgeless cubic graphs of class II are a bit thinner on the ground. By the four-color theorem, via Tait coloring, we know all of them are non-planar. Rarer still are those of them with girth at least five (and some non-triviality conditions); they are so hard to find that Martin Gardner dubbed them {\bf snarks}. The Petersen graph is one, and a few infinite families of snarks have been found.

{\footnotesize

$\dagger\quad$ $\rho=2$, so an edge-$\rho$-coloring must use alternating colors. For odd $n$ that's impossible.

$\ddagger\quad$ $\rho=n-1$, note it is also the valency of every vertex.
Fix two colors $a$ and $b$. A Kempe chain $H(a,b)$ can only terminate at a vertex where one of those colors is missing but in $K_n$ we cannot afford to miss any color at any vertex, so every $H(a,b)$ is a cycle of even length and together they visit all $n$ vertices. For odd $n$ that's impossible.
For even $n$ there are ways to construct the coloring (try it).

$\pilcrow\;\;$ $\rho = 3$ and again the valency of every vertex. For the same
reason as in the previous note, every $H(a,b)$ or $H(c,a)$ or $H(b,c)$ is a cycle. Every edge must be in two such, but a bridge cannot be part of a cycle.

}

\raggedright
\begin{thebibliography}{MST73}

\bitem{Ore67} \name{Oystein~Ore}, \book{The Four-Color Problem},\\
              Acad.~Pr.~1967, \isbn{0\,12\,528150\,1}\\
              {\it Long the standard work on its subject, but written before
              the theorem was proven. Has a wealth of other graph theory
              material, including proofs of (improvements of) Vizing's and
              Shannon's theorems.}

\bitem{FW77}  \name{S.~Fiorini} and \name{R.~J.~Wilson},
              \book{Edge-colourings of graphs}, Pitman~1977,
              \isbn{0\,273\,01129\,4}\\
              {\em The first ever book devoted to edge-colorings,
              including material previously found only in Russian
              language journal articles. Has proofs of Vizing's and
              Shannon's theorems.}

\bitem{SK77}  \name{Thomas~L.~Saaty} and \name{Paul~C.~Kainen},\\
              \book{The Four-Color Problem: assaults and conquest},\\
              McGraw-Hill 1977; repr.\ Dover 1986,
              \isbn{0\,486\,65092\,8}\\
              {\em Wonderfully broad, not only focussing on the usual
              route to the Appel-Haken proof but also giving lots of
              other material.}

\bitem{Wil02} \name{Robert~A.~Wilson},
              \book{Graphs, Colourings and the Four-colour Theorem},
              Oxford~Univ.~Pr.~2002, \isbn{0\,19\,851062\,4} (pbk),
{\tt\PMlinkexternal{http://www.maths.qmul.ac.uk/~raw/graph.html}{http://www.maths.qmul.ac.uk/~raw/graph.html}}
              (errata \&c.)\\
              {\em A good general course in graph theory, with special focus on
              the four-color theorem and details of the Appel-Haken proof.
              Has proof of Vizing's theorem.}

\end{thebibliography}
%%%%%
%%%%%
\end{document}
