\documentclass[12pt]{article}
\usepackage{pmmeta}
\pmcanonicalname{DistanceinAGraph}
\pmcreated{2013-03-22 12:31:32}
\pmmodified{2013-03-22 12:31:32}
\pmowner{CWoo}{3771}
\pmmodifier{CWoo}{3771}
\pmtitle{distance (in a graph)}
\pmrecord{11}{32765}
\pmprivacy{1}
\pmauthor{CWoo}{3771}
\pmtype{Definition}
\pmcomment{trigger rebuild}
\pmclassification{msc}{05C12}
\pmsynonym{distance}{DistanceinAGraph}
\pmrelated{Graph}
\pmrelated{Path}
\pmrelated{Diameter3}
\pmrelated{PathConnected}
\pmdefines{diameter (of a graph)}
\pmdefines{radius (of a graph)}
\pmdefines{central vertex}

% This is Cosmin's preamble.

% Packages
  \usepackage{amsmath}
  \usepackage{amssymb}
  \usepackage{amsfonts}
  \usepackage{amsthm}
  \usepackage{mathrsfs}
  %\usepackage{graphicx}
  %%%\usepackage{xypic}
  %\usepackage{babel}

% Theorem Environments
  \newtheorem*{thm}{Theorem}
  \newtheorem{thmn}{Theorem}
  \newtheorem*{lem}{Lemma}
  \newtheorem{lemn}{Lemma}
  \newtheorem*{cor}{Corollary}
  \newtheorem{corn}{Corollary}
  \newtheorem*{prop}{Proposition}
  \newtheorem{propn}{Proposition}

  \newcommand{\bbN}[1]{\mathbb{#1}}
  \DeclareMathOperator{\diam}{diam}
  \DeclareMathOperator{\rad}{rad}

  % Other Commands
    \renewcommand{\geq}{\geqslant}
    \renewcommand{\leq}{\leqslant}
    \newcommand{\vect}[1]{\boldsymbol{#1}}
    \newcommand{\mat}[1]{\mathsf{#1}}
    \renewcommand{\div}{\!\mid\!}
\begin{document}
The \emph{distance} $d(x,y)$ of two vertices $x$ and $y$ of a graph $G$ is the length of the shortest path (or, equivalently, walk) from $x$ to $y$. If there is no path from $x$ to $y$ (i.e. if they lie in different components of G), we set $d(x,y) := \infty.$ 

Two basic graph invariants involving distance are the \emph{diameter} $\diam G := \max_{(x,y)\in V(G)^2} d(x,y)$ (the maximum distance between two vertices of $G$) and the \emph{radius} $\rad G := \min_{x\in V(G)} \max_{y\in V(G)} d(x,y)$ (the maximum distance of a vertex from a \emph{central} vertex of $G$, i.e. a vertex such that the maximum distance to another vertex is minimal).
%%%%%
%%%%%
\end{document}
