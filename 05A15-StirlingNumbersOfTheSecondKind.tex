\documentclass[12pt]{article}
\usepackage{pmmeta}
\pmcanonicalname{StirlingNumbersOfTheSecondKind}
\pmcreated{2013-03-22 12:33:32}
\pmmodified{2013-03-22 12:33:32}
\pmowner{rmilson}{146}
\pmmodifier{rmilson}{146}
\pmtitle{Stirling numbers of the second kind}
\pmrecord{11}{32805}
\pmprivacy{1}
\pmauthor{rmilson}{146}
\pmtype{Definition}
\pmcomment{trigger rebuild}
\pmclassification{msc}{05A15}
\pmrelated{StirlingNumbersOfTheFirstKind}

\usepackage{amsmath}
\usepackage{amsfonts}
\usepackage{amssymb}

\newcommand{\cP}{\mathcal{P}}
\newcommand{\un}{{\underline{n}}}
\newcommand{\uk}{{\underline{k}}}
\newcommand{\reals}{\mathbb{R}}
\newcommand{\natnums}{\mathbb{N}}
\newcommand{\cnums}{\mathbb{C}}
\newcommand{\znums}{\mathbb{Z}}

\newcommand{\lp}{\left(}
\newcommand{\rp}{\right)}
\newcommand{\lb}{\left[}
\newcommand{\rb}{\right]}

\newcommand{\supth}{^{\text{th}}}


\newtheorem{proposition}{Proposition}
\begin{document}
\paragraph{Summary.}
The Stirling numbers of the second kind, 
$$S(n,k) = {n \brace k}, \quad k,n\in\natnums,\quad 1\leq k\leq n,$$
are a doubly
indexed sequence of natural numbers, enjoying a wealth of interesting
combinatorial properties.  There exist several logically equivalent
characterizations, but the starting point of the present entry will
be the following definition:
\begin{quote}
  The Stirling number $S(n,k)$ is the number of ways to partition a set
  of $n$ objects into $k$ groups.
\end{quote}
For example, $S(4,2) = 7$ because there are seven ways to partition 4
objects --- call them a, b, c, d --- into two groups, namely:
$$
(a)(bcd),\;
(b)(acd),\;
(c)(abd),\;
(d)(abc),\;
(ab)(cd),\;
(ac)(bd),\;
(ad)(bc)
$$

Four additional characterizations will be discussed in this entry:
\begin{itemize}
\item a recurrence relation
\item a generating function related to the falling factorial
\item differential operators
\item  a double-index generating function
\end{itemize}
Each of these will be discussed below, and shown to be equivalent.
\paragraph{A recurrence relation.}
The Stirling numbers of the second kind can be characterized in terms of
the following recurrence relation:
$$S(n,k) = kS(n-1,k) + S(n-1,k-1),\qquad 1\leq k < n,$$
subject to the
following initial conditions:
$$S(n,n) = S(n,1) = 1.$$
Let us now show that the recurrence formula
follows from the enumerative definition.  Evidently, there is only one
way to partition $n$ objects into $1$ group (everything is in that
group), and only one way to partition $n$ objects into $n$ groups
(every object is a group all by itself). Proceeding recursively, a
division of $n$ objects $a_1, \ldots, a_{n-1}, a_n$ into $k$ groups
can be achieved by only one of two basic maneuvers:
\begin{itemize}
\item We could partition the first $n-1$ objects into $k$ groups, and
  then add object $a_n$ into one of those groups.  There are $k
  S(n-1,k)$ ways to do this.
\item We could partition the first $n-1$ objects into $k-1$ groups and
  then add object $a_n$ as a new, 1 element group.  This gives an
  additional 
  $S(n-1,k-1)$ ways to create the desired partition.
\end{itemize}
The recursive point of view, therefore explains the connection between
the recurrence formula, and the original definition.  

Using the recurrence formula we can easily obtain a table of the
initial Stirling numbers:

\bigskip

\begin{tabular}{lccccc}
$n\backslash k$ & 1 & 2 & 3  & 4 & 5\\
1 & 1 \\
2 & 1 &  1 \\
3 & 1 & 3 & 1 \\
4 & 1 & 7 & 6 & 1 \\
5 & 1 & 15&25 & 10 & 1
\end{tabular}


\paragraph{Falling Factorials.}
Consider the vector space of polynomials in indeterminate $x$.  The
most obvious basis of this infinite-dimensional vector space is the
sequence of monomial powers: $x^n,\quad n\in\natnums.$ However, the
sequence of falling factorials:
$$x^\un=x(x-1)(x-2)\ldots(x-n+1),\quad n\in\natnums$$
is also a basis, and hence can be
used to generate the monomial basis.  Indeed, the Stirling numbers of
the second kind can be characterized as the coefficients involved
in the corresponding change of basis matrix, i.e.
$$ x^n = \sum_{k=1}^n S(n,k) x^\uk.$$
So, for example,
$$x^4 = x + 7x(x-1) + 6x(x-1)(x-2) + x(x-1)(x-2)(x-3).$$

Arguing inductively, let us prove that this characterization follows
from the recurrence relation.  Evidently the formula is true for
$n=1$.  Suppose then that the formula is true for a given $n$.  We
have
$$x x^\uk = x^{\underline{k+1}} + k x^\uk,$$ 
and hence using the recurrence relation we deduce that
\begin{align*}
x^{n+1} &= \sum_{k=1}^n S(n,k)\,x\,x^\uk \\
&= \sum_{k=1}^n \lp kS(n,k) x^\uk + S(n,k+1)\rp x^{\underline{k+1}} \\
&= \sum_{k=1}^{n+1} S(n,k) x^\uk
\end{align*}

\paragraph{Differential operators.} 
Let $D_x$ denote the ordinary derivative, applied to polynomials in
indeterminate $x$, and let $T_x$ denote the differential operator
$xD_x$.  We have the following characterization of the Stirling
numbers of the second kind in terms of these two operators:
$$(T_x)^n = \sum_{k=1}^n S(n,k)\, x^k (D_x)^k,$$
where an
exponentiated differential operator denotes the operator composed with
itself the indicated number of times.  Let us show that this follows
from the recurrence relation.  The proof is once again, inductive.
Suppose that the characterization is true for a given $n$.  We have
$$T_x(x^k (D_x)^k) =  k x^k (D_x)^k + x^{k+1} (D_x)^{k+1},$$
and hence using the recurrence relation we deduce that
\begin{align*}
(T_x)^{n+1} &= xD_x\lp \sum_{k=1}^n  S(n,k)\, x^k (D_x)^k \rp \\
&= \sum_{k=1}^n S(n,k) \lp k x^k (D_x)^k + x^{k+1} (D_x)^{k+1}\rp \\
&= \sum_{k=1}^{n+1}  S(n+1,k)\, x^k (D_x)^k
\end{align*}

\paragraph{Double index generating function.}
One can also characterize the Stirling numbers of the second kind in
terms of the following  generating function:
$$e^{x(e^t-1)} = \sum_{n=1}^\infty \sum_{k=1}^n S(n,k)\,
x^k\,\frac{t^n}{n!}.$$
Let us now prove this.  Note that the differential equation
$$\frac{d\xi}{dt} = \xi,$$
admits the general solution
$$\xi = e^t x.$$
It follows that for any polynomial $p(\xi)$ we have
$$\exp(t T_\xi)[p(\xi)]\Big|_{\xi=x} = \sum_{n=0}^\infty \frac{t^n}{n!}
(T_\xi)^n[p(\xi)]\Big|_{\xi=x} = p(e^t x).$$
The proof is simple: just take $D_t$ of both sides.  To be more
explicit,
$$D_t\lb p(e^t x) \rb = p'(e^t x) e^t x =
T_\xi[p(\xi)]\Big|_{\xi=xe^t},$$
and that is exactly equal to $D_t$ of the left-hand side.  Since this
relation holds for all polynomials, it also holds for all formal power
series.  In particular if we apply the above relation to $e^\xi$, 
use the result of the preceding section, and note that
$$D_\xi[e^\xi] = e^\xi,$$
we obtain
\begin{align*}
e^{xe^t} &= \sum_{n=1}^\infty \frac{t^n}{n!} (T_\xi)^n[e^\xi] \Big|_{\xi=x} \\
&=  \sum_{n=1}^\infty \sum_{k=1}^n S(n,k)\, \frac{t^n}{n!} \xi^k
(D_\xi)^k[e^\xi] \Big|_{\xi=x} \\
&=  e^x \sum_{n=1}^\infty \sum_{k=1}^n S(n,k)\, x^k\, \frac{t^n}{n!}\\
\end{align*}
Dividing both sides by $e^x$ we obtain the desired generating
function. Q.E.D.
%%%%%
%%%%%
\end{document}
