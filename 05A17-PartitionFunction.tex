\documentclass[12pt]{article}
\usepackage{pmmeta}
\pmcanonicalname{PartitionFunction}
\pmcreated{2013-03-22 15:57:55}
\pmmodified{2013-03-22 15:57:55}
\pmowner{silverfish}{6603}
\pmmodifier{silverfish}{6603}
\pmtitle{partition function}
\pmrecord{9}{37980}
\pmprivacy{1}
\pmauthor{silverfish}{6603}
\pmtype{Definition}
\pmcomment{trigger rebuild}
\pmclassification{msc}{05A17}
\pmrelated{IntegerPartition}
\pmrelated{NonMultiplicativeFunction}
\pmdefines{partition generating function}

% this is the default PlanetMath preamble.  as your knowledge
% of TeX increases, you will probably want to edit this, but
% it should be fine as is for beginners.

% almost certainly you want these
\usepackage{amssymb}
\usepackage{amsmath}
\usepackage{amsfonts}


% used for TeXing text within eps files
%\usepackage{psfrag}
% need this for including graphics (\includegraphics)
%\usepackage{graphicx}
% for neatly defining theorems and propositions
%\usepackage{amsthm}
% making logically defined graphics
%%%\usepackage{xypic}

% there are many more packages, add them here as you need them

% define commands here

\newtheorem{thm}{Theorem}
\begin{document}
The {\sl partition function } $p(n)$ is defined to be the number of partitions of the integer $n$.  The sequence of values $p(0), p(1), p(2),\ldots$ is Sloane's A000041 and begins $1, 1, 2, 3, 5, 7, 11, 15, 22, 30, \ldots$.  This function grows very quickly, as we see in the following theorem due to Hardy and \PMlinkname{Ramanujan}{SrinivasaRamanujan}.

\begin{thm}
As $n \rightarrow \infty$, the ratio of $p(n)$ and
\[ \frac{ e^{\pi \sqrt{ 2n/3} } } {4n \sqrt{3} } \]
approaches 1.
\end{thm}


The generating function of $p(n)$ is called $F$: by definition

\[ F(x) = \sum _{n=0} ^\infty p(n) x ^n. \]

$F$ can be written as an infinite product:

\[ F(x) = \prod _{i=1} ^\infty (1-x^i) ^{-1}. \]
To see this, expand each term in the product as a power series:

\[ \label{product} \prod _{i=1} ^\infty  (1+ x^i + x^{2i} + x^{3i} + \cdots ). \]
Now expand this as a power series.  Given a partition of $n$ with $a_i$ parts of size $i \geq 1$, we get a term $x^n$ in this expansion by choosing $x^{a_1}$ from the first term in the product, $x^{2a_2}$ from the second, $x^{3a_3}$ from the third and so on.  Clearly any term $x^n$ in the expansion arises in this way from a partition of $n$.

One can prove in the same way that the generating function $F_m$ for the number $p_m(n)$ of partitions of $n$ into at most $m$ parts (or equivalently into parts of size at most $m$) is

\[ F_m(x) = \prod _{i=1} ^m (1-x^i) ^{-1}. \]

\begin{thebibliography}{5}
\bibitem{HandW} G. H. Hardy and E. M. Wright, {\em An Introduction to the Theory of Numbers}, Oxford University Press, 2003.



\end{thebibliography}
%%%%%
%%%%%
\end{document}
