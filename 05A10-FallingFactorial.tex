\documentclass[12pt]{article}
\usepackage{pmmeta}
\pmcanonicalname{FallingFactorial}
\pmcreated{2013-03-22 12:23:58}
\pmmodified{2013-03-22 12:23:58}
\pmowner{rmilson}{146}
\pmmodifier{rmilson}{146}
\pmtitle{falling factorial}
\pmrecord{11}{32211}
\pmprivacy{1}
\pmauthor{rmilson}{146}
\pmtype{Definition}
\pmcomment{trigger rebuild}
\pmclassification{msc}{05A10}
\pmdefines{rising factorial}
\pmdefines{Pochhammer symbol}

\endmetadata

\usepackage{amsmath}
\usepackage{amsfonts}
\usepackage{amssymb}

\newcommand{\reals}{\mathbb{R}}
\newcommand{\natnums}{\mathbb{N}}
\newcommand{\cnums}{\mathbb{C}}

\newcommand{\lp}{\left(}
\newcommand{\rp}{\right)}
\newcommand{\lb}{\left[}
\newcommand{\rb}{\right]}

\newcommand{\supth}{^{\text{th}}}


\newtheorem{proposition}{Proposition}

\newcommand{\on}{{\overline{n}}}
\newcommand{\un}{{\underline{n}}}
\begin{document}
For $n\in\natnums$, the rising and falling factorials are $n\supth$
degree polynomial described, respectively, by
\begin{align*}
x^\on &= x(x+1)\ldots (x+n-1) \\
x^\un &= x(x-1)\ldots (x-n+1)
\end{align*}
The two types of polynomials are related by:
$$x^\on = (-1)^n (-x)^\un.$$


The rising factorial is often written as $(x)_n$, and referred to as
the Pochhammer symbol (see hypergeometric series). Unfortunately, the
falling factorial is also often denoted by $(x)_n$, so great care must
be taken when encountering this notation.


{\bf Notes.}  

Unfortunately, the notational conventions for the rising and falling
factorials lack a common standard, and are plagued with a fundamental
inconsistency. An examination of reference works and textbooks reveals
two fundamental sources of notation: works in combinatorics and works
dealing with hypergeometric functions.

Works of combinatorics [1,2,3] give greater focus to the falling
factorial because of its role in defining the Stirling numbers.
The symbol $(x)_n$ almost always denotes the falling factorial.  The
notation for the rising factorial varies widely; we find
$\left<x\right>_n$ in [1] and $(x)^{(n)}$ in [3].

Works focusing on special functions [4,5]  universally use $(x)_n$ to
denote the rising factorial and use this symbol in the description of
the various flavours of hypergeometric series.  Watson [5] credits
this notation to Pochhammer [6], and indeed the special functions
literature eschews ``falling factorial'' in favour of ``Pochhammer
symbol''.  Curiously, according to Knuth [7], Pochhammer himself used
$(x)_n$ to denote the binomial coefficient (Note: I haven't verified
this.)
 
The notation featured in this entry is due to D. Knuth [7,8].  Given
the fundamental inconsistency in the existing notations, it seems
sensible to break with both traditions, and to adopt new and
graphically suggestive notation for these two concepts.  The
traditional notation, especially in the hypergeometric camp, is so
deeply entrenched that, realistically, one needs to be familiar with
the traditional modes and to take care when encountering the symbol
$(x)_n$.

{\bf References}
\begin{enumerate}
\item Comtet,  {\em Advanced combinatorics.}
\item Jordan, {\em Calculus of finite differences.}
\item Riordan, {\em Introduction to combinatorial analysis.}
\item Erd\'elyi, et. al., {\em Bateman manuscript project.}
\item Watson, {\em A treatise on the theory of Bessel functions.}\ 
\item Pochhammer, ``Ueber hypergeometrische Functionen $n^{\rm ter}$
  Ordnung,'' {\sl Journal f\"ur die reine und angewandte Mathematik\/
    \bf 71} (1870), 316--352.
\item Knuth, ``Two notes on notation'' \PMlinkexternal{download}{http://www-cs-faculty.stanford.edu/~knuth/papers/tnn.tex.gz}
\item Greene, Knuth, {\em Mathematics for the analysis of algorithms.}
\end{enumerate}
%%%%%
%%%%%
\end{document}
