\documentclass[12pt]{article}
\usepackage{pmmeta}
\pmcanonicalname{EnumerativeCombinatorics}
\pmcreated{2013-03-22 16:45:58}
\pmmodified{2013-03-22 16:45:58}
\pmowner{mps}{409}
\pmmodifier{mps}{409}
\pmtitle{enumerative combinatorics}
\pmrecord{14}{38993}
\pmprivacy{1}
\pmauthor{mps}{409}
\pmtype{Topic}
\pmcomment{trigger rebuild}
\pmclassification{msc}{05-00}
\pmrelated{DijkstrasAlgorithm}
\pmrelated{RuleOfProduct}

\endmetadata

% this is the default PlanetMath preamble.  as your knowledge
% of TeX increases, you will probably want to edit this, but
% it should be fine as is for beginners.

% almost certainly you want these
\usepackage{amssymb}
\usepackage{amsmath}
\usepackage{amsfonts}

% used for TeXing text within eps files
%\usepackage{psfrag}
% need this for including graphics (\includegraphics)
%\usepackage{graphicx}
% for neatly defining theorems and propositions
\usepackage{amsthm}
% making logically defined graphics
%%%\usepackage{xypic}

% there are many more packages, add them here as you need them

% define commands here
\newtheorem{proposition}{Proposition}
\newtheorem*{proposition*}{Proposition}
\theoremstyle{remark}
\newtheorem{example}{Example}
\newtheorem*{example*}{Example}
\begin{document}
\emph{Enumerative combinatorics} deals with the question: if we know
that a set $S$ is finite, how can we determine the exact number of
elements that $S$ contains?  

Basic principles and techniques of enumerative combinatorics include:
\begin{itemize}
\item
the addition principle;

\item
the multiplication principle;

\item
the involution principle;

\item
the inclusion-exclusion principle; and

\item
the use of generating functions.
\end{itemize}

The principles listed above are disarmingly simple and seemingly
obvious.  Nonetheless, when used properly they are powerful
tools for producing bijective proofs of combinatorial identities.  On
the other hand, while generating functions can frequently be used to
give quick proofs of identities, it is sometimes difficult to extract
combinatorial proofs from such proofs.

%%% ADDITION
The most fundamental principle of enumerative combinatorics and the
basis of all counting is the \emph{addition principle}.  It says that
if $S$ is a finite set, then
\[
|S| = \sum_{x\in S} 1.
\]
More generally, if $S=A\sqcup B$, then 
\[
|S| = \sum_{x\in A\sqcup B} 1= \sum_{x\in A} 1 + \sum_{x\in B} 1 = |A| + |B|.
\]
By induction on $n$, we also get
\[
\biggl|\bigsqcup_{i\in[n]} S_i\biggr| = \sum_{i\in[n]} |S_i|.
\]
As an application of the addition principle, we prove the
\emph{multiplication principle}.

%%% MULTIPLICATION
\begin{proposition}
{\bf Multiplication principle.} If $S$ and $T$ are finite sets, then
\[
|S\times T| = |S|\cdot |T|.
\]
\end{proposition}

\begin{proof}
By the addition principle,
\[
|S\times T| = \sum_{(x,y)\in S\times T} 1.
\]
But we can write $S\times T$ as the disjoint union
\[
S\times T = \bigsqcup_{y\in T} S\times\{y\}.
\]
The projection map $S\times\{y\}\to S$ is a bijection, so
$|S\times\{y\}|=|S|$.  Hence it follows that
\begin{align*}
|S\times T| 
&= \sum_{y\in T} |S\times\{y\}| \\
&= \sum_{y\in T} |S| \\
&= |S|\cdot\sum_{y\in T} 1 \\
&= |S|\cdot |T|,
\end{align*}
which completes the proof.
\end{proof}

%%% INVOLUTION
The \emph{involution principle} says that if $\varphi\colon S\to S$ is
an involution, then for any $X\subset S$, $|X|=|\varphi(X)|$.  The
following example illustrates the involution principle.

\begin{proposition} 
{\bf Enumerating elements of the powerset.}
Let $[n]=\{1,2,\dots,n\}$ and let $2^{[n]}$ be the powerset of $[n]$.
Then the cardinality of $2^{[n]}$ is $2^n$.
\end{proposition}

\begin{proof}
Define a function $\varphi\colon 2^{[n]}\to 2^{[n]}$ by the formula
\[
\varphi(S) = S\Delta\{n\}.
\]

In other words, $\varphi(S)$ is the symmetric difference of $S$ and
$\{n\}$ --- if $n$ is an element of $S$, we remove it, and if $n$ is
not an element of $S$, we insert it.  The function $\varphi$ is an
involution, hence a bijection.  Now notice that $2^{[n-1]}\subset
2^{[n]}$ and $\varphi(2^{[n-1]}) = 2^{[n]}\setminus 2^{[n-1]}$.  In
other words,
\[
2^{[n]} = 2^{[n-1]} \sqcup \varphi(2^{[n-1]}).
\]
Since $\varphi$ is a bijection, this implies that
\[
|2^{[n]}| = 2\cdot |2^{[n-1]}|.
\]
By induction on $n$ we obtain
\[
|2^{[n]}| = 2^n\cdot |2^{[0]}|.
\]
Now $[0] = \emptyset$, so $2^{[0]} = \{\emptyset\}$.  Thus
$|2^{[0]}|=1$, implying that
\[
|2^{[n]}| = 2^n,
\]
which is what we wanted to show.
\end{proof}

%%% INCLUSION-EXCLUSION
As we have seen above, it is possible to use the addition principle to
count the number of elements in the disjoint union of finite sets.
But what if we want to count the number of elements in a non-disjoint
union of finite sets?  The \emph{inclusion-exclusion principle} gives
us a way to do this.  A common way of stating the inclusion-exclusion
principle is as the following proposition.

\begin{proposition}
{\bf Inclusion-exclusion principle.}
Let $S_1,\dots, S_n$ be finite sets.  Then
\[
\biggl|\bigcup_{i\in[n]} S_i\biggr| = \sum_{T\in 2^{[n]}} (-1)^{|T|}\biggl|\bigcap_{i\in T} S_i\biggr|.
\]
\end{proposition}

The formula in this proposition uses negative numbers.  But the core
of the inclusion-exclusion principle is a statement about natural
numbers.

\begin{proposition}
Let $S$ and $T$ be finite sets.  Then
\[
|S|+|T| = |S\cup T| + |S\cap T|.
\]
Thus the lattice of finite subsets of $\mathbb{N}$ is a modular lattice.
\end{proposition}

\begin{proof}
By the addition principle,
\begin{align*}
|S|+|T| 
&= |(S\setminus T)\sqcup(S\cap T)|+|(T\setminus S)\sqcup(S\cap T)| \\
&= |S\setminus T| + |S\cap T| + |T\setminus S| + |S\cap T|.
\end{align*}
Now observe that $S\cup T=(S\setminus T)\sqcup (S\cap T)\sqcup
(T\setminus S)$.  So by a second application of the addition principle,
\[
|S\cup T|+|S\cap T|
= |S\setminus T| + |S\cap T| + |T\setminus S| + |S\cap T|.
\]
Hence $|S|+|T|=|S\cup T| + |S\cap T|$.
\end{proof}

%%% honestly, there are too many overloaded words in mathematics
\PMlinkescapeword{algebraic}
\PMlinkescapeword{basis}
\PMlinkescapeword{branch}
\PMlinkescapeword{branches}
\PMlinkescapeword{completes}
\PMlinkescapeword{configurations}
\PMlinkescapeword{connections}
\PMlinkescapeword{core}
\PMlinkescapeword{divide}
\PMlinkescapeword{field}
\PMlinkescapeword{identities}
\PMlinkescapeword{proposition}
\PMlinkescapeword{structure}
\PMlinkescapeword{structures}
\PMlinkescapeword{subfields}
\PMlinkescapeword{theory}
\PMlinkescapeword{type}
\PMlinkescapeword{words}

%%%%%
%%%%%
\end{document}
