\documentclass[12pt]{article}
\usepackage{pmmeta}
\pmcanonicalname{WeightbalancedBinaryTreesAreUltrametric}
\pmcreated{2013-03-22 13:28:31}
\pmmodified{2013-03-22 13:28:31}
\pmowner{mathcam}{2727}
\pmmodifier{mathcam}{2727}
\pmtitle{weight-balanced binary trees are ultrametric}
\pmrecord{9}{34045}
\pmprivacy{1}
\pmauthor{mathcam}{2727}
\pmtype{Result}
\pmcomment{trigger rebuild}
\pmclassification{msc}{05C05}

% this is the default PlanetMath preamble.  as your knowledge
% of TeX increases, you will probably want to edit this, but
% it should be fine as is for beginners.

% almost certainly you want these
\usepackage{amssymb}
\usepackage{amsmath}
\usepackage{amsfonts}

% used for TeXing text within eps files
%\usepackage{psfrag}
% need this for including graphics (\includegraphics)
%\usepackage{graphicx}
% for neatly defining theorems and propositions
%\usepackage{amsthm}
% making logically defined graphics
%%%\usepackage{xypic}

% there are many more packages, add them here as you need them

% define commands here
\begin{document}
Let $X$ be the set of leaf nodes in a weight-balanced binary tree. Let the distance between leaf nodes be identified with the weighted path length between them. We will show that this distance metric on $X$ is ultrametric.  

Before we begin, let the join of any two nodes $x,y$, denoted $x \lor y$, be defined as the node $z$ which is the most immediate common ancestor of $x$ and $y$ (that is, the common ancestor which is farthest from the root). Also, we are using weight-balanced in the sense that

\begin{itemize}
\item the weighted path length from the root to each leaf node is equal, and 
\item each subtree is weight-balanced, too. 
\end{itemize}


\subsection*{Lemma: two properties of weight-balanced trees}

Because the tree is weight-balanced, the distances between any node and each of the leaf node descendents of that node are equal. So, for any leaf nodes $x,y$, 

\begin{equation} 
\label{eq:prop1} 
d(x,x \lor y) = d(y, x \lor y) \end{equation} 

Hence, 
\begin{equation} 
\label{eq:prop2} 
d(x,y) = d(x,x \lor y) + d(y, x \lor y) = 2*d(x, x \lor y) 
\end{equation}  

\subsection*{Back to the main proof}

We will now show that the ultrametric three point condition holds for any three leaf nodes in a weight-balanced binary tree. 

Consider any three points $a,b,c$ in a weight-balanced binary tree. If there exists a renaming of $a,b,c$ such that $d(a,b) \leq d(b,c) = d(a,c)$, then the three point condition holds. Now assume this is not the case. Without loss of generality, assume that $d(a,b) < d(a,c)$. 

Applying Eqn. \ref{eq:prop2}, 

\begin{eqnarray*}
2*d(a, a \lor b) &<& 2*d(a, a \lor c)
\\ d(a, a \lor b) &<& d(a, a \lor c)
\end{eqnarray*} 

Note that both $a \lor b$ and $a \lor c$ are ancestors of $a$. Hence, $a \lor c$ is a more distant ancestor of $a$ and so $a \lor c$ must be an ancestor of $a \lor b$. 

Now, consider the path between $b$ and $c$. \PMlinkescapetext{One way} to get from $b$ to $c$ is to go from $b$ up to $a \lor b$, then up to $a \lor c$, and then down to $c$. Since this is a tree, this is the only path. The highest node in this path (the ancestor of both b and c) was $a \lor c$, so the distance $d(b,c) = 2*d(b,a \lor c)$. 

But by Eqn. \ref{eq:prop1} and Eqn. \ref{eq:prop2} 
(noting that $b$ is a descendent of $a \lor c$), we have

\begin{equation*}
d(b,c) = 2*d(b,a \lor c) = 2*d(a,a \lor c) = d(a,c)
\end{equation*}

To summarize, we have $d(a,b) \leq d(b,c) = d(a,c)$, which is the desired ultrametric three point condition. So we are done.

Note that this means that, if $a,b$ are leaf nodes, and you are at a node outside the subtree under $a \lor b$, then $d(\textrm{you},a) = d(\textrm{you},b)$. In other words, (from the point of view of distance between you and them,) the \PMlinkescapetext{structure} of any subtree that is not your own doesn't matter to you. This is expressed in the three point condition as ``if two points are closer to each other than they are to you, then their distance to you is equal''.

(above, we have only proved this if you are at a leaf node, but it works for any node which is outside the subtree under $a \lor b$, because the paths to $a$ and $b$ must both pass through $a \lor b$).
%%%%%
%%%%%
\end{document}
