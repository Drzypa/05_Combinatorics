\documentclass[12pt]{article}
\usepackage{pmmeta}
\pmcanonicalname{Coloring}
\pmcreated{2013-03-22 12:55:43}
\pmmodified{2013-03-22 12:55:43}
\pmowner{Henry}{455}
\pmmodifier{Henry}{455}
\pmtitle{coloring}
\pmrecord{5}{33283}
\pmprivacy{1}
\pmauthor{Henry}{455}
\pmtype{Definition}
\pmcomment{trigger rebuild}
\pmclassification{msc}{05D10}
\pmsynonym{colouring}{Coloring}
\pmrelated{Partition}
\pmrelated{GraphTheory}

\endmetadata

% this is the default PlanetMath preamble.  as your knowledge
% of TeX increases, you will probably want to edit this, but
% it should be fine as is for beginners.

% almost certainly you want these
\usepackage{amssymb}
\usepackage{amsmath}
\usepackage{amsfonts}

% used for TeXing text within eps files
%\usepackage{psfrag}
% need this for including graphics (\includegraphics)
%\usepackage{graphicx}
% for neatly defining theorems and propositions
%\usepackage{amsthm}
% making logically defined graphics
%%%\usepackage{xypic}

% there are many more packages, add them here as you need them

% define commands here
%\PMlinkescapeword{theory}
\begin{document}
A \emph{coloring} of a set $X$ by $Y$ is just a function $f:X\rightarrow Y$.  The term coloring is used because the function can be thought of as assigning a ``color'' from $Y$ to each element of $X$.

Any coloring provides a partition of $X$: for each $y\in Y$, $f^{-1}(y)$, the set of elements $x$ such that $f(x)=y$, is one element of the partition.  Since $f$ is a function, the sets in the partition are disjoint, and since it is a total function, their union is $X$.
%%%%%
%%%%%
\end{document}
