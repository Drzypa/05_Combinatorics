\documentclass[12pt]{article}
\usepackage{pmmeta}
\pmcanonicalname{CriterionForANearlinearSpaceBeingALinearSpace}
\pmcreated{2013-03-22 14:32:47}
\pmmodified{2013-03-22 14:32:47}
\pmowner{kshum}{5987}
\pmmodifier{kshum}{5987}
\pmtitle{criterion for a near-linear space being a linear space}
\pmrecord{9}{36094}
\pmprivacy{1}
\pmauthor{kshum}{5987}
\pmtype{Theorem}
\pmcomment{trigger rebuild}
\pmclassification{msc}{05B25}

\endmetadata

% this is the default PlanetMath preamble.  as your knowledge
% of TeX increases, you will probably want to edit this, but
% it should be fine as is for beginners.

% almost certainly you want these
\usepackage{amssymb}
\usepackage{amsmath}
\usepackage{amsfonts}
\usepackage{mathrsfs}
% used for TeXing text within eps files
%\usepackage{psfrag}
% need this for including graphics (\includegraphics)
%\usepackage{graphicx}
% for neatly defining theorems and propositions
%\usepackage{amsthm}
% making logically defined graphics
%%%\usepackage{xypic}

% there are many more packages, add them here as you need them

% define commands here
\begin{document}
\PMlinkescapeword{groups}
\PMlinkescapeword{group}
\PMlinkescapeword{partition}

{\bf Theorem}

Suppose $\mathscr{S}$ is near-linear space with $v$ points and $b$ lines, and $s_i$ is the number of points in the $i$th line, for $i=1,\ldots, b$. Then
\[
\sum_{i=1}^b s_i(s_i-1) \leq v(v-1),
\] 
and equality holds if and only if $\mathscr{S}$ is a linear space.

\paragraph{Proof}

Let $N$ be the number of ordered pairs of points that are joined by a line. Clearly $N$ can be no more than $v(v-1)$, and $N=v(v-1)$ if and only if every pair of points are joined by a line. Since two points in a near-linear space are on at most one line, we can label each pair by the line to which the two points belong to. We thus have a partition of the $N$ pairs into $b$ groups, and each group is associated with a distinct line. The group corresponding to the line consisting of $s_i$ points contributes $s_i(s_i-1)$ to the total sum. Therefore
\[
 \sum_{i=1}^b s_i(s_i-1) = N \leq v(v-1).
\]
%%%%%
%%%%%
\end{document}
