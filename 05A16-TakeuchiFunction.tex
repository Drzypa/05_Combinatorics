\documentclass[12pt]{article}
\usepackage{pmmeta}
\pmcanonicalname{TakeuchiFunction}
\pmcreated{2013-03-22 17:33:07}
\pmmodified{2013-03-22 17:33:07}
\pmowner{PrimeFan}{13766}
\pmmodifier{PrimeFan}{13766}
\pmtitle{Takeuchi function}
\pmrecord{4}{39957}
\pmprivacy{1}
\pmauthor{PrimeFan}{13766}
\pmtype{Definition}
\pmcomment{trigger rebuild}
\pmclassification{msc}{05A16}

% this is the default PlanetMath preamble.  as your knowledge
% of TeX increases, you will probably want to edit this, but
% it should be fine as is for beginners.

% almost certainly you want these
\usepackage{amssymb}
\usepackage{amsmath}
\usepackage{amsfonts}

% used for TeXing text within eps files
%\usepackage{psfrag}
% need this for including graphics (\includegraphics)
%\usepackage{graphicx}
% for neatly defining theorems and propositions
%\usepackage{amsthm}
% making logically defined graphics
%%%\usepackage{xypic}

% there are many more packages, add them here as you need them

% define commands here

\begin{document}
The {\em Takeuchi function} is a triply recursive 3-parameter function originally defined by Ichiro Takeuchi in 1978 as $t(x, y, z) = y$ if $x \leq y$ and $t(x, y, z) = t(t(x - 1, y, z), t(y - 1, z, x), t(z - 1, x, y))$ otherwise. Later John McCarthy simplified the definition of the function as $t(x, y, z) = y$ if $x \leq y$, $t(x, y, z) = z$ if $y \leq z$ and $t(x, y, z) = x$ in all other cases.

For example, $t(194, 13, 5) = 194$ since 194 is not less than 13, and 13 is not less than 5. The return value of the function ``is on no practical significance,'' but the function itself ``is useful for benchmark testing of programming languages.'' (Finch, 2003) The function $T(x, y, z)$ is the number of times $t$ calls itself to obtain the return value. A properly optimized implementation of the function in a given programming language should not require more recursion than $T$ indicates.

\begin{thebibliography}{1}
\bibitem{sf} Steven R. Finch {\it Mathematical Constants} New York: Cambridge University Press (2003): 321
\end{thebibliography}
%%%%%
%%%%%
\end{document}
