\documentclass[12pt]{article}
\usepackage{pmmeta}
\pmcanonicalname{RamseysTheorem}
\pmcreated{2013-03-22 12:55:49}
\pmmodified{2013-03-22 12:55:49}
\pmowner{mathcam}{2727}
\pmmodifier{mathcam}{2727}
\pmtitle{Ramsey's theorem}
\pmrecord{9}{33285}
\pmprivacy{1}
\pmauthor{mathcam}{2727}
\pmtype{Theorem}
\pmcomment{trigger rebuild}
\pmclassification{msc}{05D10}
\pmrelated{RamseysTheorem2}
\pmrelated{GraphTheory}

% this is the default PlanetMath preamble.  as your knowledge
% of TeX increases, you will probably want to edit this, but
% it should be fine as is for beginners.

% almost certainly you want these
\usepackage{amssymb}
\usepackage{amsmath}
\usepackage{amsfonts}

% used for TeXing text within eps files
%\usepackage{psfrag}
% need this for including graphics (\includegraphics)
%\usepackage{graphicx}
% for neatly defining theorems and propositions
%\usepackage{amsthm}
% making logically defined graphics
%%%\usepackage{xypic}

% there are many more packages, add them here as you need them

% define commands here
%\PMlinkescapeword{theory}
\begin{document}
\emph{Ramsey's theorem} states that a particular arrows relation,

$$\omega\rightarrow(\omega)^n_k$$

holds for any integers $n$ and $k$.

In words, if $f$ is a function on sets of integers of size $n$ whose range is finite then there is some infinite $X\subseteq\omega$ such that $f$ is constant on the subsets of $X$ of size $n$.

As an example, consider the case where $n=k=2$, and $f\colon [\omega]^2\rightarrow\{0,1\}$ is defined by:

$$f(\{x,y\})=\left\{
\begin{array}{ll}
1&\text{ if ~} x=y^2 \text{ or ~} y=x^2\\
0&\text{otherwise}
\end{array}\right.$$

Then let $X\subseteq\omega$ be the set of integers which are not perfect squares.  This is clearly infinite, and obviously if $x,y\in X$ then neither $x=y^2$ nor $y=x^2$, so $f$ is homogeneous on $X$.
%%%%%
%%%%%
\end{document}
